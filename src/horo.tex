
% Натальная карта. Лучше оформить как пакет. Но пока и так норм.
% Использует пакет keyval. подключать его придется в главном файле.
%\usepackage{keyval}

% Номер знака зодиака от Лагны в \natal
% Возможно лучше использовать length
\newcounter{Signum}

\makeatletter
	% Доступные ключи из \natal
	\define@key{natal}{asc}{\setcounter{Signum}{#1}}
	\define@key{natal}{one}{\def\nt@one{#1}}
	\define@key{natal}{two}{\def\nt@two{#1}}
	\define@key{natal}{three}{\def\nt@three{#1}}
	\define@key{natal}{four}{\def\nt@four{#1}}
	\define@key{natal}{five}{\def\nt@five{#1}}
	\define@key{natal}{six}{\def\nt@six{#1}}
	\define@key{natal}{seven}{\def\nt@seven{#1}}
	\define@key{natal}{eight}{\def\nt@eight{#1}}
	\define@key{natal}{nine}{\def\nt@nine{#1}}
	\define@key{natal}{ten}{\def\nt@ten{#1}}
	\define@key{natal}{eleven}{\def\nt@eleven{#1}}
	\define@key{natal}{twelve}{\def\nt@twelve{#1}}

	% Доступные ключи из \planets
	\define@key{planets}{asc}{\def\pl@asc{#1}}
	\define@key{planets}{su}{\def\pl@su{#1}}
	\define@key{planets}{mo}{\def\pl@mo{#1}}
	\define@key{planets}{ma}{\def\pl@ma{#1}}
	\define@key{planets}{me}{\def\pl@me{#1}}
	\define@key{planets}{ju}{\def\pl@ju{#1}}
	\define@key{planets}{ve}{\def\pl@ve{#1}}
	\define@key{planets}{sa}{\def\pl@sa{#1}}
	\define@key{planets}{ra}{\def\pl@ra{#1}}
	\define@key{planets}{ke}{\def\pl@ke{#1}}

	% Значения ключей по умолчанию
	\setkeys{natal}{asc=1}

	% Определим новую команду \natal
	\newcommand{\natal}[1][] {{
		% Распарсим аргументы
		\setkeys{natal}{#1}

		\parindent=0
		\begin{center}
			\begin{tikzpicture}[scale=1.0]
				\draw (-8,5) -- (8,5);
				\draw (-8,-5) -- (8,-5);
				\draw (-8,-5) -- (-8,5);
				\draw (8,-5) -- (8,5);
				\draw (-8,-5) -- (8,5);
				\draw (-8,5) -- (8,-5);
				\draw (8,0) -- (0,-5) -- (-8,0) -- (0,5) -- (8,0);

				% Раскидаем знаки по домам от асцендента
				% Asc (I)
				\draw (0,0.5) node{\arabic{Signum}}; % (I)

				% (II)
				\ifthenelse{\equal{\theSignum}{12}}{\setcounter{Signum}{1}}{\addtocounter{Signum}{1}}
				\draw (-4,3) node{\arabic{Signum}};

				% (III)
				\ifthenelse{\equal{\theSignum}{12}}{\setcounter{Signum}{1}}{\addtocounter{Signum}{1}}
				\draw (-4.7,2.5) node{\arabic{Signum}};

				% (IV)
				\ifthenelse{\equal{\theSignum}{12}}{\setcounter{Signum}{1}}{\addtocounter{Signum}{1}}
				\draw (-0.7,0) node{\arabic{Signum}};

				% (V)
				\ifthenelse{\equal{\theSignum}{12}}{\setcounter{Signum}{1}}{\addtocounter{Signum}{1}}
				\draw (-4.7,-2.5) node{\arabic{Signum}};

				% (VI)
				\ifthenelse{\equal{\theSignum}{12}}{\setcounter{Signum}{1}}{\addtocounter{Signum}{1}}
				\draw (-4,-3) node{\arabic{Signum}};

				% (VII)
				\ifthenelse{\equal{\theSignum}{12}}{\setcounter{Signum}{1}}{\addtocounter{Signum}{1}}
				\draw (0,-0.5) node{\arabic{Signum}};

				% (VIII)
				\ifthenelse{\equal{\theSignum}{12}}{\setcounter{Signum}{1}}{\addtocounter{Signum}{1}}
				\draw (4,-3) node{\arabic{Signum}};

				% (IX)
				\ifthenelse{\equal{\theSignum}{12}}{\setcounter{Signum}{1}}{\addtocounter{Signum}{1}}
				\draw (4.7,-2.5) node{\arabic{Signum}};

				% (X)
				\ifthenelse{\equal{\theSignum}{12}}{\setcounter{Signum}{1}}{\addtocounter{Signum}{1}}
				\draw (0.7,0) node{\arabic{Signum}};

				% (XI)
				\ifthenelse{\equal{\theSignum}{12}}{\setcounter{Signum}{1}}{\addtocounter{Signum}{1}}
				\draw (4.7,2.5) node{\arabic{Signum}};

				% (XII)
				\ifthenelse{\equal{\theSignum}{12}}{\setcounter{Signum}{1}}{\addtocounter{Signum}{1}}
				\draw (4,3) node{\arabic{Signum}};

				% Раскидаем планеты по домам. К сожалению есть ограничение на количество аргументов команды.
				% Поэтом все 12 домов определены как отдельные переменные \houseone, \housetwo, и.т.д.
				% (I)
				\draw (0,2.5) node{\begin{minipage}[c]{6em}\nt@one\end{minipage}};

				% (II)
				\draw (-4,4.1) node{\begin{minipage}[c]{6em}\nt@two\end{minipage}};

				% (III)
				\draw (-6.5,2.5) node{\begin{minipage}[c]{6em}\nt@three\end{minipage}};

				% (IV)
				\draw (-4,0) node{\begin{minipage}[c]{6em}\nt@four\end{minipage}};

				% (V)
				\draw (-6.5,-2.5) node{\begin{minipage}[c]{6em}\nt@five\end{minipage}};

				% (VI)
				\draw (-4,-4.1) node{\begin{minipage}[c]{6em}\nt@six\end{minipage}};

				% (VII)
				\draw (0,-2.5) node{\begin{minipage}[c]{6em}\nt@seven\end{minipage}};

				% (VIII)
				\draw (4,-4.1) node{\begin{minipage}[c]{6em}\nt@eight\end{minipage}};

				% (IX)
				\draw (6.5,-2.5) node{\begin{minipage}[c]{6em}\nt@nine\end{minipage}};

				% (X)
				\draw (4,0) node{\begin{minipage}[c]{6em}\nt@ten\end{minipage}};

				% (XI)
				\draw (6.5,2.5) node{\begin{minipage}[c]{6em}\nt@eleven\end{minipage}};

				% (XII)
				\draw (4,4.1) node{\begin{minipage}[c]{6em}\nt@twelve\end{minipage}};

			\end{tikzpicture}
		\end{center}
		}}


		\newcommand{\planets}[1][] {{
			% Распарсим аргументы
			\setkeys{planets}{#1}

			\parindent=0

			\begin{table}[tph!]
				\centering

				% Заполним данными
				\begin{tabular}{ll|ll}
					Асцендент & \pl@asc & Юпитер & \pl@ju \\
					Солнце    & \pl@su  & Венера & \pl@ve \\
					Луна      & \pl@mo  & Сатурн & \pl@sa \\
					Марс      & \pl@ma  & Раху   & \pl@ra \\
					Меркурий  & \pl@me  & Кету   & \pl@ke \\
				\end{tabular}
			\end{table}
		}}

\makeatother

