Существуют две модели Солнечной системы: гелеоцентрическая и геоцентрическая. Гелеоцентрическая модель Солнечной системы --- это та модель, которая реально существует в космическом пространстве: вокруг Солнца движутся планеты. Все астрономические открытия делались на основе гелеоцентрической модели мироздания.

Астрология рассматривает движения планет в основе геоцентрической модели Солнечной системы, то есть в центр Солнечной системы условно ставится Земля и при этом движение планет относительно земного наблюдателя происходит вокруг Земли против часовой стрелки.

Древние риши определили, что весь спектр идей, относящихся к человеку и его деятельности, представлен девятью планетами, поэтому в Индийской предсказательной астрологии используются знания только об этих планетах.

Планеты бывают благоприятными и неблагоприятными по своей природе. Юпитер и Венера --- благоприятные, Сатурн, Марс, Раху, Кету и Солнце --- неблагоприятные. Меньше всего зла исходит от Солнца. Луна и Меркурий считаются нейтральными по своей природе и могут быть благоприятными и неблагоприятными в зависимости от их расположения на карте рождения. Луна растущая благоприятна, убывающая --- наоборот, неблагоприятна. Меркури, находящийся в знаке Зодиака один или в связи с благоприятной планетой, будет благоприятным. Если он находится в одном знаке Зодиака с какой-либо неблагоприятной планетой, то тоже становится неблагоприятным.

Планеты разделены на мужские, женские и нейтральные.

\begin{table}[tph!]
	% Расширить по вертикали
	\renewcommand{\arraystretch}{1}

	% Заполним данными
	\begin{tabular}{l|l|l}
		Мужские планеты & Солнце, Марс, Юпитер & активность, мужское начало \\
		Женские планеты & Луна, Венера         & мягкость, женское начало \\
		Нейтральные планеты & Меркурий, Сатурн & проявляют оба начала в соответствии \\
		                     & Раху и Кету      & с полом знака Зодиака \\
	\end{tabular}
\end{table}


Планеты кроме Раху и Кету соответствуют семи дням недели:

\begin{table}[tph!]
	% Расширить по вертикали
	\renewcommand{\arraystretch}{1}

	% Заполним данными
	\begin{tabular}{ll}
		Воскресенье & Солнце \\
		Понедельник & Луна \\
		Вторник     & Марс \\
		Среда       & Меркурий \\
		Четверг     & Юпитер \\
		Пятница     & Венера \\
		Суббота     & Сатурн \\
	\end{tabular}
\end{table}

В Индийской предсказательной астрологии каждая планета несет определенные идеи в жизнь человека. Знание их необходимо для правильного прочтения карты рождения.

\begin{myenum}[topsep=0]
	\item \textbf{Солнце:}
		\begin{mydescr}
			\item[Физиология] --- сердце, мозг, голова, кости, голосовые связки, правый глаз.
			\item[Направление] --- восток.
			\item[Идея] --- индивидуальная душа (эго), отец, представители власти, способность к самоосознанию, статус, уверенность, благородство, слава, сознание, правда, истина, храбрость, решительность, энергия, политика, химия, медицина, врачи, хирурги, целители, храмы и места поклонений, места жертвоприношений, залы для коронования, осветительные приборы, высокий пост в правительстве.
		\end{mydescr}
	\item \textbf{Луна:}
		\begin{mydescr}
			\item[Физиология] --- молочные железы, матка, кровообращение, левый глаз.
			\item[Направление] --- северо-запад.
			\item[Идея] --- мать, женщина, известность, имя, ум, вода, океан, море, реки, путешествие и путешественники, лекарственные растения, сочные плоды, женственность, фантазии, перемены, молоко, популярные курорты, фармацевты и аптекари, зеркало, рыболов, грациозность, чувства.
		\end{mydescr}
	\item \textbf{Марс:}
		\begin{mydescr}
			\item[Физиология] --- нос, мышечная ткань, сухожилия, половые органы.
			\item[Направление] --- юг.
			\item[Идея] --- братья и сестры, решительность, твердость, действия, благосостояние, недвижимость, энергия, сила, земельный участок, пожар, военные, военная операция, режущие инструменты, инженеры, математика, электроника, калькулятор, хирурги, катастрофы, горы, леса, тяжба, спор, спорт, землетрясение, раны и травмы, жар, сапфиры, кораллы, сокровища, химическая лаборатория, война, битва, соревнование, строительство, сила духа, холодное и огнестрельное оружие, ожоги, должность в армии и полиции, конструктор, хирургическая операция, разрыв, разрушение, несчастный случай.
		\end{mydescr}
	\item \textbf{Меркурий:}
		\begin{mydescr}
			\item[Физиология] --- легкие, кишечник, брюшная полость, язык, кисти рук, нервные центры.
			\item[Направление] --- север.
			\item[Идея] --- интеллект, интеллектуальная деятельность, речь, литературные способности, средства информации и связи, рационализм, секретарская работа, бухгалтер, ученый, учебные заведения, головная и желудочная боль, рекламные агенства, издательство и издатели, исследование, новости, коммерческая деятельность, друзья.
		\end{mydescr}
	\item \textbf{Юпитер:}
		\begin{mydescr}
			\item[Физиология] --- печень, нижняя часть брюшной полости, органы слуха, таз.
			\item[Направление] --- северо-восток.
			\item[Идея] --- поведение, красноречие, мудрость, философия, духовный рост, религия, духовный наставник, учитель, ведические знания, набожность, религиозные учреждения, благотворительность, профессор, склонность к полноте, синтез, аргументация, получивший высокую оценку, дети, банки и банкиры, легальная деятельность, министры, юристы, справедливость, почтенные и уважаемые люди, советники, священники, беременные женщины, муж, поддерживающий жизнь.
		\end{mydescr}
	\item \textbf{Венера:}
		\begin{mydescr}
			\item[Физиология] --- почки, яичники, выделения, половая система.
			\item[Направление] --- юго-восток
			\item[Идея] --- желания, привязанность, жадность, ревность, праздность, красота, живописные места, текстильная продукция, люди искусства, товары ``люкс'', спиртные напитки, окружающая среда, спальня, комфорт, сексуальные удовольствия, венерические болезни, парфюмерия, транспортные средства, предметы роскоши, цветы, ботаника, творческие способности, муж, жена, любовник и любовница, украшения, декорации, театры, музеи, кино, сладости, ювелирные украшения.
		\end{mydescr}
	\item \textbf{Сатурн:}
		\begin{mydescr}
			\item[Физиология] --- ноги, колени, костный мозг, мочевой пузырь, дыхательная система, принцип передвижения.
			\item[Направление] --- запад
			\item[Идея] --- время, вызывающеее изменения, смерть, лишения, результат длительных действий, аскетизм, горе, печаль, подлость, обман, воры, мошенники, слуги, работа в подчинении кого-то, рабочий, самоотречение, пожилые люди, уход в отставку, места захоронения, судебный исполнитель, нищий, настенные или настольные часы, отсрочка, заблуждение, ложный вывод, уголовный розыск, черная магия, тайные знания, практика йоги, изменение обстоятельств, разорение и крах, предательство, сила влияния на других, нефть, железо, люди низкого происхождения, должности, получаемые за услуги, консерватизм, сфера услуг, препятствие.
		\end{mydescr}
	\item \textbf{Раху:}
		\begin{mydescr}
			\item[Физиология] --- скулы, кожа, выделительная система, глотание, пищеварительный тракт, прямая кишка.
			\item[Направление] --- юго-запад
			\item[Идея] --- страсть, толпа, мятеж, восстание, ядовитые рептилии, бедствие, наводнение, люди низкой культуры, воры и мошенники, гнев, гордыня, тупость, глупость, мистические знания, таинства, тайная доктрина, грубость, путешествия, муравейник, злое предсказание, эпидемии, насилие, коррупция, незаконные действия, охотники, тяжба, инфекционные заболевания, припадок, удушье, способность, оказывать влияние на других, лишения, крупное воровство.

Раху действует подобно Сатурну и провляет энергии тех планет, которые оказывают на нее влияние.
		\end{mydescr}
	\item \textbf{Кету:}
		\begin{mydescr}
			\item[Физиология] --- позвоночник, спинно-мозговой канал, нервная система, половая система.
			\item[Направление] --- направления не имеет/
			\item[Идея] --- препятствия и помехи, несчастный случай, духовные силы, психическое состояние, астрология, банкротство, клевета, оккультизм, падение, болезнь и недомогание, огонь, математика и мат. способности, эпидемии, страх, нервозность, испуг, яд, больничная палата, мелкое воровство, неожиданность, рана.

Кету действует подобно Марсу и проявляет энергии тех планет, которые оказывают на нее влияние.
		\end{mydescr}
\end{myenum}

Главные идеи девяти планет очень важно знать напамять при анализе карты рождения. А также следует не забывать, что сильные и удачно расположенные планеты в карте рождения проявят свои хорошие качества, а слабые и неудачно расположенные планеты --- дадут плохой результат.
