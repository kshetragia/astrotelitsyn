\tocsection{ЗАКЛЮЧЕНИЕ}

Исторические записи сохранили расчеты джотишей, или астрологов, сделавших фантастические предсказания, которые не только оказались правильными, но и поразили современников сверхъестественными по точности указанием времени происшествия событий и яркости описания их качеств. Однако в них имеются и свидетельства об ошибках, допущенных профессионалами с огромным практическим опытом. Дело в том, что знания, полученные только опытным путем и определяемые в основном индивидуальными представлениями астролога, имеют видимую ограниченность. Есть сложные случаи, где таких знаний становится недостаточно, чтобы проникнуть в тайны природы или найти ключ для интерпретации гороскопа. Джотиши тогда применяют технику медитации, чтобы выйти за пределы чувств и получить ответы на свои вопросы в трансцедентальной области сознания.

Все люди, в большей или меньшей степени, наделены чувствами или суперчувствами, но благодаря культивации этих энергий, их можно обострить и развить. Чистота помыслов, ограничение мирских желания, благодеяния --- это основа этики астролога, необходимая для развития его ума и чувств. Но наиболее действенным методом для пробуждения в сознании ведических знаний является трансцедентальная медитация, предложенная учителем Махариши Махеш Йоги, о которой мы упоминали в начале этой книги.

С древних времен в Индии бытует притча о человеке, который вошел в свой дом и в темноте, наступив на веревку, принял ее за змею. Человек испугался, выскочил на улицу и позвал слуг. Один из слуг взял светильник и отправился в дом. Осветив комнату, он обнаружил вместо змеи веревку и успокоил хозяина. Суть притчи в том, что свет знания освобождает человека от страха и страданий.

Махариши Махеш Йоги в комментариях к стиху 45 главы II Бхагавад-Гиты пишет: ``Современные психологические теории исследуют причины, чтобы влиять на следствия. Они ощупью двигаются в темноте, чтобы найти причину темноты и устранить ее. В противопоставление здесь высказывается мысль принести свет, чтобы устранить тьму. Это ``принцип второго элемента''. Если вы хотите воздействовать на первый элемент, не занимайтесь этим элементом, не ищите его причину, влияйте на него непосредственно, вводя второй элемент. Прогоните темноту, принеся свет. Перенесите ум в поле счастья, чтобы избавить его от страдания''.

Принеся Джотиш--Веду --- ``свет знания'' в свою жизнь, человек разгонит темноту неведения о цели своего существования и путей, способствующих естественному развитию индивидуального процесса эволюции.

По мнению великого немецкого врача и естествоиспытателя Парацельса, ``Бог создал планеты и звезды не для того, чтобы они ему служили и были полезны, как все другие создания''.

Человек доминирует среди живых существ на земле, так как он обладает врожденным стремлением к знаниям и способностью их использовать для создания идеальных условий своего существования. В невообразимом коридоре времени звезды --- наши вреные проводники, которые дают возможность рассчитать время и регулировать свою деятельность в согласии с природой, чтобы жизнь могла быть более счастливой, мирной, а общество высокодуховным, культурно развитым.
