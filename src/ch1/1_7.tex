\section{Ключ к пониманию расположения хозяев домов}

Следующий уровень анализа гороскопа, рассматриваемый в этой главе, очень важен, индийские астрологи придают ему особое значение.

Солнце, Луна, Марс, Меркурий, Юпитер, Венера, Сатурн имеют собственные знаки:

\begin{mylist}
	\item Солнце --- Лев
	\item Луна --- Рак
	\item Марс --- Овен и Скорпион
	\item Меркурий --- Близнецы и Дева
	\item Юпитер --- Стрелец и Рыбы
	\item Венера --- Телец и Весы
	\item Сатурн --- Козерог и Водолей
\end{mylist}

Солнце и Луна имеют по одному собственному знаку, а остальные пять планет по два --- это значит, что планеты являются их хозяевами. Дом, который находится в определенном знаке Зодиака, будет подчинен планете --- хозяину этого знака, то есть планета будет хозяином дома, расположенного в собственном знаке этой планеты. Напрмер, Марс в карте рождения стоит во Льве в первом доме, а знаки Скорпион и Овен соответствуют четвертому и девятому домам, значит Марс является хозяином четвертого и девятого домов, и дела этих домов будут контролироваться Марсом.

В гороскопе хозяин какого-либо дома может располагаться в любом из двеннадцати домов и будет передавать энергию от одного дома к другому. Например, хозяин первого дома находится в пятом доме, следовательно, личностные качества человека получат развитие в его интеллектуальной деятельности или в воспитании детей.

Если хозяин какого-либо дома находится в одном и том же знаке с какой-либо планетой, то это означает, что хозяин дома связан с ней. Например, при восходящем Льве Юпитер пребывает в Стрельце в пятом доме вместе с Венерой. В данном примере Юпитер является хозяином пятого и восьмого домов, значит идеи этих домов будут связаны с идеями Венеры, что может указывать на красивых детей и благоприятные изменения в их материальной жизни.

Если хозяин дома находится под аспектом планеты, то это также надо учитывать. Например, Венера стоит в Тельце в десятом доме, а Юпитер в Скорпионе в четвертом. В данном примере Венера является хозяином третьего и десятого домов и испытывает влияние Юпитера, значит идеи, представленные третьим и десятым домами, будут нести на себе отпечаток идей Юпитера, то есть деятельность человека приведет его к наилучшим результатам, и он приобретет высокий общественный статус.

Если хозяин данного дома находится в одном знаке с хозяином другого дома или аспектирован им, то такая комбинация также анализируется. Например, связь хозяев первого и десятого домов приводит к мысли об ориентации человека на профессиональную деятельность, результат которой будет отражен тем домом, где присутствуют хозяева этих домов. Если хозяева первого и десятого домов находятся в пятом доме, то успех будет сопутствоавать делам пятого дома, то есть преподавательской или научной деятельности.

Другой пример: если хозяин первого дома находится под аспектом хозяина девятого дома, то поступки человека скорее всего будут праведными или личностные качества его получат развитие в религиозной деятельности. Связь хозяев домов из более чем двух планет, будет представлять многообразие явлений, управляемых этими хозяевами.

Ниже мы представим различные действия хозяев двеннадцати домов.

\subsubsection*{Хозяин первого дома}

Если первый дом находится в подвижном знаке и хозяин этого дома также в подвижном знаке, то человек будет ориентирован на разностороннюю деятельность.

Если первый дом находится в неподвижном знаке и хозяин первого дома также в неподвижном знаке, то человек будет ориентирован на устоявшиеся традиции.

Если первый дом находится в двойственном знаке и хозяин первого дома также в двойственном знаке, то личность человека и его действия будут изменчивыми.

Если хозяин первого дома занимает шестой, восьмой или двеннадцатый дом и связан с Солнцем, Марсом, Сатурном, Раху, Кету, то это указывает на ослабление здоровья.

Если хозяин первого дома сам является неблагоприятной планетой и занимает шестой, восьмой, или двеннадцатый дом, то это также подрывает здоровье человека. Исключение составляют позиции Марса в первом и шестом домах. Исключение составляют позиции Марса в первом и шестом домах при восходящем Скорпионе и в первом и восьмом домах при восходящем Овне. Позиция Сатурна в первом и двеннадцатом домах при восходящем Водолее.

Если хозяин первого дома находится в первом, четвертом, пятом, седьмом, девятом или десятом домах, то это благоприятно для здоровья (особенно, если хозяин первого дома связан или аспектирован благотворной планетой).

Если хозяином первого дома становится неблагоприятная планета и она находится в первом доме, то это не повлечет за собой болезнь тела.

Если хозяин первого дома стоит в шестом доме, то человека охватывают тревоги на ментальном уровне.

Если восходящими знаками являются Близнецы или Дева, а Меркури находится в Овне, Тельце, Раке, Скорпионе или Стрельце, то человек будет иметь слабое телосложение.

Если хозяин первого дома занимает дома квадранта или тригона и связан с благотворной планетой, то человек будет жить в хорошем месте и, наоборот, если хозяин первого дома занимает шестой восьмой или двеннадцатый дом и связан с неблагоприятной планетой, то эффект будет прямо противоположный.

Характер человека зависит от планетной природы хозяина первого дома, а также от того, какой дом он занимает. Например, при восходящей Деве хозяином первого дома является Меркурий, если он находится в Близнецах в десятом доме, значит характер человека будет легким, изменчивым и деятельным.

Если хозяин первого дома занимает второй дом, то человек больше ориентирован на деньги, богатство, семейные отношения.

Если хозяин первого дома занимает четвертый дом, то действия человека направлены на развитие дружеских отношений, приобретение недвижимости, транспортных средств и\,т.\,д.

Если хозяин первого дома занимает седьмой дом, то человек найдет свое место в общественной жизни и будет счастлив в браке и любви (если хозяин первого дома будет связан с Венерой, то выше перечисленные признаки будут доминировать в судьбе человека).

Если хозяин первого дома занимает восьмой дом и связан с благотворной планетой, то человек может обладать глубокими познаниями в психологии и изотерике. Если связь будет с неблагоприятной планетой, то действия его могут быть недостойны законов морали.

\subsubsection*{Хозяин второго дома}

Для накопления богатства благоприятно расположение хозяина второго дома в любом доме кроме третьего, восьмого и одиннадцатого дома. Если хозяин второго дома будет связан с Меркурием, Венерой или Юпитером, это также хорошо дл яидей второго дома.

Когда хозяин второго дома связан с Сатурном или Раху, то человек может разбогатеть незаконным путем.

Расположение хозяина второго дома в шестом, восьмом или двеннадцатом доме является неблагоприятным и не дает возможности добиться материального достатка, а также мешает осуществлению финансовых операций.

Если хозяин второго дома занимает шестой дом, то человек будет страдать зубной болью.

Если хозяин второго дома находится в восьмом доме, то человек получит наследство, но не будет удачи в накоплении денег.

Если хозяин второго дома стоит в двеннадцатом доме, то это предсказывает утрату кого-либо из членов семьи, а также денежные расходы на благотворительность.

Если хозяин второго дома занимает десятый дом, то человек сделает материальные накопления благодаря карьере.

Если хозяин второго дома расположен в пятом доме, человек получит денежные средства через детей или образовательную сферу деятельности.

\subsubsection*{Хозяин третьего дома}
Если хозяин третьего дома и Марс находятся в шестом, восьмом или двеннадцатом доме (необязательно вмесет), то  отношения с братьями и сестрами будут плохими.

Связь хозяина третьего дома с благоприятной планетой в третьем доме указывает на хорошие отношения с братьями и сестрами.

Если хозяин третьего дома занимает восьмой дом, то это часто приводит к смерти брата, особенно если хозяин третьего дома связан с неблагоприятной планетой или испытывает на себе ее аспект. Подобный результат получается, есл изозяин третьего дома находится в двеннадцатом доме.

Если хозяин третьего дома располагается в домах квадранта или тригона и тут же находится Марс, то человек будет жить в полном согласии со своими братьями и сестрами.

Если хозяин третьего дома стоит в десятом доме, то эта позиция является благоприятноя для людей, работающих в издательстве и занимающихся писательским трудом (при условии, что Меркурий будет находиться в хорошем положении).


\subsubsection*{Хозяин четвертого дома}

Когда хозяин четвертого дома связан с хозяином первого, пятого или девятого дома и это происходит в четвертом доме, человек неожиданно приобретает недвижимое имущество, а если расположение Марса будет благоприятным, то оно к тому же, будет значительным.

Хозяин четвертого дома связан с хозяином пятого или девятого дома и стоит в одиннадцатом доме --- эта комбинация дает большие шансы на приобретение автомобиля.

Если хозяин четвертого дома ослаблен по знаку, находится в шестом, восьмом или двеннадцатом доме и Марс в плохом положении, то человек будет страдать из-за отсутствия хороших жилищных условий.

\subsubsection*{Хозяин пятого дома}
Если хозяин пятого дома связан с благоприятными планетами или аспектирован ими, то все условия жизни детей будут улучшены. Эта комбинация благоприятна также для образования и развития интеллектуальных способностей человека.

Плохо для идей пятого дома, когда его хозяин занимает шестой, восьмой или двеннадцатый дом.

Расхоложение хозяина пятого дома в десятом может предсказывать карьеру преподавателя или научного работника, все зависит от других показателей гороскопа.

Позиция хозяина пятого дома в двеннадцатом доме может явиться причиной отдаления своего ребенка (например, сын или дочь может уехать в другую страну).

Хозяин пятого дома в мужском знаке указывает чаще на сыновей, в женском на дочерей.

Если хозяин пятого дома занимает третий дом, то человек может самостоятельно получать знания из книг (особенно если хозяин пятого дома находится под аспектом Меркурия).

Если хозяин пятого дома стоит в девятом, то человек может стать автором научного труда.

Всегда будет благоприятным расположения хозяина пятого дома в домах квадранта и тригона.

Если хозяин пятого дома стоит в шестом доме, то человек будет испытывать большое беспокойство от детей.

\subsubsection*{Хозяин шестого дома}

Если хозяин шестого дома сильнее хозяина первого дома, то враги будут иметь больше шансов одержать победу над данным человеком, его организм будет плохо сопротивляться болезням. Если хозяин первого дома сильнее хозяина шестого дома, то все будет наоборот.

Если хозяин шестого дома связан с Солнцем или Раху и расположен в двеннадцатом доме, то человек будет жить в доме других людей и аморально вести себя.

Расположения хозяина шестого дома в первом, шестом, седьмом или восьмом доме приносит человеку победу над врагами.

Если хозяин шестого дома находится в третьем или одиннадцатом доме, то он будет применять насилие и станет источником страданий других людей.

Если хозяин шестого дома расположен во втором или двенадцатом доме, то такая комбинация указывает на недоброжелательного человека.

Если хозяин шестого дома стоит в пятом или девятом доме, то ребенок этого человека будет противостоять ему.

Расположение хозяина шестого дома в четвертом или десятом доме указывает на боли в нижней части живота или в области пупка.

Если хозяин шестого дома находится в домах квадранта и тригона, то он негативно влияет на дела, управляемые этими домами.

Если хозяин шестого дома расположен в двенадцатом доме, то эта комбинация может привести к болезни.


\subsubsection*{Хозяин седьмого дома}
Не только планеты в седьмом доме, Венера и Юпитер (в женском гороскопе) представляют партнера по браку, но и хозяин седьмого дома. Его связь с планетами является дополнительным фактором в определении брачных дел и характера супругов.

Если хозяин седьмого дома связан с Солнцем или им аспектирован, то это указывает на обладающего властью или доминирующего партнера по браку.

Если хозяин седьмого дома связан с Луной или аспектирован ею, то это дает эмоционального партнера по браку, а также может указывать на перемену в брачной жизни, явиться причиной не одного брака. Если седьмой дом находится под влиянием благоприятных планет, то брак будет счастливым.

Когда хозяин седьмого дома связан с Марсом или им аспектирован, это предсказывает страстного или обладающего большой физической силой партнера по браку, брачная жизнь может стать сложной в зависимости от других влияний на седьмой дом и на Венеру.

Связь хозяина седьмого дома с Меркурием или аспект последнего на него, пророчит сообразительного и моложавого партнера по браку, а также дает шанс повторного брака.

Если хозяин седьмого дома связан с Юпитером или им аспектирован, это указывает на богатого или духовно близкого партнера по браку, а также на благополучный брак (если седьмой дом и Венера будут под влиянием благотворных планет).

Если хозяин седьмого дома связан с Венерой или аспектирован ею, то это говорит о красивом, моложавом, артистичном, склонном к флирту партнере по браку. Если другие указания в гороскопе будут хорошими, то такая позиция даст счастливый брак.

Если хозяин седьмого дома связан с Сатурном или аспектирован им, это свидетельствует о том, что супруг (или супруга) будет старше по возрасту, педантичным, консервативным. Брак может быть запоздалый или омраченный жизненными невзгодами.

Если хозяин седьмого дома связан с Раху или Кету или аспектирован ими, то брачные условия будут нарушены. При других неблагоприятных влияниях на седьмой дом и Венеру эта комбинация приведет к расторжению брачного контракта.

Расположение хозяина седьмого дома в первом доме является благоприятным для брака.

Если хозяин седьмого дома стоит во втором доме, то партнер по браку будет сильно ориентирован на материальные и семейные дела.

Когда хозяин седьмого дома пребывает в седьмом доме, то партнер по браку будет уделять много времени самому себе.

Если хозяин седьмого дома находится в шестом доме, то партнер по браку будет беспокойным или болезненным человеком.

Если хозяин седьмого дома расположен в восьмом доме, то условия супружеской жизни ухудшаются.

Будучи в девятом доме, хозяин седьмого дома свидетельствует о том, что партнер по браку бедт религиозным человеком.

Хозяин седьмого дома в десятом доме --- партнер по браку будет сильно ориентирован на повышение своего общественного статуса и карьеру.

Если хозяин седьмого дома находится в одиннадцатом доме, то эта комбинация предполагает самостоятельного партнера по браку.

Если хозяин седьмого дома расположен в двенадцатом доме, то это указывает на неустойчивость брачной жизни, которая впоследствии приводит к разводу.


\subsubsection*{Хозяин восьмого дома}
Если хозяин восьмого дома находится в восьмом доме, то жизнь человека будет долгой.

Если хозяева первого и восьмого домов занимают одновременно один из домов квадранта или тригона, то это указвыает на долгожителя (особенно если оба хозяина находятся под влиянием Юпитера).

Если квадранты заняты балгоприятными планетами, а хозяин первого дома в комбинации с благотворной планетой и под аспектом Юпитера, то это положениу указывает на хорошую долгую жизнь.

Хозяева певого, восьмого и десятого домов в комбинации с Сатурном занимают квадранты --- жизнь будет долгой.

Если хозяин восьмого дома сильный и находится под аспектом благоприятных планет, то продолжительность жизни не будет вызывать тревоги.

Если хозяин третьего дома в третьем доме, то это хорошо для продолжительности жизни.

Если хозяин восьмого дома расположен в третьем доме, то это также хорошо для продолжительности жизни.

Если хозяин восьмого дома стоит в двенадцатом доме вместе с неблагоприятными планетами, то это является указанием на короткую жизнь.

Если хозяева первого и восьмого домов находятся в шестом доме, то такая комбинация не свидетельствует о долгой жизни.

Если хозяин восьмого дома находится в одиннадцатом доме, то жизнь человека будет счастливая в более поздние годы.

Если хозяйкой восьмого дома является неблагоприятная планета и она не имеет связи с благоприятной планетой, то такая позиция уменьшает продолжительность жизни.

Если хозяин восьмого дома занимает второй дом, то это обычно говорит об утрате кого-либо из членов семьи.

Если хозяин восьмого дома находится в третьем доме, то будут плохие отношения с братьями и сестрами, если нет благоприятных влияний на третий дом.

ПозицияЖ хозяин восьмоо дома в четвертом доме часто пророчит утрату кого-либо из родственников.

Хозяин восьмого дома в девятом доме и хозяин восьмого дома в десятом доме --- данная позиция является указанием на смерть отца.

\subsubsection*{Хозяин девятого дома}
Если хозяин девятого дома находится в первом доме, то поступк человека будут праведными.

Если хозяин девятого дома стоит во втором доме, то будут удачно сделаны денежные накопления.

Если хозяин девятого дома расположен в третьем доме, то это указывает на успех в писательской карьере.

Нахождение хозяин девятого дома в четвертом доме приносит удачу в делах, связанных с недвижимостью, а также матерью, друзьями.

Расположение хозяина девятого дома пребывает в шестом доме, то это указывает на исполнительный характер человека и на его неудачные действия.

Если хозяин девятого дома находится в десятом доме, то такая позиция --- большая удача для профессиональных начинаний.

Если хозяин девятого дома занимает двенадцатый дом, то это говорит о возможной потере отца.

\subsubsection*{Хозяин десятого дома}
Если хозяин десятого дома расположен в первом доме, то человек будет честолюбив, и все его действия будут направлены на укрепление своего общественного статуса.

Связь хозяина десятого дома с Венерой предсказывает удачную карьеру в искусстве, швейном производстве и вообще благоприятствует профессиональным делам.

Хозяин десятого дома, связанный с Сатурном, указывает на тяжелый труд и трудности в карьере.

Если хозяин десятого дома связан с Меркурием, то человек может сделать карьеру в бизнесе, коммуникациях, на писательском поприще и вообще это --- успех в профессиональных делах.

Когда хозяин десятого дома связан или аспектирован хозяином двенадцатого дома, то следует ожидать трудности или изменения в карьере.

Если хозяин десятого дома находится в третьем доме, то занятия человека будут направлены на духовное совершенствование, особенно ярко это проявится, когда есть аспект Юпитера на хозяина десятого дома.

Хозяин десятого дома расположен в двенадцатом доме --- слабая позиция мирских дел.


\subsubsection*{Хозяин одиннадцатого дома}
Если хозяин одиннадцатого дома сильный, расположен в одном из домов квадранта или тригона и находится под влиянием Юпитера или Венеры, то это указывает на высокие заработки человека.

Когда хозяин одиннадцатого дома слабый, расположен в шестом, восьмом или двенадцатом доме и находится под влиянием неблаготворных планет, то человек будет страдать из-за низких заработков.

Если хозяин одиннадцатого дома стоит в первом доме, то это указывает на коммуникабельного дружелюбного человека, у которого время от времени будут появляться новые друзья.

Позици хозяин одиннадцатого дома во втором доме свидетельствует о различных материальных приобретениях и благоприятных финансовых возможностях.

Если хозяин одиннадцатого дома занимает седьмой дом, то доход появится благодаря взаимовыгодным партнерским отношениям.

Если хозяин одиннадцатого дома расположен в десятом доме, то это показатель хорошего дохода от деятельности человека.


\subsubsection*{Хозяин двенадцатого дома}
Дом, в котором находится хозяин двенадцатого дома и дом, на который он влияет, укажет на что будут растрачиваться денежные средства. Если хозяин двенадцатого дома будет испытывать влияние благотворных планет, то деньги уйдут на хорошие дела, если неблаготворных планет --- они будут растрачены впустую.

Если хозяин двенадцатого дома связан с Венерой, то растраты ожидаются на партнера по браку, чувственные удовольствия.

Если хозяин двенадцатого дома связан с Юпитером, то деньги будут отданы детям или вложены в религиозную деятельность.

Если хозяин двенадцатого дома связан с Меркурием, то убытки будут понесены из-за друзей или дел с другими людьми.

Если хозяин двенадцатого дома связан с Марсом, то денежные средства будут направлены на братьев и сестер или на судебные процессы и спорные дела.

Если хозяин двенадцатого дома связан с Луной, то ожидается много расходов из-за матери, если с Солнцем --- из-за отца.

Если хозяин двенадцатого дома связан с Сатурном и хозяином шестого дома, то деньги уйдут на лечение или врагов.

Если хозяин двенадцатого дома связан с Раху, то человек понесет убытки из-за нечистых на руку людей.

Нахождение хозяина двенадцатого дома в девятом доме указывает на религиозного или духовно возвышенного человека.

Позиция хозяин двенадцатого дома в двенадцатом доме показывает, что деньги будут идти на благородные дела.

Когда хозяин двенадцатого дома находится в десятом доме, это ухудшает положение в профессиональной деятельности и ведет к большим расходам.

Мы показали на примерах некоторые позиции хозяев домов и то, как можно получить информацию из индивидуальных гороскопов, пользуясь предоставленным ключом. В будущем практические навыки и личная интуиция помогут вам понять этот уровень анализа гороскопа.
