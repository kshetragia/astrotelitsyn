\section{Лунная карта и ее назначение}

Когда вы составите карту рождения, то обратите внимание на знак Зодиака, в котором находится Луна. Это второй восходящий знак. Индийские астрологи придают большое значение Луне, так как она руководит умом и миром эмоций человека.

Примите знак Зодиака, в котором находится Луна, за первый дом и составьте лунную карту. Например, в карте рождения восходящий знак в Близнецах, солнце в Стрельце в седьмом доме, Луна в Водолее в девятом доме, Марс в стрельце в седьмом доме, Меркурий в Стрельце в седьмом доме, Юпитер в Рыбах в десятом доме, Венера в Козероге в восьмом доме, Сатурн в Козероге в восьмом доме, Раху в Близнецах в первом доме и Кету в стрельце в седьмом доме. Так как Луна расположена в Водолее, то в лунной карте первый дом будет в Водолее, второй --- в Рыбах, третий --- в Овне, четвертый --- в Тельце, пятый --- в Близнецах, шестой --- в Раке, седьмой --- во Льве, восьмой --- в Деве, девятый -- в Весах, десятый --- в Скорпиона, одинадцатый --- в Стрельце и двенадцатый --- в Козероге. Исходя из выше приведенного примера карты рождения позиции планет в лунной карте будут следующие: Луна в Водолее в первом доме, Солнце в Стрельце в одинадцатом доме, Марс в Стрельце в одинадцатом доме, Меркурий в Стрельце в одинадцатом доме, Юпитер в Рыбах во втором доме, Венер в Козероге в двенадцатом доме, Сатурн в Козероге в двенадцатом доме, Раху в Близнецах в пятом доме и Кету в Стрельце в одинадцатом доме.

Лунная карта интерпретируется так же, как и карта рождения, то есть рассматриваются планеты в домах, расположение хозяев домов и на какие дома оказывают влияние планеты.

Начинающие астрологи часто задают вопрос, какая карта является приоритетной: карта рождения или лунная карта. Астрологи Индии считают: если восходящий знак будет сильнее Луны (восходящий знак под влиянием нескольких планет или они будут расположены в первом доме, а Луна в нейтральном, враждебном или ослабленном знаке), то карта рождения важнее чем лунная. И наоборот, если Луна будет сильнее, чем восходящий знак (Луна в дружесвенном знаке или знаке экзальтации и под влиянием различных планет, а восходящий знак не будет иметь подобных влияний), то ведущая роль принадлежит лунной карте.

Наша практика подсказывает, что обе карты представляют ценный материал для определения личностных качеств, способносте, привычек и различных жизненных сфер человека. Карта рождения укажет на один ряд событий, а лунная карта --- на другой. Например, по карте рождения мы определяем, что человек должен вступить в брак, и в то же самое время видим в лунной карте приятное путешествие, значит человек зарегистрирует свой брак, а потом поедет в свадебное путешествие.

Если на какое-либо событие (действие, характер) указывает карта рождения и это подтверждает лунная карта, то данной событие (действие, характер) будет усилено в определенный период времени или станет доминирующим в жизни.
