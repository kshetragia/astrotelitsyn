\section{Ключ к синтезу идей планет и домов}

Эта глава является очень важной для понимания гороскопа в целом и особенно для предсказания будущих событий. Планеты, находящиеся в домах карты рождения, указывают на сильные и слабые стороны различных сфер жизни. Любая планета в доме возбуждает идеи этого дома, а также приносит идеи, присущие ее природе. С помощью девяти планет, находящихся в определенных домах карты рождения, вы сможете понять судьбу человека  и достаточно точно предсказать ее.

Если Юпитер, Венера, Луна и Меркурий будут находиться в домах квадранта и/или тригона, то жизнь человека будет счастливой и бесхлопотной.

Если Сатурн, Марс, Раху, Кету и Солнце будут находиться в 8 и/или 12-м домах, то человек испытывает в своей жизни большие трудности, преодолеет препятствия. Самой хорошей позицией для этих планет является 6-й дом.

Когда благоприятные и нейтральные планеты находятся в 6,\,8,\,и\,12-м домах, они улучшают эти дома, но сами планет ослабляются. Если естественно неблагоприятные планеты сильны по знаку и находятся в домах квадранта и/или тригона, они принесут удачу человеку через его действия, если слабы по знаку --- неудачу и препятствия.

Присутствие планет во 2-м и 11-м домах указывает на материальное благополучие и заработки человека.

Когда планеты находятся в 3-м доме, они наделяют человека большой энергией, силой воли, способностями к писательскому труду или журналистской работе в средствах массовой информации.

Окончательные выводы астролог должен делать после тщательного изучения гороскопа, рассмотрев его со всех сторон. Когда естественно благоприятная планета сильна по знаку и находится в квадранте или тригоне, она обязательно создаст хорошие условия этому дому, и человек испытает счастье в тех сферах жизни, которыми управляет этот дом.

Если естественно неблагоприятные планеты ослаблены по знаку, то они окажут негативный эффект на жизнь человека в соответствии с тем домом, в котором находятся.

Когда естественно благоприятная планета сильна по знаку и дому, но находится под воздействием нескольких естественно неблагоприятных планет, то ее добрая сила уменьшается.

Аспекты планет на дома гороскопа могут улучшать или ухудшать дела этих домов. Ниже мы предлагаем ключ к синтезу идей планет и 12-ти домов гороскопа.

\subsubsection*{Первый дом}

Первый дом --- самый важный в карте рождения. Он указывает на характер, поведение и главные направления деятельности человека. Если первый дом сильный, то в характере человека будут преобладать качества незаурядной, яркой личности. Сила дома оценивается по планетам, занимающим этот дом, а также по аспектам, которые они на него оказывают. Если первый дом находится в основном под влиянием естественно благоприятных плаент, то характер и поведение человека будут достойны уважения, и люди с доверием будут относиться к нему. Когда первый дом находится под влиянием естественно неблагоприятных планет, то человек может отличаться грубостью, агрессивным поведением, и конституция его тела будет несколько ослаблена.

\begin{myenum}
	\item Солнце --- Внушительная внешность и способности лидера.
	\item Луна --- Привлекательная внешность и мягкие черты характера.
	\item Марс --- Воинствующее поведение, шрамы на теле или голове, вызывающий вид.
	\item Меркурий --- Дружеские отношения, легкость в общении, здравый смысл в любых действиях.
	\item Юпитер --- Величественная наружность, врожденное чувство справедливости, здоровье, интерес к религиозным занятиям.
	\item Венера --- Красивое тело, склонность к чувственным удовольствиям и изящным искусствам, любвеобильность, теплота и нежность к другим людям.
	\item Сатурн --- Серьезный вид, депрессия, печаль, озабоченность, продуманные действия, глубокие мысли.
	\item Раху --- Страстный характер, критическое настроение, сильные чувства.
	\item Кету --- Болезненность, слабая конституция тела.
\end{myenum}

Первый дом показывает наскоолько в человеке развито чувство патриотизма. Влияния естественно благоприятных планет на этот дом привязывают человека к своей родине и приносят ему успех там, где он родился.

Влияния Солнца, Марса и Раху на первый дом дает физическую силу. Для полного анализа первого дома вы должны рассмотреть Солнце со всех точек зрения, так как эта планета является определяющей в личности человека. Если Солнце хорошо расположено в гороскопе и под благоприятным влиянием других планет (а в первом доме есть добрые планеты, или они оказывают влияние на этот дом), то личность человека будет значительной, и он будет пользоваться уважением среди людей.

Более сложен для понимания результат первого дома при смешанных влияниях на него. Надо смотреть, сколько планет благотворных и неблаготворных влиют на первый дом, и предпочтение отдавать тем, которые сильнее. Например, первый дом находится в Стрельце, Меркурий в седьмом доме в Близнецах, а Сатурн в одиннадцатом доме в Весах. Меркурий имеет силу собственного знака, а Сатурн --- знака экзальтации, поэтому влияние Сатурна на первый дом будет сильнее, чем Меркурия, хотя на личности будут сказываться аспекты и Сатурна и Меркурия одновременно.

Ментальная энергия также относится к первому дому, чтобы понять, каким уровнем мышления обладает человек, необходимо рассмотреть положение Луны(ум) и Меркурия (интеллект).

Планеты в первом доме или их влияния на этот дом указывают на способности человека в соответствии с их естественной природой и природой того знака, в котором находится первый дом.

\subsubsection*{Второй дом}

По второму дому мы определяем богатство, финансовое положение, семейные дела и другие сферы жизни, управляемые этим домом.

Если Венера и Юпитер вместе находятся во втором доме, то это явно улучшает семейные и финансовые дела, и все зависит от того, насколько эти планеты или одна из них сильны по знаку. Например, Венера и Юпитер находятся в Рыбах во втором доме. Венера в знаке экзальтации, а Юпитер в собственном знаке, поэтому сила второго дома велика, и человека при таком расположении планет ждет гармоничная семейная жизнь и прекрасной финансовое положение. Друго пример: Венера и Юпитер находятся в Деве во втором доме. Венера ослаблена по знаку, Юпитер во враждебном знаке, и все условия, представленные вторым домом, будут положительными относительно силы этих планет.

Если естественно неблагоприятная планета находится во втором доме, но сильна по знаку, то финансовое положение будет хорошим, но в семье будут ощущаться трения. Если неблагоприятная планета ослаблена по знаку и стоит во втором доме, то человека ждут материальные трудности, и его отношения с членами семьи будут крайне напряженными. Например, Марс в Козероге во второ доме даст хорошее финансовое положение, хотя будут споры в семье. Положение Марса в Раке во втором доме создаст невыносимые семейные взаимоотношения и ослабление материальных возможностей.

По второму дому мы определяем манеру говорить человека и, если в этом доме находятся благоприятные планеты или оказывают на него свое влияние, то речь будет мягкой и дипломатичной. Когда во втором доме пребывают или влияют на него неблагоприятные планеты, то речь становится грубой, критичной и лживой. Смешанное влияние планет дает смешанный результат: например, Венера и Марс находятся во втором доме --- речь сладкая, но временами бывает критичной.

По второму дому видно, как питается человек. Если во втором доме стоит Юпитер, Венера или Луна, то продукты питания у него будут высокого качества. Венера в это доме указывает на пристрастие к сладостям, Марс --- на хороший аппетит, острую и горячую пищу, Раху --- на человека, которые будет неразборчив в еде.

Чтобы понять состояние финансовых дел человека, необходимо рассмотреть Юпитер и то влияние, которое на него оказывают другие планеты. Сильные второй дом и Юпитер наградят человека большим богатством и успехом в финансовых операциях. Каждая планета во втором доме будет выражать свою естественную природу. Например, Меркурий обусловит получение заработков от лекций или публичных выступлений, Марс свидетельствует о вероятности денежного вознаграждения за операции с недвижимостью или энергичную финансовую деятельность. Но если планета, находящаяся во втором доме, будет под аспектом другой планеты и та окажется сильнее, то предпочтение в выводах второго дома надо отдать более сильной планете, хотя и боле еслабая планета будет определенным образом оказывать влияние на этот дом.


\subsubsection*{Третий дом}

Главные идеи третьего дома --- это братья, сестры, поездки на небольшие расстояния, средства информации и связи. Сильный третий дом дает человеку большую выносливость и силу воли. Если в этом доме более двух планет и одна из них неблаготворная, то человек будет наделен терпеливостью.

Если одиннадцатый дом управляет старшими братьями и сестрами, то третий --- братьями и сестрами вообще, независимо от их возраста.

Марс с Сатурном или Раху в третьем доме указывают на жесткие отношения с братом или сестрой, которые часто приводят к открытой вражде. Если Марс находится в третьем доме вместе с Венерой или Юпитером, то толкование будет другим: отношения с братом или сестрой хорошие, но есть вероятность их гибели от хирургического вмешательства или несчастного случая. Все зависит от того, какие дополнительные влияния испытывает дом. Если третий дом находится под влиянием благотворных планет, то идеи его будут в основном процветать, а если под вилянием неблаготворных планет, то можно будет сказать, что третий дом серьезно поврежден, и человеку понадобится проявить большое упорство и терпение для достижения какого-либо результата.

Для того чтобы глубже понять взаимоотношения с братом и сестрой, необходимо рассмотреть Марс и те явления, которые он испытывает, а также сделать полный анализ третьего дома.

Под аспектом злых планет Раху и Сатурна Марс предсказывает на повреждение рук от оружия или огня. Луна в третьем доме указывает в основном на путешествия и возможности, связанные с написанием каких-либо текстов, а также на то, что брат или сестра будут иметь мягкий характер и привлекательную внешность. Юпитер с Венерой в третьем доме обуславливают полную гармонию в отношениях с братом или сестрой. Сатурн и Раху в третьем доме являются причиной потери как младшего, так и старшего брата.

Женские планеты в третьем доме больше говорят о сестрах, а мужские --- о братьях. Аспекты мужских и женских планет на третий дом также надо учитывать. Раху в этом доме указывает на путешествия, напряженный ручной труд, плохие привычки братьев или сестер.

\subsubsection*{Четвертый дом}

Четверты дом представляет мать, недвижимое имущество, личный транспорт, степень образованности, друзей, комфортные условия жизни и\,т.\,д. По нему мы определяем, насколько человек будет восприимчив. Естественно благоприятные планеты в этом доме указывают на хорошее жилье, прекрасных друзей, комфортную жизнь, на большие возможности для получения образования (в отличие от образования, получаемого по пятому дому, которое указывает на личные интеллектуальные достижения, здесь оно рассматривается как удача в этой жизни), на жизненные силы матери и отношения с ней. когда в четвертом доме находятся естественно неблагоприятные планеты, то результат будет противоположный. Но если сильные Марс или Сатурн стоят в этом доме, то человек будет иметь хорошие возможности для покупки дома или квартиры, чего нельзя сказать относительно приобретения личного транспорта.

Юпитер в этом доме указывает на духовно близких и честных друзей, Венера --- на друзей, связанных с искусством, Меркурий --- на друзей, занимающихся интеллектуальной деятельностью, слабый Сатурн или Раху --- на неблагоприятное окружение и\,т.\,д.

Сильная Венера в четвертом доме свидетельствует о приобретении личного автомобиля, а также о любви к природе.

Чтобы составить полное представление о жизни матери, необходимо рассмотреть этот дом вместе с Луной, так как эта планета руководит матерью. Если Луна испытывает большое влияние благотворных планет и находится в 1,\,4,\,5,\,7,\,9 или 10-м домах, то карта рождения содержит указание на хорошие условия жизни матери или на ее достойное место в обществе. Не забывайте анализировать аспекты на четвертый дом!

\subsubsection*{Пятый дом}

Этот дом является благоприятным, поскольку относится к домам тригона. Он представляет детей, образование, науку, авторство и многие другие идеи, которые не являются главными.

Меркурий в пятом доме в Близнецах, Деве или Водолее указывает на высокообразованного человека. Марс в пятом доме в воздушных знаках, кроме Весов, представляет человека, занимающегося наукой.

Все благоприятные планеты в этом доме или их аспекты создают хорошие условия для жизни детей. Неблагоприятные планеты провоцируют напряжения в отношениях с детьми. Если они ослаблены по знаку и не имеют поддержки благотворных планет, то могут привести к смерти ребенка. Вопросы, связанные с детьми, рассматриваются по пятому дому и Юпитеру. Если большее влияние на пятый до оказывают женские планеты, то рождаются девочки, если мужские --- мальчики. Если Марс в карте рождения находится в пятом доме в огненном знаке, то ребенок будет предприимчив и амбициозен. Марс в пятом доме в Раке указывает на раздражительного ребенка, он будет не раз травмирован. Если Марс стоит в пятом доме и находится под аспектом Сатурна, это может быть причиной смерти ребенка от несчастного случая.

Ослабленный Сатурн в пятом доме уеньшает интеллектуальные способности человека.

Сильная Венера в этом доме приносит красивого ребенка, а также делает его любвеобильным, обуреваемым мирскими желаниями. Юпитер указывает на его жизненные силы.

Раху в пятом доме является причиной абортов и тяжелых родов в женском гороскопе. Если Раху и Кету находятся в Овне, Тельце или Ракев пятом доме они не создают препятствий детям на их жизненном пути. В остальных знаках эти теневые планеты будут проявлять себя как проиводействующие силы.

Пятый дом отвечает за пищеварение и, если в этом доме находятся естественно неблагоприятные планеты, они вредят процессу переваривания пищи.

Марс в пятом доме обусловливает язву желудка, особенно, если он находится под влиянием неблаготворных планет. Солнце в этом же доме провоцирует боли в желудке. Вы не должны забывать, что аспекты неблаготворных планет оказывают подобные же действия.

\subsubsection*{Шестой дом}

Это неблагоприятный дом для Юпитера, Венеры, Меркурия и Луны. Для остальных планет нейтральный. Шестой до представляет болезни, хирургические операции, недоброжелателей и врагов, препятствия в жизни и\,т.\,п.

Естественно, неблагоприятные планеты в шестом доме дают силы для сопротивления врагам, но могут создать проблемы в отношении здоровья, так как поражают этот дом.

Меркурий, находящийся здесь, является причиной беспокойства, причиняемого недоброжелателями.

Луна в это доме создает ментальное беспокойство, особенно если она находится под влиянием неблагоприятных планет.

Юпитер и Венера в шестом доме явно смягчают все обстоятельства, связанные с недружелюбно настроенными людьми, и дают человеку хорошее здоровье. Однако они не в силах обеспечить ему комфортную жизнь.

\subsubsection*{Седьмой дом}

Благоприятный, так как относится к домам квадранта. Планеты, находящиеся в этом доме, обусловливают общественную и супружескую жизнь в соответствии со своей природой.

Солнце в этом доме делает человека заметным в общественной среде, но привносит определенные трудности в его брачные отношения.

Луна в седьмом доме создает условия для частых поездок. Если он находится под аспектом благоприятных планет, то является причиной счастливого брака.

Марс при таком расположении укажет на яркую личность, но в браке этому человеку будут сопутствовать частые ссоры. Если Марс находится под аспектом неблагоприятных планет, то это может привести к разводу. Многое зависит и от знака Зодиака, занимаемого Марсом. Например, Марс в Козероге представляет более благоприятную позицию для седьмого дома, чем марс в Раке. Как вы уже знаете, это является главным правилом в индийской астрологии для определения силы или слабости планет.

Меркурий в седьмом доме укажет на человека, легко вступающего в контакты, а также а перемену брачных связей.

Юпитер в этом доме покровительствует счастливому браку при условии, что он не испытывает влияния неблаготворных планет.

Сильная Венера, находящаяся в седьмом доме, представляет человека всецело ориентированного на сексуальную жизнь, а также на красивого мужа или жену. Если, например, Венера находится под влиянием Юпитера, то партнер по браку может быть религиозным человеком. Здесь все будет зависеть от других аспектов на Венеру и седьмой дом.

Сатурн при таком расположении обычно указывает на партнера по браку, который будет старше по возрасту. Он является неблагоприятной планетой, и поэтому условия брачной жизни будут не лучшими.

Раху в седьмом доме --- хорошо для владельца гороскопа, но плохо для его супружеской жизни, так как Раху проявляет себя подобно Сатурну.

Кету в этом доме создает препятствия в брачных отношениях и приносит неудачи, но для реализации себя в общественной жизни эта позиция планеты является благоприятной.

Если в седьмом доме нет планет, но он находится под влиянием благотворных планет, то брак будет удачным, если находится под влиянием неблагоприятных планет, то условия супружеской жизни будут ослаблены, если на этот дом будут влиять одновременно благотворные и неблаготворные планеты, то брак будет временами хорошим, временами плохим. В гороскопе женщины для определения брачных условий жизни и характера мужа вы должны рассмотреть седьмой дом, Венеру и Юпитера (Юпитер, как и Венера, в женском гороскопе представляет мужа). Если седьмой дом, Венера и Юпитер будут в основном находиться под аспектами благотворных планет, то брачные узы крепки, а если под аспектами неблаготворных планет, то у мужа будет плохой характер и брак окажется весьма шатким.

В гороскопе мужчины все брачные условия жизни и характер жены предопределяют седьмой дом и Венера.

венера в знаке Зодиака поможет вам определить душевные силы, поведение и характер партнера по браку. То же по Юпитеру определяется в женском гороскопе. Например, Венера в Овне служит указанием на энергичную и страстную натуру супруги.

\subsubsection*{Восьмой дом}

Это самый неблагоприятный дом гороскопа. Он относится к кончине определенного человека и смерти вообще, то есть смерти других людей, родственников, друзей. Восьмой дом указывает на трансформацию сознания человека и изменение его мировоззрения, а также психическую энергию, на основе которой развивается интуиция. Сильный восьмой дом наделяет даром предсказывания и поэтому руководит астрологией. Все тайные и эзотерические знания находятся в ведении этого дома, а также исследования в данной области. Восьмой дом ставит человека в жесткие условия и заставляет его изменить взгляды на законы мироздания. Он создает препятствия на жизненном пути и поэтому не дает быстро подняться человеку по лестнице общественного статуса.

Если здесь находятся благотворные планеты. то детство у человека будет безрадостным, серым, однако, повзрослев, он станет смелым и достигнет успеха в трудных проектах.

Восьмой дом является домом продолжительности жизни. Вопрос, связанный с определением времени смерти человека, сложный даже для индийских джйотиш--пандитов (астрологов--ученых) и, по мнению некоторых из них, чтобы описать все правила, по которым вычисляется время смерти человека, надо писать целый год. Но есть основные правила, которые берутся во внимание всеми индийскими астрологами для вероятного определения периода смерти человека (часть из них представлена в следующей главе). Здесь мы даем главное правило для определения продолжительности жизни. Если все неблагоприятные планеты находятся в восьмом доме или влияют на него, то шансы прожить долгую жизнь явно уменьшатся, особенно если эти планеты ослаблены по знаку. Благотворные планеты в этом доме или их влияние на него увеличивают продолжительность жизни. Сильный Сатурн в восьмом доме указывает на долгожительство. Знак Зодиака в восьмом доме показывает от какой болезни или разрушения какого органа наступит смерть. Например, Овен в восьмом доме --- смерть наступит от кровоизлияния в мозг или травмы головы, посольку Овен управляет головой и мозгом.

Вопросы, связанные с получением наследства, рассматриваются также по этому дому. Если в восьмом доме находятся Юпитер, Венера, Меркурий, Луна или Марс, то человек столкнется в своей жизни с проблемами наследования имущества после смерти родственников.

Луна представляет ум. Находясь в восьмом доме в момент рождения человека, эта планета будет способствовать тому, что он будет много размышлять о смерти и предназначении людей в этом мире. луна в восьмом доме под аспектом неблагоприятной планеты является причиной страданий матери.

Злой Марс в восьмом доме втягивает человека в судебные процессы, а также ослабляет психическую уравновешенность. Сильный Марс здесь наделяет человека смелостью и бесстрашием, дает после смерти родственников хорошее наследство в виде дома или квартиры.

Сильный Меркурий указывает на знание психологии и не вредит делам, связанным с образованием.

Кету в восьмом доме обусловливает хорошую интуицию, особенно, если она находится вместе с Луной или другими благоприятными планетами.

Венера в восьмом доме под влиянием неблагоприятных планет становится причиной неудачливого партнера по браку или партнера, обладающего плохими чертами характера.

Арест человека также связан с восьмым домом, так как этот дом руководит задержанием вообще.

Юпитер в этом доме указывает на трудно диагностируемые болезни.

Раху в восьмом доме часто представляет воров и людей с воровскими привычками, с которыми по воле судьбы придется столкнуться человеку.

Солнце в этом доме ослабляет волевые качества человека и в период детства делает его замкнутым, если планета находится под влиянием Сатурна или Раху.

\subsubsection*{Девятый дом}

Это благоприятный дом, так как относится к тригону. Идеи девятого дома: отец, религия, удача и неудача, судьба, начальник, дальние путешествия, иностранные языки и\,т.\,д. Он также имеет отношение к процветанию человека, если находится под влиянием сильных или естественно благоприятных планет, то характер отца будет миролюбивым и отзывчивым. Юпитер в нем указывает на долголетие отца, сильный Сатурн --- на чувство отцовского долга. Раху или луна в девятом доме представляют дальние путешествия. Если они будут находиться здесь вместе с Сатурном, то такая позиция указывает на заграничные путешествия. Праведность человека также рассматривается по девятому дому, и влияние благотворных планет подтверждает это.

Позиция Марса, Сатурна или Раху в девятом доме делает человека свободомыслящим. Если Сатурн или Раху стоят в этом доме, но не подвержены аспектам благоприятных планет, то человек далек от религии. Если же они находятся под аспектом Юпитера, то человек, напротив, может серьезно заняться теологией. Венера в Тельце, Весах или Рыбах в этом доме сулит материальный достаток. Если планеты в девятом доме расположены во враждебных или ослабленных знаках и если эти планеты являются неблагоприятными, то в судьбе человека будут большие трудности. Например, Сатурн в Овне в девятом доме приносит неудачу.

Девятый дом также представляет высшие знания и поэтому руководит духовными наставниками. Меркурий в девятом доме под аспектом Юпитера дает прекрасные шансы для получения высших знаний. Посещение церквей и храмов рассматривается по девятому дому. Юпитер вместе с Кету в этои доме наделяет человека сильными религиозными чувствами, и человек совершает паломничество. Сильные Юпитер и девятый дом пророчат человеку карьеру правоведа.

На юге Индии жизнь отца рассматривают по девятому дому, а на севере --- по десятому дому. Наша практика подсказывает, что делать это надо одновременно по девятому и по десятому домам. Синтез результатов этих домов и Солнца даст объективный портрет отца.

\subsubsection*{Десятый дом}

Один из домов квадранта, поэтому десятый дом благоприятный. Представляет профессию, занятие, репутацию, известность, славу, общественное положение, отца, власть и\,т.\,д. По этому дому мы определяем, насколько успешной может быть карьера человека и какое место он займет в обществе. Благотворные планеты в десятом доме или их влияние на этот дом дадут человеку хорошие возможности и условия в профессиональной деятельности. Неблагоприятные планеты в десятом доме, если они находятся во враждебном или ослабляющем знаке, создадут напряжение в профессиональных делах и заставять человека приложить много усилий в этой сфере. Неблаготворные планеты в десятом доме, если они находятся в дружественном или экзальтирующем знаке, принесут большую удачу в карьере, но человеку придется хорошо потрудиться. Природа планеты, находящейся в десятом доме или влияющей на него, будет определять вид профессии. Например, Солнце в Овне в десятом доме указывает на профессию политика, государственного деятеля и, если Солнце не находится под аспектом Сатурна или Раху, то человек может занимать высокий государственный пост в своей стране.

Венера в этом доме свидетельствует о поэтическом даре, музыкальных способностях, работе в текстильной промышленности и\,т.\,д., то есть природа планеты подскажет вид профессиональной деятельности. Также надо обратить внимание на знак Зодиакак в десятом доме. Если десятый дом в Близнецах, то человек больше склонен к интеллектуальным занятиям и, если пятый дом, который руководит образованием, сильный, то он может работать преподавателем в высшем учебном заведении.

Профессии и занятия, соответствующие планетам:
\begin{myenum}
	\item Солнце --- врач, хирург, человек, занимающий пост в правительстве или находящийся на общественной службе.
	\item Луна --- общественный деятель, а также работа с жидкостями, лекарственными травами.
	\item Марс --- инженер, математик, конструктор, человек, занимающий высокую должность в армии и полиции (милиции).
	\item Меркурий --- писатель, журналист, секретарь--машинистка.
	\item Юпитер --- учитель, философ, правовед, советник.
	\item Венера --- работник культуры, человек искусства, модельер, швея.
	\item Сатурн --- рабочий, а также секретные дела.
	\item Раху --- скрытые действия (теневая экономика).
	\item Кету --- духовное развитие (йога).
\end{myenum}

Этот перечень не является исччерпывающим для выше названных планет, но он послужит ключом в определении других профессий. На самом деле редко случается определить профессию по одной планете, в основном для этого используется комбинация планет. Например, Солнце, Луна и Меркурий являются главными энергиями в писательской карьере. Взаимосвязь энергий этих планет с другими планетами указывает на различные вариант приложения интеллектуальных способностей человека. Например, Луна вместе с Юпитером или под его аспектом указывает на религиозное мировоззрение, и эта комбинация подразумевает религиозную деятельность. Меркурий во взаимоотношениях с Сатурном представляет интеллект критика. Меркурий с Юпитером --- карьеру журналиста. Солнце с Венерой одаривают приятным голосом, который может способствовать карьере певца. Юпитер и Марс в паре руководят силой выражения личностных качеств. Юпитер с Сатурном, означающие ``речь во времени'', покровительствуют писателю--фантасту. Сильная Венера (красота), Сатурн (движение, время) в связке с Марсом (гибкость, мускулы) будут полезными для занятий танцами. Венера, Раху и Кету являются преобладающими энергиями для карьеры драматического артиста.

Венера (красота), Юпитер (выразительность) и Сатурн (труд, терпение) важны для людей, работающих в сфере изобразительного искусства. Марс и Сатурн (невидимые энергии) анализируются для определения работы, связанной с электричеством. Энергии Марса и Меркурия часто подходят инженеру--механику.

Чтобы определить профессиональные способности и возможности человека, необходимо сделать анализ десятого дома, Солнца, Меркурия, Юпитера, Сатурна, а потом рассмотреть оставшиеся планеты.

\subsubsection*{Одиннадцатый дом}

Главные идеи одиннадцатого дома --- заработки и доходы. Это единственный дом, в котором все планеты рассматриваются как приносящие доход человеку. Идеи этого дома отличаются от идей второго дома, представляющих богатство и платежеспособность. Одиннадцатый дом следует за десятым домом, поэтому он указывает на заработки в зависимости от профессии. Источники ожидаеых доходов в соответствии с природой планет, занимающих одиннадцатый дом или оказывающий влияние на него:

\begin{myenum}
	\item Солнце --- работа в правительстве, руководящая должность, занятие фотографией, золото, стекло.
	\item Луна --- сотрудничество с женщинами, продукты с водой, серебро, жемчуг.
	\item Марс --- производство, товары, изготавливаемые с помощью огня, служба в армии и полиции (милиции), земельное имущество, инженерное и конструкторское дело.
	\item Меркурий --- работа в учреждениях образования, публикации.
	\item Юпитер --- преподавательская работа, консультативно--совещательная деятельность, религиозная деятельность, правоведение.
	\item Венера --- культура и искусство, одежда и обувь, индустрия развлечений.
	\item Сатурн --- государственная служба, железо, уголь, азартыне игры и аферы.
\end{myenum}

Одиннадцатый дом связывают с природой Юпитера, поэтому его расположение и силу надо определять параллельно с результатами этого дома.

Одиннадцатый дом представляет друзей, которых также рассматривают по четвертому дому.

Сильный одиннадцатый дом указыват на независимого человека и обеспечивает высокий материальный уровень жизни.

\subsubsection*{Двенадцатый дом}

Неблагоприятный, по нему определяются потери, утраты. Индийские астрологи называют этот дом ``домом разрушения''. Двенадцатый дом --- это последний дом в гороскопе, он указывает на завершение эволюционного процесса в данной жизни. Потеря должности или работы рассматривается также по этому дому. Некоторые индийские школы астрологии предсказывают смерть человека по двеннадцатаму дому, не игнорируя вместе с тем идеи восьмого дома.

Если благоприятные планеты занимают двеннадцатый дом, то человек будет совершать благородные поступки и тратить деньги больше на то,  что соответствует природе планеты.

Венера показывает, что денежные средства будут идти на жену или любовницу.

Юпитер покровительствует благотворительной деятельности.

Сатурн указывает на практику аскета. если планета находится во враждебном или ослабленном знаке и под аспектами других неблагоприятных планет, то это может привести к полному материальному краху или к тюремному заключению.

Марс является причиной резкого ухудшения материального благостояния, если находится в плохом знаке. В этом доме он нарушает супружескую гармонию, так как оказывает влияние на седьмой дом.

Меркурий явно уменьшает шансы для получения высшего образования, но дает хороший результат в области трансцедентальных знаний.

Луна в этом доме (особенно, если она убывающая) указывает на ментальные страдания, желание побыть в одиночестве, дальние путешествия, вещие сны.

Раху в двеннадцатом доме становится причиной длительных зарубежных поездок, беспокойных снов, больших расходов.

Кету предопределяет желание человека освободиться от мирской суеты, посвятить себя духовной жизни.

Пребывание в госпитале или в больнице рассматривается также по двеннадцатому дому. Монахи--отшельники имеют четко выраженный двеннадцатый дом.

В этой главе мы дали ключ к пониманию значений планет в домах. Описать полностью результаты расположения планет в домах, которые проявляются в жизни человека, невозможно. Существуют тысячи планетных комбинаций, и каждая из них сугубо индивидуальна. Только гороскоп конкретного человека покажет, что именно означают позиции планет в домах. В книгах по Индийской предсказательной астрологии есть описание результатов планет в домах и знаках Зодиака, но они верны только отчасти, так как не учитывают побочные влияния на планеты и дома.
