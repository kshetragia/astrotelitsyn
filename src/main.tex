\documentclass[a4paper,oneside,12pt]{book}

% headings
\author{Станиславский С., Телицын А.}
\title{Индийская предсказательная астрология}
\date{\today}

% -------------------------------------------------------
% Языки и шрифты
% -------------------------------------------------------
\usepackage[T2A]{fontenc}
\usepackage[utf8]{inputenc}
\usepackage{csquotes}    % babel ругается без этого
\usepackage{indentfirst} % красная строка после заголовков (chapter, *section)
\usepackage[english, russian]{babel}

% Sanskrit support
\usepackage{skt}

% Zodiac and planets support
\usepackage{wasysym} % normal
\usepackage{marvosym} % bold

% Шрифты по умолчанию(computer modern)
\renewcommand{\rmdefault}{cmr}
\renewcommand{\sfdefault}{cmss}
\renewcommand{\ttdefault}{cmtt}

% Добавить бутора
\usepackage{lipsum}

% Заметки на полях FIXme
\usepackage[inline]{fixme}

% Поддержка опциональных аргументов вида ключ=значение, к примеру, для новых команд
\usepackage{keyval}

% поддержка эпиграфов
\usepackage{epigraph}

% поворот чего угодно
\usepackage{lscape}

% -------------------------------------------------------
% Графика
% -------------------------------------------------------
% Картинки
\usepackage[dvips]{graphicx}
\graphicspath{{pics/}}
%\usepackage{wrapfig} % обтекание рисунка текстом.

% Векторная графика
\usepackage{tikz}
\usetikzlibrary{calc}

% -------------------------------------------------------
% Математика
% -------------------------------------------------------
\usepackage{amsmath}

% Русские шрифты в наборе формул
\usepackage{mathtext}
\DeclareSymbolFont{T2Aletters}{T2A}{cmr}{m}{it}

% -------------------------------------------------------
% Настройка оглавления
% -------------------------------------------------------
% Отображать только один уровень без подглав.
% 0 - chapter
% 1 - chapter и section;
% 2 - chapter, section, и subsection;
% 3 - chapter, section, subsection и subsubsection;
% 4 - chapter, section, subsection, subsubsection и paragraph;
\setcounter{tocdepth}{1}

% -------------------------------------------------------
% Set bibliography support
% See the follow:
%    https://ru.sharelatex.com/blog/2013/07/31/getting-started-with-biblatex.html
% -------------------------------------------------------
\usepackage{biblatex}
\addbibresource{books}


% -------------------------------------------------------
% Интервалы, отступы, переносы и т.д.
% -------------------------------------------------------
% page style and geometry
\usepackage{geometry}
  \geometry{left=2cm}
  \geometry{right=1.5cm}
  \geometry{top=1.5cm}
  \geometry{bottom=1.5cm}

% Отступ "красной" строки
\parindent=1cm %red string

% Разрешить перенос слов.
\sloppy

% Удаление висячих строк
\clubpenalty = 10000
\widowpenalty = 10000

% Название разделов по центру
\usepackage{sectsty}
\chapterfont{\centering}
\sectionfont{\centering}
\subsectionfont{\centering}

%\righthyphenmin=2

% полуторный интервал между строк
\usepackage[onehalfspacing]{setspace}

% настройка заголовков
\usepackage{titlesec}

%\titlespacing{\section}{0em}{1em}{0em} % top, left, bottom
%\titlespacing{\subsection}{0em}{1em}{0em}
%\titlespacing{\subsubsection}{0em}{0em}{0em}

% Оформим отдельно работу с натальной картой

% Натальная карта. Лучше оформить как пакет. Но пока и так норм.
% Использует пакет keyval. подключать его придется в главном файле.
%\usepackage{keyval}

% Номер знака зодиака от Лагны в \natal
% Возможно лучше использовать length
\newcounter{Signum}

\makeatletter
	% Доступные ключи из \natal
	\define@key{natal}{asc}{\setcounter{Signum}{#1}}
	\define@key{natal}{one}{\def\nt@one{#1}}
	\define@key{natal}{two}{\def\nt@two{#1}}
	\define@key{natal}{three}{\def\nt@three{#1}}
	\define@key{natal}{four}{\def\nt@four{#1}}
	\define@key{natal}{five}{\def\nt@five{#1}}
	\define@key{natal}{six}{\def\nt@six{#1}}
	\define@key{natal}{seven}{\def\nt@seven{#1}}
	\define@key{natal}{eight}{\def\nt@eight{#1}}
	\define@key{natal}{nine}{\def\nt@nine{#1}}
	\define@key{natal}{ten}{\def\nt@ten{#1}}
	\define@key{natal}{eleven}{\def\nt@eleven{#1}}
	\define@key{natal}{twelve}{\def\nt@twelve{#1}}

	% Доступные ключи из \planets
	\define@key{planets}{asc}{\def\pl@asc{#1}}
	\define@key{planets}{su}{\def\pl@su{#1}}
	\define@key{planets}{mo}{\def\pl@mo{#1}}
	\define@key{planets}{ma}{\def\pl@ma{#1}}
	\define@key{planets}{me}{\def\pl@me{#1}}
	\define@key{planets}{ju}{\def\pl@ju{#1}}
	\define@key{planets}{ve}{\def\pl@ve{#1}}
	\define@key{planets}{sa}{\def\pl@sa{#1}}
	\define@key{planets}{ra}{\def\pl@ra{#1}}
	\define@key{planets}{ke}{\def\pl@ke{#1}}

	% Значения ключей по умолчанию
	\setkeys{natal}{asc=1}

	% Определим новую команду \natal
	\newcommand{\natal}[1][] {{
		% Распарсим аргументы
		\setkeys{natal}{#1}

		\parindent=0
		\begin{center}
			\begin{tikzpicture}[scale=1.0]
				\draw (-8,5) -- (8,5);
				\draw (-8,-5) -- (8,-5);
				\draw (-8,-5) -- (-8,5);
				\draw (8,-5) -- (8,5);
				\draw (-8,-5) -- (8,5);
				\draw (-8,5) -- (8,-5);
				\draw (8,0) -- (0,-5) -- (-8,0) -- (0,5) -- (8,0);

				% Раскидаем знаки по домам от асцендента
				% Asc (I)
				\draw (0,0.5) node{\arabic{Signum}}; % (I)

				% (II)
				\ifthenelse{\equal{\theSignum}{12}}{\setcounter{Signum}{1}}{\addtocounter{Signum}{1}}
				\draw (-4,3) node{\arabic{Signum}};

				% (III)
				\ifthenelse{\equal{\theSignum}{12}}{\setcounter{Signum}{1}}{\addtocounter{Signum}{1}}
				\draw (-4.7,2.5) node{\arabic{Signum}};

				% (IV)
				\ifthenelse{\equal{\theSignum}{12}}{\setcounter{Signum}{1}}{\addtocounter{Signum}{1}}
				\draw (-0.7,0) node{\arabic{Signum}};

				% (V)
				\ifthenelse{\equal{\theSignum}{12}}{\setcounter{Signum}{1}}{\addtocounter{Signum}{1}}
				\draw (-4.7,-2.5) node{\arabic{Signum}};

				% (VI)
				\ifthenelse{\equal{\theSignum}{12}}{\setcounter{Signum}{1}}{\addtocounter{Signum}{1}}
				\draw (-4,-3) node{\arabic{Signum}};

				% (VII)
				\ifthenelse{\equal{\theSignum}{12}}{\setcounter{Signum}{1}}{\addtocounter{Signum}{1}}
				\draw (0,-0.5) node{\arabic{Signum}};

				% (VIII)
				\ifthenelse{\equal{\theSignum}{12}}{\setcounter{Signum}{1}}{\addtocounter{Signum}{1}}
				\draw (4,-3) node{\arabic{Signum}};

				% (IX)
				\ifthenelse{\equal{\theSignum}{12}}{\setcounter{Signum}{1}}{\addtocounter{Signum}{1}}
				\draw (4.7,-2.5) node{\arabic{Signum}};

				% (X)
				\ifthenelse{\equal{\theSignum}{12}}{\setcounter{Signum}{1}}{\addtocounter{Signum}{1}}
				\draw (0.7,0) node{\arabic{Signum}};

				% (XI)
				\ifthenelse{\equal{\theSignum}{12}}{\setcounter{Signum}{1}}{\addtocounter{Signum}{1}}
				\draw (4.7,2.5) node{\arabic{Signum}};

				% (XII)
				\ifthenelse{\equal{\theSignum}{12}}{\setcounter{Signum}{1}}{\addtocounter{Signum}{1}}
				\draw (4,3) node{\arabic{Signum}};

				% Раскидаем планеты по домам. К сожалению есть ограничение на количество аргументов команды.
				% Поэтом все 12 домов определены как отдельные переменные \houseone, \housetwo, и.т.д.
				% (I)
				\draw (0,2.5) node{\begin{minipage}[c]{6em}\nt@one\end{minipage}};

				% (II)
				\draw (-4,4.1) node{\begin{minipage}[c]{6em}\nt@two\end{minipage}};

				% (III)
				\draw (-6.5,2.5) node{\begin{minipage}[c]{6em}\nt@three\end{minipage}};

				% (IV)
				\draw (-4,0) node{\begin{minipage}[c]{6em}\nt@four\end{minipage}};

				% (V)
				\draw (-6.5,-2.5) node{\begin{minipage}[c]{6em}\nt@five\end{minipage}};

				% (VI)
				\draw (-4,-4.1) node{\begin{minipage}[c]{6em}\nt@six\end{minipage}};

				% (VII)
				\draw (0,-2.5) node{\begin{minipage}[c]{6em}\nt@seven\end{minipage}};

				% (VIII)
				\draw (4,-4.1) node{\begin{minipage}[c]{6em}\nt@eight\end{minipage}};

				% (IX)
				\draw (6.5,-2.5) node{\begin{minipage}[c]{6em}\nt@nine\end{minipage}};

				% (X)
				\draw (4,0) node{\begin{minipage}[c]{6em}\nt@ten\end{minipage}};

				% (XI)
				\draw (6.5,2.5) node{\begin{minipage}[c]{6em}\nt@eleven\end{minipage}};

				% (XII)
				\draw (4,4.1) node{\begin{minipage}[c]{6em}\nt@twelve\end{minipage}};

			\end{tikzpicture}
		\end{center}
		}}


		\newcommand{\planets}[1][] {{
			% Распарсим аргументы
			\setkeys{planets}{#1}

			\parindent=0

			\begin{table}[tph!]
				\centering

				% Заполним данными
				\begin{tabular}{ll|ll}
					Асцендент & \pl@asc & Юпитер & \pl@ju \\
					Солнце    & \pl@su  & Венера & \pl@ve \\
					Луна      & \pl@mo  & Сатурн & \pl@sa \\
					Марс      & \pl@ma  & Раху   & \pl@ra \\
					Меркурий  & \pl@me  & Кету   & \pl@ke \\
				\end{tabular}
			\end{table}
		}}

\makeatother



% Begin [re]definiton
\makeatletter
	\renewcommand{\quoteit}[1]{\begin{center}\textit{#1}\end{center}}
	\newcommand{\quotenr}[1]{\begin{center}{#1}\end{center}}

	% Заголовки для введения, к читателю и пр. которые должны быть без номера
	% но при этом присутствовать в содержании на одном уровне с главами.
	% работает где есть chapter-ы (\documentclass{book})
	\newcommand{\tocsection}[1] {
		\section*{#1}
		\addcontentsline{toc}{chapter}{#1}
	}

	% Стихи/цитаты в начале глав/разделов,
	% прижатые к правому краю и выровненые по левому с поддержкой переноса слов.
	\newcommand{\quoteintro}[2]{{\epigraphrule=0em\epigraph{#1}{#2}}}

	% Вывод формул
	\newcommand{\calc}[1]{\[#1\]}

	% Вывод координат
	% ----------------------------------------------
	% Простой градус
	\newcommand{\gradus}[1]{\ensuremath{#1^\circ}}

	% Градус + минуты
	\newcommand{\cord}[2]{\gradus{#1}\ensuremath{#2{^\prime}}}

	% Градус + минуты + секунды
	\newcommand{\coord}[3]{\cord{#1}{#2}\ensuremath{#3^{\prime\prime}}}

	% Координата(градус) - знак зодиака - координата(минуты)
	\newcommand{\signum}[3]{\(#1^\circ\){\small{#3}}%
		\ifthenelse{\isempty#2}{}{\(#2^\prime\)}%
	}
	% ----------------------------------------------

	% Форматы вывода времени
	\newcommand{\timemath}[2]{#1:#2}

	\newcommand{\timeshort}[3]{
		\ifthenelse{\isempty#1}{}{#1ч}
		\ifthenelse{\isempty#2}{}{\,#2мин}
		\ifthenelse{\isempty#3}{}{\,#3сек}
	}
\makeatother

% Кастомные списки
% Как составной элемент, который может включать в себя другие новые сущности
% должны быть объявлены после этих сущностей
% https://tex.stackexchange.com/questions/300340/topsep-itemsep-partopsep-and-parsep-what-does-each-of-them-mean-and-wha
\usepackage{enumitem}

\newlist{myenum}{enumerate}{3}
\setlist[myenum]{label=\arabic*.,itemsep=0em,parsep=1em}

\newlist{myitem}{itemize}{3}
\setlist[myitem]{label=--,itemsep=0em,parsep=0em}

\newlist{mylist}{itemize}{3}
\setlist[mylist]{itemsep=0em,parsep=0em,topsep=0em}

\newlist{mydescr}{description}{3}
\setlist[mydescr]{parsep=0em,topsep=0em,itemsep=0em}

% -------------------------------------------------------
% Нумерация страниц
% Должна быть после настройки отступов иначе заголовок
% будет неверного размера
% -------------------------------------------------------
\usepackage{fancyhdr} 

\fancyhf{}
\fancyhead[LO]{}
\fancyhead[CO]{\textit{"Индийская предсказательная астрология"}}
\fancyhead[LO]{}
\fancyhead[RO]{\thepage}

\renewcommand{\headrulewidth}{0.4pt}
\renewcommand{\footrulewidth}{0pt}

% изменить стиль на страницах с началом главы и содержании
\fancypagestyle{plain} {
	\fancyhf{} % Сбросить всё
	\renewcommand{\headrulewidth}{0pt}
	\renewcommand{\footrulewidth}{0pt}
}

% -------------------------------------------------------
% Перекрестные ссылки в документе
% -------------------------------------------------------
\usepackage{color}
%\usepackage{varioref}
%\usepackage[plain]{fancyref}
\usepackage{hyperref}

\hypersetup{
	colorlinks,
	citecolor=black,
	filecolor=black,
	linkcolor=black,
	urlcolor=blue,
	linktocpage,
	unicode=true
}
%	hidelinks,
%	breaklinks,
%	unicode=true,
%	pdftitle={},
%	pdfauthor={},
%	pdfkeywords={}
%}

% -------------------------------------------------------
% Разбивка на главы/тексты
% -------------------------------------------------------

\includeonly{
	title,
	intro,
  	prescript,
	ch1,
  	ch2,
  	conclusion
}

\begin{document}
% Сбросим стили
\pagestyle{empty}

% Список продОлбов
%\listoffixmes

% содержание
% Шапка оглавление -> содержание
\renewcommand\contentsname{СОДЕРЖАНИЕ}

% обложка
\begin{titlepage}

\begin{center}
	\textbf{Станиславский\,С.A., Телицин\,А.П.}

	\vspace{18em}

	\textbf{\Large{ИНДИЙСКАЯ ПРЕДСКАЗАТЕЛЬНАЯ}}

	\vspace{1em}

	\textbf{\Large{АСТРОЛОГИЯ}}

	\vspace{20em}
\end{center}

\vspace{\fill}
\begin{center}
	1997г.
\end{center}
\end{titlepage}

\newpage

Хотите ли вы знать, что ожидает вас и ваших близких в будущем? Древнейшая и увлекательнейшая наука Джотиш--Веда, или Индийская предсказательная астрология, поможет вам самостоятельно рассчитать гороскоп, приоткрыв великую тайну грядущего.

Книга содержит практические рекомендации по астрологическому прогнозированию, основанному на философском учении В.\,Сармы и Б.\,Рамана.

Издание адресуется широкому кругу читателей.


\clearpage

\begin{spacing}{0.9}
\tableofcontents
\addtocontents{toc}{~\hfill\textbf{Стр.}\par}
\end{spacing}

% Логическая область закончилась
\clearpage

% Применим стили
\pagestyle{fancy}

% Нумерация с этой страницы
\setcounter{page}{3}

% К читателю
\tocsection{К ЧИТАТЕЛЮ}

Уважаемый читатель, для того чтобы древнейшая наука --- Индийская предсказательная астрология(ИПА) стала для вас действительно простой и доступной, не надо начинать читать эту книгу второпях, озабоченным домашними делами, работой, тем более устраивать коллективные чтения. Выберите время, приведите в порядок свои мысли, успокойте душу и постарайтесь сделать так, чтобы чтение не стало праздным развлечением. Соприкосновение с Учением, которое чудесным образом возникло с незапамятных времен истории человечества, было сохранено и донесено до наших дней народом Индии благодаря его глубокой привязанности к традициям, --- это соприкосновение с высочайшей духовностью и нравственностью, ибо только на таком уровне сознания человек мог создать совершенное Учение --- Джотиш-Веду, что в переводе с санскрита означает ``Свет Знания''. Вот как поэтично сказано об этом в Гимне из Риг-Веды:

\quoteit{
	Открой своему сознанию вечный свет \\
	и высшее блаженство! \\
	Ибо ты, кто знает путь, ведет того, \\
	Кто ищет и просит помощи. \\
}

Будущее не является загадкой для людей, владеющих методикой ИПА~\citep{ojha}.

Авторы убеждены, что, ознакомившись с содержанием этой книги, любой человек, умеющий логически мыслить и знающий математику в объеме программы средней школы, сможет легко расчитать гороскоп и самостоятельно определить основные события своей жизни: периоды удач и невезений, болезней и выздоровлений, поездок, брака, рождения детей, финансового состояния и\,т.\,д. Изучение астрологии полезно начинать с составления собственного гороскопа, так как в этом случае вероятность будущих событий гарантируется точным знанием своего прошлого и совпадением минувших событий с указаниями гороскопа.

Теоретическая часть книги подготовлена на основе трудов известных индийских астрологов Вишванат Дева Сармы~\citep{sarma} и Бангалор Венката Рамана~\citep{raman}, а также философских комментариев к Ведическому Учению величайшего ученого йогина современности Махариши Махеш Йоги~\citep{maharishi}.

Чтобы книга была понятной широкому кругу читателей, авторы старались избегать терминологии на санскрите, употребляемой в ИПА, максимально используя традиционные для западной астрологии названия, понятия и обозначения (вместе с тем читатель убедится, что индийская астрологическая система коренным образом отличается от западной). Особое внимание в ней уделено практической части, где досконально, на конкретных примерах, подтверждено каждое теоретически обоснованное правило составления гороскопа. Важную роль в этом сыграл опыт Сергея Станиславского, расчитавшего по системе ИПА более 2500 гороскопов, многие предсказания которых сбылись.

Вне сомнений, материал, изложенный в книге, --- это только первый шаг в изучении безграничного кладезя знаний древнейшей и увлекательнейшей из наук, однако с его помощью человек приоткроет для себя какую-то, пусть небольшую часть наиболее охраняемой природой тайны --- тайны будущего.


% Введение
\quoteintro{
	В одно мгновенье видеть вечность \\
	Огромный мир --- в зерне песка \\
	В единой горсти --- бесконечность \\
	И небо --- в чашечке цветка
} {Уильям Блейк}

\tocsection{ВВЕДЕНИЕ}

Ведическое Учение было принесено в Индию ариями более 5000 лет назад. Версии возникновения самого народа --- ариев и памятников древнеиндийской литературы --- Вед, появившихся на базе их культуры, пока что находятся в области преданий, догадок и эзотерики. Современная археология обнаруживает следы арийской культуры семитысячелетней давности. Так, известный украинский археолог Ю. Шилов определил основные места арийских поселений в низовьях Днепра. Находки, сделаные здесь во время раскопок курганов, свидетельствуют о том, что Ведическое Учение существовало 7000 лет назад.

Современные ученые, изучая ведические тексты, отмечают, что они содержат все знания, необходимые человеку в материальной и духовной жизни. В Ведах, написанных тысячелетия назад, есть готовые формулы квантовой физики, вычисления расстояния от Земли до Солнца и других планет Солнечной системы. Ведическая медицина --- Аюр-Веда (``Наука жизни``) считается непревзойденной до сих пор.

Еще в начале прошлого столетия французский физик, математик и астроном Пьер Лаплас писал, что для Разума, постигшего все законы, действующие во Вселенной, и ее строение, будущее будет таким же ясным, как прошлое и настоящее. В наш век, когда созданы компьютеры, способные выполнять миллиарды операций в секунду, ученые могут описать поведение любой частицы объекта и любое явление в природе. И все же при таких технических возможностях мы часто не в силах решить обычные жизненные проблемы и ежедневно совершаем ошибки.

Конечно, одному человеку не под силу изучить абсолютно все причинно-следственные связи, чтобы с математической точностью просчитать каждый шаг в течение дня. Охватить одновременно все области знаний невозможно, а найти их основу? Это была мечта многих великих ученых --- создать теорию единства всех законов природы. В конце 70-х годов нашего столетия американский профессор Джон Хэгелин предложил теорию суперсимметрии, которая сводит все законы природы в единое поле: из него возникает Вселенная и в нем же происходит взаимодействие ее объектов. Теперь каждый углубленно изучающий физику человек знает, что как бесцветный сок лежит в основе зеленого листа, коричневого стебля и розового лепестка, так и единое поле лежит в основе всех явлений Вселенной. ``Единое поле, --- говорит Джоно Хэгелин, --- есть не что иное, как простейшая форма сознания``.

В Европейском центре ядерных исследований в Женеве (Швейцария), который специализируется на исследованиях в области сознания, была апробирована методика, названная Технологией Единого Поля, или Техникой Трансцедентальной Медитации(ТМ), практикуя которую человек достигает в собственном сознании единого поля, обучаясь действовать спонтанно правильно в соответствии со всеми законами природы. Предложил эту технику индийский физик, философ и йогин Махариши Махеш Йоги. Результаты ТМ оказались действительно феноменальны. Опубликованные данные более 500 независимых научных исследований, проведенных в 27 странах мира, свидетельствуют о благотворном влиянии ТМ на все сферы человеческой жизни. Изменения в электроэнцефалограмме медитирующего человека указывали на состояние расслабления и одновременно бодрствования. Полностью когерентное, то есть согласованное, функционирование головного мозга во время трансцедентальной медитации --- это идеальная подготовка человека к активной деятельности. Институт головного мозга в Москве, проводивши исследования по программе ТМ, рекомендует ввести ее в систему образования.

Технология Единого Поля, подаренная миру Махариши, уходит корнями в Ведическое Учение. Ведические тексты описывают сознание как высшую деятельность человека и Мироздания, область чистого существования, безграничное поле энергии и разума, которое содержит все явления природы. Человек, овладевший методикой ТМ, постигает в своем сознании Единое Поле, получая информацию о прошлом и будущем.

В истории человечества всегда были люди, достигавшие путем очищения сознания таких высот духовного развития, которые позволяли им находиться на уровнях Единого Поля и нести людям Чистое Знание. Мы не будем касаться выдающихся личностей, окруженных ореолом божественного предназначения на Земле, учитывая, что книга расчитана на читателей с различными взглядами на взаимосвязь духовного и материального, но в качестве примера назовем несколько имен извстных людей, способности которых не вызывают сомнения. Это великие йогины древней Индии Тилопа и Наропа, йогины древнего Тибета Марпа и Миларепа, а также наши современники --- йогины-риши (видящие), Гуру Дев (оставил тело в 1953 году) и Махариши Махеш Йоги.

К сожалению, пока еще мало людей, имеющих возможность черпать нужную информацию из собственного сознания, а ведь знание будущего полезно и просто необходимо человеку, чтобы понять собственное предназначение в системе космической эволюции. Человек, владеющий таким знанием, становится спокойным, целеустремленным и крепким духом. Для него нет неожиданностей, нет излишних ментальных волнений, его действия гармонично сочетаются с законами природы. Значительно сокращается количество стрессов и, как следствие, улучшается психическое и физическое здоровье. Но если все-таки судьба сулит человеку неприятности, то, зная об этом, он заранее может смягчить ее удар, подготовив себя на ментальном уровне и укрепив духовно к принятию неизбежного.

Многие выдающиеся люди верили в астрологию и занимались ею. Данте признавал эту науку благороднейшей. Пифагор, Демокрит, Птолемей, Кеплер, Бэкон и Морин были известными астрологами.

Наука о влиянии звезд и планет на судьбу человека достигла наивысшего развития в Индии задолго до так называемого ``периода достоверной истории''. Общепризнанным родоначальником ее является Парашара Муни, живший более 5000 лет назад незадолго до начала эпохи Кали Юга. Его труды представляют собой наиболее древние письменные комментарии к Атхарва-Ведам (часть Вед, посвященная астрологии). Изложенные Парашарой принципы лежат в основе всех последующих трудов по астрологии, в том числе и современных.

Как течение реки ограничено ее берегами, так и действия человека регламентированы законами эволюционного развития Вселенной. Подчиняясь этим законам, звезды и планеты не меняют свой курс и с такой же неизменной силой ведут к вершине эволюции --- полному единению с природой --- каждого человека.

Часто люди, не знакомые с ведическим мировоззрением, задают вопрос: ``Каким образом планеты, находясь так далеко в космическом пространстве, могут влиять на поведение человека на Земле?'' По Ведическому Учению Вселенная --- это огромное информационно-энергетическое пространство, которое является основой всего сущего, в том числе и человека. Звезды и планеты рассматриваются не как физические тела, а как энергетические узлы в этом пространстве. Непрерывное движение и расположение планет относительно друг друга и относительно созвездий Зодиака становится причиной непрерывного изменения баланса энергии во Вселенной, в том числе в тонких энергетических структурах человека, которые, отражаясь от индивидуальной нервной системы, приводят к соответствующим действиям. Если человек знаком с процессом, происходящим в этом огромном компьютере --- Вселенной, и знает правила пользования им, он может предсказать свои действия на много лет вперед.

Различные астрологические школы придерживаются двух систем расчета гороскопа --- Саяны и Нираяны. В переводе с санскрита ``аяна'' --- продвижение, или прогресс, отсюда Саяна является системой подвижного Зодиака, а Нираяна --- неподвижного. Разница между координатами планет в этих системах определяется Аянамсой (аяна --- движение, амса --- часть). Подробнее об этом будет рассказано в соответствующей главе книги. Отметим лишь, что система Нираяна на практике дает наиболее точные результаты, поэтому предлагается читателям.

Индийская предсказательная астрология имеет несколько направлений, которые по сути образуют самостоятельные астрологические науки:

\begin{mylist}
	\item \emph{натальная астрология} --- имеет отношение к рождению и жизни конкретного человека;
	\item \emph{мунданная астрология} --- прогнозирует события в масштабах нации или государства;
	\item \emph{медицинская астрология} --- связана с течением и исходом болезней;
	\item \emph{хорарная астрология} --- отвечает на конкретно заданный вопрос в данное время;
	\item \emph{мухурта} --- помогает выбрать благоприятное время для различных видов деятельности;
	\item \emph{васту шастра} --- отностися к архитектуре;
	\item \emph{атмосферно-метеорологическая астрология} --- составляет прогноз погоды и\,т.\,д.
\end{mylist}

В этой книге основное внимание уделено натальной астрологии как наиболее популярной.

Полное название предлагаемой астрологической системы --- Вимсоттари Даша Джанана и Бикара от Нираяна Калачакра. Натальный раздел (Вимсата означает 120; Даша --- период; Джанана --- расчет времени события; Бикара --- определение сущности события в данный период времени; Калачакра --- колесо времени, или Зодиак с девятью планетами, семь из которых являются небесными телами, а две математически предсказанные).

По мнению Вишванат Дева Сармы, теория 120 лет базируется на идее прохождения эволюционного процесса в циклических фазах со стодвадцатилетним периодом (так же, как дни и ночи, сезоны и годы, которые имеют циклическую повторяемость и используется для познания объектов природы)

Несмотря на громоздкость полного названия, система довольно проста и достаточно точна на практике. В этом читатель, несомненно убедится, полностью изучив материал, изложенный в книге. Удачи вам!


% Главы
\chapter[ОСНОВОПОЛАГАЮЩИЕ ПРИНЦИПЫ]{ОСНОВОПОЛАГАЮЩИЕ ПРИНЦИПЫ ПРЕДСКАЗАТЕЛЬНОЙ АСТРОЛОГИИ}

\section{Двенадцать домов гороскопа и их идеи}

Существуют две модели Солнечной системы: гелеоцентрическая и геоцентрическая. Гелеоцентрическая модель Солнечной системы --- это та модель, которая реально существует в космическом пространстве: вокруг Солнца движутся планеты. Все астрономические открытия делались на основе гелеоцентрической модели мироздания.

Астрология рассматривает движения планет в основе геоцентрической модели Солнечной системы, то есть в центр Солнечной системы условно ставится Земля и при этом движение планет относительно земного наблюдателя происходит вокруг Земли против часовой стрелки.

Древние риши определили, что весь спектр идей, относящихся к человеку и его деятельности, представлен девятью планетами, поэтому в Индийской предсказательной астрологии используются знания только об этих планетах.

Планеты бывают благоприятными и неблагоприятными по своей природе. Юпитер и Венера --- благоприятные, Сатурн, Марс, Раху, Кету и Солнце --- неблагоприятные. Меньше всего зла исходит от Солнца. Луна и Меркурий считаются нейтральными по своей природе и могут быть благоприятными и неблагоприятными в зависимости от их расположения на карте рождения. Луна растущая благоприятна, убывающая --- наоборот, неблагоприятна. Меркури, находящийся в знаке Зодиака один или в связи с благоприятной планетой, будет благоприятным. Если он находится в одном знаке Зодиака с какой-либо неблагоприятной планетой, то тоже становится неблагоприятным.

Планеты разделены на мужские, женские и нейтральные.

\begin{table}[tph!]
	% Расширить по вертикали
	\renewcommand{\arraystretch}{1}

	% Заполним данными
	\begin{tabular}{l|l|l}
		Мужские планеты & Солнце, Марс, Юпитер & активность, мужское начало \\
		Женские планеты & Луна, Венера         & мягкость, женское начало \\
		Нейтральные планеты & Меркурий, Сатурн & проявляют оба начала в соответствии \\
		                     & Раху и Кету      & с полом знака Зодиака \\
	\end{tabular}
\end{table}


Планеты кроме Раху и Кету соответствуют семи дням недели:

\begin{table}[tph!]
	% Расширить по вертикали
	\renewcommand{\arraystretch}{1}

	% Заполним данными
	\begin{tabular}{ll}
		Воскресенье & Солнце \\
		Понедельник & Луна \\
		Вторник     & Марс \\
		Среда       & Меркурий \\
		Четверг     & Юпитер \\
		Пятница     & Венера \\
		Суббота     & Сатурн \\
	\end{tabular}
\end{table}

В Индийской предсказательной астрологии каждая планета несет определенные идеи в жизнь человека. Знание их необходимо для правильного прочтения карты рождения.

\begin{myenum}[topsep=0]
	\item \textbf{Солнце:}
		\begin{mydescr}
			\item[Физиология] --- сердце, мозг, голова, кости, голосовые связки, правый глаз.
			\item[Направление] --- восток.
			\item[Идея] --- индивидуальная душа (эго), отец, представители власти, способность к самоосознанию, статус, уверенность, благородство, слава, сознание, правда, истина, храбрость, решительность, энергия, политика, химия, медицина, врачи, хирурги, целители, храмы и места поклонений, места жертвоприношений, залы для коронования, осветительные приборы, высокий пост в правительстве.
		\end{mydescr}
	\item \textbf{Луна:}
		\begin{mydescr}
			\item[Физиология] --- молочные железы, матка, кровообращение, левый глаз.
			\item[Направление] --- северо-запад.
			\item[Идея] --- мать, женщина, известность, имя, ум, вода, океан, море, реки, путешествие и путешественники, лекарственные растения, сочные плоды, женственность, фантазии, перемены, молоко, популярные курорты, фармацевты и аптекари, зеркало, рыболов, грациозность, чувства.
		\end{mydescr}
	\item \textbf{Марс:}
		\begin{mydescr}
			\item[Физиология] --- нос, мышечная ткань, сухожилия, половые органы.
			\item[Направление] --- юг.
			\item[Идея] --- братья и сестры, решительность, твердость, действия, благосостояние, недвижимость, энергия, сила, земельный участок, пожар, военные, военная операция, режущие инструменты, инженеры, математика, электроника, калькулятор, хирурги, катастрофы, горы, леса, тяжба, спор, спорт, землетрясение, раны и травмы, жар, сапфиры, кораллы, сокровища, химическая лаборатория, война, битва, соревнование, строительство, сила духа, холодное и огнестрельное оружие, ожоги, должность в армии и полиции, конструктор, хирургическая операция, разрыв, разрушение, несчастный случай.
		\end{mydescr}
	\item \textbf{Меркурий:}
		\begin{mydescr}
			\item[Физиология] --- легкие, кишечник, брюшная полость, язык, кисти рук, нервные центры.
			\item[Направление] --- север.
			\item[Идея] --- интеллект, интеллектуальная деятельность, речь, литературные способности, средства информации и связи, рационализм, секретарская работа, бухгалтер, ученый, учебные заведения, головная и желудочная боль, рекламные агенства, издательство и издатели, исследование, новости, коммерческая деятельность, друзья.
		\end{mydescr}
	\item \textbf{Юпитер:}
		\begin{mydescr}
			\item[Физиология] --- печень, нижняя часть брюшной полости, органы слуха, таз.
			\item[Направление] --- северо-восток.
			\item[Идея] --- поведение, красноречие, мудрость, философия, духовный рост, религия, духовный наставник, учитель, ведические знания, набожность, религиозные учреждения, благотворительность, профессор, склонность к полноте, синтез, аргументация, получивший высокую оценку, дети, банки и банкиры, легальная деятельность, министры, юристы, справедливость, почтенные и уважаемые люди, советники, священники, беременные женщины, муж, поддерживающий жизнь.
		\end{mydescr}
	\item \textbf{Венера:}
		\begin{mydescr}
			\item[Физиология] --- почки, яичники, выделения, половая система.
			\item[Направление] --- юго-восток
			\item[Идея] --- желания, привязанность, жадность, ревность, праздность, красота, живописные места, текстильная продукция, люди искусства, товары ``люкс'', спиртные напитки, окружающая среда, спальня, комфорт, сексуальные удовольствия, венерические болезни, парфюмерия, транспортные средства, предметы роскоши, цветы, ботаника, творческие способности, муж, жена, любовник и любовница, украшения, декорации, театры, музеи, кино, сладости, ювелирные украшения.
		\end{mydescr}
	\item \textbf{Сатурн:}
		\begin{mydescr}
			\item[Физиология] --- ноги, колени, костный мозг, мочевой пузырь, дыхательная система, принцип передвижения.
			\item[Направление] --- запад
			\item[Идея] --- время, вызывающеее изменения, смерть, лишения, результат длительных действий, аскетизм, горе, печаль, подлость, обман, воры, мошенники, слуги, работа в подчинении кого-то, рабочий, самоотречение, пожилые люди, уход в отставку, места захоронения, судебный исполнитель, нищий, настенные или настольные часы, отсрочка, заблуждение, ложный вывод, уголовный розыск, черная магия, тайные знания, практика йоги, изменение обстоятельств, разорение и крах, предательство, сила влияния на других, нефть, железо, люди низкого происхождения, должности, получаемые за услуги, консерватизм, сфера услуг, препятствие.
		\end{mydescr}
	\item \textbf{Раху:}
		\begin{mydescr}
			\item[Физиология] --- скулы, кожа, выделительная система, глотание, пищеварительный тракт, прямая кишка.
			\item[Направление] --- юго-запад
			\item[Идея] --- страсть, толпа, мятеж, восстание, ядовитые рептилии, бедствие, наводнение, люди низкой культуры, воры и мошенники, гнев, гордыня, тупость, глупость, мистические знания, таинства, тайная доктрина, грубость, путешествия, муравейник, злое предсказание, эпидемии, насилие, коррупция, незаконные действия, охотники, тяжба, инфекционные заболевания, припадок, удушье, способность, оказывать влияние на других, лишения, крупное воровство.

Раху действует подобно Сатурну и провляет энергии тех планет, которые оказывают на нее влияние.
		\end{mydescr}
	\item \textbf{Кету:}
		\begin{mydescr}
			\item[Физиология] --- позвоночник, спинно-мозговой канал, нервная система, половая система.
			\item[Направление] --- направления не имеет/
			\item[Идея] --- препятствия и помехи, несчастный случай, духовные силы, психическое состояние, астрология, банкротство, клевета, оккультизм, падение, болезнь и недомогание, огонь, математика и мат. способности, эпидемии, страх, нервозность, испуг, яд, больничная палата, мелкое воровство, неожиданность, рана.

Кету действует подобно Марсу и проявляет энергии тех планет, которые оказывают на нее влияние.
		\end{mydescr}
\end{myenum}

Главные идеи девяти планет очень важно знать напамять при анализе карты рождения. А также следует не забывать, что сильные и удачно расположенные планеты в карте рождения проявят свои хорошие качества, а слабые и неудачно расположенные планеты --- дадут плохой результат.

\subsection{Определение силы планет}

В момент рождения человека планеты находятся в различных знаках Зодиака и в зависимости от этого проявляют свою силу или слабость.o

Существует множество способов для определения силы планет с огромным количеством вычислений и составлением дополнительно шестнадцати различных карт, аналогичных карте навамса, о которой будет рассказано далее. Однако на практике большинство астрологов пользуется для этого следующими правилами:
\begin{myenum}[itemsep=0,parsep=0]
	\item Планета в знаке экзальтации.\label{cost:1}
	\item Планета в мулатриконе.\label{cost:2}
	\item Планета в собственном знаке.\label{cost:3}
	\item Планета в знаке большого друга.\label{cost:4}
	\item Планета в дружественном знаке.\label{cost:5}
	\item Планета в нейтральном знаке.\label{cost:6}
	\item Планета во враждебном знаке.\label{cost:7}
	\item Планета в знаке большого врага.\label{cost:8}
	\item Планета в знаке ослабления.\label{cost:9}
\end{myenum}

Данные девять достоинств расположены в порядке убывания силы планет. Планета имеет наибольшую силу, если находится в знаке экзальтации, тогда она проявляет лучшие свои качества и приносит человеку счастье, удачу и доход. В знаке ослабления планеты теряют силы и сулят человеку горе, страдания и убытки. Промежуточные достоинства планет дают неоднозначный результат, в них проявляется как хорошее, так и плохое. Но чем выше достоинство планеты (мулатрикона, собственный знак, знак большого друга), тем сильнее в ней доброе начало, меньше плохих качеств. И наоборот, чем ниже достоинство планеты (враждебный знак или знак большого врага), тем больше будет проявляться зло и меньше хороших качеств.

Достоинство планет по пунктам \ref{cost:1}, \ref{cost:2}, \ref{cost:3} и \ref{cost:9} легко определить по таблице~\ref{tbl:cost}, где перечислены знаки, которые являются экзальтирующими, знаками мулатриконы, собственными и ослабляющими для каждой планеты.

Сначала надо посмотреть, в каком знаке Зодиака эта планета расположена в карте рождения, а затем по таблице определить, какое достоинство этот знак придает соответствующей планете. Например, если в карте рождения Юпитер находится в Рыбах, значит Юпитер в собственном знаке. Возле каждого знака экзальтации проставлено значение градуса (например, Овен \(10^\circ\)) --- это значит, что если планета в знаке находится именно в таком градусе, она наиболее сильная (то есть это градус наивысшей экзальтации планеты в знаке). Аналогично для знака ослабления --- указанное значение градуса определяет точку наибольшего ослабления соответствующей планеты в данном знаке.

\begin{table}[tph!]
	\caption{Определение достоинств планет}
	\label{tbl:cost}

	\centering

	% Расширить по вертикали
	\renewcommand{\arraystretch}{1}

	% Разширить вширь
	%\setlength{\tabcolsep}{.05\textwidth}

	% Заполним данными
	\begin{tabular}{|l|l|l|l|l|}
		\hline
		Планета & Знак экзальтации & Знак мулатрикона & Собственный знак & Знак ослабления \\
		\hline
		Солнце   & Овен \gradus{10}    & Лев \gradus{0}--\gradus{20}     & Лев & Весы \gradus{10}    \\
		Луна     & Телец \gradus{3}    & Телец \gradus{4}--\gradus{30}   & Рак & Скорпион \gradus{3} \\
		Марс     & Козерог \gradus{28} & Овен \gradus{1}--\gradus{12}    & Овен, Скорпион   & Рак \gradus{28}    \\
		Меркурий & Дева \gradus{15}    & Дева \gradus{16}--\gradus{20}   & Близнецы, Дева   & Рыбы \gradus{15}   \\
		Юпитер   & Рак \gradus{5}      & Стрелец \gradus{1}--\gradus{10} & Стрелец, Рыбы    & Козерог \gradus{5} \\
		Венера   & Рыбы \gradus{27}    & Весы \gradus{0}--\gradus{5}     & Телец, Весы      & Дева \gradus{27}   \\
		Сатурн   & Весы \gradus{20}    & Водолей \gradus{1}--\gradus{20} & Козерог, Водолей & Овен \gradus{20}   \\
		Раху     & Телец               & Близнецы & Водолей & Скорпион \\
		Кету     & Скорпион            & Стрелец  & Лев & Телец \\
		\hline
	\end{tabular}
\end{table}

Возле каждого знака мулатрикона также проставлены значения градусов (например, Лев \(0{^\circ}-20^\circ\)) --- это значит, что достоинство или сила мулатрикона действует в пределах указанных градусов, а в остальном пространстве знака планета будет иметь достоинство знака экзальтации или собственного знака в соответствии с таблицей\footnote{Некоторые индийские астрологические школы указывают позицию мулатрикона для Венеры от \gradus{0} Весов до \gradus{15} Весов.}.

Солнце и луна имеют по одному собственному знаку, а остальные пять планет по два. Индийские астрологи считают, что Раху и Кету не имеют собственных знаков, так как они не обладают физическими телами, но их расположение в знаках Водолея и Льва придает им силу, соответствующую достоинству собственного знака.

По терминологии, принятой в Индии, планета является хозяином собственного знака, то есть если Лев --- собственный знак для Солнца, то Солнце --- хозяин Льва и\,т.\,д.

Для того, чтобы определить достоинства планет по пунктам \ref{cost:4}, \ref{cost:5}, \ref{cost:6}, \ref{cost:7} и \ref{cost:8}, то есть в знаке большого друга, дружественном знаке, нейтральном знаке, враждебном знаке и знаке большого врага, необходимо ввести дополнительные термины, такие как \emph{постоянные} отношения и \emph{временные} отношения между планетами.

Каждая планета, кроме Раху и Кету, имеет дружественные, нейтральные или враждебные отношения с другими планетами. Их можно установить с помощью приведенной таблицы~\ref{tbl:relations}, отражающие постоянные отношения между планетами. Из этой таблицы видно, что для Солнца дружественными являются Луна, Марс и Юпитер, нейтральным --- Меркурий, а враждебными --- Сатурн и Венера. И так для каждой планеты, находящейся в левой колонке. Отношения между планетами устанавливаются через знаки, хозяевами которых они являются. Например, если Солнце находится в Раке, а хозяйкой Рака является Луна, которая дружественных отношениях с Солнцем, Значит Солнце находится в дружественном знаке. Таким образом определяют постоянные отношения для каждой планеты в гороскопе.

\begin{table}[tph!]
	\caption{Характер отношений между планетами}
	\label{tbl:relations}

	\centering

	% Расширить по вертикали
	\renewcommand{\arraystretch}{1}

	% Разширить вширь
	%\setlength{\tabcolsep}{.05\textwidth}

	% Заполним данными
	\begin{tabular}{|l|l|l|l|}
		\hline
		Планета & Дружественные & Нейтральные & Враждебные \\
		\hline
		Солнце   & Луна, Марс, Юпитер & Меркурий & Сатурн, Венера \\
		Луна     & Солнце, Меркурий & Марс, Юпитер, Венера, Сатурн & --- \\
		Марс     & Солнце, Луна, Юпитер & Венера, Сатурн & Меркурий \\
		Меркурий & Солнце, Венера & Марс, Юпитер, Сатурн & Луна \\
		Юпитер   & Солнце, Луна, Марс & Сатурн & Меркурий, Венера \\
		Венера   & Меркурий, Сатурн & Марс, Юпитер & Солнце, Луна \\
		Сатурн   & Меркурий, Венера & Юпитер & Солнце, Луна, Марс \\
		\hline
	\end{tabular}
\end{table}

Временные отношения между планетами определяются по их взаимному расположению в домах. Они бывают двух типов --- временная дружба и временная вражда. Если одна планета расположена относительно другой во 2, 3, 4, 10, 11 или 12-м доме, значит эти две планеты находятся во временных дружественных отношениях. Если две планеты находятся в одном доме или одна планета расположена относительно другой в 5, 6, 7, 8 или 9-м доме, значит эти две планеты находятся во временных враждебных отношениях. Надо обратить внимание на то, что при установлении временных отношений между планетами за первый дом принимается тот дом, где находится планета, достоинство которой вы определяете.

Установив постоянные и временные отношения между планетами в гороскопе, можно приступить к определению достоинств планет в соответствии с пунктами \ref{cost:4}, \ref{cost:5}, \ref{cost:6}, \ref{cost:7} и \ref{cost:8} из правила определения силы планет по девяти достоинствам:

\begin{myenum}[itemsep=0,parsep=0]
	\item Постоянные дружественные отношения + дружба временная = планета в знаке большого друга.
	\item Постоянные нейтральные отношения + дружба временная = планета в дружественном знаке.
	\item Постоянные враждебные отношения + дружба временная = планета в нейтральном знаке.
	\item Постоянные дружественные отношения + вражда временная = планета в нейтральном знаке.
	\item Постоянные нейтральные отношения + вражда временная = планета во враждебном знаке.
	\item Постоянные враждебные отношения + вражда временная = планета в знаке большого врага. 
\end{myenum}

Покажем как определяются достоинства планет в знаках Зодиака на нескольких примерах:

\begin{myenum}
	\item Солнце расположено в Тельце, а Венера в Овне. Солнце находится во враждебном знаке, так как хозяйка Тельца --- Венера, согласно таблице~\ref{tbl:relations}, враждебна к Солнцу. Если считать Телец, в котором находится Солнце, 1-м домом, то Венера относительно Солнца находится в Овне, то есть в 12-м доме, что говорит о временной дружбе этих планет. Теперь складываем постоянные враждебные отношения с временной дружбой и получаем, что Солнце находится в нейтральном знаке.
	\item Марс расположен во Льве, хозяином которого является Солнце. Солнце дружественно Марсу и находится в Деве, то есть во 2-м доме от Марса, что указывает на временные дружественные отношения. Сложив постоянные дружественные отношения со временной дружбой, определяем Марс как находящийся в знаке большого друга.
	\item Венера находится в Раке, хозяином которого является Луна. Луна для Венеры враждебна. Луна находится в Козероге, то есть в 7-ом доме от Венеры. Луна --- временный враг Венеры. Сложив постоянные враждебные отношения со временной враждой, определяем Венеру как находящуюся в знаке большого врага.
\end{myenum}


\subsubsection*{Определение силы планет в зависимости от их расположения в знаке Зодиака.}

Вы уже знаете, что каждый знак Зодиака занимает \gradus{30} пространства в Зодиаке, а также то, что знаки делятся на мужские и женские. Все нечетные знаки --- мужские, четные --- женские.

Если планета расположена в части знака от \gradus{12} до \gradus{18}, то это местонахождение способствует максимальному проявлению силы такой планеты. Расположение планеты в первых \gradus{6} четных женских знаков или последних \gradus{6} нечетных мужских знаков значительно ослабит ее. В остальном пространстве знака планеты чувствуют себя достаточно сильными для проявления своих идей.

Если планета находится на границе между двумя знаками, то она не в состоянии полностью проявить себя, поэтому рассматривается как ослабленная.

Если планета находится не далее чем в \gradus{5} от Солнца, то она считается сгорающей, и сила ее значительно убывает. Меркурий меньше других подвержен сгоранию, так как нахождение вблизи Солнца его естественное состояние. Планеты Раху и Кету не подвержены сгоранию, поскольку не обладают физическими телами.

В ретроградном движении планета проявляет больше силы чем в прямом.

Определение силы планет --- это важнейшая тема в Индийской предсказательной астрологии, и мы будем продолжать развивать ее в следующих главах. Конечно, при первых попытках анализа гороскопа читатель столкнется с некоторыми трудностями в определении силы планет, так как здесь нет четкого выражения относительных сил в процентах, а самих признаков силы и слабости планет достаточно много, но в дальнейшем у вас появится опыт. Практикующий астролог чувствует силу планет даже по внешнему виду человека, с которым беседует. Поэтому мы еще раз подчеркиваем, что очень важно знать напамять максимальное количество информации.

\subsection{Двадцать семь лунных созвездий(накшатр) и их идеи}

Древние риши, наблюдая за движением планет, сделали великое открытие --- определили двадцать семь равных промежутков пространства в Зодиакальном поясе --- накшатр (см. табл. \ref{tbl:nakshatras}). Каждое лунное созвездие равно \coord{13}{20}{} Зодиакального пояса. Риши обнаружили, что планеты в разных частях одного и того же знака, оказывают различное воздействие. К примеру, если Солнце проходит по знаку Овен, оно дает эффекты в зависимости от того, в каком лунном созвездии находится. Три созвездия (третье неполное) в знаке Овен и дополнительные эффекты Солнца в гороскопе должны быть связаны с тем созвездием, в котором оно находилось в момент рождения человека. Солнце в созвездии Ашвини придаст одну окраску событиям и характеру человека, в Бхарани --- другую и в Криттике --- третью.

Каждое лунное созвездие представляет определенные силы природы, которые выражаются в идеях.

\begin{myenum}
	\item \textbf{Ашвини} --- накшатра транспорта. 
		\begin{mydescr}
			\item[Протяженность] --- \signum{0}{}{\aries} -- \signum{13}{20}{\aries}
			\item[Идеи] --- владелец лошадей, наездник, кавалерист, различные средства транспорта, приезждать куда-то, достигать чего-то, посещать, получать, добиваться, совершать благородные деяния, предоставлять помощь, приносить сокровища человеку, врачеватель, обоняние. вдох-выдох, носовые звуки, нечеткое произношение, избегать горестей, уйти от несчастья.
		\end{mydescr}
	\item \textbf{Бхарани} --- накшатра сдерживающего начала.
		\begin{mydescr}
			\item[Протяженность] --- \signum{13}{20}{\aries} -- \signum{26}{40}{\aries}
			\item[Идеи] --- Бхарани заключается в себя, а потом освобождает, действие сдерживания, подавление, дисциплина, самоконтроль, моральный долг, наказывать, подчинять, управлять, злой умысел, покорять, быть преданным и непоколебимым, претерпевать и страдать, нести в чреве, наполнять желудок, питать, пища, иждивенец, чувство тяжести, нанимать кого-то, наемник, война. битва, соревнование, кричать, повышать голос, достижение, завоевание. приз, большое количество, масса, водитель, близнецы.
		\end{mydescr}
	\item \textbf{Криттика} --- накшатра огня.
		\begin{mydescr}
			\item[Протяженность] --- \signum{26}{40}{\aries} -- \signum{10}{}{\taurus}
			\item[Идеи] --- война, битва, командующий, защитник, слава, знаменитый, великие деяния, огонь, аппетит, приготовление пищи, человеческая кожа, кожаные изделия, бумага, документ, белые пятна на коже. яркий, награжденный, богатство, золото, средства передвижения, темная сторона Луны, обильный, большой, усыновленный ребенок.
		\end{mydescr}
	\item \textbf{Рохини} --- накшатра восхождения.
		\begin{mydescr}
			\item[Протяженность] --- \signum{10}{}{\taurus} -- \signum{23}{20}{\taurus}
			\item[Идеи] --- восходить, карабкаться, подниматься, заниматься альпинизмом, высота, продвижение по службе, рост, развитие, рождение, производство, распространение, потомство, сажать, сеять, выращивать, воспаление, заболевание горла, кровь, красный цвет, духи, аромат, домашний скот.
		\end{mydescr}
	\item \textbf{Мригашира} --- накшатра поиска.
		\begin{mydescr}
			\item[Протяженность] --- \signum{23}{20}{\taurus} -- \signum{6}{40}{\gemini}
			\item[Идеи] --- целеустремленно искать, исследовать, находить потерянное, стремиться, достигать, ходатайствовать, очищать, предлагать девушке выйти замуж, выслеживать, дорога, тропинка, путешествие, указывать путь, проводник, лидер, вождь, охотиться, анализировать, изучать, научные исследования.
		\end{mydescr}
	\item \textbf{Ардра} --- накшатра угнетения и притеснения.
		\begin{mydescr}
			\item[Протяженность] --- \signum{6}{40}{\gemini} -- \signum{20}{}{\gemini}
			\item[Идеи] --- угнетать, подавлять, мучительный, разрушать, убивать, раздавать, слезы, печаль, боль, горечь, жадный, жестокий, охотник, влажный, мокрый, нежный, переполненный чувствами, свежий, жидкий.
		\end{mydescr}
	\item \textbf{Пунарвасу} --- накшатра обновления.
		\begin{mydescr}
			\item[Протяженность] --- \signum{20}{}{\gemini} -- \signum{3}{20}{\cancer}
			\item[Идеи] --- место поселения, жилище, родина, восстановление богатства, собственность, ремонт в доме, возвращение из путешествия, пребывание где-то, повторение, приобретать, свободный, свобода, безопасность, бесконечность, неразрывность.
		\end{mydescr}
	\item \textbf{Пушья} --- накшатра цветения
		\begin{mydescr}
			\item[Протяженность] --- \signum{3}{20}{\cancer} -- \signum{16}{40}{\cancer}
			\item[Идеи] --- цветение, цветок, насыщение, процветать, увеличение, приобретать, получать сполна, жирность, богатство, изобилие, полнота, счастливый, благоприятный, молитва, речь, красноречие, мудрость.
		\end{mydescr}
	\item \textbf{Ашлеша} --- цепляющаяся накшатра.
		\begin{mydescr}
			\item[Протяженность] --- \signum{16}{40}{\cancer} -- \signum{0}{}{\leo}
			\item[Идеи] --- соединение, связь, союз, сексуальное партнерство, интимный контакт, объятие, пожатие, сплетение, обвиваться, скручиваться, змея, яд, мучение, жжение, боль.
		\end{mydescr}
	\item \textbf{Магха} --- накшатра восхищения и славы.
		\begin{mydescr}
			\item[Протяженность] --- \signum{0}{}{\leo} -- \signum{13}{20}{\leo}
			\item[Идеи] --- выдающийся, великий, благородный, самый уважаемый, важный, высокий, старик, величество, высочество, могущественный, возбуждать, радовать кого-то, высокая честь, поднимать настроение, богатство, власть, щедрый, отец, родители, предки, благосостояние, совершенство.
		\end{mydescr}
	\item \textbf{Пурвапхалгуни} --- накшатра удачи.
		\begin{mydescr}
			\item[Протяженность] --- \signum{13}{20}{\leo} -- \signum{26}{40}{\leo}
			\item[Идеи] --- любовь, привязанность, страсть, любовные удовольствия, флирт, праздное развлечение, производство плодов, устранение зла, исправление, очищение, получение вознаграждения, усовершенствование, реформация, опыт, следователь, практиковать, выбирать, служить, чтить, уважать.
		\end{mydescr}
	\item \textbf{Уттарпхалгуни} --- накшатра покровителя.
		\begin{mydescr}
			\item[Протяженность] --- \signum{26}{40}{\leo} -- \signum{10}{}{\virgo}
			\item[Идеи] --- покровительство, доброта, благодетельство, Друзья детства, компаньоны. Люди, к которым обращаются за помощью. Друзья, к которым приходят излить душую Лица, оказывающие финансовую поддержку. Целители, от которых ждут облегчения болезней.
		\end{mydescr}
	\item \textbf{Хаста} --- накшатра, означающая ``сжатый кулак''
		\begin{mydescr}
			\item[Протяженность] --- \signum{10}{}{\virgo} -- \signum{23}{20}{\virgo}
			\item[Идеи] --- удерживание в руке, высмеивать, оживлять, шутка, острота, почерк, мастерство, ``золотые'' руки, превосходить кого-то в чем-то, приводить в движение, команидровать, осуществлять контроль, открывать и раскрывать, расширяться, обнажать, резать, убирать урожай, косить, жать, укладывать скирды, веселье, посвящать и освящать.
		\end{mydescr}
	\item \textbf{Читра} --- накшатра ``чудесная''
		\begin{mydescr}
			\item[Протяженность] --- \signum{23}{20}{\virgo} -- \signum{6}{40}{\libra}
			\item[Идеи] --- выдающийся, великолепный, яркоокрашенный, пестрый, пятнистый, многогранный, чудесный, замечательный, многообразие, изумление, зодчий, украшенный орнаментом, картина, чертеж, план, приковывающее внимание.
		\end{mydescr}
	\item \textbf{Свати} --- накшатра самостоятельных действий.
		\begin{mydescr}
			\item[Протяженность] --- \signum{6}{40}{\libra} -- \signum{20}{}{\libra}
			\item[Идеи] --- осознание своего ``я'', идти постоянно самому, достигающий что-то собственными силами, самоподдержва, самостоятельно овладевающий каким-то навыком, провозглашать как свое собственное, материальные блага приходят и уходят сами по себе, владение и потеря богатства происходит неконтролируемым образом, воздух, ветер, шторм.
		\end{mydescr}
	\item \textbf{Висакха} --- накшатра цели.
		\begin{mydescr}
			\item[Протяженность] --- \signum{20}{}{\libra} -- \signum{3}{20}{\scorpio}
			\item[Идеи] --- идущий прямо к достижению цели, достигнутый результат, доказанный вывод, доказательство, доктрина, догма, истреблять ради достижения какой-то цели, повреждать, причинять боль, разрушать, примирять, располагать к себе, воспевать, обожать, конечная цель, просящий милостыню.
		\end{mydescr}
	\item \textbf{Анурадха} --- накшатра, призывающая к деятельности.
		\begin{mydescr}
			\item[Протяженность] --- \signum{3}{20}{\scorpio} -- \signum{16}{40}{\scorpio}
			\item[Идеи] --- друг, союзник, сотрудник, помощник. То, что учреждается и устанавливается. Нечто утверждающееся как сила и власть. Суждение, наблюдение, познание, призыв людей к активности, действовать, будучи объединенными дружбой и общей целью. Союз с кем-то ради общей цели.
		\end{mydescr}
	\item \textbf{Джиешта} --- накшатра ``главная''
		\begin{mydescr}
			\item[Протяженность] --- \signum{16}{40}{\scorpio} -- \signum{0}{}{\sagittarius}
			\item[Идеи] --- великолепный, выдающийся, первый, величайший, тот, которого прославляют, верховная власть, могущественный правитель, провозглашать что-то, старший брат, старший по должности, глава семьи, жена, любимая, возлюбленная.
		\end{mydescr}
	\item \textbf{Мула} --- накшатра ``корневая''
		\begin{mydescr}
			\item[Протяженность] --- \signum{0}{}{\sagittarius} -- \signum{13}{20}{\sagittarius}
			\item[Идеи] --- иметь корни, самая низкая часть чего-нибудь, корень растения, начало, причина, источник знаний, главный город илил столица, оригинальный текст, первоначальный, исходный, беспрецендентный, связывать, привязывать, фиксировать, удерживать, брать в плен, связь, облигация, сдерживать, подавлять, угнетать, неагрессивный, наносящий обиду.
		\end{mydescr}
	\item \textbf{Пурвашадха} --- накшатра ``непобедимая''
		\begin{mydescr}
			\item[Протяженность] --- \signum{13}{20}{\sagittarius} -- \signum{26}{40}{\sagittarius}
			\item[Идеи] --- победоносный, побеждать, превосходить, завоевание и поражение, насилие, воспротивиться чему-то, посстать, сопротивляться, переносить страдания и муки, с пониманием относиться к любому человека, терпеливо ждать подходящего времени, всепрощение, снисхождение, завершение, прийти к концу, поверхность, вода.
		\end{mydescr}
	\item \textbf{Уттарашадха} --- накшатра ``всеобщая''
		\begin{mydescr}
			\item[Протяженность] --- \signum{26}{40}{\sagittarius} -- \signum{10}{}{\capricornus}
			\item[Идеи] --- входить, поселяться, проникать, быть поглощенным, входить в соединение, входить в дом, появляться на сцене, отдыхать, возникновение мысли, принадлежать чему-то, существовать ради кого-то принимать на себя, начинать что-то помнить о каком-то деле, заставлять войти, быть причиной вхождения, универсальность, всеобщность.
		\end{mydescr}
	\item \textbf{Шравана} --- накшатра изучения.
		\begin{mydescr}
			\item[Протяженность] --- \signum{10}{}{\capricornus} -- \signum{23}{20}{\capricornus}
			\item[Идеи] --- тот, кто учился, научный работник, знание, изучение, быть внимательным и послушным, интеллектуальные способности, быть известным и почитаемым, быть услышанным, рассказывать, общаться, священные знания, откровения, таинства, слухи, сообщения новосте, слова и языки, словарный запас, заботиться о ком-то, ученик, последователь, учитель, брать с кого-то пример, ливень, выделения.
		\end{mydescr}
	\item \textbf{Дханишта} --- накшатра симфонии.
		\begin{mydescr}
			\item[Протяженность] --- \signum{23}{20}{\capricornus} -- \signum{6}{40}{\aquarius}
			\item[Идеи] --- музыка, пение, нота, звучание, богатство, драгоценные камин, драгоценности, всё что высоко ценится, влага, потеть, задняя или боковая часть чего-то.
		\end{mydescr}
	\item \textbf{Сатабиша} --- накшатра прикрытия.
		\begin{mydescr}
			\item[Протяженность] --- \signum{6}{40}{\aquarius} -- \signum{20}{}{\aquarius}
			\item[Идеи] --- скрывать, прятать, быть захваченным, удерживать в плену, иметь препятствия, океан, море, озеро, река, пруд, дожди, резервуары с водой, защитник от зла, вооружение, вся верхняя одежда, целитель, врач, лечение, нахождение лечебного средства, неизлечимые заболевания, паралич, водянка, западня.
		\end{mydescr}
	\item \textbf{Пурвабхадрапада} --- накшатра ``пара бешено несущихся лошадей''.
		\begin{mydescr}
			\item[Протяженность] --- \signum{20}{}{\aquarius} -- \signum{3}{20}{\pisces}
			\item[Идеи] --- нестись во всю прыть, гореть, сжигать, пылкий, страстный, порывистый, импульсивный, стремительный, жгучая боль, наказывать, подвергать телесному наказанию, мучать кого-то, угнетать, подавлять, ущемлять, печаль, горе, обида, падать, погибать, уходить.
		\end{mydescr}
	\item \textbf{Уттарабхадрапада} --- накшатра подобна Пурвабхадрападе. Если Пурвабхадрапада вызывает гнев, то Уттарабхадрапада дает силы его контролировать.
		\begin{mydescr}
			\item[Протяженность] --- \signum{3}{20}{\pisces} -- \signum{16}{40}{\pisces}
			\item[Идеи] --- уезжать куда-то, оставлять все дома, личность, знания, мудрость, отерчение. остальные идеи такие же, как у Пурвабхадрапады.
		\end{mydescr}
	\item \textbf{Ревати} --- накшатра указывает на того, кто содержит стадо овец.
		\begin{mydescr}
			\item[Протяженность] --- \signum{16}{40}{\pisces} -- \signum{0}{}{\apies}
			\item[Идеи] --- питание, кормление, выращивать, богатый, процветающий, роскошный, обильный.
		\end{mydescr}
\end{myenum}

К толкованию планет в лунных созвездиях карты рождения надо подходить очень осторожно, так как не все астрологи, практикующие Индийскую предсказательную астрологию, используют идеи накшатр. Для определения будущих событий вполне достаточно сделать анализ и синтез идей планет в знаках и домах гороскопа. Но если вы хотите знать детали этих событий, вы должны аккуратно подойти к использованию идей лунных созвездий. Индийские астрологи были подлинными астерами этого высокого искусства.


\section{Знаки Зодиака и их идеи}

Планеты движутся по Зодиакальному поясу, который равен \(360^\circ\). Он разделен на 12 равных частей, каждая из которых равна \(30^\circ\) пространства. Каждые \(30^\circ\) пространства представляют 12 знаков Зодиака.

Каждый знак Зодиака имеет свое название, символ, свойства и особенности (название и символ даны в таблице~\ref{tbl:signs}).
Знаки бывают подвижные, неподвижные и двойственные:

\begin{table}[tph!]
	% Расширить по вертикали
	\renewcommand{\arraystretch}{1}

	% Заполним данными
	\begin{tabular}{l|l|l}
		Подвижные    & Овен, Рак, Весы, Козерог & энергия, движение, \\
		             & & инициатива \\
		Неподвижные  & Телец, Лев, Скорпион, Водолей & устойчивость, консерватизм, \\
		             & & равновесие \\
		Двойственные & Близнецы, Дева, Стрелец, Рыбы & непостоянство, неустойчивость, \\
		             & & изменения \\
	\end{tabular}
\end{table}

Двеннадцать знаков Зодиака представляют четыре элемента природы (стихии): огонь, воздух, землю и воду. Существуют четыре группы знаков, каждая из которых соответствует определенной стихии.

\begin{table}[tph!]
	% Расширить по вертикали
	\renewcommand{\arraystretch}{1}

	% Заполним данными
	\begin{tabular}{l|l|l}
		Знаки Огня    & Овен, Лев, Стрелец & творческий дух, активность, \\
		              & & духовный рост \\
		Знаки Воздуха & Близнецы, Весы, Водолей & интеллектуальная деятельность, \\
		              & & взаимодействия, контакты \\
		Знаки Земли   & Телец, Дева, Козерог & устойчивость, прочность, \\
		              & & материальность \\
		Знаки Воды    & Рак, Скорпион, Рыбы & неустойчивость, расплывчатость, \\
		              & & эмоциональность \\
	\end{tabular}
\end{table}

Знаки Зодиака разделены на мужские и женские.

\begin{table}[tph!]
	% Расширить по вертикали
	\renewcommand{\arraystretch}{1}

	% Заполним данными
	\begin{tabular}{ll}
		Мужские(активные) знаки: & Овен, Близнецы, Лев, Весы, Стрелец, Водолей \\
		Женские(пассивные) знаки: & Телец, Рак, Дева, Скорпион, Козерог, Рыбы \\
	\end{tabular}
\end{table}

Каждый знак Зодиака выражает присущие ему идеи в проявленном материальном мире:

\begin{myenum}[topsep=0]
	\item \textbf{Овен:}
		\begin{mydescr}
			\item[Физиология] --- голова, мозг, лоб
			\item[Направление] --- восток
			\item[Идеи] --- соперничество, рискованные дела, борьба, война, соревнование, огонь, сила, мощь, энергия, сопротивление, воин, спор, сексуальность, быстрое переваривание пищи, оружие, злобность, страстные действия, травянистый участок земли, богатство, противостояние.
		\end{mydescr}
	\item \textbf{Телец:}
		\begin{mydescr}
			\item[Физиология] --- лицо, шея
			\item[Направление] --- юг
			\item[Идеи] --- материальные вещи, сельское хозяйство, земля, молоко, любовь к чувственным удовольствиям, добродетельный труд, справедливые дела, крупный рогатый скот, товары ``люкс'', приятный запах, духи, предметы, связанные с ощущениями, оплодотворение.
		\end{mydescr}
	\item \textbf{Близнецы:}
		\begin{mydescr}
			\item[Физиология] --- плечи, руки, кисти
			\item[Направление] --- запад
			\item[Идеи] ---  интеллект, результат интеллектуальной деятельности, средства информации и связи, быстрое схватывание и передача мыслей другим, нерешительность, медицина, память, гибкость.
		\end{mydescr}
	\item \textbf{Рак:}
		\begin{mydescr}
			\item[Физиология] --- сердце, грудь, легкие
			\item[Направление] --- север
			\item[Идеи] --- духовность, эмоциональные реакции, домашние дела, водянистые плоды с твердой кожурой, мазь, грязь, броня, кожа, шкура, химия, объятие.
		\end{mydescr}
	\item \textbf{Лев:}
		\begin{mydescr}
			\item[Физиология] --- желудок, живот
			\item[Направление] --- восток
			\item[Идеи] --- величие, известность, достоинство, храбрость, сила, гордость, высокомерие, хвастливость, рискованное предприятие, превосходство, власть, огонь, лавры, почести, электричество, сладости, большие достижения, победоносный.
		\end{mydescr}
	\item \textbf{Дева:}
		\begin{mydescr}
			\item[Физиология] --- таз
			\item[Направление] --- юг
			\item[Идеи] --- доброта, чистота, подчинение, служба, мастерство, любезность, воодушевление, приятное ощущение, выразительность, быть в поиске чего-то, детализация чего-либо, удовольствие, проливать свет на что-либо, агрокультура, бумага.
		\end{mydescr}
	\item \textbf{Весы:}
		\begin{mydescr}
			\item[Физиология] --- место под пупком
			\item[Направление] --- запад
			\item[Идеи] --- торговля, весы, коммерция, продавец, чувство справедливости, вынесение справедливого приговора, доказательство, третейский суд, арбитраж, полотно, цвета и оттенки, художники, артисты.
		\end{mydescr}
	\item \textbf{Скорпион:}
		\begin{mydescr}
			\item[Физиология] --- половые органы
			\item[Направление] --- север
			\item[Идеи] --- психические наклонности, скрытые действия, острая боль, порез, ранение, хирургическая операция, неожиданно нападать, бросать вызов, все режущие инструменты, отделение от чего-либо или кого-либо, разрушать, уничтожать, дематериализация, скобяные изделия, товары из воды.
		\end{mydescr}
	\item \textbf{Стрелец:}
		\begin{mydescr}
			\item[Физиология] --- бедра
			\item[Направление] --- восток
			\item[Идеи] --- высокая цель, честолюбие, сила, стрелять, лететь с силой, метать, толкать, продвигать вперед, воин, стрела, пуля, оружие, боеприпасы, артиллерия, смелый, героический, достигать цели, медицина, животные.
		\end{mydescr}
	\item \textbf{Козерог:}
		\begin{mydescr}
			\item[Физиология] --- колени, икры
			\item[Направление] --- юг
			\item[Идеи] --- медленное передвижение, производитель, строитель, создатель, находящийся в центре внимания, заключать определенную форму, знаменитый, прославленный, утверждаться в своей работе, стремиться к высоким должностям.
		\end{mydescr}
	\item \textbf{Водолей:}
		\begin{mydescr}
			\item[Физиология] --- голеностопный сустав
			\item[Направление] --- запад
			\item[Идеи] --- оккультные знания, скрытое содержание, связь, дыхательные упражнения, ухудшение и удушье, болезни глаз, болезни сердца, поток, разлив реки, изобилие воды, две стороны во время переговоров, посредник, интимность, керамика, стекло, фарфор, сосуд для воды, гроб, могила, сейф, хранилище, товары ``люкс''.
		\end{mydescr}
	\item \textbf{Рыбы:}
		\begin{mydescr}
			\item[Физиология] --- стопы
			\item[Направление] --- север
			\item[Идеи] --- мистические идеи, жертва, спасение, искушение, обольщение, соблазнение, доставлять и получать удовольствие, быть пьяным, распространение, расширение, погибать, умирать, уменьшать, продукты из воды, жиры и смазки, драгоценные камни, золото, отравление.
		\end{mydescr}
\end{myenum}

\section{Двенадцать домов гороскопа и их идеи}

Существуют две модели Солнечной системы: гелеоцентрическая и геоцентрическая. Гелеоцентрическая модель Солнечной системы --- это та модель, которая реально существует в космическом пространстве: вокруг Солнца движутся планеты. Все астрономические открытия делались на основе гелеоцентрической модели мироздания.

Астрология рассматривает движения планет в основе геоцентрической модели Солнечной системы, то есть в центр Солнечной системы условно ставится Земля и при этом движение планет относительно земного наблюдателя происходит вокруг Земли против часовой стрелки.

Древние риши определили, что весь спектр идей, относящихся к человеку и его деятельности, представлен девятью планетами, поэтому в Индийской предсказательной астрологии используются знания только об этих планетах.

Планеты бывают благоприятными и неблагоприятными по своей природе. Юпитер и Венера --- благоприятные, Сатурн, Марс, Раху, Кету и Солнце --- неблагоприятные. Меньше всего зла исходит от Солнца. Луна и Меркурий считаются нейтральными по своей природе и могут быть благоприятными и неблагоприятными в зависимости от их расположения на карте рождения. Луна растущая благоприятна, убывающая --- наоборот, неблагоприятна. Меркури, находящийся в знаке Зодиака один или в связи с благоприятной планетой, будет благоприятным. Если он находится в одном знаке Зодиака с какой-либо неблагоприятной планетой, то тоже становится неблагоприятным.

Планеты разделены на мужские, женские и нейтральные.

\begin{table}[tph!]
	% Расширить по вертикали
	\renewcommand{\arraystretch}{1}

	% Заполним данными
	\begin{tabular}{l|l|l}
		Мужские планеты & Солнце, Марс, Юпитер & активность, мужское начало \\
		Женские планеты & Луна, Венера         & мягкость, женское начало \\
		Нейтральные планеты & Меркурий, Сатурн & проявляют оба начала в соответствии \\
		                     & Раху и Кету      & с полом знака Зодиака \\
	\end{tabular}
\end{table}


Планеты кроме Раху и Кету соответствуют семи дням недели:

\begin{table}[tph!]
	% Расширить по вертикали
	\renewcommand{\arraystretch}{1}

	% Заполним данными
	\begin{tabular}{ll}
		Воскресенье & Солнце \\
		Понедельник & Луна \\
		Вторник     & Марс \\
		Среда       & Меркурий \\
		Четверг     & Юпитер \\
		Пятница     & Венера \\
		Суббота     & Сатурн \\
	\end{tabular}
\end{table}

В Индийской предсказательной астрологии каждая планета несет определенные идеи в жизнь человека. Знание их необходимо для правильного прочтения карты рождения.

\begin{myenum}[topsep=0]
	\item \textbf{Солнце:}
		\begin{mydescr}
			\item[Физиология] --- сердце, мозг, голова, кости, голосовые связки, правый глаз.
			\item[Направление] --- восток.
			\item[Идея] --- индивидуальная душа (эго), отец, представители власти, способность к самоосознанию, статус, уверенность, благородство, слава, сознание, правда, истина, храбрость, решительность, энергия, политика, химия, медицина, врачи, хирурги, целители, храмы и места поклонений, места жертвоприношений, залы для коронования, осветительные приборы, высокий пост в правительстве.
		\end{mydescr}
	\item \textbf{Луна:}
		\begin{mydescr}
			\item[Физиология] --- молочные железы, матка, кровообращение, левый глаз.
			\item[Направление] --- северо-запад.
			\item[Идея] --- мать, женщина, известность, имя, ум, вода, океан, море, реки, путешествие и путешественники, лекарственные растения, сочные плоды, женственность, фантазии, перемены, молоко, популярные курорты, фармацевты и аптекари, зеркало, рыболов, грациозность, чувства.
		\end{mydescr}
	\item \textbf{Марс:}
		\begin{mydescr}
			\item[Физиология] --- нос, мышечная ткань, сухожилия, половые органы.
			\item[Направление] --- юг.
			\item[Идея] --- братья и сестры, решительность, твердость, действия, благосостояние, недвижимость, энергия, сила, земельный участок, пожар, военные, военная операция, режущие инструменты, инженеры, математика, электроника, калькулятор, хирурги, катастрофы, горы, леса, тяжба, спор, спорт, землетрясение, раны и травмы, жар, сапфиры, кораллы, сокровища, химическая лаборатория, война, битва, соревнование, строительство, сила духа, холодное и огнестрельное оружие, ожоги, должность в армии и полиции, конструктор, хирургическая операция, разрыв, разрушение, несчастный случай.
		\end{mydescr}
	\item \textbf{Меркурий:}
		\begin{mydescr}
			\item[Физиология] --- легкие, кишечник, брюшная полость, язык, кисти рук, нервные центры.
			\item[Направление] --- север.
			\item[Идея] --- интеллект, интеллектуальная деятельность, речь, литературные способности, средства информации и связи, рационализм, секретарская работа, бухгалтер, ученый, учебные заведения, головная и желудочная боль, рекламные агенства, издательство и издатели, исследование, новости, коммерческая деятельность, друзья.
		\end{mydescr}
	\item \textbf{Юпитер:}
		\begin{mydescr}
			\item[Физиология] --- печень, нижняя часть брюшной полости, органы слуха, таз.
			\item[Направление] --- северо-восток.
			\item[Идея] --- поведение, красноречие, мудрость, философия, духовный рост, религия, духовный наставник, учитель, ведические знания, набожность, религиозные учреждения, благотворительность, профессор, склонность к полноте, синтез, аргументация, получивший высокую оценку, дети, банки и банкиры, легальная деятельность, министры, юристы, справедливость, почтенные и уважаемые люди, советники, священники, беременные женщины, муж, поддерживающий жизнь.
		\end{mydescr}
	\item \textbf{Венера:}
		\begin{mydescr}
			\item[Физиология] --- почки, яичники, выделения, половая система.
			\item[Направление] --- юго-восток
			\item[Идея] --- желания, привязанность, жадность, ревность, праздность, красота, живописные места, текстильная продукция, люди искусства, товары ``люкс'', спиртные напитки, окружающая среда, спальня, комфорт, сексуальные удовольствия, венерические болезни, парфюмерия, транспортные средства, предметы роскоши, цветы, ботаника, творческие способности, муж, жена, любовник и любовница, украшения, декорации, театры, музеи, кино, сладости, ювелирные украшения.
		\end{mydescr}
	\item \textbf{Сатурн:}
		\begin{mydescr}
			\item[Физиология] --- ноги, колени, костный мозг, мочевой пузырь, дыхательная система, принцип передвижения.
			\item[Направление] --- запад
			\item[Идея] --- время, вызывающеее изменения, смерть, лишения, результат длительных действий, аскетизм, горе, печаль, подлость, обман, воры, мошенники, слуги, работа в подчинении кого-то, рабочий, самоотречение, пожилые люди, уход в отставку, места захоронения, судебный исполнитель, нищий, настенные или настольные часы, отсрочка, заблуждение, ложный вывод, уголовный розыск, черная магия, тайные знания, практика йоги, изменение обстоятельств, разорение и крах, предательство, сила влияния на других, нефть, железо, люди низкого происхождения, должности, получаемые за услуги, консерватизм, сфера услуг, препятствие.
		\end{mydescr}
	\item \textbf{Раху:}
		\begin{mydescr}
			\item[Физиология] --- скулы, кожа, выделительная система, глотание, пищеварительный тракт, прямая кишка.
			\item[Направление] --- юго-запад
			\item[Идея] --- страсть, толпа, мятеж, восстание, ядовитые рептилии, бедствие, наводнение, люди низкой культуры, воры и мошенники, гнев, гордыня, тупость, глупость, мистические знания, таинства, тайная доктрина, грубость, путешествия, муравейник, злое предсказание, эпидемии, насилие, коррупция, незаконные действия, охотники, тяжба, инфекционные заболевания, припадок, удушье, способность, оказывать влияние на других, лишения, крупное воровство.

Раху действует подобно Сатурну и провляет энергии тех планет, которые оказывают на нее влияние.
		\end{mydescr}
	\item \textbf{Кету:}
		\begin{mydescr}
			\item[Физиология] --- позвоночник, спинно-мозговой канал, нервная система, половая система.
			\item[Направление] --- направления не имеет/
			\item[Идея] --- препятствия и помехи, несчастный случай, духовные силы, психическое состояние, астрология, банкротство, клевета, оккультизм, падение, болезнь и недомогание, огонь, математика и мат. способности, эпидемии, страх, нервозность, испуг, яд, больничная палата, мелкое воровство, неожиданность, рана.

Кету действует подобно Марсу и проявляет энергии тех планет, которые оказывают на нее влияние.
		\end{mydescr}
\end{myenum}

Главные идеи девяти планет очень важно знать напамять при анализе карты рождения. А также следует не забывать, что сильные и удачно расположенные планеты в карте рождения проявят свои хорошие качества, а слабые и неудачно расположенные планеты --- дадут плохой результат.

\subsection{Определение силы планет}

В момент рождения человека планеты находятся в различных знаках Зодиака и в зависимости от этого проявляют свою силу или слабость.o

Существует множество способов для определения силы планет с огромным количеством вычислений и составлением дополнительно шестнадцати различных карт, аналогичных карте навамса, о которой будет рассказано далее. Однако на практике большинство астрологов пользуется для этого следующими правилами:
\begin{myenum}[itemsep=0,parsep=0]
	\item Планета в знаке экзальтации.\label{cost:1}
	\item Планета в мулатриконе.\label{cost:2}
	\item Планета в собственном знаке.\label{cost:3}
	\item Планета в знаке большого друга.\label{cost:4}
	\item Планета в дружественном знаке.\label{cost:5}
	\item Планета в нейтральном знаке.\label{cost:6}
	\item Планета во враждебном знаке.\label{cost:7}
	\item Планета в знаке большого врага.\label{cost:8}
	\item Планета в знаке ослабления.\label{cost:9}
\end{myenum}

Данные девять достоинств расположены в порядке убывания силы планет. Планета имеет наибольшую силу, если находится в знаке экзальтации, тогда она проявляет лучшие свои качества и приносит человеку счастье, удачу и доход. В знаке ослабления планеты теряют силы и сулят человеку горе, страдания и убытки. Промежуточные достоинства планет дают неоднозначный результат, в них проявляется как хорошее, так и плохое. Но чем выше достоинство планеты (мулатрикона, собственный знак, знак большого друга), тем сильнее в ней доброе начало, меньше плохих качеств. И наоборот, чем ниже достоинство планеты (враждебный знак или знак большого врага), тем больше будет проявляться зло и меньше хороших качеств.

Достоинство планет по пунктам \ref{cost:1}, \ref{cost:2}, \ref{cost:3} и \ref{cost:9} легко определить по таблице~\ref{tbl:cost}, где перечислены знаки, которые являются экзальтирующими, знаками мулатриконы, собственными и ослабляющими для каждой планеты.

Сначала надо посмотреть, в каком знаке Зодиака эта планета расположена в карте рождения, а затем по таблице определить, какое достоинство этот знак придает соответствующей планете. Например, если в карте рождения Юпитер находится в Рыбах, значит Юпитер в собственном знаке. Возле каждого знака экзальтации проставлено значение градуса (например, Овен \(10^\circ\)) --- это значит, что если планета в знаке находится именно в таком градусе, она наиболее сильная (то есть это градус наивысшей экзальтации планеты в знаке). Аналогично для знака ослабления --- указанное значение градуса определяет точку наибольшего ослабления соответствующей планеты в данном знаке.

\begin{table}[tph!]
	\caption{Определение достоинств планет}
	\label{tbl:cost}

	\centering

	% Расширить по вертикали
	\renewcommand{\arraystretch}{1}

	% Разширить вширь
	%\setlength{\tabcolsep}{.05\textwidth}

	% Заполним данными
	\begin{tabular}{|l|l|l|l|l|}
		\hline
		Планета & Знак экзальтации & Знак мулатрикона & Собственный знак & Знак ослабления \\
		\hline
		Солнце   & Овен \gradus{10}    & Лев \gradus{0}--\gradus{20}     & Лев & Весы \gradus{10}    \\
		Луна     & Телец \gradus{3}    & Телец \gradus{4}--\gradus{30}   & Рак & Скорпион \gradus{3} \\
		Марс     & Козерог \gradus{28} & Овен \gradus{1}--\gradus{12}    & Овен, Скорпион   & Рак \gradus{28}    \\
		Меркурий & Дева \gradus{15}    & Дева \gradus{16}--\gradus{20}   & Близнецы, Дева   & Рыбы \gradus{15}   \\
		Юпитер   & Рак \gradus{5}      & Стрелец \gradus{1}--\gradus{10} & Стрелец, Рыбы    & Козерог \gradus{5} \\
		Венера   & Рыбы \gradus{27}    & Весы \gradus{0}--\gradus{5}     & Телец, Весы      & Дева \gradus{27}   \\
		Сатурн   & Весы \gradus{20}    & Водолей \gradus{1}--\gradus{20} & Козерог, Водолей & Овен \gradus{20}   \\
		Раху     & Телец               & Близнецы & Водолей & Скорпион \\
		Кету     & Скорпион            & Стрелец  & Лев & Телец \\
		\hline
	\end{tabular}
\end{table}

Возле каждого знака мулатрикона также проставлены значения градусов (например, Лев \(0{^\circ}-20^\circ\)) --- это значит, что достоинство или сила мулатрикона действует в пределах указанных градусов, а в остальном пространстве знака планета будет иметь достоинство знака экзальтации или собственного знака в соответствии с таблицей\footnote{Некоторые индийские астрологические школы указывают позицию мулатрикона для Венеры от \gradus{0} Весов до \gradus{15} Весов.}.

Солнце и луна имеют по одному собственному знаку, а остальные пять планет по два. Индийские астрологи считают, что Раху и Кету не имеют собственных знаков, так как они не обладают физическими телами, но их расположение в знаках Водолея и Льва придает им силу, соответствующую достоинству собственного знака.

По терминологии, принятой в Индии, планета является хозяином собственного знака, то есть если Лев --- собственный знак для Солнца, то Солнце --- хозяин Льва и\,т.\,д.

Для того, чтобы определить достоинства планет по пунктам \ref{cost:4}, \ref{cost:5}, \ref{cost:6}, \ref{cost:7} и \ref{cost:8}, то есть в знаке большого друга, дружественном знаке, нейтральном знаке, враждебном знаке и знаке большого врага, необходимо ввести дополнительные термины, такие как \emph{постоянные} отношения и \emph{временные} отношения между планетами.

Каждая планета, кроме Раху и Кету, имеет дружественные, нейтральные или враждебные отношения с другими планетами. Их можно установить с помощью приведенной таблицы~\ref{tbl:relations}, отражающие постоянные отношения между планетами. Из этой таблицы видно, что для Солнца дружественными являются Луна, Марс и Юпитер, нейтральным --- Меркурий, а враждебными --- Сатурн и Венера. И так для каждой планеты, находящейся в левой колонке. Отношения между планетами устанавливаются через знаки, хозяевами которых они являются. Например, если Солнце находится в Раке, а хозяйкой Рака является Луна, которая дружественных отношениях с Солнцем, Значит Солнце находится в дружественном знаке. Таким образом определяют постоянные отношения для каждой планеты в гороскопе.

\begin{table}[tph!]
	\caption{Характер отношений между планетами}
	\label{tbl:relations}

	\centering

	% Расширить по вертикали
	\renewcommand{\arraystretch}{1}

	% Разширить вширь
	%\setlength{\tabcolsep}{.05\textwidth}

	% Заполним данными
	\begin{tabular}{|l|l|l|l|}
		\hline
		Планета & Дружественные & Нейтральные & Враждебные \\
		\hline
		Солнце   & Луна, Марс, Юпитер & Меркурий & Сатурн, Венера \\
		Луна     & Солнце, Меркурий & Марс, Юпитер, Венера, Сатурн & --- \\
		Марс     & Солнце, Луна, Юпитер & Венера, Сатурн & Меркурий \\
		Меркурий & Солнце, Венера & Марс, Юпитер, Сатурн & Луна \\
		Юпитер   & Солнце, Луна, Марс & Сатурн & Меркурий, Венера \\
		Венера   & Меркурий, Сатурн & Марс, Юпитер & Солнце, Луна \\
		Сатурн   & Меркурий, Венера & Юпитер & Солнце, Луна, Марс \\
		\hline
	\end{tabular}
\end{table}

Временные отношения между планетами определяются по их взаимному расположению в домах. Они бывают двух типов --- временная дружба и временная вражда. Если одна планета расположена относительно другой во 2, 3, 4, 10, 11 или 12-м доме, значит эти две планеты находятся во временных дружественных отношениях. Если две планеты находятся в одном доме или одна планета расположена относительно другой в 5, 6, 7, 8 или 9-м доме, значит эти две планеты находятся во временных враждебных отношениях. Надо обратить внимание на то, что при установлении временных отношений между планетами за первый дом принимается тот дом, где находится планета, достоинство которой вы определяете.

Установив постоянные и временные отношения между планетами в гороскопе, можно приступить к определению достоинств планет в соответствии с пунктами \ref{cost:4}, \ref{cost:5}, \ref{cost:6}, \ref{cost:7} и \ref{cost:8} из правила определения силы планет по девяти достоинствам:

\begin{myenum}[itemsep=0,parsep=0]
	\item Постоянные дружественные отношения + дружба временная = планета в знаке большого друга.
	\item Постоянные нейтральные отношения + дружба временная = планета в дружественном знаке.
	\item Постоянные враждебные отношения + дружба временная = планета в нейтральном знаке.
	\item Постоянные дружественные отношения + вражда временная = планета в нейтральном знаке.
	\item Постоянные нейтральные отношения + вражда временная = планета во враждебном знаке.
	\item Постоянные враждебные отношения + вражда временная = планета в знаке большого врага. 
\end{myenum}

Покажем как определяются достоинства планет в знаках Зодиака на нескольких примерах:

\begin{myenum}
	\item Солнце расположено в Тельце, а Венера в Овне. Солнце находится во враждебном знаке, так как хозяйка Тельца --- Венера, согласно таблице~\ref{tbl:relations}, враждебна к Солнцу. Если считать Телец, в котором находится Солнце, 1-м домом, то Венера относительно Солнца находится в Овне, то есть в 12-м доме, что говорит о временной дружбе этих планет. Теперь складываем постоянные враждебные отношения с временной дружбой и получаем, что Солнце находится в нейтральном знаке.
	\item Марс расположен во Льве, хозяином которого является Солнце. Солнце дружественно Марсу и находится в Деве, то есть во 2-м доме от Марса, что указывает на временные дружественные отношения. Сложив постоянные дружественные отношения со временной дружбой, определяем Марс как находящийся в знаке большого друга.
	\item Венера находится в Раке, хозяином которого является Луна. Луна для Венеры враждебна. Луна находится в Козероге, то есть в 7-ом доме от Венеры. Луна --- временный враг Венеры. Сложив постоянные враждебные отношения со временной враждой, определяем Венеру как находящуюся в знаке большого врага.
\end{myenum}


\subsubsection*{Определение силы планет в зависимости от их расположения в знаке Зодиака.}

Вы уже знаете, что каждый знак Зодиака занимает \gradus{30} пространства в Зодиаке, а также то, что знаки делятся на мужские и женские. Все нечетные знаки --- мужские, четные --- женские.

Если планета расположена в части знака от \gradus{12} до \gradus{18}, то это местонахождение способствует максимальному проявлению силы такой планеты. Расположение планеты в первых \gradus{6} четных женских знаков или последних \gradus{6} нечетных мужских знаков значительно ослабит ее. В остальном пространстве знака планеты чувствуют себя достаточно сильными для проявления своих идей.

Если планета находится на границе между двумя знаками, то она не в состоянии полностью проявить себя, поэтому рассматривается как ослабленная.

Если планета находится не далее чем в \gradus{5} от Солнца, то она считается сгорающей, и сила ее значительно убывает. Меркурий меньше других подвержен сгоранию, так как нахождение вблизи Солнца его естественное состояние. Планеты Раху и Кету не подвержены сгоранию, поскольку не обладают физическими телами.

В ретроградном движении планета проявляет больше силы чем в прямом.

Определение силы планет --- это важнейшая тема в Индийской предсказательной астрологии, и мы будем продолжать развивать ее в следующих главах. Конечно, при первых попытках анализа гороскопа читатель столкнется с некоторыми трудностями в определении силы планет, так как здесь нет четкого выражения относительных сил в процентах, а самих признаков силы и слабости планет достаточно много, но в дальнейшем у вас появится опыт. Практикующий астролог чувствует силу планет даже по внешнему виду человека, с которым беседует. Поэтому мы еще раз подчеркиваем, что очень важно знать напамять максимальное количество информации.

\subsection{Двадцать семь лунных созвездий(накшатр) и их идеи}

Древние риши, наблюдая за движением планет, сделали великое открытие --- определили двадцать семь равных промежутков пространства в Зодиакальном поясе --- накшатр (см. табл. \ref{tbl:nakshatras}). Каждое лунное созвездие равно \coord{13}{20}{} Зодиакального пояса. Риши обнаружили, что планеты в разных частях одного и того же знака, оказывают различное воздействие. К примеру, если Солнце проходит по знаку Овен, оно дает эффекты в зависимости от того, в каком лунном созвездии находится. Три созвездия (третье неполное) в знаке Овен и дополнительные эффекты Солнца в гороскопе должны быть связаны с тем созвездием, в котором оно находилось в момент рождения человека. Солнце в созвездии Ашвини придаст одну окраску событиям и характеру человека, в Бхарани --- другую и в Криттике --- третью.

Каждое лунное созвездие представляет определенные силы природы, которые выражаются в идеях.

\begin{myenum}
	\item \textbf{Ашвини} --- накшатра транспорта. 
		\begin{mydescr}
			\item[Протяженность] --- \signum{0}{}{\aries} -- \signum{13}{20}{\aries}
			\item[Идеи] --- владелец лошадей, наездник, кавалерист, различные средства транспорта, приезждать куда-то, достигать чего-то, посещать, получать, добиваться, совершать благородные деяния, предоставлять помощь, приносить сокровища человеку, врачеватель, обоняние. вдох-выдох, носовые звуки, нечеткое произношение, избегать горестей, уйти от несчастья.
		\end{mydescr}
	\item \textbf{Бхарани} --- накшатра сдерживающего начала.
		\begin{mydescr}
			\item[Протяженность] --- \signum{13}{20}{\aries} -- \signum{26}{40}{\aries}
			\item[Идеи] --- Бхарани заключается в себя, а потом освобождает, действие сдерживания, подавление, дисциплина, самоконтроль, моральный долг, наказывать, подчинять, управлять, злой умысел, покорять, быть преданным и непоколебимым, претерпевать и страдать, нести в чреве, наполнять желудок, питать, пища, иждивенец, чувство тяжести, нанимать кого-то, наемник, война. битва, соревнование, кричать, повышать голос, достижение, завоевание. приз, большое количество, масса, водитель, близнецы.
		\end{mydescr}
	\item \textbf{Криттика} --- накшатра огня.
		\begin{mydescr}
			\item[Протяженность] --- \signum{26}{40}{\aries} -- \signum{10}{}{\taurus}
			\item[Идеи] --- война, битва, командующий, защитник, слава, знаменитый, великие деяния, огонь, аппетит, приготовление пищи, человеческая кожа, кожаные изделия, бумага, документ, белые пятна на коже. яркий, награжденный, богатство, золото, средства передвижения, темная сторона Луны, обильный, большой, усыновленный ребенок.
		\end{mydescr}
	\item \textbf{Рохини} --- накшатра восхождения.
		\begin{mydescr}
			\item[Протяженность] --- \signum{10}{}{\taurus} -- \signum{23}{20}{\taurus}
			\item[Идеи] --- восходить, карабкаться, подниматься, заниматься альпинизмом, высота, продвижение по службе, рост, развитие, рождение, производство, распространение, потомство, сажать, сеять, выращивать, воспаление, заболевание горла, кровь, красный цвет, духи, аромат, домашний скот.
		\end{mydescr}
	\item \textbf{Мригашира} --- накшатра поиска.
		\begin{mydescr}
			\item[Протяженность] --- \signum{23}{20}{\taurus} -- \signum{6}{40}{\gemini}
			\item[Идеи] --- целеустремленно искать, исследовать, находить потерянное, стремиться, достигать, ходатайствовать, очищать, предлагать девушке выйти замуж, выслеживать, дорога, тропинка, путешествие, указывать путь, проводник, лидер, вождь, охотиться, анализировать, изучать, научные исследования.
		\end{mydescr}
	\item \textbf{Ардра} --- накшатра угнетения и притеснения.
		\begin{mydescr}
			\item[Протяженность] --- \signum{6}{40}{\gemini} -- \signum{20}{}{\gemini}
			\item[Идеи] --- угнетать, подавлять, мучительный, разрушать, убивать, раздавать, слезы, печаль, боль, горечь, жадный, жестокий, охотник, влажный, мокрый, нежный, переполненный чувствами, свежий, жидкий.
		\end{mydescr}
	\item \textbf{Пунарвасу} --- накшатра обновления.
		\begin{mydescr}
			\item[Протяженность] --- \signum{20}{}{\gemini} -- \signum{3}{20}{\cancer}
			\item[Идеи] --- место поселения, жилище, родина, восстановление богатства, собственность, ремонт в доме, возвращение из путешествия, пребывание где-то, повторение, приобретать, свободный, свобода, безопасность, бесконечность, неразрывность.
		\end{mydescr}
	\item \textbf{Пушья} --- накшатра цветения
		\begin{mydescr}
			\item[Протяженность] --- \signum{3}{20}{\cancer} -- \signum{16}{40}{\cancer}
			\item[Идеи] --- цветение, цветок, насыщение, процветать, увеличение, приобретать, получать сполна, жирность, богатство, изобилие, полнота, счастливый, благоприятный, молитва, речь, красноречие, мудрость.
		\end{mydescr}
	\item \textbf{Ашлеша} --- цепляющаяся накшатра.
		\begin{mydescr}
			\item[Протяженность] --- \signum{16}{40}{\cancer} -- \signum{0}{}{\leo}
			\item[Идеи] --- соединение, связь, союз, сексуальное партнерство, интимный контакт, объятие, пожатие, сплетение, обвиваться, скручиваться, змея, яд, мучение, жжение, боль.
		\end{mydescr}
	\item \textbf{Магха} --- накшатра восхищения и славы.
		\begin{mydescr}
			\item[Протяженность] --- \signum{0}{}{\leo} -- \signum{13}{20}{\leo}
			\item[Идеи] --- выдающийся, великий, благородный, самый уважаемый, важный, высокий, старик, величество, высочество, могущественный, возбуждать, радовать кого-то, высокая честь, поднимать настроение, богатство, власть, щедрый, отец, родители, предки, благосостояние, совершенство.
		\end{mydescr}
	\item \textbf{Пурвапхалгуни} --- накшатра удачи.
		\begin{mydescr}
			\item[Протяженность] --- \signum{13}{20}{\leo} -- \signum{26}{40}{\leo}
			\item[Идеи] --- любовь, привязанность, страсть, любовные удовольствия, флирт, праздное развлечение, производство плодов, устранение зла, исправление, очищение, получение вознаграждения, усовершенствование, реформация, опыт, следователь, практиковать, выбирать, служить, чтить, уважать.
		\end{mydescr}
	\item \textbf{Уттарпхалгуни} --- накшатра покровителя.
		\begin{mydescr}
			\item[Протяженность] --- \signum{26}{40}{\leo} -- \signum{10}{}{\virgo}
			\item[Идеи] --- покровительство, доброта, благодетельство, Друзья детства, компаньоны. Люди, к которым обращаются за помощью. Друзья, к которым приходят излить душую Лица, оказывающие финансовую поддержку. Целители, от которых ждут облегчения болезней.
		\end{mydescr}
	\item \textbf{Хаста} --- накшатра, означающая ``сжатый кулак''
		\begin{mydescr}
			\item[Протяженность] --- \signum{10}{}{\virgo} -- \signum{23}{20}{\virgo}
			\item[Идеи] --- удерживание в руке, высмеивать, оживлять, шутка, острота, почерк, мастерство, ``золотые'' руки, превосходить кого-то в чем-то, приводить в движение, команидровать, осуществлять контроль, открывать и раскрывать, расширяться, обнажать, резать, убирать урожай, косить, жать, укладывать скирды, веселье, посвящать и освящать.
		\end{mydescr}
	\item \textbf{Читра} --- накшатра ``чудесная''
		\begin{mydescr}
			\item[Протяженность] --- \signum{23}{20}{\virgo} -- \signum{6}{40}{\libra}
			\item[Идеи] --- выдающийся, великолепный, яркоокрашенный, пестрый, пятнистый, многогранный, чудесный, замечательный, многообразие, изумление, зодчий, украшенный орнаментом, картина, чертеж, план, приковывающее внимание.
		\end{mydescr}
	\item \textbf{Свати} --- накшатра самостоятельных действий.
		\begin{mydescr}
			\item[Протяженность] --- \signum{6}{40}{\libra} -- \signum{20}{}{\libra}
			\item[Идеи] --- осознание своего ``я'', идти постоянно самому, достигающий что-то собственными силами, самоподдержва, самостоятельно овладевающий каким-то навыком, провозглашать как свое собственное, материальные блага приходят и уходят сами по себе, владение и потеря богатства происходит неконтролируемым образом, воздух, ветер, шторм.
		\end{mydescr}
	\item \textbf{Висакха} --- накшатра цели.
		\begin{mydescr}
			\item[Протяженность] --- \signum{20}{}{\libra} -- \signum{3}{20}{\scorpio}
			\item[Идеи] --- идущий прямо к достижению цели, достигнутый результат, доказанный вывод, доказательство, доктрина, догма, истреблять ради достижения какой-то цели, повреждать, причинять боль, разрушать, примирять, располагать к себе, воспевать, обожать, конечная цель, просящий милостыню.
		\end{mydescr}
	\item \textbf{Анурадха} --- накшатра, призывающая к деятельности.
		\begin{mydescr}
			\item[Протяженность] --- \signum{3}{20}{\scorpio} -- \signum{16}{40}{\scorpio}
			\item[Идеи] --- друг, союзник, сотрудник, помощник. То, что учреждается и устанавливается. Нечто утверждающееся как сила и власть. Суждение, наблюдение, познание, призыв людей к активности, действовать, будучи объединенными дружбой и общей целью. Союз с кем-то ради общей цели.
		\end{mydescr}
	\item \textbf{Джиешта} --- накшатра ``главная''
		\begin{mydescr}
			\item[Протяженность] --- \signum{16}{40}{\scorpio} -- \signum{0}{}{\sagittarius}
			\item[Идеи] --- великолепный, выдающийся, первый, величайший, тот, которого прославляют, верховная власть, могущественный правитель, провозглашать что-то, старший брат, старший по должности, глава семьи, жена, любимая, возлюбленная.
		\end{mydescr}
	\item \textbf{Мула} --- накшатра ``корневая''
		\begin{mydescr}
			\item[Протяженность] --- \signum{0}{}{\sagittarius} -- \signum{13}{20}{\sagittarius}
			\item[Идеи] --- иметь корни, самая низкая часть чего-нибудь, корень растения, начало, причина, источник знаний, главный город илил столица, оригинальный текст, первоначальный, исходный, беспрецендентный, связывать, привязывать, фиксировать, удерживать, брать в плен, связь, облигация, сдерживать, подавлять, угнетать, неагрессивный, наносящий обиду.
		\end{mydescr}
	\item \textbf{Пурвашадха} --- накшатра ``непобедимая''
		\begin{mydescr}
			\item[Протяженность] --- \signum{13}{20}{\sagittarius} -- \signum{26}{40}{\sagittarius}
			\item[Идеи] --- победоносный, побеждать, превосходить, завоевание и поражение, насилие, воспротивиться чему-то, посстать, сопротивляться, переносить страдания и муки, с пониманием относиться к любому человека, терпеливо ждать подходящего времени, всепрощение, снисхождение, завершение, прийти к концу, поверхность, вода.
		\end{mydescr}
	\item \textbf{Уттарашадха} --- накшатра ``всеобщая''
		\begin{mydescr}
			\item[Протяженность] --- \signum{26}{40}{\sagittarius} -- \signum{10}{}{\capricornus}
			\item[Идеи] --- входить, поселяться, проникать, быть поглощенным, входить в соединение, входить в дом, появляться на сцене, отдыхать, возникновение мысли, принадлежать чему-то, существовать ради кого-то принимать на себя, начинать что-то помнить о каком-то деле, заставлять войти, быть причиной вхождения, универсальность, всеобщность.
		\end{mydescr}
	\item \textbf{Шравана} --- накшатра изучения.
		\begin{mydescr}
			\item[Протяженность] --- \signum{10}{}{\capricornus} -- \signum{23}{20}{\capricornus}
			\item[Идеи] --- тот, кто учился, научный работник, знание, изучение, быть внимательным и послушным, интеллектуальные способности, быть известным и почитаемым, быть услышанным, рассказывать, общаться, священные знания, откровения, таинства, слухи, сообщения новосте, слова и языки, словарный запас, заботиться о ком-то, ученик, последователь, учитель, брать с кого-то пример, ливень, выделения.
		\end{mydescr}
	\item \textbf{Дханишта} --- накшатра симфонии.
		\begin{mydescr}
			\item[Протяженность] --- \signum{23}{20}{\capricornus} -- \signum{6}{40}{\aquarius}
			\item[Идеи] --- музыка, пение, нота, звучание, богатство, драгоценные камин, драгоценности, всё что высоко ценится, влага, потеть, задняя или боковая часть чего-то.
		\end{mydescr}
	\item \textbf{Сатабиша} --- накшатра прикрытия.
		\begin{mydescr}
			\item[Протяженность] --- \signum{6}{40}{\aquarius} -- \signum{20}{}{\aquarius}
			\item[Идеи] --- скрывать, прятать, быть захваченным, удерживать в плену, иметь препятствия, океан, море, озеро, река, пруд, дожди, резервуары с водой, защитник от зла, вооружение, вся верхняя одежда, целитель, врач, лечение, нахождение лечебного средства, неизлечимые заболевания, паралич, водянка, западня.
		\end{mydescr}
	\item \textbf{Пурвабхадрапада} --- накшатра ``пара бешено несущихся лошадей''.
		\begin{mydescr}
			\item[Протяженность] --- \signum{20}{}{\aquarius} -- \signum{3}{20}{\pisces}
			\item[Идеи] --- нестись во всю прыть, гореть, сжигать, пылкий, страстный, порывистый, импульсивный, стремительный, жгучая боль, наказывать, подвергать телесному наказанию, мучать кого-то, угнетать, подавлять, ущемлять, печаль, горе, обида, падать, погибать, уходить.
		\end{mydescr}
	\item \textbf{Уттарабхадрапада} --- накшатра подобна Пурвабхадрападе. Если Пурвабхадрапада вызывает гнев, то Уттарабхадрапада дает силы его контролировать.
		\begin{mydescr}
			\item[Протяженность] --- \signum{3}{20}{\pisces} -- \signum{16}{40}{\pisces}
			\item[Идеи] --- уезжать куда-то, оставлять все дома, личность, знания, мудрость, отерчение. остальные идеи такие же, как у Пурвабхадрапады.
		\end{mydescr}
	\item \textbf{Ревати} --- накшатра указывает на того, кто содержит стадо овец.
		\begin{mydescr}
			\item[Протяженность] --- \signum{16}{40}{\pisces} -- \signum{0}{}{\apies}
			\item[Идеи] --- питание, кормление, выращивать, богатый, процветающий, роскошный, обильный.
		\end{mydescr}
\end{myenum}

К толкованию планет в лунных созвездиях карты рождения надо подходить очень осторожно, так как не все астрологи, практикующие Индийскую предсказательную астрологию, используют идеи накшатр. Для определения будущих событий вполне достаточно сделать анализ и синтез идей планет в знаках и домах гороскопа. Но если вы хотите знать детали этих событий, вы должны аккуратно подойти к использованию идей лунных созвездий. Индийские астрологи были подлинными астерами этого высокого искусства.


\section{Двенадцать домов гороскопа и их идеи}

Существуют две модели Солнечной системы: гелеоцентрическая и геоцентрическая. Гелеоцентрическая модель Солнечной системы --- это та модель, которая реально существует в космическом пространстве: вокруг Солнца движутся планеты. Все астрономические открытия делались на основе гелеоцентрической модели мироздания.

Астрология рассматривает движения планет в основе геоцентрической модели Солнечной системы, то есть в центр Солнечной системы условно ставится Земля и при этом движение планет относительно земного наблюдателя происходит вокруг Земли против часовой стрелки.

Древние риши определили, что весь спектр идей, относящихся к человеку и его деятельности, представлен девятью планетами, поэтому в Индийской предсказательной астрологии используются знания только об этих планетах.

Планеты бывают благоприятными и неблагоприятными по своей природе. Юпитер и Венера --- благоприятные, Сатурн, Марс, Раху, Кету и Солнце --- неблагоприятные. Меньше всего зла исходит от Солнца. Луна и Меркурий считаются нейтральными по своей природе и могут быть благоприятными и неблагоприятными в зависимости от их расположения на карте рождения. Луна растущая благоприятна, убывающая --- наоборот, неблагоприятна. Меркури, находящийся в знаке Зодиака один или в связи с благоприятной планетой, будет благоприятным. Если он находится в одном знаке Зодиака с какой-либо неблагоприятной планетой, то тоже становится неблагоприятным.

Планеты разделены на мужские, женские и нейтральные.

\begin{table}[tph!]
	% Расширить по вертикали
	\renewcommand{\arraystretch}{1}

	% Заполним данными
	\begin{tabular}{l|l|l}
		Мужские планеты & Солнце, Марс, Юпитер & активность, мужское начало \\
		Женские планеты & Луна, Венера         & мягкость, женское начало \\
		Нейтральные планеты & Меркурий, Сатурн & проявляют оба начала в соответствии \\
		                     & Раху и Кету      & с полом знака Зодиака \\
	\end{tabular}
\end{table}


Планеты кроме Раху и Кету соответствуют семи дням недели:

\begin{table}[tph!]
	% Расширить по вертикали
	\renewcommand{\arraystretch}{1}

	% Заполним данными
	\begin{tabular}{ll}
		Воскресенье & Солнце \\
		Понедельник & Луна \\
		Вторник     & Марс \\
		Среда       & Меркурий \\
		Четверг     & Юпитер \\
		Пятница     & Венера \\
		Суббота     & Сатурн \\
	\end{tabular}
\end{table}

В Индийской предсказательной астрологии каждая планета несет определенные идеи в жизнь человека. Знание их необходимо для правильного прочтения карты рождения.

\begin{myenum}[topsep=0]
	\item \textbf{Солнце:}
		\begin{mydescr}
			\item[Физиология] --- сердце, мозг, голова, кости, голосовые связки, правый глаз.
			\item[Направление] --- восток.
			\item[Идея] --- индивидуальная душа (эго), отец, представители власти, способность к самоосознанию, статус, уверенность, благородство, слава, сознание, правда, истина, храбрость, решительность, энергия, политика, химия, медицина, врачи, хирурги, целители, храмы и места поклонений, места жертвоприношений, залы для коронования, осветительные приборы, высокий пост в правительстве.
		\end{mydescr}
	\item \textbf{Луна:}
		\begin{mydescr}
			\item[Физиология] --- молочные железы, матка, кровообращение, левый глаз.
			\item[Направление] --- северо-запад.
			\item[Идея] --- мать, женщина, известность, имя, ум, вода, океан, море, реки, путешествие и путешественники, лекарственные растения, сочные плоды, женственность, фантазии, перемены, молоко, популярные курорты, фармацевты и аптекари, зеркало, рыболов, грациозность, чувства.
		\end{mydescr}
	\item \textbf{Марс:}
		\begin{mydescr}
			\item[Физиология] --- нос, мышечная ткань, сухожилия, половые органы.
			\item[Направление] --- юг.
			\item[Идея] --- братья и сестры, решительность, твердость, действия, благосостояние, недвижимость, энергия, сила, земельный участок, пожар, военные, военная операция, режущие инструменты, инженеры, математика, электроника, калькулятор, хирурги, катастрофы, горы, леса, тяжба, спор, спорт, землетрясение, раны и травмы, жар, сапфиры, кораллы, сокровища, химическая лаборатория, война, битва, соревнование, строительство, сила духа, холодное и огнестрельное оружие, ожоги, должность в армии и полиции, конструктор, хирургическая операция, разрыв, разрушение, несчастный случай.
		\end{mydescr}
	\item \textbf{Меркурий:}
		\begin{mydescr}
			\item[Физиология] --- легкие, кишечник, брюшная полость, язык, кисти рук, нервные центры.
			\item[Направление] --- север.
			\item[Идея] --- интеллект, интеллектуальная деятельность, речь, литературные способности, средства информации и связи, рационализм, секретарская работа, бухгалтер, ученый, учебные заведения, головная и желудочная боль, рекламные агенства, издательство и издатели, исследование, новости, коммерческая деятельность, друзья.
		\end{mydescr}
	\item \textbf{Юпитер:}
		\begin{mydescr}
			\item[Физиология] --- печень, нижняя часть брюшной полости, органы слуха, таз.
			\item[Направление] --- северо-восток.
			\item[Идея] --- поведение, красноречие, мудрость, философия, духовный рост, религия, духовный наставник, учитель, ведические знания, набожность, религиозные учреждения, благотворительность, профессор, склонность к полноте, синтез, аргументация, получивший высокую оценку, дети, банки и банкиры, легальная деятельность, министры, юристы, справедливость, почтенные и уважаемые люди, советники, священники, беременные женщины, муж, поддерживающий жизнь.
		\end{mydescr}
	\item \textbf{Венера:}
		\begin{mydescr}
			\item[Физиология] --- почки, яичники, выделения, половая система.
			\item[Направление] --- юго-восток
			\item[Идея] --- желания, привязанность, жадность, ревность, праздность, красота, живописные места, текстильная продукция, люди искусства, товары ``люкс'', спиртные напитки, окружающая среда, спальня, комфорт, сексуальные удовольствия, венерические болезни, парфюмерия, транспортные средства, предметы роскоши, цветы, ботаника, творческие способности, муж, жена, любовник и любовница, украшения, декорации, театры, музеи, кино, сладости, ювелирные украшения.
		\end{mydescr}
	\item \textbf{Сатурн:}
		\begin{mydescr}
			\item[Физиология] --- ноги, колени, костный мозг, мочевой пузырь, дыхательная система, принцип передвижения.
			\item[Направление] --- запад
			\item[Идея] --- время, вызывающеее изменения, смерть, лишения, результат длительных действий, аскетизм, горе, печаль, подлость, обман, воры, мошенники, слуги, работа в подчинении кого-то, рабочий, самоотречение, пожилые люди, уход в отставку, места захоронения, судебный исполнитель, нищий, настенные или настольные часы, отсрочка, заблуждение, ложный вывод, уголовный розыск, черная магия, тайные знания, практика йоги, изменение обстоятельств, разорение и крах, предательство, сила влияния на других, нефть, железо, люди низкого происхождения, должности, получаемые за услуги, консерватизм, сфера услуг, препятствие.
		\end{mydescr}
	\item \textbf{Раху:}
		\begin{mydescr}
			\item[Физиология] --- скулы, кожа, выделительная система, глотание, пищеварительный тракт, прямая кишка.
			\item[Направление] --- юго-запад
			\item[Идея] --- страсть, толпа, мятеж, восстание, ядовитые рептилии, бедствие, наводнение, люди низкой культуры, воры и мошенники, гнев, гордыня, тупость, глупость, мистические знания, таинства, тайная доктрина, грубость, путешествия, муравейник, злое предсказание, эпидемии, насилие, коррупция, незаконные действия, охотники, тяжба, инфекционные заболевания, припадок, удушье, способность, оказывать влияние на других, лишения, крупное воровство.

Раху действует подобно Сатурну и провляет энергии тех планет, которые оказывают на нее влияние.
		\end{mydescr}
	\item \textbf{Кету:}
		\begin{mydescr}
			\item[Физиология] --- позвоночник, спинно-мозговой канал, нервная система, половая система.
			\item[Направление] --- направления не имеет/
			\item[Идея] --- препятствия и помехи, несчастный случай, духовные силы, психическое состояние, астрология, банкротство, клевета, оккультизм, падение, болезнь и недомогание, огонь, математика и мат. способности, эпидемии, страх, нервозность, испуг, яд, больничная палата, мелкое воровство, неожиданность, рана.

Кету действует подобно Марсу и проявляет энергии тех планет, которые оказывают на нее влияние.
		\end{mydescr}
\end{myenum}

Главные идеи девяти планет очень важно знать напамять при анализе карты рождения. А также следует не забывать, что сильные и удачно расположенные планеты в карте рождения проявят свои хорошие качества, а слабые и неудачно расположенные планеты --- дадут плохой результат.

\subsection{Определение силы планет}

В момент рождения человека планеты находятся в различных знаках Зодиака и в зависимости от этого проявляют свою силу или слабость.o

Существует множество способов для определения силы планет с огромным количеством вычислений и составлением дополнительно шестнадцати различных карт, аналогичных карте навамса, о которой будет рассказано далее. Однако на практике большинство астрологов пользуется для этого следующими правилами:
\begin{myenum}[itemsep=0,parsep=0]
	\item Планета в знаке экзальтации.\label{cost:1}
	\item Планета в мулатриконе.\label{cost:2}
	\item Планета в собственном знаке.\label{cost:3}
	\item Планета в знаке большого друга.\label{cost:4}
	\item Планета в дружественном знаке.\label{cost:5}
	\item Планета в нейтральном знаке.\label{cost:6}
	\item Планета во враждебном знаке.\label{cost:7}
	\item Планета в знаке большого врага.\label{cost:8}
	\item Планета в знаке ослабления.\label{cost:9}
\end{myenum}

Данные девять достоинств расположены в порядке убывания силы планет. Планета имеет наибольшую силу, если находится в знаке экзальтации, тогда она проявляет лучшие свои качества и приносит человеку счастье, удачу и доход. В знаке ослабления планеты теряют силы и сулят человеку горе, страдания и убытки. Промежуточные достоинства планет дают неоднозначный результат, в них проявляется как хорошее, так и плохое. Но чем выше достоинство планеты (мулатрикона, собственный знак, знак большого друга), тем сильнее в ней доброе начало, меньше плохих качеств. И наоборот, чем ниже достоинство планеты (враждебный знак или знак большого врага), тем больше будет проявляться зло и меньше хороших качеств.

Достоинство планет по пунктам \ref{cost:1}, \ref{cost:2}, \ref{cost:3} и \ref{cost:9} легко определить по таблице~\ref{tbl:cost}, где перечислены знаки, которые являются экзальтирующими, знаками мулатриконы, собственными и ослабляющими для каждой планеты.

Сначала надо посмотреть, в каком знаке Зодиака эта планета расположена в карте рождения, а затем по таблице определить, какое достоинство этот знак придает соответствующей планете. Например, если в карте рождения Юпитер находится в Рыбах, значит Юпитер в собственном знаке. Возле каждого знака экзальтации проставлено значение градуса (например, Овен \(10^\circ\)) --- это значит, что если планета в знаке находится именно в таком градусе, она наиболее сильная (то есть это градус наивысшей экзальтации планеты в знаке). Аналогично для знака ослабления --- указанное значение градуса определяет точку наибольшего ослабления соответствующей планеты в данном знаке.

\begin{table}[tph!]
	\caption{Определение достоинств планет}
	\label{tbl:cost}

	\centering

	% Расширить по вертикали
	\renewcommand{\arraystretch}{1}

	% Разширить вширь
	%\setlength{\tabcolsep}{.05\textwidth}

	% Заполним данными
	\begin{tabular}{|l|l|l|l|l|}
		\hline
		Планета & Знак экзальтации & Знак мулатрикона & Собственный знак & Знак ослабления \\
		\hline
		Солнце   & Овен \gradus{10}    & Лев \gradus{0}--\gradus{20}     & Лев & Весы \gradus{10}    \\
		Луна     & Телец \gradus{3}    & Телец \gradus{4}--\gradus{30}   & Рак & Скорпион \gradus{3} \\
		Марс     & Козерог \gradus{28} & Овен \gradus{1}--\gradus{12}    & Овен, Скорпион   & Рак \gradus{28}    \\
		Меркурий & Дева \gradus{15}    & Дева \gradus{16}--\gradus{20}   & Близнецы, Дева   & Рыбы \gradus{15}   \\
		Юпитер   & Рак \gradus{5}      & Стрелец \gradus{1}--\gradus{10} & Стрелец, Рыбы    & Козерог \gradus{5} \\
		Венера   & Рыбы \gradus{27}    & Весы \gradus{0}--\gradus{5}     & Телец, Весы      & Дева \gradus{27}   \\
		Сатурн   & Весы \gradus{20}    & Водолей \gradus{1}--\gradus{20} & Козерог, Водолей & Овен \gradus{20}   \\
		Раху     & Телец               & Близнецы & Водолей & Скорпион \\
		Кету     & Скорпион            & Стрелец  & Лев & Телец \\
		\hline
	\end{tabular}
\end{table}

Возле каждого знака мулатрикона также проставлены значения градусов (например, Лев \(0{^\circ}-20^\circ\)) --- это значит, что достоинство или сила мулатрикона действует в пределах указанных градусов, а в остальном пространстве знака планета будет иметь достоинство знака экзальтации или собственного знака в соответствии с таблицей\footnote{Некоторые индийские астрологические школы указывают позицию мулатрикона для Венеры от \gradus{0} Весов до \gradus{15} Весов.}.

Солнце и луна имеют по одному собственному знаку, а остальные пять планет по два. Индийские астрологи считают, что Раху и Кету не имеют собственных знаков, так как они не обладают физическими телами, но их расположение в знаках Водолея и Льва придает им силу, соответствующую достоинству собственного знака.

По терминологии, принятой в Индии, планета является хозяином собственного знака, то есть если Лев --- собственный знак для Солнца, то Солнце --- хозяин Льва и\,т.\,д.

Для того, чтобы определить достоинства планет по пунктам \ref{cost:4}, \ref{cost:5}, \ref{cost:6}, \ref{cost:7} и \ref{cost:8}, то есть в знаке большого друга, дружественном знаке, нейтральном знаке, враждебном знаке и знаке большого врага, необходимо ввести дополнительные термины, такие как \emph{постоянные} отношения и \emph{временные} отношения между планетами.

Каждая планета, кроме Раху и Кету, имеет дружественные, нейтральные или враждебные отношения с другими планетами. Их можно установить с помощью приведенной таблицы~\ref{tbl:relations}, отражающие постоянные отношения между планетами. Из этой таблицы видно, что для Солнца дружественными являются Луна, Марс и Юпитер, нейтральным --- Меркурий, а враждебными --- Сатурн и Венера. И так для каждой планеты, находящейся в левой колонке. Отношения между планетами устанавливаются через знаки, хозяевами которых они являются. Например, если Солнце находится в Раке, а хозяйкой Рака является Луна, которая дружественных отношениях с Солнцем, Значит Солнце находится в дружественном знаке. Таким образом определяют постоянные отношения для каждой планеты в гороскопе.

\begin{table}[tph!]
	\caption{Характер отношений между планетами}
	\label{tbl:relations}

	\centering

	% Расширить по вертикали
	\renewcommand{\arraystretch}{1}

	% Разширить вширь
	%\setlength{\tabcolsep}{.05\textwidth}

	% Заполним данными
	\begin{tabular}{|l|l|l|l|}
		\hline
		Планета & Дружественные & Нейтральные & Враждебные \\
		\hline
		Солнце   & Луна, Марс, Юпитер & Меркурий & Сатурн, Венера \\
		Луна     & Солнце, Меркурий & Марс, Юпитер, Венера, Сатурн & --- \\
		Марс     & Солнце, Луна, Юпитер & Венера, Сатурн & Меркурий \\
		Меркурий & Солнце, Венера & Марс, Юпитер, Сатурн & Луна \\
		Юпитер   & Солнце, Луна, Марс & Сатурн & Меркурий, Венера \\
		Венера   & Меркурий, Сатурн & Марс, Юпитер & Солнце, Луна \\
		Сатурн   & Меркурий, Венера & Юпитер & Солнце, Луна, Марс \\
		\hline
	\end{tabular}
\end{table}

Временные отношения между планетами определяются по их взаимному расположению в домах. Они бывают двух типов --- временная дружба и временная вражда. Если одна планета расположена относительно другой во 2, 3, 4, 10, 11 или 12-м доме, значит эти две планеты находятся во временных дружественных отношениях. Если две планеты находятся в одном доме или одна планета расположена относительно другой в 5, 6, 7, 8 или 9-м доме, значит эти две планеты находятся во временных враждебных отношениях. Надо обратить внимание на то, что при установлении временных отношений между планетами за первый дом принимается тот дом, где находится планета, достоинство которой вы определяете.

Установив постоянные и временные отношения между планетами в гороскопе, можно приступить к определению достоинств планет в соответствии с пунктами \ref{cost:4}, \ref{cost:5}, \ref{cost:6}, \ref{cost:7} и \ref{cost:8} из правила определения силы планет по девяти достоинствам:

\begin{myenum}[itemsep=0,parsep=0]
	\item Постоянные дружественные отношения + дружба временная = планета в знаке большого друга.
	\item Постоянные нейтральные отношения + дружба временная = планета в дружественном знаке.
	\item Постоянные враждебные отношения + дружба временная = планета в нейтральном знаке.
	\item Постоянные дружественные отношения + вражда временная = планета в нейтральном знаке.
	\item Постоянные нейтральные отношения + вражда временная = планета во враждебном знаке.
	\item Постоянные враждебные отношения + вражда временная = планета в знаке большого врага. 
\end{myenum}

Покажем как определяются достоинства планет в знаках Зодиака на нескольких примерах:

\begin{myenum}
	\item Солнце расположено в Тельце, а Венера в Овне. Солнце находится во враждебном знаке, так как хозяйка Тельца --- Венера, согласно таблице~\ref{tbl:relations}, враждебна к Солнцу. Если считать Телец, в котором находится Солнце, 1-м домом, то Венера относительно Солнца находится в Овне, то есть в 12-м доме, что говорит о временной дружбе этих планет. Теперь складываем постоянные враждебные отношения с временной дружбой и получаем, что Солнце находится в нейтральном знаке.
	\item Марс расположен во Льве, хозяином которого является Солнце. Солнце дружественно Марсу и находится в Деве, то есть во 2-м доме от Марса, что указывает на временные дружественные отношения. Сложив постоянные дружественные отношения со временной дружбой, определяем Марс как находящийся в знаке большого друга.
	\item Венера находится в Раке, хозяином которого является Луна. Луна для Венеры враждебна. Луна находится в Козероге, то есть в 7-ом доме от Венеры. Луна --- временный враг Венеры. Сложив постоянные враждебные отношения со временной враждой, определяем Венеру как находящуюся в знаке большого врага.
\end{myenum}


\subsubsection*{Определение силы планет в зависимости от их расположения в знаке Зодиака.}

Вы уже знаете, что каждый знак Зодиака занимает \gradus{30} пространства в Зодиаке, а также то, что знаки делятся на мужские и женские. Все нечетные знаки --- мужские, четные --- женские.

Если планета расположена в части знака от \gradus{12} до \gradus{18}, то это местонахождение способствует максимальному проявлению силы такой планеты. Расположение планеты в первых \gradus{6} четных женских знаков или последних \gradus{6} нечетных мужских знаков значительно ослабит ее. В остальном пространстве знака планеты чувствуют себя достаточно сильными для проявления своих идей.

Если планета находится на границе между двумя знаками, то она не в состоянии полностью проявить себя, поэтому рассматривается как ослабленная.

Если планета находится не далее чем в \gradus{5} от Солнца, то она считается сгорающей, и сила ее значительно убывает. Меркурий меньше других подвержен сгоранию, так как нахождение вблизи Солнца его естественное состояние. Планеты Раху и Кету не подвержены сгоранию, поскольку не обладают физическими телами.

В ретроградном движении планета проявляет больше силы чем в прямом.

Определение силы планет --- это важнейшая тема в Индийской предсказательной астрологии, и мы будем продолжать развивать ее в следующих главах. Конечно, при первых попытках анализа гороскопа читатель столкнется с некоторыми трудностями в определении силы планет, так как здесь нет четкого выражения относительных сил в процентах, а самих признаков силы и слабости планет достаточно много, но в дальнейшем у вас появится опыт. Практикующий астролог чувствует силу планет даже по внешнему виду человека, с которым беседует. Поэтому мы еще раз подчеркиваем, что очень важно знать напамять максимальное количество информации.

\subsection{Двадцать семь лунных созвездий(накшатр) и их идеи}

Древние риши, наблюдая за движением планет, сделали великое открытие --- определили двадцать семь равных промежутков пространства в Зодиакальном поясе --- накшатр (см. табл. \ref{tbl:nakshatras}). Каждое лунное созвездие равно \coord{13}{20}{} Зодиакального пояса. Риши обнаружили, что планеты в разных частях одного и того же знака, оказывают различное воздействие. К примеру, если Солнце проходит по знаку Овен, оно дает эффекты в зависимости от того, в каком лунном созвездии находится. Три созвездия (третье неполное) в знаке Овен и дополнительные эффекты Солнца в гороскопе должны быть связаны с тем созвездием, в котором оно находилось в момент рождения человека. Солнце в созвездии Ашвини придаст одну окраску событиям и характеру человека, в Бхарани --- другую и в Криттике --- третью.

Каждое лунное созвездие представляет определенные силы природы, которые выражаются в идеях.

\begin{myenum}
	\item \textbf{Ашвини} --- накшатра транспорта. 
		\begin{mydescr}
			\item[Протяженность] --- \signum{0}{}{\aries} -- \signum{13}{20}{\aries}
			\item[Идеи] --- владелец лошадей, наездник, кавалерист, различные средства транспорта, приезждать куда-то, достигать чего-то, посещать, получать, добиваться, совершать благородные деяния, предоставлять помощь, приносить сокровища человеку, врачеватель, обоняние. вдох-выдох, носовые звуки, нечеткое произношение, избегать горестей, уйти от несчастья.
		\end{mydescr}
	\item \textbf{Бхарани} --- накшатра сдерживающего начала.
		\begin{mydescr}
			\item[Протяженность] --- \signum{13}{20}{\aries} -- \signum{26}{40}{\aries}
			\item[Идеи] --- Бхарани заключается в себя, а потом освобождает, действие сдерживания, подавление, дисциплина, самоконтроль, моральный долг, наказывать, подчинять, управлять, злой умысел, покорять, быть преданным и непоколебимым, претерпевать и страдать, нести в чреве, наполнять желудок, питать, пища, иждивенец, чувство тяжести, нанимать кого-то, наемник, война. битва, соревнование, кричать, повышать голос, достижение, завоевание. приз, большое количество, масса, водитель, близнецы.
		\end{mydescr}
	\item \textbf{Криттика} --- накшатра огня.
		\begin{mydescr}
			\item[Протяженность] --- \signum{26}{40}{\aries} -- \signum{10}{}{\taurus}
			\item[Идеи] --- война, битва, командующий, защитник, слава, знаменитый, великие деяния, огонь, аппетит, приготовление пищи, человеческая кожа, кожаные изделия, бумага, документ, белые пятна на коже. яркий, награжденный, богатство, золото, средства передвижения, темная сторона Луны, обильный, большой, усыновленный ребенок.
		\end{mydescr}
	\item \textbf{Рохини} --- накшатра восхождения.
		\begin{mydescr}
			\item[Протяженность] --- \signum{10}{}{\taurus} -- \signum{23}{20}{\taurus}
			\item[Идеи] --- восходить, карабкаться, подниматься, заниматься альпинизмом, высота, продвижение по службе, рост, развитие, рождение, производство, распространение, потомство, сажать, сеять, выращивать, воспаление, заболевание горла, кровь, красный цвет, духи, аромат, домашний скот.
		\end{mydescr}
	\item \textbf{Мригашира} --- накшатра поиска.
		\begin{mydescr}
			\item[Протяженность] --- \signum{23}{20}{\taurus} -- \signum{6}{40}{\gemini}
			\item[Идеи] --- целеустремленно искать, исследовать, находить потерянное, стремиться, достигать, ходатайствовать, очищать, предлагать девушке выйти замуж, выслеживать, дорога, тропинка, путешествие, указывать путь, проводник, лидер, вождь, охотиться, анализировать, изучать, научные исследования.
		\end{mydescr}
	\item \textbf{Ардра} --- накшатра угнетения и притеснения.
		\begin{mydescr}
			\item[Протяженность] --- \signum{6}{40}{\gemini} -- \signum{20}{}{\gemini}
			\item[Идеи] --- угнетать, подавлять, мучительный, разрушать, убивать, раздавать, слезы, печаль, боль, горечь, жадный, жестокий, охотник, влажный, мокрый, нежный, переполненный чувствами, свежий, жидкий.
		\end{mydescr}
	\item \textbf{Пунарвасу} --- накшатра обновления.
		\begin{mydescr}
			\item[Протяженность] --- \signum{20}{}{\gemini} -- \signum{3}{20}{\cancer}
			\item[Идеи] --- место поселения, жилище, родина, восстановление богатства, собственность, ремонт в доме, возвращение из путешествия, пребывание где-то, повторение, приобретать, свободный, свобода, безопасность, бесконечность, неразрывность.
		\end{mydescr}
	\item \textbf{Пушья} --- накшатра цветения
		\begin{mydescr}
			\item[Протяженность] --- \signum{3}{20}{\cancer} -- \signum{16}{40}{\cancer}
			\item[Идеи] --- цветение, цветок, насыщение, процветать, увеличение, приобретать, получать сполна, жирность, богатство, изобилие, полнота, счастливый, благоприятный, молитва, речь, красноречие, мудрость.
		\end{mydescr}
	\item \textbf{Ашлеша} --- цепляющаяся накшатра.
		\begin{mydescr}
			\item[Протяженность] --- \signum{16}{40}{\cancer} -- \signum{0}{}{\leo}
			\item[Идеи] --- соединение, связь, союз, сексуальное партнерство, интимный контакт, объятие, пожатие, сплетение, обвиваться, скручиваться, змея, яд, мучение, жжение, боль.
		\end{mydescr}
	\item \textbf{Магха} --- накшатра восхищения и славы.
		\begin{mydescr}
			\item[Протяженность] --- \signum{0}{}{\leo} -- \signum{13}{20}{\leo}
			\item[Идеи] --- выдающийся, великий, благородный, самый уважаемый, важный, высокий, старик, величество, высочество, могущественный, возбуждать, радовать кого-то, высокая честь, поднимать настроение, богатство, власть, щедрый, отец, родители, предки, благосостояние, совершенство.
		\end{mydescr}
	\item \textbf{Пурвапхалгуни} --- накшатра удачи.
		\begin{mydescr}
			\item[Протяженность] --- \signum{13}{20}{\leo} -- \signum{26}{40}{\leo}
			\item[Идеи] --- любовь, привязанность, страсть, любовные удовольствия, флирт, праздное развлечение, производство плодов, устранение зла, исправление, очищение, получение вознаграждения, усовершенствование, реформация, опыт, следователь, практиковать, выбирать, служить, чтить, уважать.
		\end{mydescr}
	\item \textbf{Уттарпхалгуни} --- накшатра покровителя.
		\begin{mydescr}
			\item[Протяженность] --- \signum{26}{40}{\leo} -- \signum{10}{}{\virgo}
			\item[Идеи] --- покровительство, доброта, благодетельство, Друзья детства, компаньоны. Люди, к которым обращаются за помощью. Друзья, к которым приходят излить душую Лица, оказывающие финансовую поддержку. Целители, от которых ждут облегчения болезней.
		\end{mydescr}
	\item \textbf{Хаста} --- накшатра, означающая ``сжатый кулак''
		\begin{mydescr}
			\item[Протяженность] --- \signum{10}{}{\virgo} -- \signum{23}{20}{\virgo}
			\item[Идеи] --- удерживание в руке, высмеивать, оживлять, шутка, острота, почерк, мастерство, ``золотые'' руки, превосходить кого-то в чем-то, приводить в движение, команидровать, осуществлять контроль, открывать и раскрывать, расширяться, обнажать, резать, убирать урожай, косить, жать, укладывать скирды, веселье, посвящать и освящать.
		\end{mydescr}
	\item \textbf{Читра} --- накшатра ``чудесная''
		\begin{mydescr}
			\item[Протяженность] --- \signum{23}{20}{\virgo} -- \signum{6}{40}{\libra}
			\item[Идеи] --- выдающийся, великолепный, яркоокрашенный, пестрый, пятнистый, многогранный, чудесный, замечательный, многообразие, изумление, зодчий, украшенный орнаментом, картина, чертеж, план, приковывающее внимание.
		\end{mydescr}
	\item \textbf{Свати} --- накшатра самостоятельных действий.
		\begin{mydescr}
			\item[Протяженность] --- \signum{6}{40}{\libra} -- \signum{20}{}{\libra}
			\item[Идеи] --- осознание своего ``я'', идти постоянно самому, достигающий что-то собственными силами, самоподдержва, самостоятельно овладевающий каким-то навыком, провозглашать как свое собственное, материальные блага приходят и уходят сами по себе, владение и потеря богатства происходит неконтролируемым образом, воздух, ветер, шторм.
		\end{mydescr}
	\item \textbf{Висакха} --- накшатра цели.
		\begin{mydescr}
			\item[Протяженность] --- \signum{20}{}{\libra} -- \signum{3}{20}{\scorpio}
			\item[Идеи] --- идущий прямо к достижению цели, достигнутый результат, доказанный вывод, доказательство, доктрина, догма, истреблять ради достижения какой-то цели, повреждать, причинять боль, разрушать, примирять, располагать к себе, воспевать, обожать, конечная цель, просящий милостыню.
		\end{mydescr}
	\item \textbf{Анурадха} --- накшатра, призывающая к деятельности.
		\begin{mydescr}
			\item[Протяженность] --- \signum{3}{20}{\scorpio} -- \signum{16}{40}{\scorpio}
			\item[Идеи] --- друг, союзник, сотрудник, помощник. То, что учреждается и устанавливается. Нечто утверждающееся как сила и власть. Суждение, наблюдение, познание, призыв людей к активности, действовать, будучи объединенными дружбой и общей целью. Союз с кем-то ради общей цели.
		\end{mydescr}
	\item \textbf{Джиешта} --- накшатра ``главная''
		\begin{mydescr}
			\item[Протяженность] --- \signum{16}{40}{\scorpio} -- \signum{0}{}{\sagittarius}
			\item[Идеи] --- великолепный, выдающийся, первый, величайший, тот, которого прославляют, верховная власть, могущественный правитель, провозглашать что-то, старший брат, старший по должности, глава семьи, жена, любимая, возлюбленная.
		\end{mydescr}
	\item \textbf{Мула} --- накшатра ``корневая''
		\begin{mydescr}
			\item[Протяженность] --- \signum{0}{}{\sagittarius} -- \signum{13}{20}{\sagittarius}
			\item[Идеи] --- иметь корни, самая низкая часть чего-нибудь, корень растения, начало, причина, источник знаний, главный город илил столица, оригинальный текст, первоначальный, исходный, беспрецендентный, связывать, привязывать, фиксировать, удерживать, брать в плен, связь, облигация, сдерживать, подавлять, угнетать, неагрессивный, наносящий обиду.
		\end{mydescr}
	\item \textbf{Пурвашадха} --- накшатра ``непобедимая''
		\begin{mydescr}
			\item[Протяженность] --- \signum{13}{20}{\sagittarius} -- \signum{26}{40}{\sagittarius}
			\item[Идеи] --- победоносный, побеждать, превосходить, завоевание и поражение, насилие, воспротивиться чему-то, посстать, сопротивляться, переносить страдания и муки, с пониманием относиться к любому человека, терпеливо ждать подходящего времени, всепрощение, снисхождение, завершение, прийти к концу, поверхность, вода.
		\end{mydescr}
	\item \textbf{Уттарашадха} --- накшатра ``всеобщая''
		\begin{mydescr}
			\item[Протяженность] --- \signum{26}{40}{\sagittarius} -- \signum{10}{}{\capricornus}
			\item[Идеи] --- входить, поселяться, проникать, быть поглощенным, входить в соединение, входить в дом, появляться на сцене, отдыхать, возникновение мысли, принадлежать чему-то, существовать ради кого-то принимать на себя, начинать что-то помнить о каком-то деле, заставлять войти, быть причиной вхождения, универсальность, всеобщность.
		\end{mydescr}
	\item \textbf{Шравана} --- накшатра изучения.
		\begin{mydescr}
			\item[Протяженность] --- \signum{10}{}{\capricornus} -- \signum{23}{20}{\capricornus}
			\item[Идеи] --- тот, кто учился, научный работник, знание, изучение, быть внимательным и послушным, интеллектуальные способности, быть известным и почитаемым, быть услышанным, рассказывать, общаться, священные знания, откровения, таинства, слухи, сообщения новосте, слова и языки, словарный запас, заботиться о ком-то, ученик, последователь, учитель, брать с кого-то пример, ливень, выделения.
		\end{mydescr}
	\item \textbf{Дханишта} --- накшатра симфонии.
		\begin{mydescr}
			\item[Протяженность] --- \signum{23}{20}{\capricornus} -- \signum{6}{40}{\aquarius}
			\item[Идеи] --- музыка, пение, нота, звучание, богатство, драгоценные камин, драгоценности, всё что высоко ценится, влага, потеть, задняя или боковая часть чего-то.
		\end{mydescr}
	\item \textbf{Сатабиша} --- накшатра прикрытия.
		\begin{mydescr}
			\item[Протяженность] --- \signum{6}{40}{\aquarius} -- \signum{20}{}{\aquarius}
			\item[Идеи] --- скрывать, прятать, быть захваченным, удерживать в плену, иметь препятствия, океан, море, озеро, река, пруд, дожди, резервуары с водой, защитник от зла, вооружение, вся верхняя одежда, целитель, врач, лечение, нахождение лечебного средства, неизлечимые заболевания, паралич, водянка, западня.
		\end{mydescr}
	\item \textbf{Пурвабхадрапада} --- накшатра ``пара бешено несущихся лошадей''.
		\begin{mydescr}
			\item[Протяженность] --- \signum{20}{}{\aquarius} -- \signum{3}{20}{\pisces}
			\item[Идеи] --- нестись во всю прыть, гореть, сжигать, пылкий, страстный, порывистый, импульсивный, стремительный, жгучая боль, наказывать, подвергать телесному наказанию, мучать кого-то, угнетать, подавлять, ущемлять, печаль, горе, обида, падать, погибать, уходить.
		\end{mydescr}
	\item \textbf{Уттарабхадрапада} --- накшатра подобна Пурвабхадрападе. Если Пурвабхадрапада вызывает гнев, то Уттарабхадрапада дает силы его контролировать.
		\begin{mydescr}
			\item[Протяженность] --- \signum{3}{20}{\pisces} -- \signum{16}{40}{\pisces}
			\item[Идеи] --- уезжать куда-то, оставлять все дома, личность, знания, мудрость, отерчение. остальные идеи такие же, как у Пурвабхадрапады.
		\end{mydescr}
	\item \textbf{Ревати} --- накшатра указывает на того, кто содержит стадо овец.
		\begin{mydescr}
			\item[Протяженность] --- \signum{16}{40}{\pisces} -- \signum{0}{}{\apies}
			\item[Идеи] --- питание, кормление, выращивать, богатый, процветающий, роскошный, обильный.
		\end{mydescr}
\end{myenum}

К толкованию планет в лунных созвездиях карты рождения надо подходить очень осторожно, так как не все астрологи, практикующие Индийскую предсказательную астрологию, используют идеи накшатр. Для определения будущих событий вполне достаточно сделать анализ и синтез идей планет в знаках и домах гороскопа. Но если вы хотите знать детали этих событий, вы должны аккуратно подойти к использованию идей лунных созвездий. Индийские астрологи были подлинными астерами этого высокого искусства.


\section{Ключ к синтезу идей планет и знаков Зодиака}

Планеты, находящиеся в знаках Зодиака, в карте рождения могут быть сильными, средними или слабыми. Когда планета в знаке экзальтации --- она сильна, в нейтральном знаке --- средняя по силе, в ослабленном знаке --- слабая. Если планета пребывает в собственном знаке, то она также сильна и немного будет уступать по силе планете в позиции мулатрикона. Когда планета расположена в дружеском знаке, она сильна относительно планеты в нейтральном знаке и ослаблена относительно планеты в знаке большой дружбы. Планета, находящаяся во враждебном знаке, сильнее планеты в знаке большой вражды, но слабее планеты в нейтральном знаке. Чем сильнее планета по знаку, тем благоприятнее ее влияние на судьбу человека. Чем слабее планета по знаку, тем больше неблагоприятных моментов принемет она в жизнь человека.

Знак Зодиака, в отором находится планета, имеет меньшее значение, чем дом, в котором она расположена. Это знак Зодиака, придаст окраску событиям, а дом --- отразит те стороны жизни, в которых планета будет особенно проявлять свое действие. Мы даем, как пример, краткое описание влияние Солнца, находящегося в разных Знаках Зодиака, на характер человека.

\begin{myenum}
	\item \emph{Солнце в Овне} будет сильным, так как Овен является знаком экзальтации для Солнца. Солнце (энергия, душа) указывает на энергичного человека, поскольку хозяин Овна --- Марс.
	\item \emph{Солнце в Тельце} (миролюбие, упрямство, добродетель) высветит в человеке любовь к произведениям искусства, чувственным удовольствиям, а также укажет на упрямый характер. Это легко становится понятным из того, что хозяйкой Тельца выступает Венера (красота, любовь, желания, искусство), которая окрашивает Солнце, находящееся в ее знаке.
	\item \emph{Солнце в Близнецах} (разнообразие, двойственность, интеллектуальные занятия) укажет на человека, способного схыватывать на лету мысли одних людей и передавать их другим. Время от времени характер и поведение этого человека будет изменяться: он становится то решительным, то нерешительным. Так как хозяином Близнецов является Меркурий, то Солнце в этом знаке даст энергию для интеллектуальной деятельности.
	\item \emph{Солнце в Раке} (эмоции, чувства), хозяин которого Луна, наделяет человека эмоциональным характером и тонкими чувствами.
	\item \emph{Солнце во Льве} (гордость, сила, руководство, благородство) является сильным (поскольку оно управляет этим знаком) и дарит человеку чувство превосходства над другими.
	\item \emph{Солнце в Деве} (чистота, порядок, аналитичность), которой управляет Меркурий, указывает н ачеловека опрятного, тонкого, смышленого и любящего во всем порядок.
	\item \emph{Солнце в Весах} (равновесие, справедливость, гармония) ослаблено, поэтому характер человека будет мягким, он будет во всем искать равновесия и справедливости. Так как хозяином этого знака является Венера, душа человека будет пребывать в гармонии с окружающим миром.
	\item \emph{Солнце в Скорпионе} (скрытность, психическая энергия, мстительность), хозяин которого Марс, указывает на характер критика. Такой человек обладает хорошей интуицией, энергичен и предпочитает действовать тайно.
	\item \emph{Солнце в Стрельце} (высокая цель, лидерство) дает много жизненных сил и благих устремлений. Хозяин Стрельца --- Юпитер делает человека высокодуховной личностью.
	\item \emph{Солнце в Козероге} (настойчивость, упорство), хозяином которого является Сатурн, наделяет человека настойчивым характером, а также честолюбивыми помыслами и трудолюбием.
	\item \emph{Солнце в Водолее} (реформисткие цели, прогрессивные мысли), хозяином которого выступает Сатурн, указывает на подчиненный характер человека, на его новые идеи, стремление помогать другим.
	\item \emph{Солнце в Рыбах} (восприимчивость, психическая энергия, духовность), которыми управляет Юпитер, обусловливает беспокойный характер, развитую интуицию и духовные силы.
\end{myenum}

Таким же путем, исходя из положения других планет в знаках Зодиака, вы сможете выявить черты характера человека, предопределенные каждой планетой. Для этого необходимо, научиться объединять идеи планет и знаков Зодиака.

Планета, находящаяся в знаке Зодиака, может иметь связь с другими планетами в этом же знаке или их аспект. В данном случае планете приписываются дополнительные качества через связь с другими планетами.

Солнце во Льве находится в хорошей позиции, но если в этом знаке одновременно присутствует Раху(страсть, грубость), то в характере человека проявятся гордыня, безнравственность, склонность к плохим привычкам, неуважение к другим людям. Раху является естественно неблагоприятной планетой и поэтому ``повреждает'' Солнце.

Луна в Тельце сильна, а если на нее влияет благотворная планета Юпитер, то она становится еще сильнее.

Планеты, находящиеся под влиянием Меркурия, Луны, Венеры и особенно Юпитера, будут увеличивать свое благотворное воздействие на жизнь человека и его характер.

Планеты, под влиянием Сатурна, Марса, Раху, Кету будут ухудшать условия жизни, характер и поведение человека.

Планеты, аспектируемые Солнцем, большого вреда не нанесут, так как Солнце по своей природе много зла не приносит.

Вы должны научиться объединять идеи знаков Зодиака и планет, оказывающих влияние, но только тех идей, которые имеют смысл в результате синтеза.

\section{Ключ к синтезу идей планет и домов}

Эта глава является очень важной для понимания гороскопа в целом и особенно для предсказания будущих событий. Планеты, находящиеся в домах карты рождения, указывают на сильные и слабые стороны различных сфер жизни. Любая планета в доме возбуждает идеи этого дома, а также приносит идеи, присущие ее природе. С помощью девяти планет, находящихся в определенных домах карты рождения, вы сможете понять судьбу человека  и достаточно точно предсказать ее.

Если Юпитер, Венера, Луна и Меркурий будут находиться в домах квадранта и/или тригона, то жизнь человека будет счастливой и бесхлопотной.

Если Сатурн, Марс, Раху, Кету и Солнце будут находиться в 8 и/или 12-м домах, то человек испытывает в своей жизни большие трудности, преодолеет препятствия. Самой хорошей позицией для этих планет является 6-й дом.

Когда благоприятные и нейтральные планеты находятся в 6,\,8,\,и\,12-м домах, они улучшают эти дома, но сами планет ослабляются. Если естественно неблагоприятные планеты сильны по знаку и находятся в домах квадранта и/или тригона, они принесут удачу человеку через его действия, если слабы по знаку --- неудачу и препятствия.

Присутствие планет во 2-м и 11-м домах указывает на материальное благополучие и заработки человека.

Когда планеты находятся в 3-м доме, они наделяют человека большой энергией, силой воли, способностями к писательскому труду или журналистской работе в средствах массовой информации.

Окончательные выводы астролог должен делать после тщательного изучения гороскопа, рассмотрев его со всех сторон. Когда естественно благоприятная планета сильна по знаку и находится в квадранте или тригоне, она обязательно создаст хорошие условия этому дому, и человек испытает счастье в тех сферах жизни, которыми управляет этот дом.

Если естественно неблагоприятные планеты ослаблены по знаку, то они окажут негативный эффект на жизнь человека в соответствии с тем домом, в котором находятся.

Когда естественно благоприятная планета сильна по знаку и дому, но находится под воздействием нескольких естественно неблагоприятных планет, то ее добрая сила уменьшается.

Аспекты планет на дома гороскопа могут улучшать или ухудшать дела этих домов. Ниже мы предлагаем ключ к синтезу идей планет и 12-ти домов гороскопа.

\subsubsection*{Первый дом}

Первый дом --- самый важный в карте рождения. Он указывает на характер, поведение и главные направления деятельности человека. Если первый дом сильный, то в характере человека будут преобладать качества незаурядной, яркой личности. Сила дома оценивается по планетам, занимающим этот дом, а также по аспектам, которые они на него оказывают. Если первый дом находится в основном под влиянием естественно благоприятных плаент, то характер и поведение человека будут достойны уважения, и люди с доверием будут относиться к нему. Когда первый дом находится под влиянием естественно неблагоприятных планет, то человек может отличаться грубостью, агрессивным поведением, и конституция его тела будет несколько ослаблена.

\begin{myenum}
	\item Солнце --- Внушительная внешность и способности лидера.
	\item Луна --- Привлекательная внешность и мягкие черты характера.
	\item Марс --- Воинствующее поведение, шрамы на теле или голове, вызывающий вид.
	\item Меркурий --- Дружеские отношения, легкость в общении, здравый смысл в любых действиях.
	\item Юпитер --- Величественная наружность, врожденное чувство справедливости, здоровье, интерес к религиозным занятиям.
	\item Венера --- Красивое тело, склонность к чувственным удовольствиям и изящным искусствам, любвеобильность, теплота и нежность к другим людям.
	\item Сатурн --- Серьезный вид, депрессия, печаль, озабоченность, продуманные действия, глубокие мысли.
	\item Раху --- Страстный характер, критическое настроение, сильные чувства.
	\item Кету --- Болезненность, слабая конституция тела.
\end{myenum}

Первый дом показывает наскоолько в человеке развито чувство патриотизма. Влияния естественно благоприятных планет на этот дом привязывают человека к своей родине и приносят ему успех там, где он родился.

Влияния Солнца, Марса и Раху на первый дом дает физическую силу. Для полного анализа первого дома вы должны рассмотреть Солнце со всех точек зрения, так как эта планета является определяющей в личности человека. Если Солнце хорошо расположено в гороскопе и под благоприятным влиянием других планет (а в первом доме есть добрые планеты, или они оказывают влияние на этот дом), то личность человека будет значительной, и он будет пользоваться уважением среди людей.

Более сложен для понимания результат первого дома при смешанных влияниях на него. Надо смотреть, сколько планет благотворных и неблаготворных влиют на первый дом, и предпочтение отдавать тем, которые сильнее. Например, первый дом находится в Стрельце, Меркурий в седьмом доме в Близнецах, а Сатурн в одиннадцатом доме в Весах. Меркурий имеет силу собственного знака, а Сатурн --- знака экзальтации, поэтому влияние Сатурна на первый дом будет сильнее, чем Меркурия, хотя на личности будут сказываться аспекты и Сатурна и Меркурия одновременно.

Ментальная энергия также относится к первому дому, чтобы понять, каким уровнем мышления обладает человек, необходимо рассмотреть положение Луны(ум) и Меркурия (интеллект).

Планеты в первом доме или их влияния на этот дом указывают на способности человека в соответствии с их естественной природой и природой того знака, в котором находится первый дом.

\subsubsection*{Второй дом}

По второму дому мы определяем богатство, финансовое положение, семейные дела и другие сферы жизни, управляемые этим домом.

Если Венера и Юпитер вместе находятся во втором доме, то это явно улучшает семейные и финансовые дела, и все зависит от того, насколько эти планеты или одна из них сильны по знаку. Например, Венера и Юпитер находятся в Рыбах во втором доме. Венера в знаке экзальтации, а Юпитер в собственном знаке, поэтому сила второго дома велика, и человека при таком расположении планет ждет гармоничная семейная жизнь и прекрасной финансовое положение. Друго пример: Венера и Юпитер находятся в Деве во втором доме. Венера ослаблена по знаку, Юпитер во враждебном знаке, и все условия, представленные вторым домом, будут положительными относительно силы этих планет.

Если естественно неблагоприятная планета находится во втором доме, но сильна по знаку, то финансовое положение будет хорошим, но в семье будут ощущаться трения. Если неблагоприятная планета ослаблена по знаку и стоит во втором доме, то человека ждут материальные трудности, и его отношения с членами семьи будут крайне напряженными. Например, Марс в Козероге во второ доме даст хорошее финансовое положение, хотя будут споры в семье. Положение Марса в Раке во втором доме создаст невыносимые семейные взаимоотношения и ослабление материальных возможностей.

По второму дому мы определяем манеру говорить человека и, если в этом доме находятся благоприятные планеты или оказывают на него свое влияние, то речь будет мягкой и дипломатичной. Когда во втором доме пребывают или влияют на него неблагоприятные планеты, то речь становится грубой, критичной и лживой. Смешанное влияние планет дает смешанный результат: например, Венера и Марс находятся во втором доме --- речь сладкая, но временами бывает критичной.

По второму дому видно, как питается человек. Если во втором доме стоит Юпитер, Венера или Луна, то продукты питания у него будут высокого качества. Венера в это доме указывает на пристрастие к сладостям, Марс --- на хороший аппетит, острую и горячую пищу, Раху --- на человека, которые будет неразборчив в еде.

Чтобы понять состояние финансовых дел человека, необходимо рассмотреть Юпитер и то влияние, которое на него оказывают другие планеты. Сильные второй дом и Юпитер наградят человека большим богатством и успехом в финансовых операциях. Каждая планета во втором доме будет выражать свою естественную природу. Например, Меркурий обусловит получение заработков от лекций или публичных выступлений, Марс свидетельствует о вероятности денежного вознаграждения за операции с недвижимостью или энергичную финансовую деятельность. Но если планета, находящаяся во втором доме, будет под аспектом другой планеты и та окажется сильнее, то предпочтение в выводах второго дома надо отдать более сильной планете, хотя и боле еслабая планета будет определенным образом оказывать влияние на этот дом.


\subsubsection*{Третий дом}

Главные идеи третьего дома --- это братья, сестры, поездки на небольшие расстояния, средства информации и связи. Сильный третий дом дает человеку большую выносливость и силу воли. Если в этом доме более двух планет и одна из них неблаготворная, то человек будет наделен терпеливостью.

Если одиннадцатый дом управляет старшими братьями и сестрами, то третий --- братьями и сестрами вообще, независимо от их возраста.

Марс с Сатурном или Раху в третьем доме указывают на жесткие отношения с братом или сестрой, которые часто приводят к открытой вражде. Если Марс находится в третьем доме вместе с Венерой или Юпитером, то толкование будет другим: отношения с братом или сестрой хорошие, но есть вероятность их гибели от хирургического вмешательства или несчастного случая. Все зависит от того, какие дополнительные влияния испытывает дом. Если третий дом находится под влиянием благотворных планет, то идеи его будут в основном процветать, а если под вилянием неблаготворных планет, то можно будет сказать, что третий дом серьезно поврежден, и человеку понадобится проявить большое упорство и терпение для достижения какого-либо результата.

Для того чтобы глубже понять взаимоотношения с братом и сестрой, необходимо рассмотреть Марс и те явления, которые он испытывает, а также сделать полный анализ третьего дома.

Под аспектом злых планет Раху и Сатурна Марс предсказывает на повреждение рук от оружия или огня. Луна в третьем доме указывает в основном на путешествия и возможности, связанные с написанием каких-либо текстов, а также на то, что брат или сестра будут иметь мягкий характер и привлекательную внешность. Юпитер с Венерой в третьем доме обуславливают полную гармонию в отношениях с братом или сестрой. Сатурн и Раху в третьем доме являются причиной потери как младшего, так и старшего брата.

Женские планеты в третьем доме больше говорят о сестрах, а мужские --- о братьях. Аспекты мужских и женских планет на третий дом также надо учитывать. Раху в этом доме указывает на путешествия, напряженный ручной труд, плохие привычки братьев или сестер.

\subsubsection*{Четвертый дом}

Четверты дом представляет мать, недвижимое имущество, личный транспорт, степень образованности, друзей, комфортные условия жизни и\,т.\,д. По нему мы определяем, насколько человек будет восприимчив. Естественно благоприятные планеты в этом доме указывают на хорошее жилье, прекрасных друзей, комфортную жизнь, на большие возможности для получения образования (в отличие от образования, получаемого по пятому дому, которое указывает на личные интеллектуальные достижения, здесь оно рассматривается как удача в этой жизни), на жизненные силы матери и отношения с ней. когда в четвертом доме находятся естественно неблагоприятные планеты, то результат будет противоположный. Но если сильные Марс или Сатурн стоят в этом доме, то человек будет иметь хорошие возможности для покупки дома или квартиры, чего нельзя сказать относительно приобретения личного транспорта.

Юпитер в этом доме указывает на духовно близких и честных друзей, Венера --- на друзей, связанных с искусством, Меркурий --- на друзей, занимающихся интеллектуальной деятельностью, слабый Сатурн или Раху --- на неблагоприятное окружение и\,т.\,д.

Сильная Венера в четвертом доме свидетельствует о приобретении личного автомобиля, а также о любви к природе.

Чтобы составить полное представление о жизни матери, необходимо рассмотреть этот дом вместе с Луной, так как эта планета руководит матерью. Если Луна испытывает большое влияние благотворных планет и находится в 1,\,4,\,5,\,7,\,9 или 10-м домах, то карта рождения содержит указание на хорошие условия жизни матери или на ее достойное место в обществе. Не забывайте анализировать аспекты на четвертый дом!

\subsubsection*{Пятый дом}

Этот дом является благоприятным, поскольку относится к домам тригона. Он представляет детей, образование, науку, авторство и многие другие идеи, которые не являются главными.

Меркурий в пятом доме в Близнецах, Деве или Водолее указывает на высокообразованного человека. Марс в пятом доме в воздушных знаках, кроме Весов, представляет человека, занимающегося наукой.

Все благоприятные планеты в этом доме или их аспекты создают хорошие условия для жизни детей. Неблагоприятные планеты провоцируют напряжения в отношениях с детьми. Если они ослаблены по знаку и не имеют поддержки благотворных планет, то могут привести к смерти ребенка. Вопросы, связанные с детьми, рассматриваются по пятому дому и Юпитеру. Если большее влияние на пятый до оказывают женские планеты, то рождаются девочки, если мужские --- мальчики. Если Марс в карте рождения находится в пятом доме в огненном знаке, то ребенок будет предприимчив и амбициозен. Марс в пятом доме в Раке указывает на раздражительного ребенка, он будет не раз травмирован. Если Марс стоит в пятом доме и находится под аспектом Сатурна, это может быть причиной смерти ребенка от несчастного случая.

Ослабленный Сатурн в пятом доме уеньшает интеллектуальные способности человека.

Сильная Венера в этом доме приносит красивого ребенка, а также делает его любвеобильным, обуреваемым мирскими желаниями. Юпитер указывает на его жизненные силы.

Раху в пятом доме является причиной абортов и тяжелых родов в женском гороскопе. Если Раху и Кету находятся в Овне, Тельце или Ракев пятом доме они не создают препятствий детям на их жизненном пути. В остальных знаках эти теневые планеты будут проявлять себя как проиводействующие силы.

Пятый дом отвечает за пищеварение и, если в этом доме находятся естественно неблагоприятные планеты, они вредят процессу переваривания пищи.

Марс в пятом доме обусловливает язву желудка, особенно, если он находится под влиянием неблаготворных планет. Солнце в этом же доме провоцирует боли в желудке. Вы не должны забывать, что аспекты неблаготворных планет оказывают подобные же действия.

\subsubsection*{Шестой дом}

Это неблагоприятный дом для Юпитера, Венеры, Меркурия и Луны. Для остальных планет нейтральный. Шестой до представляет болезни, хирургические операции, недоброжелателей и врагов, препятствия в жизни и\,т.\,п.

Естественно, неблагоприятные планеты в шестом доме дают силы для сопротивления врагам, но могут создать проблемы в отношении здоровья, так как поражают этот дом.

Меркурий, находящийся здесь, является причиной беспокойства, причиняемого недоброжелателями.

Луна в это доме создает ментальное беспокойство, особенно если она находится под влиянием неблагоприятных планет.

Юпитер и Венера в шестом доме явно смягчают все обстоятельства, связанные с недружелюбно настроенными людьми, и дают человеку хорошее здоровье. Однако они не в силах обеспечить ему комфортную жизнь.

\subsubsection*{Седьмой дом}

Благоприятный, так как относится к домам квадранта. Планеты, находящиеся в этом доме, обусловливают общественную и супружескую жизнь в соответствии со своей природой.

Солнце в этом доме делает человека заметным в общественной среде, но привносит определенные трудности в его брачные отношения.

Луна в седьмом доме создает условия для частых поездок. Если он находится под аспектом благоприятных планет, то является причиной счастливого брака.

Марс при таком расположении укажет на яркую личность, но в браке этому человеку будут сопутствовать частые ссоры. Если Марс находится под аспектом неблагоприятных планет, то это может привести к разводу. Многое зависит и от знака Зодиака, занимаемого Марсом. Например, Марс в Козероге представляет более благоприятную позицию для седьмого дома, чем марс в Раке. Как вы уже знаете, это является главным правилом в индийской астрологии для определения силы или слабости планет.

Меркурий в седьмом доме укажет на человека, легко вступающего в контакты, а также а перемену брачных связей.

Юпитер в этом доме покровительствует счастливому браку при условии, что он не испытывает влияния неблаготворных планет.

Сильная Венера, находящаяся в седьмом доме, представляет человека всецело ориентированного на сексуальную жизнь, а также на красивого мужа или жену. Если, например, Венера находится под влиянием Юпитера, то партнер по браку может быть религиозным человеком. Здесь все будет зависеть от других аспектов на Венеру и седьмой дом.

Сатурн при таком расположении обычно указывает на партнера по браку, который будет старше по возрасту. Он является неблагоприятной планетой, и поэтому условия брачной жизни будут не лучшими.

Раху в седьмом доме --- хорошо для владельца гороскопа, но плохо для его супружеской жизни, так как Раху проявляет себя подобно Сатурну.

Кету в этом доме создает препятствия в брачных отношениях и приносит неудачи, но для реализации себя в общественной жизни эта позиция планеты является благоприятной.

Если в седьмом доме нет планет, но он находится под влиянием благотворных планет, то брак будет удачным, если находится под влиянием неблагоприятных планет, то условия супружеской жизни будут ослаблены, если на этот дом будут влиять одновременно благотворные и неблаготворные планеты, то брак будет временами хорошим, временами плохим. В гороскопе женщины для определения брачных условий жизни и характера мужа вы должны рассмотреть седьмой дом, Венеру и Юпитера (Юпитер, как и Венера, в женском гороскопе представляет мужа). Если седьмой дом, Венера и Юпитер будут в основном находиться под аспектами благотворных планет, то брачные узы крепки, а если под аспектами неблаготворных планет, то у мужа будет плохой характер и брак окажется весьма шатким.

В гороскопе мужчины все брачные условия жизни и характер жены предопределяют седьмой дом и Венера.

венера в знаке Зодиака поможет вам определить душевные силы, поведение и характер партнера по браку. То же по Юпитеру определяется в женском гороскопе. Например, Венера в Овне служит указанием на энергичную и страстную натуру супруги.

\subsubsection*{Восьмой дом}

Это самый неблагоприятный дом гороскопа. Он относится к кончине определенного человека и смерти вообще, то есть смерти других людей, родственников, друзей. Восьмой дом указывает на трансформацию сознания человека и изменение его мировоззрения, а также психическую энергию, на основе которой развивается интуиция. Сильный восьмой дом наделяет даром предсказывания и поэтому руководит астрологией. Все тайные и эзотерические знания находятся в ведении этого дома, а также исследования в данной области. Восьмой дом ставит человека в жесткие условия и заставляет его изменить взгляды на законы мироздания. Он создает препятствия на жизненном пути и поэтому не дает быстро подняться человеку по лестнице общественного статуса.

Если здесь находятся благотворные планеты. то детство у человека будет безрадостным, серым, однако, повзрослев, он станет смелым и достигнет успеха в трудных проектах.

Восьмой дом является домом продолжительности жизни. Вопрос, связанный с определением времени смерти человека, сложный даже для индийских джйотиш--пандитов (астрологов--ученых) и, по мнению некоторых из них, чтобы описать все правила, по которым вычисляется время смерти человека, надо писать целый год. Но есть основные правила, которые берутся во внимание всеми индийскими астрологами для вероятного определения периода смерти человека (часть из них представлена в следующей главе). Здесь мы даем главное правило для определения продолжительности жизни. Если все неблагоприятные планеты находятся в восьмом доме или влияют на него, то шансы прожить долгую жизнь явно уменьшатся, особенно если эти планеты ослаблены по знаку. Благотворные планеты в этом доме или их влияние на него увеличивают продолжительность жизни. Сильный Сатурн в восьмом доме указывает на долгожительство. Знак Зодиака в восьмом доме показывает от какой болезни или разрушения какого органа наступит смерть. Например, Овен в восьмом доме --- смерть наступит от кровоизлияния в мозг или травмы головы, посольку Овен управляет головой и мозгом.

Вопросы, связанные с получением наследства, рассматриваются также по этому дому. Если в восьмом доме находятся Юпитер, Венера, Меркурий, Луна или Марс, то человек столкнется в своей жизни с проблемами наследования имущества после смерти родственников.

Луна представляет ум. Находясь в восьмом доме в момент рождения человека, эта планета будет способствовать тому, что он будет много размышлять о смерти и предназначении людей в этом мире. луна в восьмом доме под аспектом неблагоприятной планеты является причиной страданий матери.

Злой Марс в восьмом доме втягивает человека в судебные процессы, а также ослабляет психическую уравновешенность. Сильный Марс здесь наделяет человека смелостью и бесстрашием, дает после смерти родственников хорошее наследство в виде дома или квартиры.

Сильный Меркурий указывает на знание психологии и не вредит делам, связанным с образованием.

Кету в восьмом доме обусловливает хорошую интуицию, особенно, если она находится вместе с Луной или другими благоприятными планетами.

Венера в восьмом доме под влиянием неблагоприятных планет становится причиной неудачливого партнера по браку или партнера, обладающего плохими чертами характера.

Арест человека также связан с восьмым домом, так как этот дом руководит задержанием вообще.

Юпитер в этом доме указывает на трудно диагностируемые болезни.

Раху в восьмом доме часто представляет воров и людей с воровскими привычками, с которыми по воле судьбы придется столкнуться человеку.

Солнце в этом доме ослабляет волевые качества человека и в период детства делает его замкнутым, если планета находится под влиянием Сатурна или Раху.

\subsubsection*{Девятый дом}

Это благоприятный дом, так как относится к тригону. Идеи девятого дома: отец, религия, удача и неудача, судьба, начальник, дальние путешествия, иностранные языки и\,т.\,д. Он также имеет отношение к процветанию человека, если находится под влиянием сильных или естественно благоприятных планет, то характер отца будет миролюбивым и отзывчивым. Юпитер в нем указывает на долголетие отца, сильный Сатурн --- на чувство отцовского долга. Раху или луна в девятом доме представляют дальние путешествия. Если они будут находиться здесь вместе с Сатурном, то такая позиция указывает на заграничные путешествия. Праведность человека также рассматривается по девятому дому, и влияние благотворных планет подтверждает это.

Позиция Марса, Сатурна или Раху в девятом доме делает человека свободомыслящим. Если Сатурн или Раху стоят в этом доме, но не подвержены аспектам благоприятных планет, то человек далек от религии. Если же они находятся под аспектом Юпитера, то человек, напротив, может серьезно заняться теологией. Венера в Тельце, Весах или Рыбах в этом доме сулит материальный достаток. Если планеты в девятом доме расположены во враждебных или ослабленных знаках и если эти планеты являются неблагоприятными, то в судьбе человека будут большие трудности. Например, Сатурн в Овне в девятом доме приносит неудачу.

Девятый дом также представляет высшие знания и поэтому руководит духовными наставниками. Меркурий в девятом доме под аспектом Юпитера дает прекрасные шансы для получения высших знаний. Посещение церквей и храмов рассматривается по девятому дому. Юпитер вместе с Кету в этои доме наделяет человека сильными религиозными чувствами, и человек совершает паломничество. Сильные Юпитер и девятый дом пророчат человеку карьеру правоведа.

На юге Индии жизнь отца рассматривают по девятому дому, а на севере --- по десятому дому. Наша практика подсказывает, что делать это надо одновременно по девятому и по десятому домам. Синтез результатов этих домов и Солнца даст объективный портрет отца.

\subsubsection*{Десятый дом}

Один из домов квадранта, поэтому десятый дом благоприятный. Представляет профессию, занятие, репутацию, известность, славу, общественное положение, отца, власть и\,т.\,д. По этому дому мы определяем, насколько успешной может быть карьера человека и какое место он займет в обществе. Благотворные планеты в десятом доме или их влияние на этот дом дадут человеку хорошие возможности и условия в профессиональной деятельности. Неблагоприятные планеты в десятом доме, если они находятся во враждебном или ослабляющем знаке, создадут напряжение в профессиональных делах и заставять человека приложить много усилий в этой сфере. Неблаготворные планеты в десятом доме, если они находятся в дружественном или экзальтирующем знаке, принесут большую удачу в карьере, но человеку придется хорошо потрудиться. Природа планеты, находящейся в десятом доме или влияющей на него, будет определять вид профессии. Например, Солнце в Овне в десятом доме указывает на профессию политика, государственного деятеля и, если Солнце не находится под аспектом Сатурна или Раху, то человек может занимать высокий государственный пост в своей стране.

Венера в этом доме свидетельствует о поэтическом даре, музыкальных способностях, работе в текстильной промышленности и\,т.\,д., то есть природа планеты подскажет вид профессиональной деятельности. Также надо обратить внимание на знак Зодиакак в десятом доме. Если десятый дом в Близнецах, то человек больше склонен к интеллектуальным занятиям и, если пятый дом, который руководит образованием, сильный, то он может работать преподавателем в высшем учебном заведении.

Профессии и занятия, соответствующие планетам:
\begin{myenum}
	\item Солнце --- врач, хирург, человек, занимающий пост в правительстве или находящийся на общественной службе.
	\item Луна --- общественный деятель, а также работа с жидкостями, лекарственными травами.
	\item Марс --- инженер, математик, конструктор, человек, занимающий высокую должность в армии и полиции (милиции).
	\item Меркурий --- писатель, журналист, секретарь--машинистка.
	\item Юпитер --- учитель, философ, правовед, советник.
	\item Венера --- работник культуры, человек искусства, модельер, швея.
	\item Сатурн --- рабочий, а также секретные дела.
	\item Раху --- скрытые действия (теневая экономика).
	\item Кету --- духовное развитие (йога).
\end{myenum}

Этот перечень не является исччерпывающим для выше названных планет, но он послужит ключом в определении других профессий. На самом деле редко случается определить профессию по одной планете, в основном для этого используется комбинация планет. Например, Солнце, Луна и Меркурий являются главными энергиями в писательской карьере. Взаимосвязь энергий этих планет с другими планетами указывает на различные вариант приложения интеллектуальных способностей человека. Например, Луна вместе с Юпитером или под его аспектом указывает на религиозное мировоззрение, и эта комбинация подразумевает религиозную деятельность. Меркурий во взаимоотношениях с Сатурном представляет интеллект критика. Меркурий с Юпитером --- карьеру журналиста. Солнце с Венерой одаривают приятным голосом, который может способствовать карьере певца. Юпитер и Марс в паре руководят силой выражения личностных качеств. Юпитер с Сатурном, означающие ``речь во времени'', покровительствуют писателю--фантасту. Сильная Венера (красота), Сатурн (движение, время) в связке с Марсом (гибкость, мускулы) будут полезными для занятий танцами. Венера, Раху и Кету являются преобладающими энергиями для карьеры драматического артиста.

Венера (красота), Юпитер (выразительность) и Сатурн (труд, терпение) важны для людей, работающих в сфере изобразительного искусства. Марс и Сатурн (невидимые энергии) анализируются для определения работы, связанной с электричеством. Энергии Марса и Меркурия часто подходят инженеру--механику.

Чтобы определить профессиональные способности и возможности человека, необходимо сделать анализ десятого дома, Солнца, Меркурия, Юпитера, Сатурна, а потом рассмотреть оставшиеся планеты.

\subsubsection*{Одиннадцатый дом}

Главные идеи одиннадцатого дома --- заработки и доходы. Это единственный дом, в котором все планеты рассматриваются как приносящие доход человеку. Идеи этого дома отличаются от идей второго дома, представляющих богатство и платежеспособность. Одиннадцатый дом следует за десятым домом, поэтому он указывает на заработки в зависимости от профессии. Источники ожидаеых доходов в соответствии с природой планет, занимающих одиннадцатый дом или оказывающий влияние на него:

\begin{myenum}
	\item Солнце --- работа в правительстве, руководящая должность, занятие фотографией, золото, стекло.
	\item Луна --- сотрудничество с женщинами, продукты с водой, серебро, жемчуг.
	\item Марс --- производство, товары, изготавливаемые с помощью огня, служба в армии и полиции (милиции), земельное имущество, инженерное и конструкторское дело.
	\item Меркурий --- работа в учреждениях образования, публикации.
	\item Юпитер --- преподавательская работа, консультативно--совещательная деятельность, религиозная деятельность, правоведение.
	\item Венера --- культура и искусство, одежда и обувь, индустрия развлечений.
	\item Сатурн --- государственная служба, железо, уголь, азартыне игры и аферы.
\end{myenum}

Одиннадцатый дом связывают с природой Юпитера, поэтому его расположение и силу надо определять параллельно с результатами этого дома.

Одиннадцатый дом представляет друзей, которых также рассматривают по четвертому дому.

Сильный одиннадцатый дом указыват на независимого человека и обеспечивает высокий материальный уровень жизни.

\subsubsection*{Двенадцатый дом}

Неблагоприятный, по нему определяются потери, утраты. Индийские астрологи называют этот дом ``домом разрушения''. Двенадцатый дом --- это последний дом в гороскопе, он указывает на завершение эволюционного процесса в данной жизни. Потеря должности или работы рассматривается также по этому дому. Некоторые индийские школы астрологии предсказывают смерть человека по двеннадцатаму дому, не игнорируя вместе с тем идеи восьмого дома.

Если благоприятные планеты занимают двеннадцатый дом, то человек будет совершать благородные поступки и тратить деньги больше на то,  что соответствует природе планеты.

Венера показывает, что денежные средства будут идти на жену или любовницу.

Юпитер покровительствует благотворительной деятельности.

Сатурн указывает на практику аскета. если планета находится во враждебном или ослабленном знаке и под аспектами других неблагоприятных планет, то это может привести к полному материальному краху или к тюремному заключению.

Марс является причиной резкого ухудшения материального благостояния, если находится в плохом знаке. В этом доме он нарушает супружескую гармонию, так как оказывает влияние на седьмой дом.

Меркурий явно уменьшает шансы для получения высшего образования, но дает хороший результат в области трансцедентальных знаний.

Луна в этом доме (особенно, если она убывающая) указывает на ментальные страдания, желание побыть в одиночестве, дальние путешествия, вещие сны.

Раху в двеннадцатом доме становится причиной длительных зарубежных поездок, беспокойных снов, больших расходов.

Кету предопределяет желание человека освободиться от мирской суеты, посвятить себя духовной жизни.

Пребывание в госпитале или в больнице рассматривается также по двеннадцатому дому. Монахи--отшельники имеют четко выраженный двеннадцатый дом.

В этой главе мы дали ключ к пониманию значений планет в домах. Описать полностью результаты расположения планет в домах, которые проявляются в жизни человека, невозможно. Существуют тысячи планетных комбинаций, и каждая из них сугубо индивидуальна. Только гороскоп конкретного человека покажет, что именно означают позиции планет в домах. В книгах по Индийской предсказательной астрологии есть описание результатов планет в домах и знаках Зодиака, но они верны только отчасти, так как не учитывают побочные влияния на планеты и дома.

\section{Ключ к пониманию расположения хозяев домов}

Следующий уровень анализа гороскопа, рассматриваемый в этой главе, очень важен, индийские астрологи придают ему особое значение.

Солнце, Луна, Марс, Меркурий, Юпитер, Венера, Сатурн имеют собственные знаки:

\begin{mylist}
	\item Солнце --- Лев
	\item Луна --- Рак
	\item Марс --- Овен и Скорпион
	\item Меркурий --- Близнецы и Дева
	\item Юпитер --- Стрелец и Рыбы
	\item Венера --- Телец и Весы
	\item Сатурн --- Козерог и Водолей
\end{mylist}

Солнце и Луна имеют по одному собственному знаку, а остальные пять планет по два --- это значит, что планеты являются их хозяевами. Дом, который находится в определенном знаке Зодиака, будет подчинен планете --- хозяину этого знака, то есть планета будет хозяином дома, расположенного в собственном знаке этой планеты. Напрмер, Марс в карте рождения стоит во Льве в первом доме, а знаки Скорпион и Овен соответствуют четвертому и девятому домам, значит Марс является хозяином четвертого и девятого домов, и дела этих домов будут контролироваться Марсом.

В гороскопе хозяин какого-либо дома может располагаться в любом из двеннадцати домов и будет передавать энергию от одного дома к другому. Например, хозяин первого дома находится в пятом доме, следовательно, личностные качества человека получат развитие в его интеллектуальной деятельности или в воспитании детей.

Если хозяин какого-либо дома находится в одном и том же знаке с какой-либо планетой, то это означает, что хозяин дома связан с ней. Например, при восходящем Льве Юпитер пребывает в Стрельце в пятом доме вместе с Венерой. В данном примере Юпитер является хозяином пятого и восьмого домов, значит идеи этих домов будут связаны с идеями Венеры, что может указывать на красивых детей и благоприятные изменения в их материальной жизни.

Если хозяин дома находится под аспектом планеты, то это также надо учитывать. Например, Венера стоит в Тельце в десятом доме, а Юпитер в Скорпионе в четвертом. В данном примере Венера является хозяином третьего и десятого домов и испытывает влияние Юпитера, значит идеи, представленные третьим и десятым домами, будут нести на себе отпечаток идей Юпитера, то есть деятельность человека приведет его к наилучшим результатам, и он приобретет высокий общественный статус.

Если хозяин данного дома находится в одном знаке с хозяином другого дома или аспектирован им, то такая комбинация также анализируется. Например, связь хозяев первого и десятого домов приводит к мысли об ориентации человека на профессиональную деятельность, результат которой будет отражен тем домом, где присутствуют хозяева этих домов. Если хозяева первого и десятого домов находятся в пятом доме, то успех будет сопутствоавать делам пятого дома, то есть преподавательской или научной деятельности.

Другой пример: если хозяин первого дома находится под аспектом хозяина девятого дома, то поступки человека скорее всего будут праведными или личностные качества его получат развитие в религиозной деятельности. Связь хозяев домов из более чем двух планет, будет представлять многообразие явлений, управляемых этими хозяевами.

Ниже мы представим различные действия хозяев двеннадцати домов.

\subsubsection*{Хозяин первого дома}

Если первый дом находится в подвижном знаке и хозяин этого дома также в подвижном знаке, то человек будет ориентирован на разностороннюю деятельность.

Если первый дом находится в неподвижном знаке и хозяин первого дома также в неподвижном знаке, то человек будет ориентирован на устоявшиеся традиции.

Если первый дом находится в двойственном знаке и хозяин первого дома также в двойственном знаке, то личность человека и его действия будут изменчивыми.

Если хозяин первого дома занимает шестой, восьмой или двеннадцатый дом и связан с Солнцем, Марсом, Сатурном, Раху, Кету, то это указывает на ослабление здоровья.

Если хозяин первого дома сам является неблагоприятной планетой и занимает шестой, восьмой, или двеннадцатый дом, то это также подрывает здоровье человека. Исключение составляют позиции Марса в первом и шестом домах. Исключение составляют позиции Марса в первом и шестом домах при восходящем Скорпионе и в первом и восьмом домах при восходящем Овне. Позиция Сатурна в первом и двеннадцатом домах при восходящем Водолее.

Если хозяин первого дома находится в первом, четвертом, пятом, седьмом, девятом или десятом домах, то это благоприятно для здоровья (особенно, если хозяин первого дома связан или аспектирован благотворной планетой).

Если хозяином первого дома становится неблагоприятная планета и она находится в первом доме, то это не повлечет за собой болезнь тела.

Если хозяин первого дома стоит в шестом доме, то человека охватывают тревоги на ментальном уровне.

Если восходящими знаками являются Близнецы или Дева, а Меркури находится в Овне, Тельце, Раке, Скорпионе или Стрельце, то человек будет иметь слабое телосложение.

Если хозяин первого дома занимает дома квадранта или тригона и связан с благотворной планетой, то человек будет жить в хорошем месте и, наоборот, если хозяин первого дома занимает шестой восьмой или двеннадцатый дом и связан с неблагоприятной планетой, то эффект будет прямо противоположный.

Характер человека зависит от планетной природы хозяина первого дома, а также от того, какой дом он занимает. Например, при восходящей Деве хозяином первого дома является Меркурий, если он находится в Близнецах в десятом доме, значит характер человека будет легким, изменчивым и деятельным.

Если хозяин первого дома занимает второй дом, то человек больше ориентирован на деньги, богатство, семейные отношения.

Если хозяин первого дома занимает четвертый дом, то действия человека направлены на развитие дружеских отношений, приобретение недвижимости, транспортных средств и\,т.\,д.

Если хозяин первого дома занимает седьмой дом, то человек найдет свое место в общественной жизни и будет счастлив в браке и любви (если хозяин первого дома будет связан с Венерой, то выше перечисленные признаки будут доминировать в судьбе человека).

Если хозяин первого дома занимает восьмой дом и связан с благотворной планетой, то человек может обладать глубокими познаниями в психологии и изотерике. Если связь будет с неблагоприятной планетой, то действия его могут быть недостойны законов морали.

\subsubsection*{Хозяин второго дома}

Для накопления богатства благоприятно расположение хозяина второго дома в любом доме кроме третьего, восьмого и одиннадцатого дома. Если хозяин второго дома будет связан с Меркурием, Венерой или Юпитером, это также хорошо дл яидей второго дома.

Когда хозяин второго дома связан с Сатурном или Раху, то человек может разбогатеть незаконным путем.

Расположение хозяина второго дома в шестом, восьмом или двеннадцатом доме является неблагоприятным и не дает возможности добиться материального достатка, а также мешает осуществлению финансовых операций.

Если хозяин второго дома занимает шестой дом, то человек будет страдать зубной болью.

Если хозяин второго дома находится в восьмом доме, то человек получит наследство, но не будет удачи в накоплении денег.

Если хозяин второго дома стоит в двеннадцатом доме, то это предсказывает утрату кого-либо из членов семьи, а также денежные расходы на благотворительность.

Если хозяин второго дома занимает десятый дом, то человек сделает материальные накопления благодаря карьере.

Если хозяин второго дома расположен в пятом доме, человек получит денежные средства через детей или образовательную сферу деятельности.

\subsubsection*{Хозяин третьего дома}
Если хозяин третьего дома и Марс находятся в шестом, восьмом или двеннадцатом доме (необязательно вмесет), то  отношения с братьями и сестрами будут плохими.

Связь хозяина третьего дома с благоприятной планетой в третьем доме указывает на хорошие отношения с братьями и сестрами.

Если хозяин третьего дома занимает восьмой дом, то это часто приводит к смерти брата, особенно если хозяин третьего дома связан с неблагоприятной планетой или испытывает на себе ее аспект. Подобный результат получается, есл изозяин третьего дома находится в двеннадцатом доме.

Если хозяин третьего дома располагается в домах квадранта или тригона и тут же находится Марс, то человек будет жить в полном согласии со своими братьями и сестрами.

Если хозяин третьего дома стоит в десятом доме, то эта позиция является благоприятноя для людей, работающих в издательстве и занимающихся писательским трудом (при условии, что Меркурий будет находиться в хорошем положении).


\subsubsection*{Хозяин четвертого дома}

Когда хозяин четвертого дома связан с хозяином первого, пятого или девятого дома и это происходит в четвертом доме, человек неожиданно приобретает недвижимое имущество, а если расположение Марса будет благоприятным, то оно к тому же, будет значительным.

Хозяин четвертого дома связан с хозяином пятого или девятого дома и стоит в одиннадцатом доме --- эта комбинация дает большие шансы на приобретение автомобиля.

Если хозяин четвертого дома ослаблен по знаку, находится в шестом, восьмом или двеннадцатом доме и Марс в плохом положении, то человек будет страдать из-за отсутствия хороших жилищных условий.

\subsubsection*{Хозяин пятого дома}
Если хозяин пятого дома связан с благоприятными планетами или аспектирован ими, то все условия жизни детей будут улучшены. Эта комбинация благоприятна также для образования и развития интеллектуальных способностей человека.

Плохо для идей пятого дома, когда его хозяин занимает шестой, восьмой или двеннадцатый дом.

Расхоложение хозяина пятого дома в десятом может предсказывать карьеру преподавателя или научного работника, все зависит от других показателей гороскопа.

Позиция хозяина пятого дома в двеннадцатом доме может явиться причиной отдаления своего ребенка (например, сын или дочь может уехать в другую страну).

Хозяин пятого дома в мужском знаке указывает чаще на сыновей, в женском на дочерей.

Если хозяин пятого дома занимает третий дом, то человек может самостоятельно получать знания из книг (особенно если хозяин пятого дома находится под аспектом Меркурия).

Если хозяин пятого дома стоит в девятом, то человек может стать автором научного труда.

Всегда будет благоприятным расположения хозяина пятого дома в домах квадранта и тригона.

Если хозяин пятого дома стоит в шестом доме, то человек будет испытывать большое беспокойство от детей.

\subsubsection*{Хозяин шестого дома}

Если хозяин шестого дома сильнее хозяина первого дома, то враги будут иметь больше шансов одержать победу над данным человеком, его организм будет плохо сопротивляться болезням. Если хозяин первого дома сильнее хозяина шестого дома, то все будет наоборот.

Если хозяин шестого дома связан с Солнцем или Раху и расположен в двеннадцатом доме, то человек будет жить в доме других людей и аморально вести себя.

Расположения хозяина шестого дома в первом, шестом, седьмом или восьмом доме приносит человеку победу над врагами.

Если хозяин шестого дома находится в третьем или одиннадцатом доме, то он будет применять насилие и станет источником страданий других людей.

Если хозяин шестого дома расположен во втором или двенадцатом доме, то такая комбинация указывает на недоброжелательного человека.

Если хозяин шестого дома стоит в пятом или девятом доме, то ребенок этого человека будет противостоять ему.

Расположение хозяина шестого дома в четвертом или десятом доме указывает на боли в нижней части живота или в области пупка.

Если хозяин шестого дома находится в домах квадранта и тригона, то он негативно влияет на дела, управляемые этими домами.

Если хозяин шестого дома расположен в двенадцатом доме, то эта комбинация может привести к болезни.


\subsubsection*{Хозяин седьмого дома}
Не только планеты в седьмом доме, Венера и Юпитер (в женском гороскопе) представляют партнера по браку, но и хозяин седьмого дома. Его связь с планетами является дополнительным фактором в определении брачных дел и характера супругов.

Если хозяин седьмого дома связан с Солнцем или им аспектирован, то это указывает на обладающего властью или доминирующего партнера по браку.

Если хозяин седьмого дома связан с Луной или аспектирован ею, то это дает эмоционального партнера по браку, а также может указывать на перемену в брачной жизни, явиться причиной не одного брака. Если седьмой дом находится под влиянием благоприятных планет, то брак будет счастливым.

Когда хозяин седьмого дома связан с Марсом или им аспектирован, это предсказывает страстного или обладающего большой физической силой партнера по браку, брачная жизнь может стать сложной в зависимости от других влияний на седьмой дом и на Венеру.

Связь хозяина седьмого дома с Меркурием или аспект последнего на него, пророчит сообразительного и моложавого партнера по браку, а также дает шанс повторного брака.

Если хозяин седьмого дома связан с Юпитером или им аспектирован, это указывает на богатого или духовно близкого партнера по браку, а также на благополучный брак (если седьмой дом и Венера будут под влиянием благотворных планет).

Если хозяин седьмого дома связан с Венерой или аспектирован ею, то это говорит о красивом, моложавом, артистичном, склонном к флирту партнере по браку. Если другие указания в гороскопе будут хорошими, то такая позиция даст счастливый брак.

Если хозяин седьмого дома связан с Сатурном или аспектирован им, это свидетельствует о том, что супруг (или супруга) будет старше по возрасту, педантичным, консервативным. Брак может быть запоздалый или омраченный жизненными невзгодами.

Если хозяин седьмого дома связан с Раху или Кету или аспектирован ими, то брачные условия будут нарушены. При других неблагоприятных влияниях на седьмой дом и Венеру эта комбинация приведет к расторжению брачного контракта.

Расположение хозяина седьмого дома в первом доме является благоприятным для брака.

Если хозяин седьмого дома стоит во втором доме, то партнер по браку будет сильно ориентирован на материальные и семейные дела.

Когда хозяин седьмого дома пребывает в седьмом доме, то партнер по браку будет уделять много времени самому себе.

Если хозяин седьмого дома находится в шестом доме, то партнер по браку будет беспокойным или болезненным человеком.

Если хозяин седьмого дома расположен в восьмом доме, то условия супружеской жизни ухудшаются.

Будучи в девятом доме, хозяин седьмого дома свидетельствует о том, что партнер по браку бедт религиозным человеком.

Хозяин седьмого дома в десятом доме --- партнер по браку будет сильно ориентирован на повышение своего общественного статуса и карьеру.

Если хозяин седьмого дома находится в одиннадцатом доме, то эта комбинация предполагает самостоятельного партнера по браку.

Если хозяин седьмого дома расположен в двенадцатом доме, то это указывает на неустойчивость брачной жизни, которая впоследствии приводит к разводу.


\subsubsection*{Хозяин восьмого дома}
Если хозяин восьмого дома находится в восьмом доме, то жизнь человека будет долгой.

Если хозяева первого и восьмого домов занимают одновременно один из домов квадранта или тригона, то это указвыает на долгожителя (особенно если оба хозяина находятся под влиянием Юпитера).

Если квадранты заняты балгоприятными планетами, а хозяин первого дома в комбинации с благотворной планетой и под аспектом Юпитера, то это положениу указывает на хорошую долгую жизнь.

Хозяева певого, восьмого и десятого домов в комбинации с Сатурном занимают квадранты --- жизнь будет долгой.

Если хозяин восьмого дома сильный и находится под аспектом благоприятных планет, то продолжительность жизни не будет вызывать тревоги.

Если хозяин третьего дома в третьем доме, то это хорошо для продолжительности жизни.

Если хозяин восьмого дома расположен в третьем доме, то это также хорошо для продолжительности жизни.

Если хозяин восьмого дома стоит в двенадцатом доме вместе с неблагоприятными планетами, то это является указанием на короткую жизнь.

Если хозяева первого и восьмого домов находятся в шестом доме, то такая комбинация не свидетельствует о долгой жизни.

Если хозяин восьмого дома находится в одиннадцатом доме, то жизнь человека будет счастливая в более поздние годы.

Если хозяйкой восьмого дома является неблагоприятная планета и она не имеет связи с благоприятной планетой, то такая позиция уменьшает продолжительность жизни.

Если хозяин восьмого дома занимает второй дом, то это обычно говорит об утрате кого-либо из членов семьи.

Если хозяин восьмого дома находится в третьем доме, то будут плохие отношения с братьями и сестрами, если нет благоприятных влияний на третий дом.

ПозицияЖ хозяин восьмоо дома в четвертом доме часто пророчит утрату кого-либо из родственников.

Хозяин восьмого дома в девятом доме и хозяин восьмого дома в десятом доме --- данная позиция является указанием на смерть отца.

\subsubsection*{Хозяин девятого дома}
Если хозяин девятого дома находится в первом доме, то поступк человека будут праведными.

Если хозяин девятого дома стоит во втором доме, то будут удачно сделаны денежные накопления.

Если хозяин девятого дома расположен в третьем доме, то это указывает на успех в писательской карьере.

Нахождение хозяин девятого дома в четвертом доме приносит удачу в делах, связанных с недвижимостью, а также матерью, друзьями.

Расположение хозяина девятого дома пребывает в шестом доме, то это указывает на исполнительный характер человека и на его неудачные действия.

Если хозяин девятого дома находится в десятом доме, то такая позиция --- большая удача для профессиональных начинаний.

Если хозяин девятого дома занимает двенадцатый дом, то это говорит о возможной потере отца.

\subsubsection*{Хозяин десятого дома}
Если хозяин десятого дома расположен в первом доме, то человек будет честолюбив, и все его действия будут направлены на укрепление своего общественного статуса.

Связь хозяина десятого дома с Венерой предсказывает удачную карьеру в искусстве, швейном производстве и вообще благоприятствует профессиональным делам.

Хозяин десятого дома, связанный с Сатурном, указывает на тяжелый труд и трудности в карьере.

Если хозяин десятого дома связан с Меркурием, то человек может сделать карьеру в бизнесе, коммуникациях, на писательском поприще и вообще это --- успех в профессиональных делах.

Когда хозяин десятого дома связан или аспектирован хозяином двенадцатого дома, то следует ожидать трудности или изменения в карьере.

Если хозяин десятого дома находится в третьем доме, то занятия человека будут направлены на духовное совершенствование, особенно ярко это проявится, когда есть аспект Юпитера на хозяина десятого дома.

Хозяин десятого дома расположен в двенадцатом доме --- слабая позиция мирских дел.


\subsubsection*{Хозяин одиннадцатого дома}
Если хозяин одиннадцатого дома сильный, расположен в одном из домов квадранта или тригона и находится под влиянием Юпитера или Венеры, то это указывает на высокие заработки человека.

Когда хозяин одиннадцатого дома слабый, расположен в шестом, восьмом или двенадцатом доме и находится под влиянием неблаготворных планет, то человек будет страдать из-за низких заработков.

Если хозяин одиннадцатого дома стоит в первом доме, то это указывает на коммуникабельного дружелюбного человека, у которого время от времени будут появляться новые друзья.

Позици хозяин одиннадцатого дома во втором доме свидетельствует о различных материальных приобретениях и благоприятных финансовых возможностях.

Если хозяин одиннадцатого дома занимает седьмой дом, то доход появится благодаря взаимовыгодным партнерским отношениям.

Если хозяин одиннадцатого дома расположен в десятом доме, то это показатель хорошего дохода от деятельности человека.


\subsubsection*{Хозяин двенадцатого дома}
Дом, в котором находится хозяин двенадцатого дома и дом, на который он влияет, укажет на что будут растрачиваться денежные средства. Если хозяин двенадцатого дома будет испытывать влияние благотворных планет, то деньги уйдут на хорошие дела, если неблаготворных планет --- они будут растрачены впустую.

Если хозяин двенадцатого дома связан с Венерой, то растраты ожидаются на партнера по браку, чувственные удовольствия.

Если хозяин двенадцатого дома связан с Юпитером, то деньги будут отданы детям или вложены в религиозную деятельность.

Если хозяин двенадцатого дома связан с Меркурием, то убытки будут понесены из-за друзей или дел с другими людьми.

Если хозяин двенадцатого дома связан с Марсом, то денежные средства будут направлены на братьев и сестер или на судебные процессы и спорные дела.

Если хозяин двенадцатого дома связан с Луной, то ожидается много расходов из-за матери, если с Солнцем --- из-за отца.

Если хозяин двенадцатого дома связан с Сатурном и хозяином шестого дома, то деньги уйдут на лечение или врагов.

Если хозяин двенадцатого дома связан с Раху, то человек понесет убытки из-за нечистых на руку людей.

Нахождение хозяина двенадцатого дома в девятом доме указывает на религиозного или духовно возвышенного человека.

Позиция хозяин двенадцатого дома в двенадцатом доме показывает, что деньги будут идти на благородные дела.

Когда хозяин двенадцатого дома находится в десятом доме, это ухудшает положение в профессиональной деятельности и ведет к большим расходам.

Мы показали на примерах некоторые позиции хозяев домов и то, как можно получить информацию из индивидуальных гороскопов, пользуясь предоставленным ключом. В будущем практические навыки и личная интуиция помогут вам понять этот уровень анализа гороскопа.

\section{Важнейшие планетные комбинации}

Две и более планет, находящихся во взаимосвязи, представляют планетную комбинацию, которая бывает благоприятной или неблагоприятной. Планетные комбинации могут усиливать или ослаблять планеты и дома. Те, что усиливают, являются благоприятными, те что ослабляют --- неблагоприятными.

\subsection*{Благоприятные комбинации планет}

Луна выше по долготе, но в одном знаке с Марсом дает большое богатство (например, Луна \signum{8}{20}{\cancer}, а Марс \signum{4}{15}{\cancer}). Меркурий выше Солнца и прошел точку сгорания (т.\,е. более \gradus{5} от Солнца) и Солнце в одном знаке с Меркурием --- такая комбинация указывает на высокие интеллектуальные способности (например, Солнце в \gradus{10} Тельца, а Меркурий в \gradus{19} Тельца).

Планеты занимают все ниже перечисленные дома: второй, шестой, восьмой и двенадцатый. Эта комбинация дает большое богатство. По крайней мере одна планета занимает второй дом от Луны --- хорошо для материального положения (например, Луна расположена в Овне, а Марс --- в Тельце). Если хотя бы одна планета занимает двенадцатый дом от Луны --- материальное положение улучшится (например, Луна расположена в Близнецах, а Венера --- в Тельце).



Планеты занимают все ниже перечисленные дома: второй, шестой, восьмой и двенадцатый. Эта комбинация дает большое богатство. По крайней мере одна планета занимает второй дом от Луны --- хорошо для материального положения (например, Луна расположена в Овне, а Марс --- в Тельце). Если хотя бы одна планета занимает двенадцатый дом от Луны --- материальное положение улучшится (например, Луна расположена в Близнецах, а Венера --- в Тельце).

Еще более благоприятная комбинация, если одна планета расположена во втором доме, а другая в двенадцатом доме от Луны --- высокое материальное положение (например, Луна находится в Раке, Юпитер --- во Льве, а Солнце --- в Близнецах).

Венера с Марсом в одном знаке и в любом квадранте делает человека лидером в его семье, а также наделяет его большой сексуальной энергией (например, Венера с Марсом во Льве в первом доме).

Венера в одном знаке с Сатурном свидетельствует о таланте или удачной карьере в искусстве, Сатурн в двенадцатом доме от Венеры дает подобный результат (например, Венера с Сатурном в Весах или Венера в Близнецах, а Сатурн в Тельце).

Раху в одном знаке с Венерой и Юпитером сулит все радости жизни.

Солнце с Луной в одном знаке и в любом квадранте говорит о способностях лидера (например, Солнце и Луна во Льве в седьмом доме).

Луна в одном знаке с Юпитером или во взаимной аспектной связи указывает на религиозное мировоззрение или набожность человека (например, Луна и Юпитер в Раке или Луна в Рыбах, а Юпитер в Деве).

Луна в одном знаке с Меркурием или во взаимной аспектной связи предопределяет высокий интеллект человека (например, Луна и Меркурий во Льве или Луна во Льве, а Меркурий в Водолее).

Луна в одном знаке с Кету свидетельствует о стремлении к духовной жизни.

Позиция Меркурия в знаке Марса или Марса в знаке Меркурия хороша для развития интеллекта (например, Меркурий в Овне или Марс в Близнецах).

Юпитер в одном знаке с Кету способствует духовным достижениям.

Венера, аспектированная или в связи с Юпитером, указывает на исселдовательские способности человека и обусловливает моральные устои семейной жизни.

Венера в знаке Марса или Марс в знаке Венеры свидетельство необузданных сексуальных желаний (например, Венера в Скорпионе, а Марс в Тельце).

Венера в одном знаке с Меркурием в любом квадранте дает большие материальные блага, но также может наделить коварной женой.

Венера в квадранте от Луны говорит о сексуальных устремлениях (например Луна в Тельце, а Венера во Льве).

Венера и Марс, аспектирующие друг друга, предсказывают активную сексуальную жизнь (например, Венера в Тельце, а Марс в Скорпионе).

Расположение Меркурия с Солнца в Козероге в первом доме указывает на интерес к медицине.

Позиция Марса и Солнца в Овне в десятом доме от восходящего знака или от Луны пропрочит человеку высокий государственный пост.

Если хозяева квадрантов связаны с хозяевами тригона, то это очень благоприятно для того дома, в котором имеется такая связь (например, Венера как хозяйка девятого дома расположена в Близнецах в десятом доме вместе с Меркурием, который является хозяином десятого дома. В данном примере усилен десятый дом. Другой пример: Меркурий как хозяин девятого дома расположен в водолее в пятом доме вместе с Сатурном, который является хозяином четвертого дома, следовательно, усилен пятый дом).

Если хозяева квадрантов обмениваются домами с хозяевами тригона, то это очень благоприятно для домов (например, Венера как хозяйка девятого дома расположена в Близнецах в десятом доме, а Меркурий как хозяин десятого дома находится в Тельце в девятом доме. В данном примере усилены девятый и десятый дома).

Если хозяева квадрантов и тригона одновременно аспектируют какой-либо дом, то этот дом становится сильным (например, при восходящем Раке Марс занимает седьмой дом, а Луна четвертый дом как хозяева пятого и первого домов). В данном примере десятый дом усилен, поэтому человек найдет удовлетворение в профессиональных делах.

Планета, которая расположена в собственном знаке или знаке экзальтации в квадранте от восходящего знака или от Луны, является очень благоприятной (например, Марс в Козероге в седьмом доме от восходящего Рака. Меркурий в Близнецах в десятом доме от восходящей Девы. Юпитер в Раке, а Луна в Овне. Марс в Скорпионе, а Луна во Льве).

Если Луна, Меркурий, Венера или Юпитер окружают отдельный дом или планету, то это является хорошим условием для данного дома или планеты (например, Солнце в Тельце, Луна в Овне, а Юпитер в Близнецах. В данном примере Марс находится между Венерой и Меркурием, поэтому на нем сказывается благотворное воздействие этих планет).

Если планета находится в ослабленном знаке, но хозяин этого знака расположен в знаке экзальтации, то ослабленная планет несколько усиливается (например, Меркурий ослаблен в Рыбах, но хозяин Рыб, Юпитер, расположен в Раке, значит, Меркурий набирает некоторую силу за счет экзальтированного Юпитера).


\subsection*{Неблагоприятные комбинации планет}

Если нет планет перед знаком, где находится Луна, и в следующем знаке от Луны, то человек часто будет впадать в депрессию, и окружающие не будут понимать его состояние.

Луна или асцендент, находящиеся в последних двух градусах водного знака, являются указанием на смерть в детском возрасте.

Солнце в одном знаке с Раху ведет к душевному разладу, и у других людей не будет взаимопонимания с этим человеком.

Солнце в одном знаке с Марсом указывает на опасность огня.

Венера в одном знаке с Раху сулит побочные сексуальные связи.

Юпитер, аспектированный Сатурном, выявляет ``вечного'' студента (ученика).

Несколько ретроградных планет в гороскопе говорят о человеке, критически относящемся ко всему на свете, склонному к отрицательным суждениям о людях.

Венера и Юпитер будут испорчены влиянием злой Раху, если находятся с ней в одном знаке.

Взаимосвязь Марса и Сатурна приносит трудности в те дома, в которых они расположены.

Венере вредит расположение в одном знаке с Солнцем.

Взаимосвязь Солнца и Сатурна создает трудности в жизни, но наделяет человека такими качествами, как целеустремленность, ответственность, смирение и духовность.

Луна в одном знаке с Сатурном порождает у человека чувство одиночества, но настраивает его на медитацию.

Планета будет уменьшать силу в том знаке, хозяин короткого расположен в знаке ослабления (например, Юпитер в Весах, а Венера в Деве).

Связь хозяев шестого, восьмого и двенадцатого домов в домах квадрантов или тригона является крайне неблагоприятной (например, при восходящем скорпионе Марс как хозяин шестого дома и Меркурий как хозяин восьмого дома находятся в Водолее в четвертом доме.

Общее влияние хозяев шестого, восьмого и двенадцатого домов на дома квадранта или тригона также неблагоприятно и поражает любой из этих домов (например, при восходящих Близнецах Сатурн из Козерога как хозяин восьмого дома и Марс из Девы как хозяин шестого дома аспектируют десятый дом в Рыбах. В данном примере десятый дом становится пораженным).

Если Сатурн, Марс, Раху, Кету или Солнце окружают отдельный дом или планету, то это является неблагоприятным условием для данного дома или планеты (например, Солнце в Тельце в седьмом доме, Марс в Близнецах в восьмом, а Сатурн в Овне в шестом доме. В этой ситуации уменьшается сила Солнца и седьмого дома, так как они находятся в тисках Марса и Сатурна. Другой пример: Венера в \gradus{12} Тельца, Раху в \gradus{5} Тельца, а Солнце в \gradus{20} Тельца. Итак, Венера зажата неблаготворными Раху и Солнцем, поэтому брачный союз, который представлен Венерой, будет нарушен).

Благотворная планета, находящаяся под аспектами двух и более неблаготворных планет, явно уменьшает свою позитивную силу и сзодает трудности в жизни человека (например, Юпитер расположен в Весах во втором доме, Сатурн --- в Овне в восьмом доме, а Марс в Раке в одинадцатом доме. В данном примере Юпитер аспектирован Сатурном и Марсом, поэтому он не принесет удачу в материальных делах и семейного счастья).

Существует большое количество других планетных комбинаций в индийской астрологии, которые мы не стали описывать из--за второстепенного их значения.

\section{Лунная карта и ее назначение}

Когда вы составите карту рождения, то обратите внимание на знак Зодиака, в котором находится Луна. Это второй восходящий знак. Индийские астрологи придают большое значение Луне, так как она руководит умом и миром эмоций человека.

Примите знак Зодиака, в котором находится Луна, за первый дом и составьте лунную карту. Например, в карте рождения восходящий знак в Близнецах, солнце в Стрельце в седьмом доме, Луна в Водолее в девятом доме, Марс в стрельце в седьмом доме, Меркурий в Стрельце в седьмом доме, Юпитер в Рыбах в десятом доме, Венера в Козероге в восьмом доме, Сатурн в Козероге в восьмом доме, Раху в Близнецах в первом доме и Кету в стрельце в седьмом доме. Так как Луна расположена в Водолее, то в лунной карте первый дом будет в Водолее, второй --- в Рыбах, третий --- в Овне, четвертый --- в Тельце, пятый --- в Близнецах, шестой --- в Раке, седьмой --- во Льве, восьмой --- в Деве, девятый -- в Весах, десятый --- в Скорпиона, одинадцатый --- в Стрельце и двенадцатый --- в Козероге. Исходя из выше приведенного примера карты рождения позиции планет в лунной карте будут следующие: Луна в Водолее в первом доме, Солнце в Стрельце в одинадцатом доме, Марс в Стрельце в одинадцатом доме, Меркурий в Стрельце в одинадцатом доме, Юпитер в Рыбах во втором доме, Венер в Козероге в двенадцатом доме, Сатурн в Козероге в двенадцатом доме, Раху в Близнецах в пятом доме и Кету в Стрельце в одинадцатом доме.

Лунная карта интерпретируется так же, как и карта рождения, то есть рассматриваются планеты в домах, расположение хозяев домов и на какие дома оказывают влияние планеты.

Начинающие астрологи часто задают вопрос, какая карта является приоритетной: карта рождения или лунная карта. Астрологи Индии считают: если восходящий знак будет сильнее Луны (восходящий знак под влиянием нескольких планет или они будут расположены в первом доме, а Луна в нейтральном, враждебном или ослабленном знаке), то карта рождения важнее чем лунная. И наоборот, если Луна будет сильнее, чем восходящий знак (Луна в дружесвенном знаке или знаке экзальтации и под влиянием различных планет, а восходящий знак не будет иметь подобных влияний), то ведущая роль принадлежит лунной карте.

Наша практика подсказывает, что обе карты представляют ценный материал для определения личностных качеств, способносте, привычек и различных жизненных сфер человека. Карта рождения укажет на один ряд событий, а лунная карта --- на другой. Например, по карте рождения мы определяем, что человек должен вступить в брак, и в то же самое время видим в лунной карте приятное путешествие, значит человек зарегистрирует свой брак, а потом поедет в свадебное путешествие.

Если на какое-либо событие (действие, характер) указывает карта рождения и это подтверждает лунная карта, то данной событие (действие, характер) будет усилено в определенный период времени или станет доминирующим в жизни.

\section{Карта навамcа и ее назначение}

В Индийской предсказательной астрологии существует шестнадцать карт, которые могут быть составлены для любого человека. Каждая из них содержит специфическую информацию, и большинство астрологов используют эти карты для определения достоинств планет. Карта навамса --- наиболее важная после карты рождения и лунной карты и составляется из одной девятой части зодиакального знака. Знак Зодиака, равный \gradus{30}, делится на девять равных частей, каждая из который равно \coord{3}{20}{0}.

\begin{landscape}
	\begin{table}[tph!]
		\centering
		\caption{Положение планет и домов в карте навамса.}
		\label{tbl:navamsa}

		% Расширить по вертикали
		\renewcommand{\arraystretch}{1.5}

		% Заполним данными
		\begin{tabular}{|l|c|c|c|c|c|c|c|c|c|c|c|c|}
			\hline
			 & \gradus{0}--\cord{3}{20} & \cord{3}{20}--\cord{6}{40} & \cord{6}{40}--\gradus{10} & \gradus{10}--\cord{13}{20} & \cord{13}{20}--\cord{16}{40} & \cord{16}{40}--\gradus{20} & \gradus{20}--\cord{23}{20} & \cord{23}{20}--\cord{26}{40} & \cord{26}{40}--\gradus{30} \\
			\hline
			Овен     & Овен & Телец & Близнецы & Рак & Лев & Дева & Весы & Скорпион & Стрелец \\
			Телец    & Козерог & Водолей & Рыбы & Овен & Телец & Близнецы & Рак & Лев & Дева \\
			Близнецы & Весы & Скорпион & Стрелец & Козерог & Водолей & Рыбы & Овен & Телец & Близнецы \\
			Рак      & Рак & Лев & Дева & Весы & Скорпион & Стрелец & Козерог & Водолей & Рыбы \\
			Лев      & Овен & Телец & Близнецы & Рак & Лев & Дева & Весы & Скорпион & Стрелец \\
			Дева     & Козерог & Водолей & Рыбы & Овен & Телец & Близнецы & Рак & Лев & Дева \\
			Весы     & Весы & Скорпион & Стрелец & Козерог & Водолей & Рыбы & Овен & Телец & Близнецы \\
			Скорпион & Рак & Лев & Дева & Весы & Скорпион & Стрелец & Козерог & Водолей & Рыбы \\
			Стрелец  & Овен & Телец & Близнецы & Рак & Лев & Дева & Весы & Скорпион & Стрелец \\
			Козерог  & Козерог & Водолей & Рыбы & Овен & Телец & Близнецы & Рак & Лев & Дева \\
			Водолей  & Весы & Скорпион & Стрелец & Козерог & Водолей & Рыбы & Овен & Телец & Близнецы \\
			Рыбы     & Рак & Лев & Дева & Весы & Скорпион & Стрелец & Козерог & Водолей & Рыбы \\
			\hline
		\end{tabular}
	\end{table}
\end{landscape}

Картa навамса составляется из координат восходящего знака (асцендента) и планет карты рождения. Например, карта рождения имеет следующие координаты:

\begin{table}[tph!]
	% Расширить по вертикали
	%\renewcommand{\arraystretch}{1.5}

	% Заполним данными
	\begin{tabular}{|lll|}
		\hline
		Асцендент & \cord{28}{02} & Льва \\
		Солнце   & \cord{25}{13} & Девы \\
		Луна     & \cord{9}{54} & Скорпиона \\
		Марс     & \cord{23}{59} & Льва \\
		Меркурий & \cord{17}{05} & Весов \\
		Юпитер   & \cord{3}{16} & Близнецов \\
		Венера   & \cord{28}{40} & Льва \\
		Сатурн   & \cord{5}{31} & Весов \\
		Раху     & \cord{6}{14} & Козерога \\
		Кету     & \cord{6}{14} & Рака \\ \hline
	\end{tabular}
\end{table}

Определяем по таблице, в какой навамсе находятся асцендент и девять планет с учетом их координат в карте рождения.

Составляем карту навамса и получаем следующие координаты расположения планет и домов:

\begin{mylist}
	\item Стрелец --- первый дом.
	\item Солнце --- Лев и девятый дом.
	\item Луна --- Дева и десятый дом.
	\item Марс --- Скорпион и двенадцатый дом.
	\item Меркурий --- Рыбы и четвертый дом.
	\item Юпитер --- Весы и одинадцатый дом.
	\item Венера --- Стрелец и первый дом.
	\item Сатурн --- Скорпион и двенадцатый дом.
	\item Раху --- Водолей и третий дом.
	\item Кету --- Лев и девятый дом.
\end{mylist}

В карте навамса градусы и минуты долготы отсутствуют, учитываются только знаки Зодиака.

Карта навамса дает дополнительную информацию о брачных отношениях наряду с картой рождения и лунной картой.

Эта карта поможет определить дополнительные достоинства планет (например, планеты в экзальтации, собственном и ослабленном знаках).

В выше приведенном примере в карте навамса Солнце расположено во Льве, Марс --- в Скорпионе, а Меркурий в Рыбах, поэтому Солнце и Марс, находящиеся в собственных знаках, будут усилены, Меркурий, который стоит в ослабленном знаке, будет ослаблен.

Если планета расположена в одном и том же знаке в карте рождения и в карте навамса, то ее сила приравнивается к силе собственного знака, и на это надо обратить особое внимание, поскольку такая позиция планеты оказывает благотворное влияние на жизнь человека (например, в картах рождения и навамса Солнце находится в Близнецах, значит Солнце в гороскопе будет достаточно сильным и его влияние на тот дом, в котором оно стоит в карте рождения, будет благоприятным).

Если планета расположена в карте навамса в ослабленном знаке, то она будет ослаблять тот дом, которым управляет в карте рождения (например, в карте рождения Меркурий находится в Весах в третьем доме и является хозяином второго и одинадцатого домов, а в карте навамса он пребывает в знаке ослабления, то есть в Рыбах, значит идеи второго и одинадцатого домов будут ослаблены, следовательно, платежеспособность и заработки человека будут низкими.

Если планета расположена в карте навамса в собственном знаке или в знаке экзальтации, то она будет усиливать тот, дом, которым управляет в карте рождения (например, в карте рождения Солнце расположено в Деве во втором доме и является хозяином первого дома, а в карте навамса оно находится во Льве, то есть в собственном знаке, значит идеи первого дома (внешность, характер и жизнестойкость человека) будут улучшаться.

Нельзя рассматривать карту навамса отдельно от карты рождения и лунной карты, так как именно в них содержатся главные условия брачных отношений. Карта навамса их только дополняет.

Некоторые индийские астрологи предсказывают события исходя из позиции планет в карте навамса. Но, учитывая, что она является более тонким инструментом в предсказании, чем карта рождения и лунная карта, советуем пока ею не пользоваться. Когда вы будете иметь опыт в прочтении двух главных карт, вы сами почувствуете, что наступило время делать предсказания, используя карту навамса. Надо принять к сведению еще и такой факт: если планета расположена в одном и том же доме в карте рождения и карте навамша, то этот дом будет себя сильно проявлять в зависимости от природы планеты (например, в карте рождения Юпитер находится в Близнецах в одинадцатом доме, а в карте навамса он в Весах в одинадцатом доме --- значит, во--первых, отношения с друзьями будут прекрасные или друзья будут высокодуховными людьми, во--вторых, заработки будут хорошие и все дела, связанные с одинадцатым домом, будут процветать).

Если в картах рождения и навамса в одном и том же доме находится неблагоприятная планета, то дела этого дома придут в упадок (например, в карте рождения Сатурн расположен в Скорпионе в третьем доме, а в карте навамса во Льве в третьем доме, значит, идеи третьего дома ухудшатся, то есть отношения с братьями и сестрами будут напряженными).

Таки образом, научившись пользоваться картой навамса, вы получите дополнительные сведения не только о супружеских взаимоотношениях, но и о других областях жизни.

\section{Порядок интерпретации гороскопа с использованием трех карт}

Для того, чтобы правильно интерпретировать гороскоп, надо прежде всего определить силу каждой из девяти планет, рассматривая их с учетом всех правил, которые были даны в предыдущих главах:
\begin{myenum}
	\item Определить достоинство планеты в знаке Зодиака.
	\item Определить расположение планеты в домах квадранта, тригона и неблагоприятных домах(6,\,8,\,12) в карте рождения.
	\item Оценить планету с точки зрения естественно благоприятной или естественно неблагоприятной.
	\item Отметить связь планеты в одном знаке с благотворными или неблаготворными планетами.
	\item Рассмотреть планету с точки зрения благоприятных или неблагоприятных аспектов.
	\item Обратить внимание на сгоревшие, ретроградные, находящиеся в первом и в последнем градусе знака, в позициях от \gradus{12} до \gradus{18}, а также в последних \gradus{6} мужских знаков и в первых \gradus{6} женских знаков планеты.
	\item Оценить планету с точки зрения ее диспозитора (сильный диспозитор усиливает рассматриваемую планету, а слабый --- ослабляет), например: Луна находится в Овне, хозяином которого является Марс, а Марс находится в Козероге, т.\,\е. в знаке экзальтации. В данном примере Марс является диспозитором Луны и будет ее усиливать.
	\item Отметить расположение планеты в домах квадранта, тригона и неблагоприятных домах в лунной карте.
	\item Рассмотреть главные достоинства планеты в карте навамса (в экзальтации, собственном и ослабленном знаках).
	\item Обратить внимание на расположение планеты в одном и том же знаке в карте рождения и карте навамса.
\end{myenum}


После того, как вы оцените силу каждой планеты исходя из десяти пунктов, перейдите к интерпретации гороскопа.

Если планета положительно оценивается по наибольшему количеству пунктов, то она рассматривается в гороскопе как благоприятная (даже если она является естественно неблагоприятной). Если планета отрицательно оценивается по наибольшему количеству пунктов, то она рассматривается в гороскопе как неблагоприятная (даже будучи благотворной по своей природе). Если планета оценивается как положительная и отрицательная в равной мере, то она рассматривается в гороскопе как нейтральная, то есть производит позитивные и негативные эффекты.

Известно, что двенадцать домов гороскопа дают знания о всех сферах жизни человека, но нет такого гороскопа, где бы все эти дома рассматривались положительно. Например, у человека прекрасная карьера, хорошие дети, комфортная жизнь, он пользуется всеобщим уважением, но несчастлив в супружеской жизни. Другой пример: человек может испытывать затруднения в работе, получать небольшую зарплату, переживать семейные конфликты и вместе с тем обладать отличным здоровьем, иметь большую продолжительность жизни. Все это вы можете увидеть в индивидуальном гороскопе.

Чтобы определить уровень какой--либо жизненной сферы, необходимо сделать анализ каждого дома в следующем порядке:

\begin{myenum}
	\item Рассмотреть планету в доме.
	\item Обратить внимание на эту планету как на хозяина дома или домов (Солнце и Луна являются хозяевами только одного дома).
	\item Оценить хозяина рассматриваемого дома (напрмер, если седьмой дом находится в Скорпионе, то рассмотреть положение Марса).
	\item Проанализировать дом с точки зрения аспектной связи (например, если пятый дом под аспектом Юпитера, то учесть благотворное влияние этой планеты).
	\item Обратить внимание на окружение дома благотворными и неблаготворными планетами (например, в шестом доме Марс, а в восьмом доме Раху, значит, седьмой дома окружен неблаготворными плаентами).
	\item Проанализировать состояние планеты, идеи которой тождественны идеям рассматриваемого дома (например, если анализируем первый дом, то параллельно надо рассматривать Солнце, если второй дом --- Юпитер, третий дом --- Марс, четвертый дом --- Луну, пятый дом --- Юпитер, шестой дом --- Марс, седьмой дом --- Венеру, восьмой дом --- Сатурн, девятый дом --- Юпитер, десятый дом --- Солнце, одиннадцатый дом --- Юпитер, двенадцатый дом --- Сатурн).
	\item С учетом выше изложенного рассмотреть дома в лунной карте.
	\item Рассмотреть дополнительный дом от планеты, идеи которой соответствуют идеям анализируемого дома (например, Юпитер и пятый дом отвечают за детей, поэтому надо рассмотреть пятый дом от Юпитера. Другой пример: Луна и четвертый дом отвечают за мать, значит, надо рассмотреть четвертый дом от Луны).
\end{myenum}

Используя предложенную методику определения достоинств планет и домов гороскопа, вы сможете глубоко понять как самого человека, так и его судьбу. Вначале это будет сделать нелегко из--за кажущегося противоречия перечисленных для анализа пунктов. Поэтому вы должны в первую очередь обратить внимание на те факты, которые не один раз подтверждаются в гороскопе. Чем больше одни и те же факты указывают на силу или слабость каких--либо домов и планет, там явственнее их идеи будут выступать в жизни человека как позитивные или негативные. Покажем это на примере гороскопа женщины:

\begin{table}[tph!]
	% Расширить по вертикали
	%\renewcommand{\arraystretch}{1.5}

	% Заполним данными
	\begin{tabular}{|lll|}
		\hline
		Асцендент & \cord{8}{38} & Тельца \\
		Солнце   & \cord{1}{40}  & Скорпиона \\
		Луна     & \cord{28}{55} & Водолея \\
		Марс     & \cord{9}{19}  & Скорпиона \\
		Меркурий & \cord{15}{56} & Весов \\
		Юпитер   & \cord{8}{29}  & Козерога \\
		Венера   & \cord{14}{38} & Весов \\
		Сатурн   & \cord{1}{56}  & Козерога \\
		Раху     & \cord{29}{15} & Рака \\
		Кету     & \cord{29}{15} & Козерога \\ \hline
	\end{tabular}
\end{table}

\begin{table}
	\caption{Карта рождения}
	\natal[asc=2,three=РАХУ,six=ВЕНЕРА\\МЕРКУРИЙ,seven=СОЛНЦЕ\\МАРС,nine=КЕТУ\\ЮПИТЕР\\САТУРН,ten=ЛУНА]{}
\end{table}

\begin{table}
	\caption{Лунная карта}
	\natal[asc=11,one=ЛУНА,six=РАХУ,nine=МЕРКУРИЙ\\ВЕНЕРА,ten=МАРС\\СОЛНЦЕ,twelve=САТУРН\\ЮПИТЕР\\КЕТУ]{}
\end{table}

\begin{table}
	\caption{Карта навамса}
	\natal[asc=12,one=РАХУ\\ЮПИТЕР,four=ЛУНА,five=СОЛНЦЕ,seven=МАРС\\КЕТУ,ten=САТУРН,twelve=МЕРКУРИЙ\\ВЕНЕРА]{}
\end{table}

\subsubsection*{Определение силы планет}

Солнце --- планета в нейтральном знаке, естественно неблагоприятная, расположена в неблагоприятном промежутке от \gradus{0} до \gradus{6} женского знака, в квадранте карты рождения, в квадранте лунной карты, связана с неблагоприятным по своей природе Марсом, находится под аспектом неблагоприятной по своей природе Раху, диспозитор --- Марс (Марс в собственном знаке и в квадранте от восходящего знака и от Луны является сильным).

Луна --- планета в дружественном знаке, естественно благоприятная (растущая Луна), расположена в неблагоприятном промежутке от \gradus{24} до \gradus{30} мужского знака, в квадранте карты рождения (расположение Луны в первом доме лунной карты не берется в расчет, поскольку во всех картах она будет иметь такую же позицию), находится под аспектом естественно неблагоприятного Марса, диспозитор --- Сатурн.

Марс --- планет в собственном знаке, естественно неблагоприятная, в квадранте карты рождения, в квадранте лунной карты, связана с неблагоприятным по своей природе Солнцем, находится под аспектом естественно неблагоприятной Раху, диспозитора не имеет, так как стоит в собственном знаке.

Меркурий --- планета в нейтральном знаке, естественно благоприятная, находится в благоприятном промежутке от \gradus{12} до \gradus{18} знака, расположена в неблагоприятном шестом доме карты рождения, находится в тригоне лунной карты, связан с Венерой, имеет аспект от неблагоприятного Сатурна, диспозитор --- Венера.

Юпитер --- планета в ослабленном знаке, естественно благоприятная, расположена в тригоне карты рождения, находится в неблагоприятном двенадцатом доме лунной карты, связана с естественно неблагоприятными Сатурном и Кету, испытывает аспект естественно неблагоприятной Раху, в карте навамса расположена в собственно мзнаке, диспозитор --- Сатурн, зажат Сатурном и Кету.

Венера --- планета в собственном знаке, естественно благоприятная, находится в благоприятном промежутке от от \gradus{12} до \gradus{18} знака, расположена неблагоприятном шестом доме карты рождения, находится в тригоне лунной карты, связана с Меркурием, испытывает аспект Сатурна, диспозитора не имеет, так как стоит в собственном знаке.


Сатурн --- планета в собственном знаке, естественно неблагоприятная, находится в неблагоприятном промежутке от \gradus{0} до \gradus{6} женского знака, расположена в тригоне карты рождения, в неблагоприятном двенадцатом доме лунной карты, связана с естественно благоприятным Юпитером и неблагоприятной Кету, стоит под аспектом естественно неблагоприятной Раху, в карте навамса пребывает в собственном знаке, диспозитора не имеет, так как расположена в собсвтенном знаке.

Раху --- по знаку планета силы не имеет, естественно неблагоприятная, находится в последнем градусе знака, что является неблагоприятным для нее, расположена в третьем доме карты рождения, в шестом доме лунной карты, испытывает аспект естественно благоприятного Юпитера и неблагоприятного Сатурна (аспект Кету не учитывается, так как Кету всегда расположена противоположно Раху), диспозитор --- Луна.

Кету --- по знаку планета силы не имеет, естественно неблагоприятная, находится в последнем градусе знака, расположена в тригоне карты рождения, в неблагоприятном двенадцатом доме лунной карты, связана с естественно благоприятным Юпитером и неблагоприятным Сатурном, диспозитор --- Сатурн.

На основании данного анализа планет очень легко определить, какие из них --- сильные, нейтральные и слабые.

Солнце имеет позитивные и негативные результаты анализа, но так как оно находится в квадранте в двух картах, то будет обладать силой. Поэтому сила Солнца будет превышать среднюю силу планеты.

Луна имеет больше положительных признаков и поэтому является достаточно сильной в гороскопе.

Марс, как и Луна, также сильная планета.

Меркурий имеет нейтральную силу.

Юпитер --- самая слабая планета этого гороскопа.

Венера по многим признакам благоприятна и поэтому является сильной планетой.

Сатурн также обладает силой.

Раху присуще больше неблагоприятных признаков, и поэтому она не имеет благотворной силы.

Кету сильнее, чем Раху, так как находится в благоприятном девятом доме и Юпитер ближе, чем Сатурн, расположен к ней.

Сильные планеты окажут благотворное воздействие в тех областях жизни, которые находятся под их управлением. Ослабленные планеты принесут отрицательные результаты.

Теперь сделаем анализ каждого дома и определим, как они будут  себя проявлять в жизни этой женщины.

\subsubsection*{Первый дом}
Восходят Телец и созвездие Криттика. Венера как хозяйка Тельца находится в Весах в шестом доме вместе с Меркурием и под аспектом Сатурна (надо учесть, что Меркурий является хозяином второго и пятого домов, а Сатурн --- хозяином девятого и десятого домов). Первый дом находится под аспектами Солнца, Марса, Юпитера и Кету. Солнце, как и первый дом, представляет личность и характер человека, поэтому рассмотрим его со всех точек зрения. Обратим внимание на знак, занимаемый Луной. Это --- Водолей, который является вторым восходящим знаком и находится под воздействием Марса.

Выводы: эта женщина имеет привлекательную внешность и красивую фигуру (восходящие Телец и Криттика, Телец под влиянием Юпитера и Солнца). Она упряма, но терпелива (восходящий Телец), способна выдерживать большие физические нагрузки (Марс и Солнце аспектируют первый дом). Имеет хорошее здоровье и жизнеспособность (хозяин первого дома, испытывает влияние хозяев девятого и десятого домов). Склонна к возражениям и критике (Солнце в Скорпионе под влиянием Марса).

\subsubsection*{Второй дом}
В этом доме нет планет, но он находится под аспектами Марса и Раху. Марс оказывает влияние на второй дом, как хозяин седьмого и двенадцатого домов. Второй дом расположен в Близнецах, которыми правит Меркурий, находящийся в неблагоприятном шестом доме. Хозяин второго дома испытывает влияние Венеры и Сатурна. Юпитер, как и второй дом, представляет богатство и деньги, он находится в ослабленном положении. В лунной карте второй дом стоит в Рыбах, хозяином которых является Юпитер, расположенный в двенадцатом доме между Сатурном и Кету. Второй дом в лунной карте находится под влиянием Раху и Сатурна. Во втором доме от Юпитера --- Луна под аспектом Марса.

Выводы: этот дом ослаблен. Женщина будет испытывать материальные трудности, страдать зубной болью, устраивать конфликты в семье. Но, так как хозяин второго дома связан с сильной Венерой, это дает ей шанс в определенные периоды жизни улучшить свое материальное положение.

\subsubsection*{Третий дом}
Раху стоит в третьем доме. Луна является хозяйкой третьего дома и расположена в десятом доме под аспектом огненного Марса. Третий дом находится под аспектом Сатурна, Юпитера и Кету. Идеи третьего дома схожи с идеями Марса, поэтому проанализируем положение последнего. Марс хорошо расположен в гороскопе, он достаточно силен. В лунной карте третий до находится в Овне, а его хозяином является Марс, который знанимает десятый дом. Венера и Меркурий в лунной карте аспектируют третий дом. В третьем доме от Марса находятся Сатурн, Юпитер, Кету, а Раху аспектирует его.

Выводы: третий дом представляет жизнестойкость, мужество, решительность, руки, братьев, сестер и\,т.\,д. Из анализа третьего дома мы видим, что он в основном имеет связь с естественно неблагопрятными планетами, которые являются стимулом для развития решительности, жизнестойкости. Этот дом указывает н ато, что в своей жизни женщина много будет заниматься ручным трудом, и ее сила воли с каждым годом будет укрепляться. Родных братьев и сестер она не имеет, т.\,к. третий дом в основном имеет жесткие энергии.

\subsubsection*{Четвертый дом}
Расположен во Льве, хозяином которого является Солнце, находящееся в благоприятном седьмом доме. Четвертый дом испытывает влияние от Луны. Эта планета, как и четвертый дом, представляет мать, поэтому сделаем ее анализ. В лунной карте четвертый дом находится в Тельце, хозяином которого является Венера, расположенная в благоприятном девятом доме. Четвертый лунный дом находится под аспектами Солнца, Марса, Юпитера и Кету.

Выводы: в основном четвертый дом состоит из положительных признаков, которые являются указанием на хорошую мать и дружеские отношения с ней. Владелица гороскопа имеет в своем распоряжении значительной недвижимое имущество и хороших родственников.

\subsubsection*{Пятый дом}
Этот дом расположен в Деве, хозяином которого являетс Меркурий, стоящий в шестом доме рядом с Венерой и под аспектом Сатурна. Юпитер и Кету оказывают влияние на пятый дом. Юпитер как планета, управляющая детьми, зажат неблагоприятным Сатурном и Кету и испытывает на себе влияние Раху. В лунной карте пятый дом расположен в Близнецах, хозяином которого является Меркурий, находящийся в благоприятном девятом доме вместе с Венерой и под аспектом Сатурна. Пятый дом в лунной карте находится под аспектом Марса и Раху. Рассмотрим пятый дом от Юпитера, который расположен в Тельце и находится под влиянием Солнца, Марса, Кету и Юпитера. Хозяин пятого дома от Юпитера расположен в десятом доме вместе с Меркурием и находится под аспектом Сатурна.

\subsubsection*{Шестой дом}
Здесь расположены Венера и Меркурий. Первыя является хозяйкой первого и шестого домов, а второй --- хозяином второго и пятого домов. Венера усиливает шестой дом, поскольку его хозяйка находится в нем. Сильный Сатурн влияет на шестой дом. Рассмотрим Марс, так как он является планетой споров и конфликтов. В лунной карте в шестом доме расположена Раху под аспектами Сатурна и Юпитера, а хозяин шестого дома находится в первом доме. Обратим внимание на шестой дом от Марса, который стоит в Овне под влиянием Венеры и Меркурия и хозяин которого расположен в первом доме вместе с Солнцем и под влиянием Раху.

Выводы: в данном гороскопе этот дом очень сильный и указывает на высокий служебный долг женщины и на ее большие возможности в победе над недоброжелателями.

\subsubsection*{Седьмой дом}
Две огненные планеты --- Солнце и Марс расположены в седьмом доме. Солнце является хозяином четвертого дома, а Марс --- седьмого и двенадцатого. Марс активизирует идеи седьмого дома, так как является его хозяином и находится в нем. Раху влияет на седьмой дом. Венера, которая отражает главные идеи седьмого дома, расположена достаточно хорошо. В лунной карте хозяин седьмого дома находится в десятом доме вместе с Марсом и под аспектом Раху.

Выводы: этот дом очень активный и имеет жесткие энергии, поэтому супружеская жизнь женщины будет не лишена споров и конфликтов. В общественной сфере она будет пользоваться вниманием, что будет приносить ей радость.


\subsubsection*{Восьмой дом}
Он находится в Стрельце и не имеет планет. Хозяином восьмого дома является Юпитер, который расположен в девятом доме между Сатурном и Кету. Только Кету влияет на восьмой дом. Он окружен такими неблагоприятными планетами, как Сатурн, Кету, Марс и Солнце. Но планета Сатурн, которая отвечает за долголетие, достаточно сильна и имеет связь с благоприятным по природе Юпитером. Юпитер по долготе ближе, чем Кету и Раху, находится к Сатурну. В лунной карте хозяин восьмого дома стоит в девятом доме и дублирует позицию хозяина восьмого дома в карте рождения. Восьмой дом в лунной карте испытывает влияние Кету и Юпитера. Рассмотрим также восьмой дом от Сатурна, который аспектирует растущая Луна и хозяин которого находится в одинадцатом доме.

Выводы: если рассматривать восьмой дом с точки зрения долголетия, то сильный Сатурн свидетельствует о хорошей продолжительности жизни. Юпитер, который ослаблен и является хозяином восьмого дома, говорит о том, что женщине предстоит пережить смерть других людей, в том числе и ее родственников.

\subsubsection*{Девятый дом}
Сатурн, Юпитер и Кету занимают девятый дом. Сатурн является хозяином девятого и десятого домов, а Юпитер --- восьмого и одинадцатого. Сатурн усиливает девятый дом, он хозяин этого дома и находится в нем. В лунной карте в девятом доме расположены Венера и Меркурий, которые находятся под аспектом Сатурна. Венера, как хозяйка девятого дома, находится в нем и улучшает его. Рассмотрите девятый дом от Юпитера и узнаете дополнительно о духовной жизни человека. Оцените девятый дом от Солнца и вы получите дополнительную информацию об отце.

Выводы: идеи этого дома будут ярко проявляться в жизни женщины. Сильный девятый дом обещает долгую жизнь и ее отцу. Так как благотворный Юпитер, руководящий духовной жизнью, ослаблен, а неблагопритяный Сатурн, который также отвечает за духовность, усилен и является хозяином девятого дома, то в жизни этой женщины будет много противоречий в отношении ее духовного развития.

\subsubsection*{Десятый дом}
Луна находится в десятом доме под влиянием Марса и является хозяйкой третьего дома. Хозяин десятого дома расположен в девятом доме и испытывает влияние Юпитера, Раху, Кету. Десятый дом, как и девятый, руководит отцом, поэтому надо рассмотреть положение Солнца. В лунной карте в десятом доме находятся Солнце и Марс. Солнце является хозяином седьмого дома, а Раху аспектирует десятый дом. Рассмотрим десятый дом от СолнцаЖ он находится во Льве под аспектом Луны, а его хозяин расположен в первом доме вместе с Марсом и под аспектом Раху.

Выводы: в гороскопе этой женщины благоприятный десятый дом указывает на удачу в работе, а также на хорошего и внимательного отца.

\subsubsection*{Одинадцатый дом}
В одинадцатом доме нет планет, но хозяин этого дома --- Юпитер расположен в девятом доме между Сатурном и Кету. Юпитер --- самая слабая планета в данном гороскопе. Сатур и Раху влияют на одинадцатый дом. Юпитер --- хозяин второго дома, в лунной карте находится в двенадцатом доме. Так как эта планета управляет богатством и приобретениями, то рассмотрим одинадцатый дом от нее, в которо находятся Солнце и Марс под влиянием Раху.

Выводы: этот дом ослаблен, и поэтому у женщины нет никакой возможности хорошо зарабатывать ни на своей работе, ни на какой-нибудь другой производственной или непроизводственной сфере. Одинадцатый дом также указывает на взаимоотношения с друзьями, а поскольку он слабый, женщина не будет испытывать особой привязанности к ним.

\subsubsection*{Двенадцатый дом}
Здесь нет планет, но хозяин дома расположен в седьмом доме вместе с Солнцем и находится под аспектом Раху. Венера и Меркурий аспектируют двенадцатый дом. Идеи двенадцатого дома схожи с идеями Сатурна, поэтому проанализируем положение Сатурна. В лунной карте в двенадцатом доме стоят Сатурн, Юпитер и Кету. Сатурн в лунной карте является хозяином двенадцатого дома и находится в нем. Двенадцатый дом лунной карты испытывает на себе влияние Раху. Двенадцатый дом от Сатурна не имеет планет, но его хозяин (Юпитер) ослаблен и занимает первый дом.

Выводы: так как хозяином двенадцатого дома является Марс, то он приведет к материальным потерям. Будучи расположенным в седьмом доме он указывает на то, что женщина будет тратить деньги на своего мужа. Ослабленный Юпитер в двенадцатом доме лунной карты свидетельствует об обременительных расходах, связанных с детьми. Юпитер в женском гороскопе представляет мужа и поэтому еще раз подтверждает, что деньги будут уходить на него. Венера, которая влияет на двенадцатый дом карты рождения, в третий раз подтверждает расходы на супруга.

На примере гороскопа женщины мы кратко описали жизненные сферы, не заостряя внимание на деталях. Когда вы поймете, как делаются главные выводы из гороскопа, вы сможете приблизиться к изучению деталей и расширить со временем свои знания в области индийской астрологии.

Итак, мы рассмотрели судьбу женщины в общих чертах, без учета времени событий. Но, то, что показывает гороскоп, должно проявиться как результат кармической реакции в определенный период жизни, и об этом мы расскажем в следующих главах
Здесь нет планет, но хозяин дома расположен в седьмом доме вместе с Солнцем и находится под аспектом Раху. Венера и Меркурий аспектируют двенадцатый дом. Идеи двенадцатого дома схожи с идеями Сатурна, поэтому проанализируем положение Сатурна. В лунной карте в двенадцатом доме стоят Сатурн, Юпитер и Кету. Сатурн в лунной карте является хозяином двенадцатого дома и находится в нем. Двенадцатый дом лунной карты испытывает на себе влияние Раху. Двенадцатый дом от Сатурна не имеет планет, но его хозяин (Юпитер) ослаблен и занимает первый дом.

Выводы: так как хозяином двенадцатого дома является Марс, то он приведет к материальным потерям. Будучи расположенным в седьмом доме он указывает на то, что женщина будет тратить деньги на своего мужа. Ослабленный Юпитер в двенадцатом доме лунной карты свидетельствует об обременительных расходах, связанных с детьми. Юпитер в женском гороскопе представляет мужа и поэтому еще раз подтверждает, что деньги будут уходить на него. Венера, которая влияет на двенадцатый дом карты рождения, в третий раз подтверждает расходы на супруга.

На примере гороскопа женщины мы кратко описали жизненные сферы, не заостряя внимание на деталях. Когда вы поймете, как делаются главные выводы из гороскопа, вы сможете приблизиться к изучению деталей и расширить со временем свои знания в области индийской астрологии.

Итак, мы рассмотрели судьбу женщины в общих чертах, без учета времени событий. Но, то, что показывает гороскоп, должно проявиться как результат кармической реакции в определенный период жизни, и об этом мы расскажем в следующих главах.




\chapter{МЕТОДИКА ПРЕДСКАЗАНИЯ}

\section{Анализ главного планетного периода}

События, происходящие в жизни человека, можно рассматривать как звенья одной непрерывной цепи, которые выражаются во времени. Каждый период во времени играет особую роль в эволюции сознания человека.

Среди индийских астрологов бытует мнение, что предсказательная система базируется на теори 120-летней продолжительности человеческой жизни. Это утверждение, как говорят некоторое знатоки индийской астрологии, по крайней мерен, некорректно. В основе предсказательной системы лежит 120-летний цикл эволюции жизненного опыта человека. Это количество лет получено из постоянной величины накшатр, равных \cord{13}{20}, и девяти планет, т.\,е. \cord{13}{20} * 9 = 120.

Жизнь человека делится на главные периоды, подпериоды, подподпериоды и на еще более маленькие периоды, которые находятся под управлением различных планет (расчеты главных периодов и подпериодов даны в главе 1). Каждая планета в гороскопе обладает силой, она особенно ярко проявляется в главный период этой планеты и оказывает влияние на жизнь человека. После общего анализа гороскопа вы должны приступить к интерпретации планеты, которая руководит главным периодом.

Если планета главного периода находится в хорошей позиции, то она оказывает благоприятный эффект в определенное время в тех областях жизни, которыми управляет и на которые влияет. Например, рассмотрим главный период Сатурна, который находится в Козероге в четвертом доме и испытывает влияние Венеры из Рака и Юпитера из Девы. В данном примере Сатурн расположен в квадранте в собственном знаке, является хозяином четвертого и пятого домов и находится под аспектом естественно благоприятных Венеры и Юпитера. Отсюда следует, что Сатурн обладает благоприятными силами, а потому у человека есть вероятность получить хорошую квартиру, дачный участок, автомобиль, а дети принесут счастье.

Если планета главного периода занимает плохую позицию, то она оказывает неблагоприятный эффект в определенное время в тех областях жизни, которыми управляет и на которые влияет. Например, рассмотрим главный период Марса, который находится в Раке в восьмом доме вместе с Сатурном. Поскольку Марс расположен в ослабленном знаке, в плохом доме, является хозяином двенадцатого дома и испытывает влияние естественно неблагоприятного Сатурна, его семилетний период принесет человеку большие страдания, несчастья, унижения, материальные потери, ухудшит условия жизни.

Итак, мы провели два противоположных по своему результату примера, которые наблюдаются в астрологической практике крайне редко. По мнению индийских астрологов, нет ни одного абсолютно благоприятного или абсолютно неблагоприятного главного планетного периода, так как подпериоды будут указывать на различные варианты воздействия тех или иных эффектов в определенное время жизни человека. В своей практике астрологам приходится в основном сталкиваться с главными планетными периодами, дающими смешанный результат. Начинающим оценка планеты главного периода может показаться слишком противоречивой, но на самом деле никакого противоречия здесь нет и все, что связано с этой планетой, должно проявиться в ее главный период. Например, планета главного периода стоит в ослабленном знаке, но в хорошем доме (1, 4, 5, 7, 9 или 10-м доме), значит, благотворный эффект планеты в доме не будет выраженным. Другой пример: планета главного периода находится в плохо доме (6, 8 или 12-м доме), но в знаке экзальтации, поэтому ее неблагоприятный эффект в доме будет явно уменьшен, и в конечном итоге все в жизни человека будет зависеть от дополнительных влияний благоприятных или неблагоприятных планет.

Таким образом, эффект планеты главного периода проявится:

\begin{myitem}
	\item в доме, который она занимает.
	\item в доме, хозяином которого она является
	\item в знаке, в котором она находится
	\item в доме, который она аспектирует
	\item в естественной природе самой планеты
	\item в связи с планетами в одном доме
	\item в связи с планетами через аспект
	\item в лунной карте: в доме, в котором находится эта планета, в доме, зозяином которого она является, и в доме, который она аспектирует.
\end{myitem}

Существует большое количество книг по индийской астрологии, в которых есть главы с описанием эффектов планет главного периода, но, так как нет одинаковых гороскопов, то и предсказания не могут быть одинаковыми, давать общие шаблонные описания действий планет, руководящих главным периодом, нет смысла.

Надо отметить и такой факт, что если естественно неблагоприятная планета главного периода является одновременно хозяином квадранта и тригона, то она становится благотворной в течение данного периода и принесет в жизнь человека хорошие события. Если эта планета расположена в доме квадранта или тригона, то ее добрая сила явно возрастает. Например, при восходящем Раке Марс находится в Овне в десятом доме. Другой пример: при восходящих Весах Сатурн находится в Козероге в четвертом доме.

Естественно благоприятная планета главного периода, если она является одновременно хозяином квадранта и тригона, становится очень благоприятной. Например, при восходящем Водолее Венера является хозяином четвертого и девятого домов.

Естественно благоприятная планета главного периода, если она является хозяином шестого, восьмого или двенадцатого дома, проявит в течение этого периода негативный эффект как хозяин неблагоприятного дома. Например, при восходящем Раке Меркурий является хозяином двенадцатого дома.

Естественно неблагоприятная планета главного периода, если она является хозяином шестого, восьмого или двенадцатого дома, окажет в течение этого периода очень негативный эффект на жизнь человека. Например, при восходящей Деве Марс является хозяином восьмого дома.

Если с учетом выше приведенных правил планета главного периода находится в хорошем знаке и доме, то ее добрая сила возрастет. Если планета главного периода находится в плохом знаке и доме, то ее отрицательный эффект увеличится.

Если естественно неблагоприятная планета главного периода является хозяином хороших домов, то она не нанесет большого вреда тому дому, который она занимает. Например, при восходящем Раке Марс является хозяином пятого и десятого домов и расположен в третьем доме. В данном примере третий дом серьезно страдать не будет.

Каждая планета в соответствии со своей естественной природой создает определенные эффекты в течение главного периода. Если эта планета в результате анализа окажется сильной, то она принесет благотворные результаты в соответствии со своей планетной природой. Например, если в гороскопе Солнце определено как сильная планета, то в его главный период будут процветать дела отца, наладятся хорошие отношения с людьми, представляющими власть, человек будет заниматься самосовершенствованием, успешными будут путешествия в восточном направлении и\,т.\,д.

Если планета главного периода в результате анализа окажется слабой, то она принесет неблагоприятные результаты в соответствии со своей планетной природой. Например, если в гороскопе Солнце определено как слабая планета, то в его главный период дела отца будут в упадке, здоровье пошатнется и даже возникнет угроза смерти, будут складываться плохие отношения с начальством или представителями власти, человек не будет верно оценивать свои поступки и путешествия на восток не принесут ему пользы.

Ниже мы представляем гороскопы с анализом главного планетного периода.

\subsubsection*{Гороскоп мужчины}

\planets[%
	asc=\signum{4}{11}{\scorpio},
	su=\signum{0}{48}{\leo},
	mo=\signum{10}{58}{\pisces},
	ma=\signum{4}{46}{\sagittarius}, 
	me=\signum{26}{13}{\cancer},
	ju=\signum{25}{37}{\gemini},
	ve=\signum{15}{54}{\virgo}, 
	sa=\signum{10}{48}{\libra},
	ra=\signum{20}{52}{\sagittarius},
	ke=\signum{20}{52}{\gemini},
]{}

\natal[%
	asc=8,
	two=МАРС\\РАХУ,
	five=ЛУНА,
	eight=КЕТУ\\ЮПИТЕР,
	nine=МЕРКУРИЙ,
	ten=СОЛНЦЕ,
	eleven=ВЕНЕРА,
	twelve=САТУРН
]{}

Рассмотрим главные период Меркурия, который будет длиться семнадцать лет.

\begin{myenum}[itemsep=0,parsep=0]
	\item Меркурий расположен в девятом доме карты рождения.
	\item Меркурий является хозяином восьмого и одинадцатого домов в карте рождения.
	\item Меркурий находится в знаке большого врага (Рак).
	\item Меркурий аспектирует третий дом.
	\item Меркурий в данной позиции является естественно благоприятной планетой.
	\item Меркурий связи с планетами в одном доме не имеет.
	\item Меркурий находится под аспектом естественно неблагоприятных Марса и Сатурна.
	\item Меркурий расположен в пятом доме лунной карты.
	\item Меркурий является хозяином четвертого и седьмого домов в лунной карте.
\end{myenum}

Главный период Меркурия в данном гороскопе охватывает возраст с восьми до двадцатипяти лет. Этот период не будет сильным, так как Меркурий расположен в знаке большого врага, является хозяином восьмого дома и находится под аспектом неблаготворных Марса и Сатурна. Положение Меркурия в девятом доме благоприятно, но этого недостаточно для погашения отрицательных эффектов, поэтому не следует ожидать большой удачи в главный период этой планеты. Но в лунной карте Меркурий расположен в пятом доме и является хозяином квадрантов, значит он в лунной карте будет способствовать успешной учебе и обеспечит хорошие результаты, связанные с вопросами образования. Меркурий в лунной карте --- хозяин седьмого дома (любовные и брачные дела), что дает все основания предполагать возможность заключения брака. Так как Меркурий в карте рождения является хозяином восьмого дома (критическая ситуация, травмы, трансформация) и находится под аспектом злого Марса (травмы, порезы, физическая боль), то вполне можно ожидать хирургического вмешательства или травму в главный период планеты. То, что Меркурий испытывает аспект Сатурна, указывает на взаимоотношения с нехорошими людьми и уменьшение материальных возможностей. Меркурий в карте рождения является хозяином одинадцатого дома (друзья, наджды), значит в главный планетный период человек будет иметь друзей и надежды не покинут его сердце. В лунной карте Меркурий становится хозяином четвертого дома (мать, друзья, родственники, недвижимость), что подтверждает вышеописанное и свидетельствует о наличии скромного жилья и тесных взаимоотношениях с матерью и родственниками. Меркурий в карте рождения аспектирует третий дом (информация, литература), а в лунной --- одинадцатый дом (друзья, доходы), поэтому на протяжении его главного периода человек будет увлекаться литературой, оказывать влияние на друзей и иметь средние доходы.

Из анализа главного периода Меркурия видно, что основной эффект этой планеты будет проявлен в делах образования и во взаимоотношениях с друзьями.

\subsubsection*{Гороскоп женщины}

\planets[%
	asc=\signum{17}{37}{\scorpio},
	su=\signum{17}{58}{\pisces},
	mo=\signum{4}{07}{\capricornus},
	me=\signum{11}{55}{\pisces}\,(ретроградный),
	ma=\signum{1}{54}{\gemini},
	ju=\signum{8}{24}{\scorpio},
	ve=\signum{21}{01}{\aries},
	sa=\signum{13}{36}{\sagittarius},
	ra=\signum{19}{48}{\virgo},
	ke=\signum{19}{48}{\pisces}
]{}

\natal[%
	asc=8,
	one=ЮПИТЕР,
	two=САТУРН,
	three=ЛУНА,
	five=СОЛНЦЕ\\меркурий\\КЕТУ,
	six=ВЕНЕРА,
	eight=МАРС,
	eleven=РАХУ
]{}

Рассмотрим главный период Луны, который длится десять лет:

\begin{myenum}[parsep=0,itemsep=0]
	\item Луна занимает третий дом.
	\item Луна является хозяином девятого дома.
	\item Луна находится в знаке друга (Козерог).
	\item Луна аспектирует девятый дом.
	\item Луна убывающая, значит, она является естественно неблагоприятной планетой.
	\item Луна не связана ни с одной планетой в том доме, в котором она находится.
	\item Луна находится под аспектом Марса и Раху.
	\item Луна не рассматривается в лунной карте как планета, находящаяся в первом доме (во всех лунных картах Луна будет занимать первый дом).
	\item Луна в лунной карте является хозяином седьмого дома и одновременно его аспектирует.
\end{end}

Главный период Луны в данном гороскопе охватывает возраст девочки с двух до двенадцати лет. Планета имеет средние силы и в свой главный период принесет владелице гороскопа хорошие и плохие результаты. Луна в третьем доме указывает на частые непродолжительные поездки, отношения с братьями и сестрами. Поскольку Луна хозяин девятого дома и одновременно его аспектирует, это свидетельствует о дальних путешествиях и удаче. Подтверждением этого факта является то, что Луна находится в подвижном знаке (Козероге). Эта планета по своей природе представляет мать, женщину, места купания, общественную среду, и поэтому ее идеи будут ярко проявлены в десятилетний период жизни. Луна находится под воздействием Марса (порезы, операция, ушиб) и Раху (инфекционные болезни), значит, в главный ее период следует ожидать хирургической операции, травмы, инфекционных заболеваний. В лунной карте Луна является хозяином седьмого дома и одновременно его аспектирует, а так как седьмой дом представляет общественную жизнь, то этот ребенок в школьные годы будет заниматься общественными делами.

Итак, все, с чем связана Луна в гороскопе, будет проявляено в течение десяти лет.

По двум вышеприведенным гороскопам мы сделали основной анализ эффектов планет главного периода, но они не являются исчерпывающими, так как подпериоды укажут на дополнительные варианты событий в судьбе человека.

\section[Анализ планетного подпериода]{Анализ планетного подпериода в основе главного планетного периода}

Каждый главный период имеет девять подпериодов и определяется при помощи таблицы~\ref{tbl:stargroups}. Принципы и правила, которые применяются для определения качества планет главного периода, остаются теми же и для определения качества планеты подпериода. Если планета главного периода оценивается как сильная, то планета подпериода будет тоже сильной. Все зависит от того, в основе какого главного планетного периода будет действовать планетный подпериод. Если планеты главного периода и подпериода являются сильными, то человека ждеть большая радость, счастье, удача и процветание, когда планеты этих периодов вступят в силу. Если планеты главного периода и подпериода слабые, то человек испытывает большие неудачи и потрясения, когда планеты этих периодов вступят в силу. Если планета главного периода сильная, а планета подпериода слабая, то преимущество надо отдать планете главного периода, хотя планета подпериода в свои сроки принесет некоторые неудобства. Если планета главного периода является слабой, а планета подпериода сильной, то последняя не сможет коренным образом изменить судьбу человека к лучшему, хотя и даст ему возможность ощутить позитивные изменения. Если планета подпериода находится в домах квадранта или тригона, считая от планеты главного периода, то подпериод этой планеты улучшится. Если планета подпериода находится в шестом, восьмом или двенадцатом доме, считая от планеты главного периода, то подпериод этой планеты ухудшится. Если планета главного периода враждебна планете подпериода, то подпериод ухудшится. Если планета главного периода дружественна планете подпериода, то подпериод улучшится.

Если планета главного периода является хозяином квадранта, а планета подпериода --- хозяином тригона, то подпериод улучшается. То же произойдет, если планета главного периода является хозяином тригона, а планета подпериода --- хозяином квадранта. Если планета главного периода является хозяином шестого, восьмого или двенадцатого дома, а планета подпериода является хозяином одного из этих же домов, то подпериод ухудшается.

Если планета главного периода естественно благоприятная и планета подпериода также естественно благоприятная, то подпериод улучшается. Если планеты главного периода и подпериода естественно неблагоприятные, то подпериод ухудшается.

Чтобы определить дополнительные эффекты в подпериод какой--либо планеты, надо планету главного периода поставить в первый дом и рассмотреть планету подпериода относительно планеты главного периода, учитывая занимаемый дом и то, хозяином каких домов она является. Например, рассмотри главный период Марса, который расположен в Овне в десятом доме, и подпериод Сатурна, находящегося в Козероге в седьмом доме. Помимо того, что надо рассмотреть положение Марса по десятому дому, а Сатурна по седьмому дому, вы также должны определить положение Сатурна относительно позиции Марса, то есть составить дополнительную карту, где Марс как планета главного периода будет находиться в первом доме, тогда Сатурн относительно марса бедт стоять в десятом доме и являться хозяином десятого и одинадцатого домов.

В предыдущей главе мы сделали анализ главных планетных периодов в двух гороскопах. В этой главе мы рассмотрим их планетные подпериоды, определив природу и время событий. 

\subsubsection*{Гороскоп мужчины}

\natal[%
	asc=8,
	two=МАРС\\РАХУ,
	five=ЛУНА,
	eight=КЕТУ\\ЮПИТЕР,
	nine=МЕРКУРИЙ,
	ten=СОЛНЦЕ,
	eleven=ВЕНЕРА,
	twelve=САТУРН
]{}

Рассмотрим подпериод Меркурия в главном периоде Меркурия (возраст 9--10 лет). В это время школьник станет более серьезно относиться к учебе, и его успеваемость повысится. На это указывает расположение Меркурия в пятом доме лунной карты. Так как Меркурий представляет друзей и в карте рождения является хозяином одинадцатого дома (друзья), то в этот период ребенок будет находиться в тесных контактах со своими друзьями и иметь благоприятный шанс обрести новых друзей. Меркурий в карте рождения находится под влиянием естественно неблагоприятного Марса (травмы, ушибы, порезы, хирургическая операция) и является хозяином восьмого дома (критическая ситуация, травмы, хирургическая операция), поэтому он, возможно, в этот период перенесет хирургическую операцию.

Рассмотрим подпериод Луны в главном периоде Меркурия (возраст 16--17лет). Луна расположена в карте рождения в пятом доме (образование), является хозяином девятого дома (удача, путешествие) и находится под аспектом естественно благоприятной Венеры. В лунной карте Луна является хозяином пятого дома (образование). В главный период Меркурия и подпериод Луны человек поступил учиться в высшее учебное заведение. Луна находится в девятом доме (путешествие, удача) от Меркурия и является хозяином первого дома (тело, личность), поэтому в главный период Меркурия и подпериод Луны человек испытывал удовольствие от путешествий и имел хорошее здоровье.

Рассмотрим подпериод Сатурна в главном периоде Меркурия (возраст 23--25 лет). Сатурн в карте рождения расположен в двенадцатом доме (расходы, потери), а в лунной карте --- в восьмом доме (трансформация).

Сатурн находится под влиянием Кету (препятствия, помехи) и Юпитера (дети). В лунной карте Сатурн аспектирует пятый дом (дети), где располагается Меркурий, значит, в главный период Меркурия и подпериод Сатурна произойдет рождение ребенка, что будет сопровождаться большими расходами и трудностями семейной жизни. Сатурн (смерть) стоит в восьмом доме (смерть), что предполагает утрату кого--либо из родных.

В карте рождения Сатурн находится в двенадцатом доме (утрата) и является хозяином четвертого дома (родственники), что подтверждает вероятность смерти кого--либо из родственников. Об этом свидетельствует также и то, что в лунной карте Сатурн является хозяином двенадцатого дома (утрата, потеря). Если учесть, что Меркурий хозяин восьмого дома (смерть, трансформация), которыей несет в себе силы главного периода, то вполне очевидно, что в подпериод Сатурна должна произойти трансформация, которая будет связана со смертью родственника.

\subsubsection*{Гороскоп женщины}

\natal[%
	asc=8,
	one=ЮПИТЕР,
	two=САТУРН,
	three=ЛУНА,
	five=СОЛНЦЕ\\меркурий\\КЕТУ,
	six=ВЕНЕРА,
	eight=МАРС,
	eleven=РАХУ
]{}

Рассмотрим подпериод Марса в главном периоде Луны (возраст 3 года). Марс расположен в восьмом доме (критическая ситуация, травма, и\,т.\,д.) и является хозяином первого (тело) и шестого (болезнь, боль, травма) домов. По своей природе эта планета неблагоприятна и несет идеи, связанные с разрушением. Марс испытывает влияние от естествнно неблагоприятного Сатурна (горе, неудача). Луна как планета главного периода находится под аспектом естественно неблагоприятного Марса (травма, физическая боль), поэтому в главный период Луны подпериод Марса у девочки была травмирована голова. Марс находится в шестом доме (болезнь) от Луны, что является указанием на преодоление физической боли.

Рассмотрим подпериод Раху в главном периоде Луны (возраст 4--5лет). Раху находится в одинадатом доме (друзья, надежды) в карте рождения и в девятом доме (путешествие) в лунной карте. Раху испытывает влияния от Марса, Сатурна, Солнца и Меркурия. Раху в соответствии со своей природой сначала заключает в себя, а потом освобождает и управляет путешествиями. Луна как планета главного периода находится под аспектом Раху, и поэтому в главный период Луны и подпериод Раху произошел переезд в другой город.

Рассмотрим подпериод Сатурна в главном периоде Луны (возраст 8--9 лет). Сатурн расположен во втором доме (семья) в карте рождения и в двенадцатом доме (упадок, изоляция, больница) в лунной карте. Он находится под влиянием естественно неблагоприятного Марса (физическая боль). В главный период Луны и подпериод Сатурна женщина в детском возрасте болела, но так как Луна является хозяином девятого дома (удача), она быстро поправилась, хотя трудности в семье ощущались (Сатурн во втором доме в карте рождения).

На примерах--гороскопах мы описали метод, с помощью которого вы сможете добиться хороших результатов в предсказании будущих событий. Если вы будете знать вышеприведенные правила напамять, это облегчит не только понимание гороскопа в целом, но также приведет вас к знанию высшего закона Кармы.

\section[Предсказание событий на примерах]{Предсказание различных событий на примерах индивидуальных гороскопов}

В этой главе мы рассмотрим индивидуальные гороскопы с результатами предсказаний в основе главного планетного периода и планетного подпериода. Эта хрестоматийная часть книги поможет закрепить свои знания в искусстве предсказания будущих событий.

\subsubsection*{Гороскоп женщины}

\planets[%
	asc=\signum{13}{54}{\gemini},
	su=\signum{3}{01}{\cancer},
	mo=\signum{10}{57}{\taurus},
	ma=\signum{27}{05}{\aries},
	me=\signum{29}{50}{\gemini}(ретро),
	ju=\signum{2}{05}{\sagittarius}(ретро),
	ve=\signum{10}{17}{\cancer},
	sa=\signum{20}{58}{\sagittarius}(ретро),
	ra=\signum{23}{37}{\leo},
	ke=\signum{23}{37}{\aquarius}
]{}

\natal[%
	asc=3,
	one=меркурий,
	two=СОЛНЦЕ\\ВЕНЕРА,
	three=РАХУ,
	seven=юпитер\\сатурн,
	nine=КЕТУ,
	eleven=МАРС,
	twelve=ЛУНА
]{}

Рассмотрим подпериод Луны в главном периоде Раху (возраст 33 года).

Раху аспектирует седьмой, девятый, одинадцатый и второй дома и стоит в третьем доме. Она испытывает влияние Юпитера (муж, дети). Луна находится в двенадцатом доме (потеря, утрата) и является хозяином второго дома (члены семьи). От Раху Луна является хозяином двенадцатого дома (потеря, утрата).

В главный период Раху и подпериод Луны эта женщина потеряла мужа. Если учесть расположение Сатурна (смерть) в седьмом доме (муж) и то, что он является хозяином восьмого дома (смерть), становится очевидным такой исход.

Рассмотрим подпериод Юпитера в главном периоде Юпитера (возраст 34--36 лет). Юпитер (муж) расположен в седьмом доме (брак) и является хозяином седьмого дома. Юпитера находится под влиянием Сатурна (время, старший по возрасту и Раху (подобна Сатурну). В главный период Юпитера и подпериод Юпитера эта женщина получила предложение от мужчины, который значительно старше ее по возрасту.


\subsubsection*{Гороскоп мужчины}

\planets[%
	asc=\signum{17}{03}{\libra},
	su=\signum{7}{35}{\scorpio},
	mo=\signum{19}{55}{\capricornus},
	ma=\signum{7}{48}{\capricornus},
	me=\signum{22}{39}{\scorpio}(ретро),
	ju=\signum{20}{46}{\aries}(ретро),
	ve=\signum{15}{38}{\sagittarius},
	sa=\signum{29}{45}{\virgo},
	ra=\signum{22}{12}{\capricornus},
	ke=\signum{22}{12}{\cancer}
]{}

\natal[%
	asc=7,
	two=меркурий\\СОЛНЦЕ,
	three=ВЕНЕРА,
	four=МАРС\\ЛУНА\\РАХУ,
	seven=юпитер,
	ten=КЕТУ,
	twelve=САТУРН
]{}

Рассмотрим подпериод Луны в главном периоде Юпитера (возраст 39--40 лет). Юпитер находится в седьмом доме (общественная деятельность) и является хозяином третьего (личные усилия) и шестого (препятствия, долги, враги) домов. Юпитер подврежен аспекту Марса (рискованные дела, недвижимость). Луна стоит в четвертом доме (недвижимость) и является хозяином десятого дома (статус, карьера). Луна находится под влиянием Марса (недвижимость, рискованные дела), Раху (незаконные действия, люди низкой культуры) и Кету (секретность). Луна занимает десятый дом (статус, карьера) от Юпитера и является хозяином четвертого дома (недвижимость). В главный период Юпитера и подпериод Луны владелец гороскопа организовал свою фирму и приобрел офис.

Рассмотрим подпериод Марса в главном периоде Юпитера (возраст 40--41 год). Марс (недвижимость) расположен в четвертом доме (недвижимость) и является хозяином второго (богатство, деньги) и седьмого (партнерство, бизнес) домов. Марс находится в десятом доме (статус, карьера) от Юпитера и является хозяином первого (личность) и восьмого (трансформация) домов. В главный период Юпитера и подпериод Марса у этого человека увеличились доходы, и он получил прекрасную возможность приобрести новый офис в другом месте.

Рассмотрим подпериод Раху в главном периоде Юпитера (возраст 41-43 года). Раху находится в четвертом доме (окружение человека). Раху является естественно неблагоприятной планетой и аспектирует десятый дом (карьера). Раху (плохие действия) находится в десятом доме (статус) от Юпитера. В главный период Юпитера и подпериод Раху этот человек неправильно организовал свою работу, обманул своих партнеров и привел в упадок дела своей фирмы. Надо отметить, что Юпитер как планета, управляющая главным периодом, не указывает на повышение общественного статуса, и поэтому профессиональный подъем этого человека был временным, то есть приоритет всегда сохраняется за планетой главного периода, а в карте рождения Кету (падение) расположена в десятом доме (карьера), что является прямым указанием на негативный результат в профессиональных делах.


\subsubsection*{Гороскоп мужчины}

\planets[%
	asc=\signum{17}{24}{\scorpio},
	su=\signum{29}{22}{\gemini},
	mo=\signum{4}{02}{\cancer},
	ma=\signum{22}{07}{\virgo},
	me=\signum{4}{33}{\cancer},
	ju=\signum{13}{45}{\aquarius}(ретр),
	ve=\signum{28}{20}{\taurus},
	sa=\signum{22}{26}{\leo},
	ra=\signum{7}{34}{\pisces},
	ke=\signum{7}{34}{\virgo}
]{}

\natal[%
	asc=8,
	four=юпитер,
	five=РАХУ,
	seven=ВЕНЕРА,
	eight=СОЛНЦЕ,
	nine=МЕРКУРИЙ\\ЛУНА,
	ten=САТУРН,
	eleven=МАРС\\КЕТУ
]{}

Рассмотрим подпериод Венеры в главном периоде Венеры (возраст 42--45 лет). Венера (комфорт) находится в собственном знаке, в седьмом доме (супружеская жизнь) и является хозяином седьмого  (жена) и двенадцатого (расходы) домов. Венера находится под аспектом Сатурна (низкие люди) и Кету (секретные действия).

Рассмотрим Венеру (комфортные условия) в лунной карте, которая расположена в одинадцатом доме (доходы, приобретения) и является хозяином четвертого (недвижимость) и одинадцатого (доходы, приобретения) домов.

В главный период Венеры и подпериод Венеры человек испытывал удовольствия от брачной жизни, имел большие расходы, связанные с приобретением квартиры, встречался с низкими людьми и занимался секретными делами.


\subsubsection*{Гороскоп женщины}

\planets[%
	asc=\signum{29}{36}{\libra},
	su=\signum{7}{08}{\libra},
	mo=\signum{28}{58}{\virgo},
	ma=\signum{27}{17}{\libra},
	me=\signum{29}{43}{\libra},
	ju=\signum{13}{01}{\libra},
	ve=\signum{9}{05}{\scorpio},
	sa=\signum{15}{05}{\cancer},
	ra=\signum{19}{17}{\taurus},
	ke=\signum{19}{17}{\scorpio}
]{}

\natal[%
	asc=7,
	one=СОЛНЦЕ\\ЮПИТЕР\\МЕРКУРИЙ\\МАРС,
	two=ВЕНЕРА\\КЕТУ,
	eight=РАХУ,
	ten=САТУРН,
	twelve=ЛУНА
]{}

Рассмотрим подпериод Луны в главном периоде Юпитера (возраст 34 года). Юпитер находится в первом доме и является хозяином третьего и шестого (горе, несчастье) домов. Юпитер зажат Солнцем и Марсом, что является неблагоприятным для Юпитера. Луна (мать) расположена в двенадцатом доме (утрата) и находится под аспектом естественно неблагоприятного Сатурна (смерть, горе) и Раху (несчастья).

Луна находится в двенадцатом доме (утрата) от Юпитера.

В главный период Юпитера и подпериод Луны у женщины умерла мать.


\subsubsection*{Гороскоп мужчины}

\planets[%
	asc=\signum{22}{17}{\gemini},
	su=\signum{13}{20}{\capricornus},
	mo=\signum{21}{35}{\capricornus},
	ma=\signum{10}{10}{\libra},
	me=\signum{26}{45}{\sagittarius},
	ju=\signum{16}{25}{\pisces},
	ve=\signum{6}{38}{\sagitarius},
	sa=\signum{21}{47}{\virgo}(ретро),
	ra=\signum{7}{28}{\aquarius},
	ke=\signum{7}{28}{\leo}
]{}

\natal[%
	asc=3,
	three=КЕТУ,
	four=сатурн,
	five=МАРС,
	seven=ВЕНЕРА\\МЕРКУРИЙ,
	eight=ЛУНА\\СОЛНЦЕ,
	nine=РАХУ,
	ten=ЮПИТЕР
]{}

Рассмотрим подпериод Сатурна в главном периоде Сатурна (возраст 43--45 лет). Сатурн в знаке большого друга, в четвертом доме (недвижимость, транспортные средства) находится под аспектом Юпитера (богатство, благосостояние). Сатурн представляет старые вещи и является хозяином восьмого (трансформация) и девятого (удача) домов.

В главный период Сатурна и подпериод Сатурна этот мужчина приобрел неновую квартиру и подержанный автомобиль.


\subsubsection*{Гороскоп женщины}

\planets[%
	asc=\signum{11}{45}{\capricornus},
	su=\signum{14}{50}{\capricornus},
	mo=\signum{14}{01}{\cancer},
	ma=\signum{23}{46}{\sagittarius},
	me=\signum{24}{50}{\sagittarius},
	ju=\signum{4}{00}{\virgo}(ретро),
	ve=\signum{1}{37}{\pisces},
	sa=\signum{11}{57}{\gemini}(ретро),
	ra=\signum{25}{20}{\gemini},
	ke=\signum{25}{20}{\sagittarius}
]{}

\natal[%
	asc=10,
	one=СОЛНЦЕ,
	three=ВЕНЕРА,
	six=сатурн\\РАХУ,
	seven=ЛУНА,
	nine=юпитер,
	twelve=МЕРКУРИЙ\\КЕТУ\\МАРС
]{}

Рассмотрим подпериод Меркурия в главном периоде Кету (возраст 27 лет). Кету стоит в двенадцатом доме (потеря, утрата) и находится под влиянием Меркурия, Марса (разрыв) и Сатурна (горе, смерть). Меркурий расположен в двенадцатом доме (потеря, утрата) и зажат естественно наблагоприятными Кету и Марсом и аспектируется Раху и Сатурном. Меркурий является хозяином шестого дома (горе, препятствия). Меркурий как хозяин девятого дома (удача) зажат Кету и Марсом. Меркурий является хозяином седьмого дома от Кету (неожиданность) и зажат Кету и Марсом (несчастный случай).

В главный период Кету и подпериод Меркурия эта женщина перенесла смерть своего мужа, которая произошла от несчастного случая.


\subsubsection*{Гороскоп женщины}

\planets[%
	asc=\signum{7}{42}{\aquarius},
	su=\signum{10}{16}{\capricornus},
	mo=\signum{12}{56}{\capricornus},
	ma=\signum{13}{17}{\pisces},
	me=\signum{25}{74}{\capricornus},
	ju=\signum{0}{33}{\cancer},
	ve=\signum{23}{27}{\scorpio},
	sa=\signum{26}{51}{\libra},
	ra=\signum{12}{11}{\sagittarius},
	ke=\signum{12}{11}{\gemini}
]{}

\natal[%
	asc=11,
	two=МАРС,
	five=КЕТУ,
	six=ЮПИТЕР,
	nine=САТУРН,
	ten=ВЕНЕРА,
	eleven=РАХУ,
	twelve=МЕРКУРИЙ\\СОЛНЦЕ\\ЛУНА
]{}

Рассмотрим подпериод Меркурия в главном периоде Юпитера (возраст 37--38 лет). Юпитер в лунной карте находится в седьмом доме (муж) и является хозяином двенадцатого дома (разрыв, отделение). Юпитер испытывает неблаготворное влияние Сатурна.

Меркурий расположен в двенадцатом доме (разрыв, отделение) карты рождения и является хозяином восьмого дома (трансформация). Меркурий стоит в седьмом доме (брак) от Юпитера и является хозяином двенадцатого дома (разрыв, отделение).

В главный период Юпитера и подпериод Меркурия эта женщина разорвала брачные отношения со своим мужем.


\subsubsection*{Гороскоп мужчины}

\planets[%
	asc=\signum{21}{33}{\taurus},
	su=\signum{19}{15}{\libra},
	mo=\signum{18}{30}{\cancer},
	ma=\signum{25}{55}{\cancer},
	me=\signum{19}{57}{\libra}(ретро),
	ju=\signum{9}{32}{\scorpio},
	ve=\signum{5}{40}{\scorpio},
	sa=\signum{28}{48}{\cancer},
	ra=\signum{0}{22}{\taurus},
	ke=\signum{0}{22}{\scorpio}
]{}

\natal[%
	asc=2,
	one=РАХУ,
	three=ЛУНА\\САТУРН\\МАРС,
	six=СОЛНЦЕ\\меркурий,
	seven=ВЕНЕРА\\ЮПИТЕР\\КЕТУ
]{}

Рассмотрим подпериод Меркурия в главном периоде Солнца (возраст 46 лет). Солнце (врачи, медицина) ослаблено в Весах и расположено в шестом доме (болезни). Оно находится под влиянием Меркурия, Марса и Кету. Сгоревший Меркурий расположен в шестом доме (болезни) и находится под влиянием Солнца (врачи), Марса (жар) и Кету (болезненное состояние). Меркурий занимает первый дом (тело) от Солнца и является хозяином девятого (удача) и двенадцатого (больница) домов.

В главный период Солнца и подпериод Меркурия этот мужчина перенес тяжелое заболевание и ощутил признаки выздоровления через четыре месяца от начала болезни.


\subsubsection*{oops}

\planets[%
	asc=\signum{18}{03}{\scorpio},
	su=\signum{8}{05}{\scorpio},
	mo=\signum{13}{44}{\virgo},
	ma=\signum{28}{33}{\virgo},
	me=\signum{23}{26}{\scorpio},
	ju=\signum{26}{18}{\virgo},
	ve=\signum{17}{23}{\virgo}(ретро),
	sa=\signum{24}{41}{\aries}(ретро),
	ra=\signum{4}{40}{\aquarius},
	ke=\signum{4}{40}{\leo}
]{}

\natal[%
	asc=8,
	one=МЕРКУРИЙ\\СОЛНЦЕ,
	four=РАХУ,
	six=сатурн,
	ten=КЕТУ,
	eleven=МАРС\\ЛУНА,
	twelve=ЮПИТЕР\\ВЕНЕРА
]{}

Рассмотрим подпериод Кету в главном периоде Раху (возраст 25 лет). Раху в карте рождения аспектирует двенадцатый дом (дальние путешествия). Раху управляет путешествиями и стоит в четвертом доме (транспортные средства). Раху в лунной карте аспектирует двенадцатый дом (иностранные государства) и находится в шестом доме (препятствия). Кету в карте рождения занимает десятый дом (карьера), в лунной карте она находится в двенадцатом доме (дальние путешествия за границу). Кету стоит в седьмом доме (брак) от Раху.

В главный период Раху и подпериод Кету эта женщина вышла за границей замуж за иностранца. Юпитер (муж в женском гороскопе) и Венера (муж) расположены в двенадцатом доме (иностраные государства) и хозяин седьмого дома (муж, брак) находится в двенадцатом доме, что является главным условием для подобного события.


\subsubsection*{Гороскоп мужчины}

\planets[%
	asc=\signum{22}{56}{\cancer},
	su=\signum{12}{22}{\cancer},
	mo=\signum{3}{12}{\cancer},
	ma=\signum{29}{31}{\pisces},
	me=\signum{29}{32}{\gemini}(ретро),
	ju=\signum{13}{53}{\capricornus}(ретро),
	ve=\signum{11}{22}{\leo},
	sa=\signum{6}{04}{\gemini},
	ra=\signum{14}{00}{\sagittarius},
	ke=\signum{14}{00}{\gemini}
]{}

\natal[%
	asc=4,
	one=ЛУНА\\СОЛНЦЕ,
	two=ВЕНЕРА,
	six=РАХУ,
	seven=юпитер,
	nine=МАРС,
	twelve=КЕТУ\\САТУРН\\меркурий
]{}

Рассмотрим подпериод Меркурия в главном периоде Меркурия (возраст 20--21 год). Меркурий расположен в двенадцатом доме (потеря, иностранные государства) и является хозяином третьего и двенадцатого (иностранные государства, потеря) домов. Меркурий находится под влиянием естественно неблагоприятных планет: Сатурна, Раху, Кету и Марса. Меркурий управляет образованием и учебными заведениями.

В главный период Меркурия и подпериод Меркурия этот молодой человек оставил учебу в высшем учебном заведении и уехал жить в другую страну.


\subsubsection*{Гороскоп женщины}

\planets[%
	asc=\signum{5}{44}{\virgo},
	su=\signum{12}{27}{\scorpio},
	mo=\signum{19}{43}{\virgo},
	ma=\signum{2}{42}{\pisces},
	me=\signum{20}{51}{\scorpio},
	ju=\signum{4}{57}{\virgo},
	ve=\signum{9}{43}{\libra},
	sa=\signum{12}{08}{\scorpio},
	ra=\signum{5}{46}{\scorpio},
	ke=\signum{5}{46}{\taurus}
]{}

\natal[%
	asc=6,
	one=ЮПИТЕР\\ЛУНА,
	two=ВЕНЕРА,
	three=РАХУ\\САТУРН\\МЕРКУРИЙ\\СОЛНЦЕ,
	seven=МАРС,
	nine=КЕТУ
]{}

Рассмотрим подпериод Сатурна в главном периоде Юпитера (возраст 30--31 год). Юпитер расположен в первом доме (тело) и является хозяином четвертого и седьмого (брачная связь) домов. Он представляет мужа, детей и стоит в одном доме с Луной (материнство, чрево).

Юпитер при тех же условиях находится в лунной карте и аспектирует пятый (дети), седьмой (брачная связь) и девятый (удача) дома.

Сатурн занимает третий дом, является хозяином пятого (дети, рождение) и шестого (физическая боль) домов. Сатурн аспектирует пятый дом (рождение, дети), девятый дом (удача) и двенадцатый дом (больница).

Сатурн находится в третьем доме от Юпитера и является хозяином пятого (рождение, дети) и шестого (физическая боль) домов.

В главный период Юпитера и подпериод Сатурна эта женщина родила ребенка.

\subsubsection*{Гороскоп женщины}

\planets[%
	asc=\signum{21}{04}{\libra},
	su=\signum{27}{57}{\leo},
	mo=\signum{7}{19}{\aries},
	ma=\signum{22}{53}{\virgo},
	me=\signum{16}{27}{\leo},
	ju=\signum{5}{57}{\gemini},
	ve=\signum{7}{16}{\libra},
	sa=\signum{19}{51}{\aquarius},
	ra=\signum{14}{16}{\taurus},
	ke=\signum{14}{16}{\scorpio}
]{}

\natal[%
	asc=7,
	one=ВЕНЕРА\\МАРС,
	two=КЕТУ,
	five=САТУРН,
	seven=ЛУНА,
	eight=РАХУ,
	nine=ЮПИТЕР,
	eleven=СОЛНЦЕ\\МЕРКУРИЙ
]{}


Рассмотрим подпериод Марса в главном периоде Венеры (возраст 10 лет). Венера расположена в собственном знаке в первом доме (тело), является хозяином первого (тело) и восьмого (опасность, риск) домов.

Венера находится в одном доме с Марсом (порезы, хирургическая операция). Хозяин первого дома связан с Марсом. На Венеру воздействует аспект Луны (мать), Кету (болезнь, рана) и Юпитер (поддерживающий жизнь). Марс расположен в первом доме (тело) и испытывает влияние от тех же планет, что и Венера. Марс находится в первом доме (тело) от Венеры.

В главный период Венеры и подпериод Марса эта женщина в детском возрасте успешно перенесла хирургическую операцию и постоянно ощущала внимание и заботу матери.

Рассмотрим подпериод Меркурия в главном периоде Венеры (возраст 21--22 года). Венера является хозяином восьмого дома (смерть), а Меркурий --- двенадцатого дома (утрата). Меркурия испытывает влияние Сатурна (горе, смерть) и связан с неблагоприятным Солнцем. Меркурий является хозяином двенадцатого дома (утрата) от Венеры.

В главный период Венеры и подпериод Меркурия произошла смерть матери.


\subsubsection*{Гороскоп женщины}

\planets[%
	asc=\signum{18}{28}{\pisces},
	su=\signum{11}{58}{\taurus},
	mo=\signum{22}{40}{\leo},
	ma=\signum{28}{38}{\aquarius},
	me=\signum{19}{47}{\aries},
	ju=\signum{29}{17}{\virgo}(ретро),
	ve=\signum{0}{50}{\aries},
	sa=\signum{0}{27}{\scorpio}(ретро),
	ra=\signum{7}{32}{\libra},
	ke=\signum{7}{32}{\aries}
]{}

\natal[%
	asc=12,
	two=ВЕНЕРА\\МЕРКУРИЙ\\КЕТУ,
	three=СОЛНЦЕ,
	six=ЛУНА,
	seven=юпитер,
	eight=РАХУ,
	ten=САТУРН,
	twelve=МАРС
]{}

Рассмотрим подпериод Раху в главном периоде Марса (возраст 22 года). Марс расположен в двенадцатом доме (утрата) и является хозяином второго (члены семьи) и девятого (отец) домов. Он находится под аспектами Сатурна (горе, смерть) и Раху (подобен Сатурну). Раху расположен в восьмом дом (смерть) и аспектирует двенадцатый дом (утрата), в котором стоит Марс. Раху занимает девятый дом (отец) от Марса.

В главный период Марса и подпериод Раху произошла смерть отца.

Рассмотрим подпериод Юпитера в главном периоде Марса (возраст 23 года). Марс является хозяином второго (семья) и девятого (удача) домов, а Юпитер --- первого (тело) и десятого (статус) домов. Юпитер расположен в седьмом доме (брак) и является хозяином пятого (дети) дома от Луны. Юпитер представляет мужа, детей и находится в восьмом доме (трансформация) от Марса (страсть), а также явялется хозяином второго (семья) и одинадцатого (надежды, стремления) домов.

В главный период Марса и подпериод Юпитера эта женщина вышла замуж и родила ребенка.

Рассмотрим подпериод Раху в главном периоде Раху (возраст 31--33 года). Раху представляет путешествия, находится в подвижном знаке и аспектирует двенадцатый дом (зарубежные поездки). Она испытывает влияние от Венеры (желания, удовольствия) и Меркурия (друзья).

В главный период Раху и подпериод Раху женщина с друзьями совершила туристическую поездку за рубеж.

В этой главе мы через гороскопы показали различные события, которые происходят в жизни почти у каждого человека, и надеемся, что это послужит для вас ключом для верного предсказания в основе главного плаентного периода и планетного подпериода.



% Заключение
\tocsection{ЗАКЛЮЧЕНИЕ}

Исторические записи сохранили расчеты джотишей, или астрологов, сделавших фантастические предсказания, которые не только оказались правильными, но и поразили современников сверхъестественными по точности указанием времени происшествия событий и яркости описания их качеств. Однако в них имеются и свидетельства об ошибках, допущенных профессионалами с огромным практическим опытом. Дело в том, что знания, полученные только опытным путем и определяемые в основном индивидуальными представлениями астролога, имеют видимую ограниченность. Есть сложные случаи, где таких знаний становится недостаточно, чтобы проникнуть в тайны природы или найти ключ для интерпретации гороскопа. Джотиши тогда применяют технику медитации, чтобы выйти за пределы чувств и получить ответы на свои вопросы в трансцедентальной области сознания.

Все люди, в большей или меньшей степени, наделены чувствами или суперчувствами, но благодаря культивации этих энергий, их можно обострить и развить. Чистота помыслов, ограничение мирских желания, благодеяния --- это основа этики астролога, необходимая для развития его ума и чувств. Но наиболее действенным методом для пробуждения в сознании ведических знаний является трансцедентальная медитация, предложенная учителем Махариши Махеш Йоги, о которой мы упоминали в начале этой книги.

С древних времен в Индии бытует притча о человеке, который вошел в свой дом и в темноте, наступив на веревку, принял ее за змею. Человек испугался, выскочил на улицу и позвал слуг. Один из слуг взял светильник и отправился в дом. Осветив комнату, он обнаружил вместо змеи веревку и успокоил хозяина. Суть притчи в том, что свет знания освобождает человека от страха и страданий.

Махариши Махеш Йоги в комментариях к стиху 45 главы II Бхагавад-Гиты пишет: ``Современные психологические теории исследуют причины, чтобы влиять на следствия. Они ощупью двигаются в темноте, чтобы найти причину темноты и устранить ее. В противопоставление здесь высказывается мысль принести свет, чтобы устранить тьму. Это ``принцип второго элемента''. Если вы хотите воздействовать на первый элемент, не занимайтесь этим элементом, не ищите его причину, влияйте на него непосредственно, вводя второй элемент. Прогоните темноту, принеся свет. Перенесите ум в поле счастья, чтобы избавить его от страдания''.

Принеся Джотиш--Веду --- ``свет знания'' в свою жизнь, человек разгонит темноту неведения о цели своего существования и путей, способствующих естественному развитию индивидуального процесса эволюции.

По мнению великого немецкого врача и естествоиспытателя Парацельса, ``Бог создал планеты и звезды не для того, чтобы они ему служили и были полезны, как все другие создания''.

Человек доминирует среди живых существ на земле, так как он обладает врожденным стремлением к знаниям и способностью их использовать для создания идеальных условий своего существования. В невообразимом коридоре времени звезды --- наши вреные проводники, которые дают возможность рассчитать время и регулировать свою деятельность в согласии с природой, чтобы жизнь могла быть более счастливой, мирной, а общество высокодуховным, культурно развитым.


% Список литературы
\newpage
\printbibliography

\end{document}
