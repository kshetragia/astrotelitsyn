\subsection{Определение силы планет}

В момент рождения человека планеты находятся в различных знаках Зодиака и в зависимости от этого проявляют свою силу или слабость.o

Существует множество способов для определения силы планет с огромным количеством вычислений и составлением дополнительно шестнадцати различных карт, аналогичных карте навамса, о которой будет рассказано далее. Однако на практике большинство астрологов пользуется для этого следующими правилами:
\begin{myenum}[itemsep=0,parsep=0]
	\item Планета в знаке экзальтации.\label{cost:1}
	\item Планета в мулатриконе.\label{cost:2}
	\item Планета в собственном знаке.\label{cost:3}
	\item Планета в знаке большого друга.\label{cost:4}
	\item Планета в дружественном знаке.\label{cost:5}
	\item Планета в нейтральном знаке.\label{cost:6}
	\item Планета во враждебном знаке.\label{cost:7}
	\item Планета в знаке большого врага.\label{cost:8}
	\item Планета в знаке ослабления.\label{cost:9}
\end{myenum}

Данные девять достоинств расположены в порядке убывания силы планет. Планета имеет наибольшую силу, если находится в знаке экзальтации, тогда она проявляет лучшие свои качества и приносит человеку счастье, удачу и доход. В знаке ослабления планеты теряют силы и сулят человеку горе, страдания и убытки. Промежуточные достоинства планет дают неоднозначный результат, в них проявляется как хорошее, так и плохое. Но чем выше достоинство планеты (мулатрикона, собственный знак, знак большого друга), тем сильнее в ней доброе начало, меньше плохих качеств. И наоборот, чем ниже достоинство планеты (враждебный знак или знак большого врага), тем больше будет проявляться зло и меньше хороших качеств.

Достоинство планет по пунктам \ref{cost:1}, \ref{cost:2}, \ref{cost:3} и \ref{cost:9} легко определить по таблице~\ref{tbl:cost}, где перечислены знаки, которые являются экзальтирующими, знаками мулатриконы, собственными и ослабляющими для каждой планеты.

Сначала надо посмотреть, в каком знаке Зодиака эта планета расположена в карте рождения, а затем по таблице определить, какое достоинство этот знак придает соответствующей планете. Например, если в карте рождения Юпитер находится в Рыбах, значит Юпитер в собственном знаке. Возле каждого знака экзальтации проставлено значение градуса (например, Овен \(10^\circ\)) --- это значит, что если планета в знаке находится именно в таком градусе, она наиболее сильная (то есть это градус наивысшей экзальтации планеты в знаке). Аналогично для знака ослабления --- указанное значение градуса определяет точку наибольшего ослабления соответствующей планеты в данном знаке.

\begin{table}[tph!]
	\caption{Определение достоинств планет}
	\label{tbl:cost}

	\centering

	% Расширить по вертикали
	\renewcommand{\arraystretch}{1}

	% Разширить вширь
	%\setlength{\tabcolsep}{.05\textwidth}

	% Заполним данными
	\begin{tabular}{|l|l|l|l|l|}
		\hline
		Планета & Знак экзальтации & Знак мулатрикона & Собственный знак & Знак ослабления \\
		\hline
		Солнце   & Овен \gradus{10}    & Лев \gradus{0}--\gradus{20}     & Лев & Весы \gradus{10}    \\
		Луна     & Телец \gradus{3}    & Телец \gradus{4}--\gradus{30}   & Рак & Скорпион \gradus{3} \\
		Марс     & Козерог \gradus{28} & Овен \gradus{1}--\gradus{12}    & Овен, Скорпион   & Рак \gradus{28}    \\
		Меркурий & Дева \gradus{15}    & Дева \gradus{16}--\gradus{20}   & Близнецы, Дева   & Рыбы \gradus{15}   \\
		Юпитер   & Рак \gradus{5}      & Стрелец \gradus{1}--\gradus{10} & Стрелец, Рыбы    & Козерог \gradus{5} \\
		Венера   & Рыбы \gradus{27}    & Весы \gradus{0}--\gradus{5}     & Телец, Весы      & Дева \gradus{27}   \\
		Сатурн   & Весы \gradus{20}    & Водолей \gradus{1}--\gradus{20} & Козерог, Водолей & Овен \gradus{20}   \\
		Раху     & Телец               & Близнецы & Водолей & Скорпион \\
		Кету     & Скорпион            & Стрелец  & Лев & Телец \\
		\hline
	\end{tabular}
\end{table}

Возле каждого знака мулатрикона также проставлены значения градусов (например, Лев \(0{^\circ}-20^\circ\)) --- это значит, что достоинство или сила мулатрикона действует в пределах указанных градусов, а в остальном пространстве знака планета будет иметь достоинство знака экзальтации или собственного знака в соответствии с таблицей\footnote{Некоторые индийские астрологические школы указывают позицию мулатрикона для Венеры от \gradus{0} Весов до \gradus{15} Весов.}.

Солнце и луна имеют по одному собственному знаку, а остальные пять планет по два. Индийские астрологи считают, что Раху и Кету не имеют собственных знаков, так как они не обладают физическими телами, но их расположение в знаках Водолея и Льва придает им силу, соответствующую достоинству собственного знака.

По терминологии, принятой в Индии, планета является хозяином собственного знака, то есть если Лев --- собственный знак для Солнца, то Солнце --- хозяин Льва и\,т.\,д.

Для того, чтобы определить достоинства планет по пунктам \ref{cost:4}, \ref{cost:5}, \ref{cost:6}, \ref{cost:7} и \ref{cost:8}, то есть в знаке большого друга, дружественном знаке, нейтральном знаке, враждебном знаке и знаке большого врага, необходимо ввести дополнительные термины, такие как \emph{постоянные} отношения и \emph{временные} отношения между планетами.

Каждая планета, кроме Раху и Кету, имеет дружественные, нейтральные или враждебные отношения с другими планетами. Их можно установить с помощью приведенной таблицы~\ref{tbl:relations}, отражающие постоянные отношения между планетами. Из этой таблицы видно, что для Солнца дружественными являются Луна, Марс и Юпитер, нейтральным --- Меркурий, а враждебными --- Сатурн и Венера. И так для каждой планеты, находящейся в левой колонке. Отношения между планетами устанавливаются через знаки, хозяевами которых они являются. Например, если Солнце находится в Раке, а хозяйкой Рака является Луна, которая дружественных отношениях с Солнцем, Значит Солнце находится в дружественном знаке. Таким образом определяют постоянные отношения для каждой планеты в гороскопе.

\begin{table}[tph!]
	\caption{Характер отношений между планетами}
	\label{tbl:relations}

	\centering

	% Расширить по вертикали
	\renewcommand{\arraystretch}{1}

	% Разширить вширь
	%\setlength{\tabcolsep}{.05\textwidth}

	% Заполним данными
	\begin{tabular}{|l|l|l|l|}
		\hline
		Планета & Дружественные & Нейтральные & Враждебные \\
		\hline
		Солнце   & Луна, Марс, Юпитер & Меркурий & Сатурн, Венера \\
		Луна     & Солнце, Меркурий & Марс, Юпитер, Венера, Сатурн & --- \\
		Марс     & Солнце, Луна, Юпитер & Венера, Сатурн & Меркурий \\
		Меркурий & Солнце, Венера & Марс, Юпитер, Сатурн & Луна \\
		Юпитер   & Солнце, Луна, Марс & Сатурн & Меркурий, Венера \\
		Венера   & Меркурий, Сатурн & Марс, Юпитер & Солнце, Луна \\
		Сатурн   & Меркурий, Венера & Юпитер & Солнце, Луна, Марс \\
		\hline
	\end{tabular}
\end{table}

Временные отношения между планетами определяются по их взаимному расположению в домах. Они бывают двух типов --- временная дружба и временная вражда. Если одна планета расположена относительно другой во 2, 3, 4, 10, 11 или 12-м доме, значит эти две планеты находятся во временных дружественных отношениях. Если две планеты находятся в одном доме или одна планета расположена относительно другой в 5, 6, 7, 8 или 9-м доме, значит эти две планеты находятся во временных враждебных отношениях. Надо обратить внимание на то, что при установлении временных отношений между планетами за первый дом принимается тот дом, где находится планета, достоинство которой вы определяете.

Установив постоянные и временные отношения между планетами в гороскопе, можно приступить к определению достоинств планет в соответствии с пунктами \ref{cost:4}, \ref{cost:5}, \ref{cost:6}, \ref{cost:7} и \ref{cost:8} из правила определения силы планет по девяти достоинствам:

\begin{myenum}[itemsep=0,parsep=0]
	\item Постоянные дружественные отношения + дружба временная = планета в знаке большого друга.
	\item Постоянные нейтральные отношения + дружба временная = планета в дружественном знаке.
	\item Постоянные враждебные отношения + дружба временная = планета в нейтральном знаке.
	\item Постоянные дружественные отношения + вражда временная = планета в нейтральном знаке.
	\item Постоянные нейтральные отношения + вражда временная = планета во враждебном знаке.
	\item Постоянные враждебные отношения + вражда временная = планета в знаке большого врага. 
\end{myenum}

Покажем как определяются достоинства планет в знаках Зодиака на нескольких примерах:

\begin{myenum}
	\item Солнце расположено в Тельце, а Венера в Овне. Солнце находится во враждебном знаке, так как хозяйка Тельца --- Венера, согласно таблице~\ref{tbl:relations}, враждебна к Солнцу. Если считать Телец, в котором находится Солнце, 1-м домом, то Венера относительно Солнца находится в Овне, то есть в 12-м доме, что говорит о временной дружбе этих планет. Теперь складываем постоянные враждебные отношения с временной дружбой и получаем, что Солнце находится в нейтральном знаке.
	\item Марс расположен во Льве, хозяином которого является Солнце. Солнце дружественно Марсу и находится в Деве, то есть во 2-м доме от Марса, что указывает на временные дружественные отношения. Сложив постоянные дружественные отношения со временной дружбой, определяем Марс как находящийся в знаке большого друга.
	\item Венера находится в Раке, хозяином которого является Луна. Луна для Венеры враждебна. Луна находится в Козероге, то есть в 7-ом доме от Венеры. Луна --- временный враг Венеры. Сложив постоянные враждебные отношения со временной враждой, определяем Венеру как находящуюся в знаке большого врага.
\end{myenum}


\subsubsection*{Определение силы планет в зависимости от их расположения в знаке Зодиака.}

Вы уже знаете, что каждый знак Зодиака занимает \gradus{30} пространства в Зодиаке, а также то, что знаки делятся на мужские и женские. Все нечетные знаки --- мужские, четные --- женские.

Если планета расположена в части знака от \gradus{12} до \gradus{18}, то это местонахождение способствует максимальному проявлению силы такой планеты. Расположение планеты в первых \gradus{6} четных женских знаков или последних \gradus{6} нечетных мужских знаков значительно ослабит ее. В остальном пространстве знака планеты чувствуют себя достаточно сильными для проявления своих идей.

Если планета находится на границе между двумя знаками, то она не в состоянии полностью проявить себя, поэтому рассматривается как ослабленная.

Если планета находится не далее чем в \gradus{5} от Солнца, то она считается сгорающей, и сила ее значительно убывает. Меркурий меньше других подвержен сгоранию, так как нахождение вблизи Солнца его естественное состояние. Планеты Раху и Кету не подвержены сгоранию, поскольку не обладают физическими телами.

В ретроградном движении планета проявляет больше силы чем в прямом.

Определение силы планет --- это важнейшая тема в Индийской предсказательной астрологии, и мы будем продолжать развивать ее в следующих главах. Конечно, при первых попытках анализа гороскопа читатель столкнется с некоторыми трудностями в определении силы планет, так как здесь нет четкого выражения относительных сил в процентах, а самих признаков силы и слабости планет достаточно много, но в дальнейшем у вас появится опыт. Практикующий астролог чувствует силу планет даже по внешнему виду человека, с которым беседует. Поэтому мы еще раз подчеркиваем, что очень важно знать напамять максимальное количество информации.
