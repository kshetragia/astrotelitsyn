\section{Карта навамcа и ее назначение}

В Индийской предсказательной астрологии существует шестнадцать карт, которые могут быть составлены для любого человека. Каждая из них содержит специфическую информацию, и большинство астрологов используют эти карты для определения достоинств планет. Карта навамса --- наиболее важная после карты рождения и лунной карты и составляется из одной девятой части зодиакального знака. Знак Зодиака, равный \gradus{30}, делится на девять равных частей, каждая из который равно \coord{3}{20}{0}.

\begin{landscape}
	\begin{table}[tph!]
		\centering
		\caption{Положение планет и домов в карте навамса.}
		\label{tbl:navamsa}

		% Расширить по вертикали
		\renewcommand{\arraystretch}{1.5}

		% Заполним данными
		\begin{tabular}{|l|c|c|c|c|c|c|c|c|c|c|c|c|}
			\hline
			 & \gradus{0}--\cord{3}{20} & \cord{3}{20}--\cord{6}{40} & \cord{6}{40}--\gradus{10} & \gradus{10}--\cord{13}{20} & \cord{13}{20}--\cord{16}{40} & \cord{16}{40}--\gradus{20} & \gradus{20}--\cord{23}{20} & \cord{23}{20}--\cord{26}{40} & \cord{26}{40}--\gradus{30} \\
			\hline
			Овен     & Овен & Телец & Близнецы & Рак & Лев & Дева & Весы & Скорпион & Стрелец \\
			Телец    & Козерог & Водолей & Рыбы & Овен & Телец & Близнецы & Рак & Лев & Дева \\
			Близнецы & Весы & Скорпион & Стрелец & Козерог & Водолей & Рыбы & Овен & Телец & Близнецы \\
			Рак      & Рак & Лев & Дева & Весы & Скорпион & Стрелец & Козерог & Водолей & Рыбы \\
			Лев      & Овен & Телец & Близнецы & Рак & Лев & Дева & Весы & Скорпион & Стрелец \\
			Дева     & Козерог & Водолей & Рыбы & Овен & Телец & Близнецы & Рак & Лев & Дева \\
			Весы     & Весы & Скорпион & Стрелец & Козерог & Водолей & Рыбы & Овен & Телец & Близнецы \\
			Скорпион & Рак & Лев & Дева & Весы & Скорпион & Стрелец & Козерог & Водолей & Рыбы \\
			Стрелец  & Овен & Телец & Близнецы & Рак & Лев & Дева & Весы & Скорпион & Стрелец \\
			Козерог  & Козерог & Водолей & Рыбы & Овен & Телец & Близнецы & Рак & Лев & Дева \\
			Водолей  & Весы & Скорпион & Стрелец & Козерог & Водолей & Рыбы & Овен & Телец & Близнецы \\
			Рыбы     & Рак & Лев & Дева & Весы & Скорпион & Стрелец & Козерог & Водолей & Рыбы \\
			\hline
		\end{tabular}
	\end{table}
\end{landscape}

Картa навамса составляется из координат восходящего знака (асцендента) и планет карты рождения. Например, карта рождения имеет следующие координаты:

\begin{table}[tph!]
	% Расширить по вертикали
	%\renewcommand{\arraystretch}{1.5}

	% Заполним данными
	\begin{tabular}{|lll|}
		\hline
		Асцендент & \cord{28}{02} & Льва \\
		Солнце   & \cord{25}{13} & Девы \\
		Луна     & \cord{9}{54} & Скорпиона \\
		Марс     & \cord{23}{59} & Льва \\
		Меркурий & \cord{17}{05} & Весов \\
		Юпитер   & \cord{3}{16} & Близнецов \\
		Венера   & \cord{28}{40} & Льва \\
		Сатурн   & \cord{5}{31} & Весов \\
		Раху     & \cord{6}{14} & Козерога \\
		Кету     & \cord{6}{14} & Рака \\ \hline
	\end{tabular}
\end{table}

Определяем по таблице, в какой навамсе находятся асцендент и девять планет с учетом их координат в карте рождения.

Составляем карту навамса и получаем следующие координаты расположения планет и домов:

\begin{mylist}
	\item Стрелец --- первый дом.
	\item Солнце --- Лев и девятый дом.
	\item Луна --- Дева и десятый дом.
	\item Марс --- Скорпион и двенадцатый дом.
	\item Меркурий --- Рыбы и четвертый дом.
	\item Юпитер --- Весы и одинадцатый дом.
	\item Венера --- Стрелец и первый дом.
	\item Сатурн --- Скорпион и двенадцатый дом.
	\item Раху --- Водолей и третий дом.
	\item Кету --- Лев и девятый дом.
\end{mylist}

В карте навамса градусы и минуты долготы отсутствуют, учитываются только знаки Зодиака.

Карта навамса дает дополнительную информацию о брачных отношениях наряду с картой рождения и лунной картой.

Эта карта поможет определить дополнительные достоинства планет (например, планеты в экзальтации, собственном и ослабленном знаках).

В выше приведенном примере в карте навамса Солнце расположено во Льве, Марс --- в Скорпионе, а Меркурий в Рыбах, поэтому Солнце и Марс, находящиеся в собственных знаках, будут усилены, Меркурий, который стоит в ослабленном знаке, будет ослаблен.

Если планета расположена в одном и том же знаке в карте рождения и в карте навамса, то ее сила приравнивается к силе собственного знака, и на это надо обратить особое внимание, поскольку такая позиция планеты оказывает благотворное влияние на жизнь человека (например, в картах рождения и навамса Солнце находится в Близнецах, значит Солнце в гороскопе будет достаточно сильным и его влияние на тот дом, в котором оно стоит в карте рождения, будет благоприятным).

Если планета расположена в карте навамса в ослабленном знаке, то она будет ослаблять тот дом, которым управляет в карте рождения (например, в карте рождения Меркурий находится в Весах в третьем доме и является хозяином второго и одинадцатого домов, а в карте навамса он пребывает в знаке ослабления, то есть в Рыбах, значит идеи второго и одинадцатого домов будут ослаблены, следовательно, платежеспособность и заработки человека будут низкими.

Если планета расположена в карте навамса в собственном знаке или в знаке экзальтации, то она будет усиливать тот, дом, которым управляет в карте рождения (например, в карте рождения Солнце расположено в Деве во втором доме и является хозяином первого дома, а в карте навамса оно находится во Льве, то есть в собственном знаке, значит идеи первого дома (внешность, характер и жизнестойкость человека) будут улучшаться.

Нельзя рассматривать карту навамса отдельно от карты рождения и лунной карты, так как именно в них содержатся главные условия брачных отношений. Карта навамса их только дополняет.

Некоторые индийские астрологи предсказывают события исходя из позиции планет в карте навамса. Но, учитывая, что она является более тонким инструментом в предсказании, чем карта рождения и лунная карта, советуем пока ею не пользоваться. Когда вы будете иметь опыт в прочтении двух главных карт, вы сами почувствуете, что наступило время делать предсказания, используя карту навамса. Надо принять к сведению еще и такой факт: если планета расположена в одном и том же доме в карте рождения и карте навамша, то этот дом будет себя сильно проявлять в зависимости от природы планеты (например, в карте рождения Юпитер находится в Близнецах в одинадцатом доме, а в карте навамса он в Весах в одинадцатом доме --- значит, во--первых, отношения с друзьями будут прекрасные или друзья будут высокодуховными людьми, во--вторых, заработки будут хорошие и все дела, связанные с одинадцатым домом, будут процветать).

Если в картах рождения и навамса в одном и том же доме находится неблагоприятная планета, то дела этого дома придут в упадок (например, в карте рождения Сатурн расположен в Скорпионе в третьем доме, а в карте навамса во Льве в третьем доме, значит, идеи третьего дома ухудшатся, то есть отношения с братьями и сестрами будут напряженными).

Таки образом, научившись пользоваться картой навамса, вы получите дополнительные сведения не только о супружеских взаимоотношениях, но и о других областях жизни.
