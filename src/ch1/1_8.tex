\section{Важнейшие планетные комбинации}

Две и более планет, находящихся во взаимосвязи, представляют планетную комбинацию, которая бывает благоприятной или неблагоприятной. Планетные комбинации могут усиливать или ослаблять планеты и дома. Те, что усиливают, являются благоприятными, те что ослабляют --- неблагоприятными.

\subsection*{Благоприятные комбинации планет}

Луна выше по долготе, но в одном знаке с Марсом дает большое богатство (например, Луна \signum{8}{20}{\cancer}, а Марс \signum{4}{15}{\cancer}). Меркурий выше Солнца и прошел точку сгорания (т.\,е. более \gradus{5} от Солнца) и Солнце в одном знаке с Меркурием --- такая комбинация указывает на высокие интеллектуальные способности (например, Солнце в \gradus{10} Тельца, а Меркурий в \gradus{19} Тельца).

Планеты занимают все ниже перечисленные дома: второй, шестой, восьмой и двенадцатый. Эта комбинация дает большое богатство. По крайней мере одна планета занимает второй дом от Луны --- хорошо для материального положения (например, Луна расположена в Овне, а Марс --- в Тельце). Если хотя бы одна планета занимает двенадцатый дом от Луны --- материальное положение улучшится (например, Луна расположена в Близнецах, а Венера --- в Тельце).



Планеты занимают все ниже перечисленные дома: второй, шестой, восьмой и двенадцатый. Эта комбинация дает большое богатство. По крайней мере одна планета занимает второй дом от Луны --- хорошо для материального положения (например, Луна расположена в Овне, а Марс --- в Тельце). Если хотя бы одна планета занимает двенадцатый дом от Луны --- материальное положение улучшится (например, Луна расположена в Близнецах, а Венера --- в Тельце).

Еще более благоприятная комбинация, если одна планета расположена во втором доме, а другая в двенадцатом доме от Луны --- высокое материальное положение (например, Луна находится в Раке, Юпитер --- во Льве, а Солнце --- в Близнецах).

Венера с Марсом в одном знаке и в любом квадранте делает человека лидером в его семье, а также наделяет его большой сексуальной энергией (например, Венера с Марсом во Льве в первом доме).

Венера в одном знаке с Сатурном свидетельствует о таланте или удачной карьере в искусстве, Сатурн в двенадцатом доме от Венеры дает подобный результат (например, Венера с Сатурном в Весах или Венера в Близнецах, а Сатурн в Тельце).

Раху в одном знаке с Венерой и Юпитером сулит все радости жизни.

Солнце с Луной в одном знаке и в любом квадранте говорит о способностях лидера (например, Солнце и Луна во Льве в седьмом доме).

Луна в одном знаке с Юпитером или во взаимной аспектной связи указывает на религиозное мировоззрение или набожность человека (например, Луна и Юпитер в Раке или Луна в Рыбах, а Юпитер в Деве).

Луна в одном знаке с Меркурием или во взаимной аспектной связи предопределяет высокий интеллект человека (например, Луна и Меркурий во Льве или Луна во Льве, а Меркурий в Водолее).

Луна в одном знаке с Кету свидетельствует о стремлении к духовной жизни.

Позиция Меркурия в знаке Марса или Марса в знаке Меркурия хороша для развития интеллекта (например, Меркурий в Овне или Марс в Близнецах).

Юпитер в одном знаке с Кету способствует духовным достижениям.

Венера, аспектированная или в связи с Юпитером, указывает на исселдовательские способности человека и обусловливает моральные устои семейной жизни.

Венера в знаке Марса или Марс в знаке Венеры свидетельство необузданных сексуальных желаний (например, Венера в Скорпионе, а Марс в Тельце).

Венера в одном знаке с Меркурием в любом квадранте дает большие материальные блага, но также может наделить коварной женой.

Венера в квадранте от Луны говорит о сексуальных устремлениях (например Луна в Тельце, а Венера во Льве).

Венера и Марс, аспектирующие друг друга, предсказывают активную сексуальную жизнь (например, Венера в Тельце, а Марс в Скорпионе).

Расположение Меркурия с Солнца в Козероге в первом доме указывает на интерес к медицине.

Позиция Марса и Солнца в Овне в десятом доме от восходящего знака или от Луны пропрочит человеку высокий государственный пост.

Если хозяева квадрантов связаны с хозяевами тригона, то это очень благоприятно для того дома, в котором имеется такая связь (например, Венера как хозяйка девятого дома расположена в Близнецах в десятом доме вместе с Меркурием, который является хозяином десятого дома. В данном примере усилен десятый дом. Другой пример: Меркурий как хозяин девятого дома расположен в водолее в пятом доме вместе с Сатурном, который является хозяином четвертого дома, следовательно, усилен пятый дом).

Если хозяева квадрантов обмениваются домами с хозяевами тригона, то это очень благоприятно для домов (например, Венера как хозяйка девятого дома расположена в Близнецах в десятом доме, а Меркурий как хозяин десятого дома находится в Тельце в девятом доме. В данном примере усилены девятый и десятый дома).

Если хозяева квадрантов и тригона одновременно аспектируют какой-либо дом, то этот дом становится сильным (например, при восходящем Раке Марс занимает седьмой дом, а Луна четвертый дом как хозяева пятого и первого домов). В данном примере десятый дом усилен, поэтому человек найдет удовлетворение в профессиональных делах.

Планета, которая расположена в собственном знаке или знаке экзальтации в квадранте от восходящего знака или от Луны, является очень благоприятной (например, Марс в Козероге в седьмом доме от восходящего Рака. Меркурий в Близнецах в десятом доме от восходящей Девы. Юпитер в Раке, а Луна в Овне. Марс в Скорпионе, а Луна во Льве).

Если Луна, Меркурий, Венера или Юпитер окружают отдельный дом или планету, то это является хорошим условием для данного дома или планеты (например, Солнце в Тельце, Луна в Овне, а Юпитер в Близнецах. В данном примере Марс находится между Венерой и Меркурием, поэтому на нем сказывается благотворное воздействие этих планет).

Если планета находится в ослабленном знаке, но хозяин этого знака расположен в знаке экзальтации, то ослабленная планет несколько усиливается (например, Меркурий ослаблен в Рыбах, но хозяин Рыб, Юпитер, расположен в Раке, значит, Меркурий набирает некоторую силу за счет экзальтированного Юпитера).


\subsection*{Неблагоприятные комбинации планет}

Если нет планет перед знаком, где находится Луна, и в следующем знаке от Луны, то человек часто будет впадать в депрессию, и окружающие не будут понимать его состояние.

Луна или асцендент, находящиеся в последних двух градусах водного знака, являются указанием на смерть в детском возрасте.

Солнце в одном знаке с Раху ведет к душевному разладу, и у других людей не будет взаимопонимания с этим человеком.

Солнце в одном знаке с Марсом указывает на опасность огня.

Венера в одном знаке с Раху сулит побочные сексуальные связи.

Юпитер, аспектированный Сатурном, выявляет ``вечного'' студента (ученика).

Несколько ретроградных планет в гороскопе говорят о человеке, критически относящемся ко всему на свете, склонному к отрицательным суждениям о людях.

Венера и Юпитер будут испорчены влиянием злой Раху, если находятся с ней в одном знаке.

Взаимосвязь Марса и Сатурна приносит трудности в те дома, в которых они расположены.

Венере вредит расположение в одном знаке с Солнцем.

Взаимосвязь Солнца и Сатурна создает трудности в жизни, но наделяет человека такими качествами, как целеустремленность, ответственность, смирение и духовность.

Луна в одном знаке с Сатурном порождает у человека чувство одиночества, но настраивает его на медитацию.

Планета будет уменьшать силу в том знаке, хозяин короткого расположен в знаке ослабления (например, Юпитер в Весах, а Венера в Деве).

Связь хозяев шестого, восьмого и двенадцатого домов в домах квадрантов или тригона является крайне неблагоприятной (например, при восходящем скорпионе Марс как хозяин шестого дома и Меркурий как хозяин восьмого дома находятся в Водолее в четвертом доме.

Общее влияние хозяев шестого, восьмого и двенадцатого домов на дома квадранта или тригона также неблагоприятно и поражает любой из этих домов (например, при восходящих Близнецах Сатурн из Козерога как хозяин восьмого дома и Марс из Девы как хозяин шестого дома аспектируют десятый дом в Рыбах. В данном примере десятый дом становится пораженным).

Если Сатурн, Марс, Раху, Кету или Солнце окружают отдельный дом или планету, то это является неблагоприятным условием для данного дома или планеты (например, Солнце в Тельце в седьмом доме, Марс в Близнецах в восьмом, а Сатурн в Овне в шестом доме. В этой ситуации уменьшается сила Солнца и седьмого дома, так как они находятся в тисках Марса и Сатурна. Другой пример: Венера в \gradus{12} Тельца, Раху в \gradus{5} Тельца, а Солнце в \gradus{20} Тельца. Итак, Венера зажата неблаготворными Раху и Солнцем, поэтому брачный союз, который представлен Венерой, будет нарушен).

Благотворная планета, находящаяся под аспектами двух и более неблаготворных планет, явно уменьшает свою позитивную силу и сзодает трудности в жизни человека (например, Юпитер расположен в Весах во втором доме, Сатурн --- в Овне в восьмом доме, а Марс в Раке в одинадцатом доме. В данном примере Юпитер аспектирован Сатурном и Марсом, поэтому он не принесет удачу в материальных делах и семейного счастья).

Существует большое количество других планетных комбинаций в индийской астрологии, которые мы не стали описывать из--за второстепенного их значения.
