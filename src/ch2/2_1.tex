\section{Анализ главного планетного периода}

События, происходящие в жизни человека, можно рассматривать как звенья одной непрерывной цепи, которые выражаются во времени. Каждый период во времени играет особую роль в эволюции сознания человека.

Среди индийских астрологов бытует мнение, что предсказательная система базируется на теори 120-летней продолжительности человеческой жизни. Это утверждение, как говорят некоторое знатоки индийской астрологии, по крайней мерен, некорректно. В основе предсказательной системы лежит 120-летний цикл эволюции жизненного опыта человека. Это количество лет получено из постоянной величины накшатр, равных \cord{13}{20}, и девяти планет, т.\,е. \cord{13}{20} * 9 = 120.

Жизнь человека делится на главные периоды, подпериоды, подподпериоды и на еще более маленькие периоды, которые находятся под управлением различных планет (расчеты главных периодов и подпериодов даны в главе 1). Каждая планета в гороскопе обладает силой, она особенно ярко проявляется в главный период этой планеты и оказывает влияние на жизнь человека. После общего анализа гороскопа вы должны приступить к интерпретации планеты, которая руководит главным периодом.

Если планета главного периода находится в хорошей позиции, то она оказывает благоприятный эффект в определенное время в тех областях жизни, которыми управляет и на которые влияет. Например, рассмотрим главный период Сатурна, который находится в Козероге в четвертом доме и испытывает влияние Венеры из Рака и Юпитера из Девы. В данном примере Сатурн расположен в квадранте в собственном знаке, является хозяином четвертого и пятого домов и находится под аспектом естественно благоприятных Венеры и Юпитера. Отсюда следует, что Сатурн обладает благоприятными силами, а потому у человека есть вероятность получить хорошую квартиру, дачный участок, автомобиль, а дети принесут счастье.

Если планета главного периода занимает плохую позицию, то она оказывает неблагоприятный эффект в определенное время в тех областях жизни, которыми управляет и на которые влияет. Например, рассмотрим главный период Марса, который находится в Раке в восьмом доме вместе с Сатурном. Поскольку Марс расположен в ослабленном знаке, в плохом доме, является хозяином двенадцатого дома и испытывает влияние естественно неблагоприятного Сатурна, его семилетний период принесет человеку большие страдания, несчастья, унижения, материальные потери, ухудшит условия жизни.

Итак, мы провели два противоположных по своему результату примера, которые наблюдаются в астрологической практике крайне редко. По мнению индийских астрологов, нет ни одного абсолютно благоприятного или абсолютно неблагоприятного главного планетного периода, так как подпериоды будут указывать на различные варианты воздействия тех или иных эффектов в определенное время жизни человека. В своей практике астрологам приходится в основном сталкиваться с главными планетными периодами, дающими смешанный результат. Начинающим оценка планеты главного периода может показаться слишком противоречивой, но на самом деле никакого противоречия здесь нет и все, что связано с этой планетой, должно проявиться в ее главный период. Например, планета главного периода стоит в ослабленном знаке, но в хорошем доме (1, 4, 5, 7, 9 или 10-м доме), значит, благотворный эффект планеты в доме не будет выраженным. Другой пример: планета главного периода находится в плохо доме (6, 8 или 12-м доме), но в знаке экзальтации, поэтому ее неблагоприятный эффект в доме будет явно уменьшен, и в конечном итоге все в жизни человека будет зависеть от дополнительных влияний благоприятных или неблагоприятных планет.

Таким образом, эффект планеты главного периода проявится:

\begin{myitem}
	\item в доме, который она занимает.
	\item в доме, хозяином которого она является
	\item в знаке, в котором она находится
	\item в доме, который она аспектирует
	\item в естественной природе самой планеты
	\item в связи с планетами в одном доме
	\item в связи с планетами через аспект
	\item в лунной карте: в доме, в котором находится эта планета, в доме, зозяином которого она является, и в доме, который она аспектирует.
\end{myitem}

Существует большое количество книг по индийской астрологии, в которых есть главы с описанием эффектов планет главного периода, но, так как нет одинаковых гороскопов, то и предсказания не могут быть одинаковыми, давать общие шаблонные описания действий планет, руководящих главным периодом, нет смысла.

Надо отметить и такой факт, что если естественно неблагоприятная планета главного периода является одновременно хозяином квадранта и тригона, то она становится благотворной в течение данного периода и принесет в жизнь человека хорошие события. Если эта планета расположена в доме квадранта или тригона, то ее добрая сила явно возрастает. Например, при восходящем Раке Марс находится в Овне в десятом доме. Другой пример: при восходящих Весах Сатурн находится в Козероге в четвертом доме.

Естественно благоприятная планета главного периода, если она является одновременно хозяином квадранта и тригона, становится очень благоприятной. Например, при восходящем Водолее Венера является хозяином четвертого и девятого домов.

Естественно благоприятная планета главного периода, если она является хозяином шестого, восьмого или двенадцатого дома, проявит в течение этого периода негативный эффект как хозяин неблагоприятного дома. Например, при восходящем Раке Меркурий является хозяином двенадцатого дома.

Естественно неблагоприятная планета главного периода, если она является хозяином шестого, восьмого или двенадцатого дома, окажет в течение этого периода очень негативный эффект на жизнь человека. Например, при восходящей Деве Марс является хозяином восьмого дома.

Если с учетом выше приведенных правил планета главного периода находится в хорошем знаке и доме, то ее добрая сила возрастет. Если планета главного периода находится в плохом знаке и доме, то ее отрицательный эффект увеличится.

Если естественно неблагоприятная планета главного периода является хозяином хороших домов, то она не нанесет большого вреда тому дому, который она занимает. Например, при восходящем Раке Марс является хозяином пятого и десятого домов и расположен в третьем доме. В данном примере третий дом серьезно страдать не будет.

Каждая планета в соответствии со своей естественной природой создает определенные эффекты в течение главного периода. Если эта планета в результате анализа окажется сильной, то она принесет благотворные результаты в соответствии со своей планетной природой. Например, если в гороскопе Солнце определено как сильная планета, то в его главный период будут процветать дела отца, наладятся хорошие отношения с людьми, представляющими власть, человек будет заниматься самосовершенствованием, успешными будут путешествия в восточном направлении и\,т.\,д.

Если планета главного периода в результате анализа окажется слабой, то она принесет неблагоприятные результаты в соответствии со своей планетной природой. Например, если в гороскопе Солнце определено как слабая планета, то в его главный период дела отца будут в упадке, здоровье пошатнется и даже возникнет угроза смерти, будут складываться плохие отношения с начальством или представителями власти, человек не будет верно оценивать свои поступки и путешествия на восток не принесут ему пользы.

Ниже мы представляем гороскопы с анализом главного планетного периода.

\subsubsection*{Гороскоп мужчины}

\planets[%
	asc=\signum{4}{11}{\scorpio},
	su=\signum{0}{48}{\leo},
	mo=\signum{10}{58}{\pisces},
	ma=\signum{4}{46}{\sagittarius}, 
	me=\signum{26}{13}{\cancer},
	ju=\signum{25}{37}{\gemini},
	ve=\signum{15}{54}{\virgo}, 
	sa=\signum{10}{48}{\libra},
	ra=\signum{20}{52}{\sagittarius},
	ke=\signum{20}{52}{\gemini},
]{}

\natal[%
	asc=8,
	two=МАРС\\РАХУ,
	five=ЛУНА,
	eight=КЕТУ\\ЮПИТЕР,
	nine=МЕРКУРИЙ,
	ten=СОЛНЦЕ,
	eleven=ВЕНЕРА,
	twelve=САТУРН
]{}

Рассмотрим главные период Меркурия, который будет длиться семнадцать лет.

\begin{myenum}[itemsep=0,parsep=0]
	\item Меркурий расположен в девятом доме карты рождения.
	\item Меркурий является хозяином восьмого и одинадцатого домов в карте рождения.
	\item Меркурий находится в знаке большого врага (Рак).
	\item Меркурий аспектирует третий дом.
	\item Меркурий в данной позиции является естественно благоприятной планетой.
	\item Меркурий связи с планетами в одном доме не имеет.
	\item Меркурий находится под аспектом естественно неблагоприятных Марса и Сатурна.
	\item Меркурий расположен в пятом доме лунной карты.
	\item Меркурий является хозяином четвертого и седьмого домов в лунной карте.
\end{myenum}

Главный период Меркурия в данном гороскопе охватывает возраст с восьми до двадцатипяти лет. Этот период не будет сильным, так как Меркурий расположен в знаке большого врага, является хозяином восьмого дома и находится под аспектом неблаготворных Марса и Сатурна. Положение Меркурия в девятом доме благоприятно, но этого недостаточно для погашения отрицательных эффектов, поэтому не следует ожидать большой удачи в главный период этой планеты. Но в лунной карте Меркурий расположен в пятом доме и является хозяином квадрантов, значит он в лунной карте будет способствовать успешной учебе и обеспечит хорошие результаты, связанные с вопросами образования. Меркурий в лунной карте --- хозяин седьмого дома (любовные и брачные дела), что дает все основания предполагать возможность заключения брака. Так как Меркурий в карте рождения является хозяином восьмого дома (критическая ситуация, травмы, трансформация) и находится под аспектом злого Марса (травмы, порезы, физическая боль), то вполне можно ожидать хирургического вмешательства или травму в главный период планеты. То, что Меркурий испытывает аспект Сатурна, указывает на взаимоотношения с нехорошими людьми и уменьшение материальных возможностей. Меркурий в карте рождения является хозяином одинадцатого дома (друзья, наджды), значит в главный планетный период человек будет иметь друзей и надежды не покинут его сердце. В лунной карте Меркурий становится хозяином четвертого дома (мать, друзья, родственники, недвижимость), что подтверждает вышеописанное и свидетельствует о наличии скромного жилья и тесных взаимоотношениях с матерью и родственниками. Меркурий в карте рождения аспектирует третий дом (информация, литература), а в лунной --- одинадцатый дом (друзья, доходы), поэтому на протяжении его главного периода человек будет увлекаться литературой, оказывать влияние на друзей и иметь средние доходы.

Из анализа главного периода Меркурия видно, что основной эффект этой планеты будет проявлен в делах образования и во взаимоотношениях с друзьями.

\subsubsection*{Гороскоп женщины}

\planets[%
	asc=\signum{17}{37}{\scorpio},
	su=\signum{17}{58}{\pisces},
	mo=\signum{4}{07}{\capricornus},
	me=\signum{11}{55}{\pisces}\,(ретроградный),
	ma=\signum{1}{54}{\gemini},
	ju=\signum{8}{24}{\scorpio},
	ve=\signum{21}{01}{\aries},
	sa=\signum{13}{36}{\sagittarius},
	ra=\signum{19}{48}{\virgo},
	ke=\signum{19}{48}{\pisces}
]{}

\natal[%
	asc=8,
	one=ЮПИТЕР,
	two=САТУРН,
	three=ЛУНА,
	five=СОЛНЦЕ\\меркурий\\КЕТУ,
	six=ВЕНЕРА,
	eight=МАРС,
	eleven=РАХУ
]{}

Рассмотрим главный период Луны, который длится десять лет:

\begin{myenum}[parsep=0,itemsep=0]
	\item Луна занимает третий дом.
	\item Луна является хозяином девятого дома.
	\item Луна находится в знаке друга (Козерог).
	\item Луна аспектирует девятый дом.
	\item Луна убывающая, значит, она является естественно неблагоприятной планетой.
	\item Луна не связана ни с одной планетой в том доме, в котором она находится.
	\item Луна находится под аспектом Марса и Раху.
	\item Луна не рассматривается в лунной карте как планета, находящаяся в первом доме (во всех лунных картах Луна будет занимать первый дом).
	\item Луна в лунной карте является хозяином седьмого дома и одновременно его аспектирует.
\end{end}

Главный период Луны в данном гороскопе охватывает возраст девочки с двух до двенадцати лет. Планета имеет средние силы и в свой главный период принесет владелице гороскопа хорошие и плохие результаты. Луна в третьем доме указывает на частые непродолжительные поездки, отношения с братьями и сестрами. Поскольку Луна хозяин девятого дома и одновременно его аспектирует, это свидетельствует о дальних путешествиях и удаче. Подтверждением этого факта является то, что Луна находится в подвижном знаке (Козероге). Эта планета по своей природе представляет мать, женщину, места купания, общественную среду, и поэтому ее идеи будут ярко проявлены в десятилетний период жизни. Луна находится под воздействием Марса (порезы, операция, ушиб) и Раху (инфекционные болезни), значит, в главный ее период следует ожидать хирургической операции, травмы, инфекционных заболеваний. В лунной карте Луна является хозяином седьмого дома и одновременно его аспектирует, а так как седьмой дом представляет общественную жизнь, то этот ребенок в школьные годы будет заниматься общественными делами.

Итак, все, с чем связана Луна в гороскопе, будет проявляено в течение десяти лет.

По двум вышеприведенным гороскопам мы сделали основной анализ эффектов планет главного периода, но они не являются исчерпывающими, так как подпериоды укажут на дополнительные варианты событий в судьбе человека.
