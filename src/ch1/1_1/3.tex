\subsection{Расчет главных периодов и подпериодов планет}

Мы знаем, что Зодиак разделен на 12 равных знаков. Но в индийской предсказательной астрологии он еще делится на 27 равных частей, которые называются \emph{лунными созвездиями} или \emph{накшатрами}. Координаты лунных созвездий относительно знаков Зодиака даны в таблице~\ref{tbl:nakshatras}. Индийские риши разделили 27 созвездий на 9 групп по 3 созвездия в каждой. Каждая группа имеет своего ``хозяина'' --- одну из девяти планет. Каждая планета оказывает влияние на жизнь человека в определенный период времени.


\begin{table}[tph!]
	\caption{27 лунных созвездий(накшатр)}
	\label{tbl:nakshatras}

	\centering

	% Расширить по вертикали
	\renewcommand{\arraystretch}{1}

	% Заполним данными
	\begin{tabular}{|p{.01\textwidth}p{.18\textwidth}|p{.01\textwidth}p{.18\textwidth}|p{.01\textwidth}p{.21\textwidth}|p{.09\textwidth}|p{.12\textwidth}|}
		\hline
		\multicolumn{6}{|c|}{} & Хозяева & Продол-\\
		\multicolumn{6}{|c|}{Накшатры и их протяженность} & групп созвездий & житель-ность главных периодов \\
		\hline
		1. & Ашвини & 10. & Магха & 19. & Мула & Кету & 7 лет /\\
		& \signum{0}{13}{\aries}--\signum{13}{20}{\aries}& & \signum{0}{13}{\leo}--\signum{13}{20}{\leo} & & \signum{0}{13}{\sagittarius}--\signum{13}{20}{\sagittarius} & & 2520\,дней\\
		& & & & & & & \\

		2. & Бхарани & 11. & Пурва"=Пхалунги & 20. & Пурвашадха & Венера & 20 лет /\\
		& \signum{13}{20}{\aries}--\signum{26}{40}{\aries}& & \signum{13}{20}{\leo}--\signum{26}{40}{\leo} & & \signum{13}{20}{\sagittarius}--\signum{26}{40}{\sagittarius} & & 7200\,дней\\
		& & & & & & & \\

		3. & Криттика & 12. & Уттар"=Пхалунги & 21. & Уттарашадха & Солнце & 6 лет /\\
		& \signum{26}{40}{\aries}--\signum{10}{}{\taurus}& & \signum{26}{40}{\leo}--\signum{10}{}{\virgo} & & \signum{26}{40}{\sagittarius}--\signum{10}{}{\capricornus} & & 2160\,дней\\
		& & & & & & & \\

		4. & Рохини & 13. & Хаста & 22. & Шравана & Луна & 10 лет /\\
		& \signum{10}{}{\taurus}--\signum{23}{20}{\taurus}& & \signum{10}{}{\virgo}--\signum{23}{20}{\virgo} & & \signum{10}{}{\capricornus}--\signum{23}{20}{\capricornus} & & 3600\,дней\\
		& & & & & & & \\

		5. & Мригашира & 14. & Читра & 23. & Дханишта & Марс & 7 лет /\\
		& \signum{23}{20}{\taurus}--\signum{6}{40}{\gemini}& & \signum{23}{20}{\virgo}--\signum{6}{40}{\libra} & & \signum{23}{20}{\capricornus}--\signum{6}{40}{\aquarius} & & 2520\,дней\\
		& & & & & & & \\

		6. & Ардра & 15. & Свати & 24. & Сатабиша & Раху & 18 лет /\\
		& \signum{6}{40}{\gemini}--\signum{20}{}{\gemini}& & \signum{6}{40}{\virgo}--\signum{20}{}{\virgo} & & \signum{6}{40}{\aquarius}--\signum{20}{}{\aquarius} & & 6480\,дней\\
		& & & & & & & \\

		7. & Пунарвасу & 16. & Висакха & 25. & Пурва"=Бхадрапада & Юпитер & 16 лет /\\
		& \signum{20}{}{\gemini}--\signum{3}{20}{\cancer}& & \signum{20}{}{\virgo}--\signum{3}{20}{\scorpio} & & \signum{20}{}{\aquarius}--\signum{3}{20}{\pisces} & & 5760\,дней\\
		& & & & & & & \\

		8. & Пушья & 17. & Анурадха & 26. & Уттар"=Бхадрапада & Сатурн & 19 лет /\\
		& \signum{3}{20}{\cancer}--\signum{16}{40}{\cancer}& & \signum{3}{20}{\scorpio}--\signum{16}{40}{\scorpio} & & \signum{3}{20}{\pisces}--\signum{16}{40}{\pisces} & & 6840\,дней\\
		& & & & & & & \\

		9. & Ашлеша & 18. & Джиешта & 27. & Ревати & Мер- & 17 лет /\\
		& \signum{16}{40}{\cancer}--\signum{0}{}{\leo}& & \signum{16}{40}{\scorpio}--\signum{0}{}{\sagittarius} & & \signum{16}{40}{\pisces}--\signum{0}{}{\aries} & курий & 6120\,дней\\

		\hline
	\end{tabular}
\end{table}

Группы лунных созвездий, их хозяева, главные периоды и подпериоды планет даны в таблице~\ref{tbl:stargroups}

Для определения времени, в котором будут действовать различные главные периоды и подпериоды планет, необходимо знать хозяев лунных созвездий, точные координаты Луны (с точностью до минут) и иметь под рукой таблицу главных периодов и подпериодов планет.

В нашем гороскопе-примере координаты Луны, равны \coord{14}{51}{12} Овна. Округлим до минут и получим \coord{14}{51}{0} Овна.

По таблице~\ref{tbl:nakshatras} определяем, что Луна в момент рождения человека находилась под созвездием Бхарани, границы которого проходят от \coord{13}{20}{0} Овна до \coord{26}{40}{0} Овна. Протяженность каждого созвездия равна \coord{13}{20}{0}, или \(800^\prime\).


\begin{table}[tph!]
	\caption{Подпериоды планет в годах, месяцах, днях}
	\label{tbl:stargroups}

	\centering

	% Расширить по вертикали
	\renewcommand{\arraystretch}{1}

	% Заполним данными
	\begin{tabular}{|p{.1\textwidth}|c|c|c|c|c|c|}
		\hline
		Главные & & & & & & \\
		периоды планет & Солнце & Луна & Марс & Раху & Юпитер & Венера \\
		\hline
		Солнце   & 3м 18д  & 6м     & 4м 6д    & 10м 24д   & 9м 18д    & 11м 12д   \\
		Луна     & 6м      & 10м 9д & 7м       & 1г 6м     & 1г 4м     & 1г 7м     \\
		Марс     & 4м 6д   & 7м     & 4м 27д   & 1г 18д    & 11м 6д    & 1г 1м 9д  \\
		Раху     & 10м 24д & 1г 6м  & 1г 18д   & 2г 8м 12д & 2г 4м 24д & 2г 10м 6д \\
		Юпитер   & 9м 18д  & 1г 4м  & 11м 6д   & 2г 4м 24д & 2г 1м 18д & 2г 6м 12д \\
		Сатурн   & 11м 12д & 1г 7м  & 1г 1м 9д & 2г 10м 6д & 2г 6м 12д & 3г 3д     \\
		Меркурий & 10м 6д  & 1г 5м  & 11м 27д  & 2г 6м 18д & 2г 3м 6д  & 2г 8м 9д  \\
		Кету     & 4м 6д   & 7м     & 4м 27д   & 1г 18д    & 11м 6д    & 1г 1м 9д  \\
		Венера   & 1г      & 1г 8м  & 1г 2м    & 3г        & 2г 8м     & 3г 2м     \\
		\hline
	\end{tabular}
	\newline
	\begin{tabular}{|p{.1\textwidth}|c|c|c|}
		\hline
		Главные & & & \\
		периоды планет & Сатурн & Меркурий & Кету \\
		\hline
		Солнце   & 10м 6д    & 4м 6д    & 1г \\
		Луна     & 1г 5м     & 7м       & 1г 8м \\
		Марс     & 11м 27д   & 4м 27д   & 1г 2м \\
		Раху     & 2г 6м 18д & 1г 18д   & 3г \\
		Юпитер   & 2г 3м 6д  & 11м 6д   & 2г 8м \\
		Сатурн   & 2г 8м 9д  & 1г 1м 9д & 3г 2м \\
		Меркурий & 2г 4м 27д & 11м 27д  & 2г 10м \\
		Кету     & 11м 27д   & 4м 27д   & 1г 2м \\
		Венера   & 2г 10м    & 1г 2м    & 3г 4м \\
		\hline
	\end{tabular}
\end{table}

Из таблицы~\ref{tbl:nakshatras} следует, что хозяином Бхарани является Венера и ее главный период равен 20 годам, или 7200 дням. Если бы, предположим, в момент рождения Луна оказалась в \coord{13}{20}{0} Овна, то человек вступил бы в двадцатилетний главный период Венеры. В нашем примере Луна находилась в \coord{14}{51}{0} Овна, а это значит, что главный период Венеры для данного человека будет неполный. Нам нужно определить какой путь Луна должна пройти, чтобы достичь отметки \coord{26}{40}{0} Овна. Найдем остаточный путь Луны:

\calc{\coord{26}{40}{0} - \coord{14}{51}{0} = \coord{11}{49} = 709^\prime}

Теперь высчитаем, сколько человеку осталось прожить в главном периоде Венеры от своего дня рождения:

\calc{\dfrac{7200}{800^\prime} = 9}

9 --- сколько дней приходится в главном периоде Венеры на одну угловую минуту пути Луны. Зная остаточный путь Луны в угловых минутах (\(709^\prime\)), определим, сколько дней осталось жить в главном периоде Венеры:

\calc{9 * 709^\prime = 6381}

В индийской астрологии используется система, в которой 1 год равен 360 дням и 1 месяц --- 30 дням. Эта система основана на периоде годового обращения Солнца (\(1^\circ\) равен 1 дню).

Для удобства дальнейшего подсчета переводим полученные дни в годы имесяцы:

\calc{\dfrac{6381}{360} = 17:8:21}

В нашем примере человек родился 4 декабря 1957 года. К этой дате рождения прибавим 17 лет 8 месяцев и 21 день и в результате получим дату, до которой будет действовать главный период Венеры. Затем в силу вступит главный период Солнца (6 лет), потом главный период Луны (10 лет), главный период Марса (7 лет), главный период Раху (18 лет), главный период Юпитера (16 лет) и\,т.\,д. В таблице~\ref{tbl:stargroups} вы увидите их последовательность.

После определения дат главных периодов приступим к определению дат подпериодов планет. Главный период Солнца начнется с подпериода Солнца, потом наступит подпериод Луны, подпериод Марса, подпериод Раху, подпериод Юпитера, подпериод Сатурна, подпериод Меркурия, подпериод Кету и последний подпериод Венеры главного периода Солнца. Таким же образом определяются даты подпериодов главного периода Луны, главного периода Марса, главного периода Раху, главного периода Юпитера и\,т.\,д. Для примера дадим даты девяти подпериодов главного периода Солнца:

\begin{table}[tph!]
	\centering

	% Расширить по вертикали
	\renewcommand{\arraystretch}{1.5}

	% Заполним данными
	\begin{tabular}{|lllll|}
		\hline
		гл. период & подпериод & день & месяц & год \\
		\hline
		Солнце & Солнце & 25 & август & 1975 \\
		Солнце & Луна & 13 & декабрь & 1975 \\
		Солнце & Марс & 13 & июнь & 1976 \\
		Солнце & Раху & 19 & октябрь & 1976 \\
		Солнце & Юпитер & 13 & сентябрь & 1977 \\
		Солнце & Сатурн & 1 & июль & 1978 \\
		Солнце & Меркурий & 13 & июнь & 1979 \\
		Солнце & Кету & 19 & апрель & 1980 \\
		Солнце & Венера & 25 & август & 1980 \\
		Луна & Луна & 25& август & 1981 \\
		\hline
	\end{tabular}
\end{table}

После того как вы определите даты подпериодов последующих главных периодов, предлагаемый гороскоп-пример будет приведен в соответствие с периодами жизни.
