\section{Порядок интерпретации гороскопа с использованием трех карт}

Для того, чтобы правильно интерпретировать гороскоп, надо прежде всего определить силу каждой из девяти планет, рассматривая их с учетом всех правил, которые были даны в предыдущих главах:
\begin{myenum}
	\item Определить достоинство планеты в знаке Зодиака.
	\item Определить расположение планеты в домах квадранта, тригона и неблагоприятных домах(6,\,8,\,12) в карте рождения.
	\item Оценить планету с точки зрения естественно благоприятной или естественно неблагоприятной.
	\item Отметить связь планеты в одном знаке с благотворными или неблаготворными планетами.
	\item Рассмотреть планету с точки зрения благоприятных или неблагоприятных аспектов.
	\item Обратить внимание на сгоревшие, ретроградные, находящиеся в первом и в последнем градусе знака, в позициях от \gradus{12} до \gradus{18}, а также в последних \gradus{6} мужских знаков и в первых \gradus{6} женских знаков планеты.
	\item Оценить планету с точки зрения ее диспозитора (сильный диспозитор усиливает рассматриваемую планету, а слабый --- ослабляет), например: Луна находится в Овне, хозяином которого является Марс, а Марс находится в Козероге, т.\,\е. в знаке экзальтации. В данном примере Марс является диспозитором Луны и будет ее усиливать.
	\item Отметить расположение планеты в домах квадранта, тригона и неблагоприятных домах в лунной карте.
	\item Рассмотреть главные достоинства планеты в карте навамса (в экзальтации, собственном и ослабленном знаках).
	\item Обратить внимание на расположение планеты в одном и том же знаке в карте рождения и карте навамса.
\end{myenum}


После того, как вы оцените силу каждой планеты исходя из десяти пунктов, перейдите к интерпретации гороскопа.

Если планета положительно оценивается по наибольшему количеству пунктов, то она рассматривается в гороскопе как благоприятная (даже если она является естественно неблагоприятной). Если планета отрицательно оценивается по наибольшему количеству пунктов, то она рассматривается в гороскопе как неблагоприятная (даже будучи благотворной по своей природе). Если планета оценивается как положительная и отрицательная в равной мере, то она рассматривается в гороскопе как нейтральная, то есть производит позитивные и негативные эффекты.

Известно, что двенадцать домов гороскопа дают знания о всех сферах жизни человека, но нет такого гороскопа, где бы все эти дома рассматривались положительно. Например, у человека прекрасная карьера, хорошие дети, комфортная жизнь, он пользуется всеобщим уважением, но несчастлив в супружеской жизни. Другой пример: человек может испытывать затруднения в работе, получать небольшую зарплату, переживать семейные конфликты и вместе с тем обладать отличным здоровьем, иметь большую продолжительность жизни. Все это вы можете увидеть в индивидуальном гороскопе.

Чтобы определить уровень какой--либо жизненной сферы, необходимо сделать анализ каждого дома в следующем порядке:

\begin{myenum}
	\item Рассмотреть планету в доме.
	\item Обратить внимание на эту планету как на хозяина дома или домов (Солнце и Луна являются хозяевами только одного дома).
	\item Оценить хозяина рассматриваемого дома (напрмер, если седьмой дом находится в Скорпионе, то рассмотреть положение Марса).
	\item Проанализировать дом с точки зрения аспектной связи (например, если пятый дом под аспектом Юпитера, то учесть благотворное влияние этой планеты).
	\item Обратить внимание на окружение дома благотворными и неблаготворными планетами (например, в шестом доме Марс, а в восьмом доме Раху, значит, седьмой дома окружен неблаготворными плаентами).
	\item Проанализировать состояние планеты, идеи которой тождественны идеям рассматриваемого дома (например, если анализируем первый дом, то параллельно надо рассматривать Солнце, если второй дом --- Юпитер, третий дом --- Марс, четвертый дом --- Луну, пятый дом --- Юпитер, шестой дом --- Марс, седьмой дом --- Венеру, восьмой дом --- Сатурн, девятый дом --- Юпитер, десятый дом --- Солнце, одиннадцатый дом --- Юпитер, двенадцатый дом --- Сатурн).
	\item С учетом выше изложенного рассмотреть дома в лунной карте.
	\item Рассмотреть дополнительный дом от планеты, идеи которой соответствуют идеям анализируемого дома (например, Юпитер и пятый дом отвечают за детей, поэтому надо рассмотреть пятый дом от Юпитера. Другой пример: Луна и четвертый дом отвечают за мать, значит, надо рассмотреть четвертый дом от Луны).
\end{myenum}

Используя предложенную методику определения достоинств планет и домов гороскопа, вы сможете глубоко понять как самого человека, так и его судьбу. Вначале это будет сделать нелегко из--за кажущегося противоречия перечисленных для анализа пунктов. Поэтому вы должны в первую очередь обратить внимание на те факты, которые не один раз подтверждаются в гороскопе. Чем больше одни и те же факты указывают на силу или слабость каких--либо домов и планет, там явственнее их идеи будут выступать в жизни человека как позитивные или негативные. Покажем это на примере гороскопа женщины:

\begin{table}[tph!]
	% Расширить по вертикали
	%\renewcommand{\arraystretch}{1.5}

	% Заполним данными
	\begin{tabular}{|lll|}
		\hline
		Асцендент & \cord{8}{38} & Тельца \\
		Солнце   & \cord{1}{40}  & Скорпиона \\
		Луна     & \cord{28}{55} & Водолея \\
		Марс     & \cord{9}{19}  & Скорпиона \\
		Меркурий & \cord{15}{56} & Весов \\
		Юпитер   & \cord{8}{29}  & Козерога \\
		Венера   & \cord{14}{38} & Весов \\
		Сатурн   & \cord{1}{56}  & Козерога \\
		Раху     & \cord{29}{15} & Рака \\
		Кету     & \cord{29}{15} & Козерога \\ \hline
	\end{tabular}
\end{table}

\begin{table}
	\caption{Карта рождения}
	\natal[asc=2,three=РАХУ,six=ВЕНЕРА\\МЕРКУРИЙ,seven=СОЛНЦЕ\\МАРС,nine=КЕТУ\\ЮПИТЕР\\САТУРН,ten=ЛУНА]{}
\end{table}

\begin{table}
	\caption{Лунная карта}
	\natal[asc=11,one=ЛУНА,six=РАХУ,nine=МЕРКУРИЙ\\ВЕНЕРА,ten=МАРС\\СОЛНЦЕ,twelve=САТУРН\\ЮПИТЕР\\КЕТУ]{}
\end{table}

\begin{table}
	\caption{Карта навамса}
	\natal[asc=12,one=РАХУ\\ЮПИТЕР,four=ЛУНА,five=СОЛНЦЕ,seven=МАРС\\КЕТУ,ten=САТУРН,twelve=МЕРКУРИЙ\\ВЕНЕРА]{}
\end{table}

\subsubsection*{Определение силы планет}

Солнце --- планета в нейтральном знаке, естественно неблагоприятная, расположена в неблагоприятном промежутке от \gradus{0} до \gradus{6} женского знака, в квадранте карты рождения, в квадранте лунной карты, связана с неблагоприятным по своей природе Марсом, находится под аспектом неблагоприятной по своей природе Раху, диспозитор --- Марс (Марс в собственном знаке и в квадранте от восходящего знака и от Луны является сильным).

Луна --- планета в дружественном знаке, естественно благоприятная (растущая Луна), расположена в неблагоприятном промежутке от \gradus{24} до \gradus{30} мужского знака, в квадранте карты рождения (расположение Луны в первом доме лунной карты не берется в расчет, поскольку во всех картах она будет иметь такую же позицию), находится под аспектом естественно неблагоприятного Марса, диспозитор --- Сатурн.

Марс --- планет в собственном знаке, естественно неблагоприятная, в квадранте карты рождения, в квадранте лунной карты, связана с неблагоприятным по своей природе Солнцем, находится под аспектом естественно неблагоприятной Раху, диспозитора не имеет, так как стоит в собственном знаке.

Меркурий --- планета в нейтральном знаке, естественно благоприятная, находится в благоприятном промежутке от \gradus{12} до \gradus{18} знака, расположена в неблагоприятном шестом доме карты рождения, находится в тригоне лунной карты, связан с Венерой, имеет аспект от неблагоприятного Сатурна, диспозитор --- Венера.

Юпитер --- планета в ослабленном знаке, естественно благоприятная, расположена в тригоне карты рождения, находится в неблагоприятном двенадцатом доме лунной карты, связана с естественно неблагоприятными Сатурном и Кету, испытывает аспект естественно неблагоприятной Раху, в карте навамса расположена в собственно мзнаке, диспозитор --- Сатурн, зажат Сатурном и Кету.

Венера --- планета в собственном знаке, естественно благоприятная, находится в благоприятном промежутке от от \gradus{12} до \gradus{18} знака, расположена неблагоприятном шестом доме карты рождения, находится в тригоне лунной карты, связана с Меркурием, испытывает аспект Сатурна, диспозитора не имеет, так как стоит в собственном знаке.


Сатурн --- планета в собственном знаке, естественно неблагоприятная, находится в неблагоприятном промежутке от \gradus{0} до \gradus{6} женского знака, расположена в тригоне карты рождения, в неблагоприятном двенадцатом доме лунной карты, связана с естественно благоприятным Юпитером и неблагоприятной Кету, стоит под аспектом естественно неблагоприятной Раху, в карте навамса пребывает в собственном знаке, диспозитора не имеет, так как расположена в собсвтенном знаке.

Раху --- по знаку планета силы не имеет, естественно неблагоприятная, находится в последнем градусе знака, что является неблагоприятным для нее, расположена в третьем доме карты рождения, в шестом доме лунной карты, испытывает аспект естественно благоприятного Юпитера и неблагоприятного Сатурна (аспект Кету не учитывается, так как Кету всегда расположена противоположно Раху), диспозитор --- Луна.

Кету --- по знаку планета силы не имеет, естественно неблагоприятная, находится в последнем градусе знака, расположена в тригоне карты рождения, в неблагоприятном двенадцатом доме лунной карты, связана с естественно благоприятным Юпитером и неблагоприятным Сатурном, диспозитор --- Сатурн.

На основании данного анализа планет очень легко определить, какие из них --- сильные, нейтральные и слабые.

Солнце имеет позитивные и негативные результаты анализа, но так как оно находится в квадранте в двух картах, то будет обладать силой. Поэтому сила Солнца будет превышать среднюю силу планеты.

Луна имеет больше положительных признаков и поэтому является достаточно сильной в гороскопе.

Марс, как и Луна, также сильная планета.

Меркурий имеет нейтральную силу.

Юпитер --- самая слабая планета этого гороскопа.

Венера по многим признакам благоприятна и поэтому является сильной планетой.

Сатурн также обладает силой.

Раху присуще больше неблагоприятных признаков, и поэтому она не имеет благотворной силы.

Кету сильнее, чем Раху, так как находится в благоприятном девятом доме и Юпитер ближе, чем Сатурн, расположен к ней.

Сильные планеты окажут благотворное воздействие в тех областях жизни, которые находятся под их управлением. Ослабленные планеты принесут отрицательные результаты.

Теперь сделаем анализ каждого дома и определим, как они будут  себя проявлять в жизни этой женщины.

\subsubsection*{Первый дом}
Восходят Телец и созвездие Криттика. Венера как хозяйка Тельца находится в Весах в шестом доме вместе с Меркурием и под аспектом Сатурна (надо учесть, что Меркурий является хозяином второго и пятого домов, а Сатурн --- хозяином девятого и десятого домов). Первый дом находится под аспектами Солнца, Марса, Юпитера и Кету. Солнце, как и первый дом, представляет личность и характер человека, поэтому рассмотрим его со всех точек зрения. Обратим внимание на знак, занимаемый Луной. Это --- Водолей, который является вторым восходящим знаком и находится под воздействием Марса.

Выводы: эта женщина имеет привлекательную внешность и красивую фигуру (восходящие Телец и Криттика, Телец под влиянием Юпитера и Солнца). Она упряма, но терпелива (восходящий Телец), способна выдерживать большие физические нагрузки (Марс и Солнце аспектируют первый дом). Имеет хорошее здоровье и жизнеспособность (хозяин первого дома, испытывает влияние хозяев девятого и десятого домов). Склонна к возражениям и критике (Солнце в Скорпионе под влиянием Марса).

\subsubsection*{Второй дом}
В этом доме нет планет, но он находится под аспектами Марса и Раху. Марс оказывает влияние на второй дом, как хозяин седьмого и двенадцатого домов. Второй дом расположен в Близнецах, которыми правит Меркурий, находящийся в неблагоприятном шестом доме. Хозяин второго дома испытывает влияние Венеры и Сатурна. Юпитер, как и второй дом, представляет богатство и деньги, он находится в ослабленном положении. В лунной карте второй дом стоит в Рыбах, хозяином которых является Юпитер, расположенный в двенадцатом доме между Сатурном и Кету. Второй дом в лунной карте находится под влиянием Раху и Сатурна. Во втором доме от Юпитера --- Луна под аспектом Марса.

Выводы: этот дом ослаблен. Женщина будет испытывать материальные трудности, страдать зубной болью, устраивать конфликты в семье. Но, так как хозяин второго дома связан с сильной Венерой, это дает ей шанс в определенные периоды жизни улучшить свое материальное положение.

\subsubsection*{Третий дом}
Раху стоит в третьем доме. Луна является хозяйкой третьего дома и расположена в десятом доме под аспектом огненного Марса. Третий дом находится под аспектом Сатурна, Юпитера и Кету. Идеи третьего дома схожи с идеями Марса, поэтому проанализируем положение последнего. Марс хорошо расположен в гороскопе, он достаточно силен. В лунной карте третий до находится в Овне, а его хозяином является Марс, который знанимает десятый дом. Венера и Меркурий в лунной карте аспектируют третий дом. В третьем доме от Марса находятся Сатурн, Юпитер, Кету, а Раху аспектирует его.

Выводы: третий дом представляет жизнестойкость, мужество, решительность, руки, братьев, сестер и\,т.\,д. Из анализа третьего дома мы видим, что он в основном имеет связь с естественно неблагопрятными планетами, которые являются стимулом для развития решительности, жизнестойкости. Этот дом указывает н ато, что в своей жизни женщина много будет заниматься ручным трудом, и ее сила воли с каждым годом будет укрепляться. Родных братьев и сестер она не имеет, т.\,к. третий дом в основном имеет жесткие энергии.

\subsubsection*{Четвертый дом}
Расположен во Льве, хозяином которого является Солнце, находящееся в благоприятном седьмом доме. Четвертый дом испытывает влияние от Луны. Эта планета, как и четвертый дом, представляет мать, поэтому сделаем ее анализ. В лунной карте четвертый дом находится в Тельце, хозяином которого является Венера, расположенная в благоприятном девятом доме. Четвертый лунный дом находится под аспектами Солнца, Марса, Юпитера и Кету.

Выводы: в основном четвертый дом состоит из положительных признаков, которые являются указанием на хорошую мать и дружеские отношения с ней. Владелица гороскопа имеет в своем распоряжении значительной недвижимое имущество и хороших родственников.

\subsubsection*{Пятый дом}
Этот дом расположен в Деве, хозяином которого являетс Меркурий, стоящий в шестом доме рядом с Венерой и под аспектом Сатурна. Юпитер и Кету оказывают влияние на пятый дом. Юпитер как планета, управляющая детьми, зажат неблагоприятным Сатурном и Кету и испытывает на себе влияние Раху. В лунной карте пятый дом расположен в Близнецах, хозяином которого является Меркурий, находящийся в благоприятном девятом доме вместе с Венерой и под аспектом Сатурна. Пятый дом в лунной карте находится под аспектом Марса и Раху. Рассмотрим пятый дом от Юпитера, который расположен в Тельце и находится под влиянием Солнца, Марса, Кету и Юпитера. Хозяин пятого дома от Юпитера расположен в десятом доме вместе с Меркурием и находится под аспектом Сатурна.

\subsubsection*{Шестой дом}
Здесь расположены Венера и Меркурий. Первыя является хозяйкой первого и шестого домов, а второй --- хозяином второго и пятого домов. Венера усиливает шестой дом, поскольку его хозяйка находится в нем. Сильный Сатурн влияет на шестой дом. Рассмотрим Марс, так как он является планетой споров и конфликтов. В лунной карте в шестом доме расположена Раху под аспектами Сатурна и Юпитера, а хозяин шестого дома находится в первом доме. Обратим внимание на шестой дом от Марса, который стоит в Овне под влиянием Венеры и Меркурия и хозяин которого расположен в первом доме вместе с Солнцем и под влиянием Раху.

Выводы: в данном гороскопе этот дом очень сильный и указывает на высокий служебный долг женщины и на ее большие возможности в победе над недоброжелателями.

\subsubsection*{Седьмой дом}
Две огненные планеты --- Солнце и Марс расположены в седьмом доме. Солнце является хозяином четвертого дома, а Марс --- седьмого и двенадцатого. Марс активизирует идеи седьмого дома, так как является его хозяином и находится в нем. Раху влияет на седьмой дом. Венера, которая отражает главные идеи седьмого дома, расположена достаточно хорошо. В лунной карте хозяин седьмого дома находится в десятом доме вместе с Марсом и под аспектом Раху.

Выводы: этот дом очень активный и имеет жесткие энергии, поэтому супружеская жизнь женщины будет не лишена споров и конфликтов. В общественной сфере она будет пользоваться вниманием, что будет приносить ей радость.


\subsubsection*{Восьмой дом}
Он находится в Стрельце и не имеет планет. Хозяином восьмого дома является Юпитер, который расположен в девятом доме между Сатурном и Кету. Только Кету влияет на восьмой дом. Он окружен такими неблагоприятными планетами, как Сатурн, Кету, Марс и Солнце. Но планета Сатурн, которая отвечает за долголетие, достаточно сильна и имеет связь с благоприятным по природе Юпитером. Юпитер по долготе ближе, чем Кету и Раху, находится к Сатурну. В лунной карте хозяин восьмого дома стоит в девятом доме и дублирует позицию хозяина восьмого дома в карте рождения. Восьмой дом в лунной карте испытывает влияние Кету и Юпитера. Рассмотрим также восьмой дом от Сатурна, который аспектирует растущая Луна и хозяин которого находится в одинадцатом доме.

Выводы: если рассматривать восьмой дом с точки зрения долголетия, то сильный Сатурн свидетельствует о хорошей продолжительности жизни. Юпитер, который ослаблен и является хозяином восьмого дома, говорит о том, что женщине предстоит пережить смерть других людей, в том числе и ее родственников.

\subsubsection*{Девятый дом}
Сатурн, Юпитер и Кету занимают девятый дом. Сатурн является хозяином девятого и десятого домов, а Юпитер --- восьмого и одинадцатого. Сатурн усиливает девятый дом, он хозяин этого дома и находится в нем. В лунной карте в девятом доме расположены Венера и Меркурий, которые находятся под аспектом Сатурна. Венера, как хозяйка девятого дома, находится в нем и улучшает его. Рассмотрите девятый дом от Юпитера и узнаете дополнительно о духовной жизни человека. Оцените девятый дом от Солнца и вы получите дополнительную информацию об отце.

Выводы: идеи этого дома будут ярко проявляться в жизни женщины. Сильный девятый дом обещает долгую жизнь и ее отцу. Так как благотворный Юпитер, руководящий духовной жизнью, ослаблен, а неблагопритяный Сатурн, который также отвечает за духовность, усилен и является хозяином девятого дома, то в жизни этой женщины будет много противоречий в отношении ее духовного развития.

\subsubsection*{Десятый дом}
Луна находится в десятом доме под влиянием Марса и является хозяйкой третьего дома. Хозяин десятого дома расположен в девятом доме и испытывает влияние Юпитера, Раху, Кету. Десятый дом, как и девятый, руководит отцом, поэтому надо рассмотреть положение Солнца. В лунной карте в десятом доме находятся Солнце и Марс. Солнце является хозяином седьмого дома, а Раху аспектирует десятый дом. Рассмотрим десятый дом от СолнцаЖ он находится во Льве под аспектом Луны, а его хозяин расположен в первом доме вместе с Марсом и под аспектом Раху.

Выводы: в гороскопе этой женщины благоприятный десятый дом указывает на удачу в работе, а также на хорошего и внимательного отца.

\subsubsection*{Одинадцатый дом}
В одинадцатом доме нет планет, но хозяин этого дома --- Юпитер расположен в девятом доме между Сатурном и Кету. Юпитер --- самая слабая планета в данном гороскопе. Сатур и Раху влияют на одинадцатый дом. Юпитер --- хозяин второго дома, в лунной карте находится в двенадцатом доме. Так как эта планета управляет богатством и приобретениями, то рассмотрим одинадцатый дом от нее, в которо находятся Солнце и Марс под влиянием Раху.

Выводы: этот дом ослаблен, и поэтому у женщины нет никакой возможности хорошо зарабатывать ни на своей работе, ни на какой-нибудь другой производственной или непроизводственной сфере. Одинадцатый дом также указывает на взаимоотношения с друзьями, а поскольку он слабый, женщина не будет испытывать особой привязанности к ним.

\subsubsection*{Двенадцатый дом}
Здесь нет планет, но хозяин дома расположен в седьмом доме вместе с Солнцем и находится под аспектом Раху. Венера и Меркурий аспектируют двенадцатый дом. Идеи двенадцатого дома схожи с идеями Сатурна, поэтому проанализируем положение Сатурна. В лунной карте в двенадцатом доме стоят Сатурн, Юпитер и Кету. Сатурн в лунной карте является хозяином двенадцатого дома и находится в нем. Двенадцатый дом лунной карты испытывает на себе влияние Раху. Двенадцатый дом от Сатурна не имеет планет, но его хозяин (Юпитер) ослаблен и занимает первый дом.

Выводы: так как хозяином двенадцатого дома является Марс, то он приведет к материальным потерям. Будучи расположенным в седьмом доме он указывает на то, что женщина будет тратить деньги на своего мужа. Ослабленный Юпитер в двенадцатом доме лунной карты свидетельствует об обременительных расходах, связанных с детьми. Юпитер в женском гороскопе представляет мужа и поэтому еще раз подтверждает, что деньги будут уходить на него. Венера, которая влияет на двенадцатый дом карты рождения, в третий раз подтверждает расходы на супруга.

На примере гороскопа женщины мы кратко описали жизненные сферы, не заостряя внимание на деталях. Когда вы поймете, как делаются главные выводы из гороскопа, вы сможете приблизиться к изучению деталей и расширить со временем свои знания в области индийской астрологии.

Итак, мы рассмотрели судьбу женщины в общих чертах, без учета времени событий. Но, то, что показывает гороскоп, должно проявиться как результат кармической реакции в определенный период жизни, и об этом мы расскажем в следующих главах
Здесь нет планет, но хозяин дома расположен в седьмом доме вместе с Солнцем и находится под аспектом Раху. Венера и Меркурий аспектируют двенадцатый дом. Идеи двенадцатого дома схожи с идеями Сатурна, поэтому проанализируем положение Сатурна. В лунной карте в двенадцатом доме стоят Сатурн, Юпитер и Кету. Сатурн в лунной карте является хозяином двенадцатого дома и находится в нем. Двенадцатый дом лунной карты испытывает на себе влияние Раху. Двенадцатый дом от Сатурна не имеет планет, но его хозяин (Юпитер) ослаблен и занимает первый дом.

Выводы: так как хозяином двенадцатого дома является Марс, то он приведет к материальным потерям. Будучи расположенным в седьмом доме он указывает на то, что женщина будет тратить деньги на своего мужа. Ослабленный Юпитер в двенадцатом доме лунной карты свидетельствует об обременительных расходах, связанных с детьми. Юпитер в женском гороскопе представляет мужа и поэтому еще раз подтверждает, что деньги будут уходить на него. Венера, которая влияет на двенадцатый дом карты рождения, в третий раз подтверждает расходы на супруга.

На примере гороскопа женщины мы кратко описали жизненные сферы, не заостряя внимание на деталях. Когда вы поймете, как делаются главные выводы из гороскопа, вы сможете приблизиться к изучению деталей и расширить со временем свои знания в области индийской астрологии.

Итак, мы рассмотрели судьбу женщины в общих чертах, без учета времени событий. Но, то, что показывает гороскоп, должно проявиться как результат кармической реакции в определенный период жизни, и об этом мы расскажем в следующих главах.


