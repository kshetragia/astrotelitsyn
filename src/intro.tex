\tocsection{К ЧИТАТЕЛЮ}

Уважаемый читатель, для того чтобы древнейшая наука --- Индийская предсказательная астрология(ИПА) стала для вас действительно простой и доступной, не надо начинать читать эту книгу второпях, озабоченным домашними делами, работой, тем более устраивать коллективные чтения. Выберите время, приведите в порядок свои мысли, успокойте душу и постарайтесь сделать так, чтобы чтение не стало праздным развлечением. Соприкосновение с Учением, которое чудесным образом возникло с незапамятных времен истории человечества, было сохранено и донесено до наших дней народом Индии благодаря его глубокой привязанности к традициям, --- это соприкосновение с высочайшей духовностью и нравственностью, ибо только на таком уровне сознания человек мог создать совершенное Учение --- Джотиш-Веду, что в переводе с санскрита означает ``Свет Знания''. Вот как поэтично сказано об этом в Гимне из Риг-Веды:

\quoteit{
	Открой своему сознанию вечный свет \\
	и высшее блаженство! \\
	Ибо ты, кто знает путь, ведет того, \\
	Кто ищет и просит помощи. \\
}

Будущее не является загадкой для людей, владеющих методикой ИПА~\citep{ojha}.

Авторы убеждены, что, ознакомившись с содержанием этой книги, любой человек, умеющий логически мыслить и знающий математику в объеме программы средней школы, сможет легко расчитать гороскоп и самостоятельно определить основные события своей жизни: периоды удач и невезений, болезней и выздоровлений, поездок, брака, рождения детей, финансового состояния и\,т.\,д. Изучение астрологии полезно начинать с составления собственного гороскопа, так как в этом случае вероятность будущих событий гарантируется точным знанием своего прошлого и совпадением минувших событий с указаниями гороскопа.

Теоретическая часть книги подготовлена на основе трудов известных индийских астрологов Вишванат Дева Сармы~\citep{sarma} и Бангалор Венката Рамана~\citep{raman}, а также философских комментариев к Ведическому Учению величайшего ученого йогина современности Махариши Махеш Йоги~\citep{maharishi}.

Чтобы книга была понятной широкому кругу читателей, авторы старались избегать терминологии на санскрите, употребляемой в ИПА, максимально используя традиционные для западной астрологии названия, понятия и обозначения (вместе с тем читатель убедится, что индийская астрологическая система коренным образом отличается от западной). Особое внимание в ней уделено практической части, где досконально, на конкретных примерах, подтверждено каждое теоретически обоснованное правило составления гороскопа. Важную роль в этом сыграл опыт Сергея Станиславского, расчитавшего по системе ИПА более 2500 гороскопов, многие предсказания которых сбылись.

Вне сомнений, материал, изложенный в книге, --- это только первый шаг в изучении безграничного кладезя знаний древнейшей и увлекательнейшей из наук, однако с его помощью человек приоткроет для себя какую-то, пусть небольшую часть наиболее охраняемой природой тайны --- тайны будущего.
