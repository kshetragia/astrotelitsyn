\quoteintro{
	В одно мгновенье видеть вечность \\
	Огромный мир --- в зерне песка \\
	В единой горсти --- бесконечность \\
	И небо --- в чашечке цветка
} {Уильям Блейк}

\tocsection{ВВЕДЕНИЕ}

Ведическое Учение было принесено в Индию ариями более 5000 лет назад. Версии возникновения самого народа --- ариев и памятников древнеиндийской литературы --- Вед, появившихся на базе их культуры, пока что находятся в области преданий, догадок и эзотерики. Современная археология обнаруживает следы арийской культуры семитысячелетней давности. Так, известный украинский археолог Ю. Шилов определил основные места арийских поселений в низовьях Днепра. Находки, сделаные здесь во время раскопок курганов, свидетельствуют о том, что Ведическое Учение существовало 7000 лет назад.

Современные ученые, изучая ведические тексты, отмечают, что они содержат все знания, необходимые человеку в материальной и духовной жизни. В Ведах, написанных тысячелетия назад, есть готовые формулы квантовой физики, вычисления расстояния от Земли до Солнца и других планет Солнечной системы. Ведическая медицина --- Аюр-Веда (``Наука жизни``) считается непревзойденной до сих пор.

Еще в начале прошлого столетия французский физик, математик и астроном Пьер Лаплас писал, что для Разума, постигшего все законы, действующие во Вселенной, и ее строение, будущее будет таким же ясным, как прошлое и настоящее. В наш век, когда созданы компьютеры, способные выполнять миллиарды операций в секунду, ученые могут описать поведение любой частицы объекта и любое явление в природе. И все же при таких технических возможностях мы часто не в силах решить обычные жизненные проблемы и ежедневно совершаем ошибки.

Конечно, одному человеку не под силу изучить абсолютно все причинно-следственные связи, чтобы с математической точностью просчитать каждый шаг в течение дня. Охватить одновременно все области знаний невозможно, а найти их основу? Это была мечта многих великих ученых --- создать теорию единства всех законов природы. В конце 70-х годов нашего столетия американский профессор Джон Хэгелин предложил теорию суперсимметрии, которая сводит все законы природы в единое поле: из него возникает Вселенная и в нем же происходит взаимодействие ее объектов. Теперь каждый углубленно изучающий физику человек знает, что как бесцветный сок лежит в основе зеленого листа, коричневого стебля и розового лепестка, так и единое поле лежит в основе всех явлений Вселенной. ``Единое поле, --- говорит Джоно Хэгелин, --- есть не что иное, как простейшая форма сознания``.

В Европейском центре ядерных исследований в Женеве (Швейцария), который специализируется на исследованиях в области сознания, была апробирована методика, названная Технологией Единого Поля, или Техникой Трансцедентальной Медитации(ТМ), практикуя которую человек достигает в собственном сознании единого поля, обучаясь действовать спонтанно правильно в соответствии со всеми законами природы. Предложил эту технику индийский физик, философ и йогин Махариши Махеш Йоги. Результаты ТМ оказались действительно феноменальны. Опубликованные данные более 500 независимых научных исследований, проведенных в 27 странах мира, свидетельствуют о благотворном влиянии ТМ на все сферы человеческой жизни. Изменения в электроэнцефалограмме медитирующего человека указывали на состояние расслабления и одновременно бодрствования. Полностью когерентное, то есть согласованное, функционирование головного мозга во время трансцедентальной медитации --- это идеальная подготовка человека к активной деятельности. Институт головного мозга в Москве, проводивши исследования по программе ТМ, рекомендует ввести ее в систему образования.

Технология Единого Поля, подаренная миру Махариши, уходит корнями в Ведическое Учение. Ведические тексты описывают сознание как высшую деятельность человека и Мироздания, область чистого существования, безграничное поле энергии и разума, которое содержит все явления природы. Человек, овладевший методикой ТМ, постигает в своем сознании Единое Поле, получая информацию о прошлом и будущем.

В истории человечества всегда были люди, достигавшие путем очищения сознания таких высот духовного развития, которые позволяли им находиться на уровнях Единого Поля и нести людям Чистое Знание. Мы не будем касаться выдающихся личностей, окруженных ореолом божественного предназначения на Земле, учитывая, что книга расчитана на читателей с различными взглядами на взаимосвязь духовного и материального, но в качестве примера назовем несколько имен извстных людей, способности которых не вызывают сомнения. Это великие йогины древней Индии Тилопа и Наропа, йогины древнего Тибета Марпа и Миларепа, а также наши современники --- йогины-риши (видящие), Гуру Дев (оставил тело в 1953 году) и Махариши Махеш Йоги.

К сожалению, пока еще мало людей, имеющих возможность черпать нужную информацию из собственного сознания, а ведь знание будущего полезно и просто необходимо человеку, чтобы понять собственное предназначение в системе космической эволюции. Человек, владеющий таким знанием, становится спокойным, целеустремленным и крепким духом. Для него нет неожиданностей, нет излишних ментальных волнений, его действия гармонично сочетаются с законами природы. Значительно сокращается количество стрессов и, как следствие, улучшается психическое и физическое здоровье. Но если все-таки судьба сулит человеку неприятности, то, зная об этом, он заранее может смягчить ее удар, подготовив себя на ментальном уровне и укрепив духовно к принятию неизбежного.

Многие выдающиеся люди верили в астрологию и занимались ею. Данте признавал эту науку благороднейшей. Пифагор, Демокрит, Птолемей, Кеплер, Бэкон и Морин были известными астрологами.

Наука о влиянии звезд и планет на судьбу человека достигла наивысшего развития в Индии задолго до так называемого ``периода достоверной истории''. Общепризнанным родоначальником ее является Парашара Муни, живший более 5000 лет назад незадолго до начала эпохи Кали Юга. Его труды представляют собой наиболее древние письменные комментарии к Атхарва-Ведам (часть Вед, посвященная астрологии). Изложенные Парашарой принципы лежат в основе всех последующих трудов по астрологии, в том числе и современных.

Как течение реки ограничено ее берегами, так и действия человека регламентированы законами эволюционного развития Вселенной. Подчиняясь этим законам, звезды и планеты не меняют свой курс и с такой же неизменной силой ведут к вершине эволюции --- полному единению с природой --- каждого человека.

Часто люди, не знакомые с ведическим мировоззрением, задают вопрос: ``Каким образом планеты, находясь так далеко в космическом пространстве, могут влиять на поведение человека на Земле?'' По Ведическому Учению Вселенная --- это огромное информационно-энергетическое пространство, которое является основой всего сущего, в том числе и человека. Звезды и планеты рассматриваются не как физические тела, а как энергетические узлы в этом пространстве. Непрерывное движение и расположение планет относительно друг друга и относительно созвездий Зодиака становится причиной непрерывного изменения баланса энергии во Вселенной, в том числе в тонких энергетических структурах человека, которые, отражаясь от индивидуальной нервной системы, приводят к соответствующим действиям. Если человек знаком с процессом, происходящим в этом огромном компьютере --- Вселенной, и знает правила пользования им, он может предсказать свои действия на много лет вперед.

Различные астрологические школы придерживаются двух систем расчета гороскопа --- Саяны и Нираяны. В переводе с санскрита ``аяна'' --- продвижение, или прогресс, отсюда Саяна является системой подвижного Зодиака, а Нираяна --- неподвижного. Разница между координатами планет в этих системах определяется Аянамсой (аяна --- движение, амса --- часть). Подробнее об этом будет рассказано в соответствующей главе книги. Отметим лишь, что система Нираяна на практике дает наиболее точные результаты, поэтому предлагается читателям.

Индийская предсказательная астрология имеет несколько направлений, которые по сути образуют самостоятельные астрологические науки:

\begin{mylist}
	\item \emph{натальная астрология} --- имеет отношение к рождению и жизни конкретного человека;
	\item \emph{мунданная астрология} --- прогнозирует события в масштабах нации или государства;
	\item \emph{медицинская астрология} --- связана с течением и исходом болезней;
	\item \emph{хорарная астрология} --- отвечает на конкретно заданный вопрос в данное время;
	\item \emph{мухурта} --- помогает выбрать благоприятное время для различных видов деятельности;
	\item \emph{васту шастра} --- отностися к архитектуре;
	\item \emph{атмосферно-метеорологическая астрология} --- составляет прогноз погоды и\,т.\,д.
\end{mylist}

В этой книге основное внимание уделено натальной астрологии как наиболее популярной.

Полное название предлагаемой астрологической системы --- Вимсоттари Даша Джанана и Бикара от Нираяна Калачакра. Натальный раздел (Вимсата означает 120; Даша --- период; Джанана --- расчет времени события; Бикара --- определение сущности события в данный период времени; Калачакра --- колесо времени, или Зодиак с девятью планетами, семь из которых являются небесными телами, а две математически предсказанные).

По мнению Вишванат Дева Сармы, теория 120 лет базируется на идее прохождения эволюционного процесса в циклических фазах со стодвадцатилетним периодом (так же, как дни и ночи, сезоны и годы, которые имеют циклическую повторяемость и используется для познания объектов природы)

Несмотря на громоздкость полного названия, система довольно проста и достаточно точна на практике. В этом читатель, несомненно убедится, полностью изучив материал, изложенный в книге. Удачи вам!
