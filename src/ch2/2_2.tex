\section[Анализ планетного подпериода]{Анализ планетного подпериода в основе главного планетного периода}

Каждый главный период имеет девять подпериодов и определяется при помощи таблицы~\ref{tbl:stargroups}. Принципы и правила, которые применяются для определения качества планет главного периода, остаются теми же и для определения качества планеты подпериода. Если планета главного периода оценивается как сильная, то планета подпериода будет тоже сильной. Все зависит от того, в основе какого главного планетного периода будет действовать планетный подпериод. Если планеты главного периода и подпериода являются сильными, то человека ждеть большая радость, счастье, удача и процветание, когда планеты этих периодов вступят в силу. Если планеты главного периода и подпериода слабые, то человек испытывает большие неудачи и потрясения, когда планеты этих периодов вступят в силу. Если планета главного периода сильная, а планета подпериода слабая, то преимущество надо отдать планете главного периода, хотя планета подпериода в свои сроки принесет некоторые неудобства. Если планета главного периода является слабой, а планета подпериода сильной, то последняя не сможет коренным образом изменить судьбу человека к лучшему, хотя и даст ему возможность ощутить позитивные изменения. Если планета подпериода находится в домах квадранта или тригона, считая от планеты главного периода, то подпериод этой планеты улучшится. Если планета подпериода находится в шестом, восьмом или двенадцатом доме, считая от планеты главного периода, то подпериод этой планеты ухудшится. Если планета главного периода враждебна планете подпериода, то подпериод ухудшится. Если планета главного периода дружественна планете подпериода, то подпериод улучшится.

Если планета главного периода является хозяином квадранта, а планета подпериода --- хозяином тригона, то подпериод улучшается. То же произойдет, если планета главного периода является хозяином тригона, а планета подпериода --- хозяином квадранта. Если планета главного периода является хозяином шестого, восьмого или двенадцатого дома, а планета подпериода является хозяином одного из этих же домов, то подпериод ухудшается.

Если планета главного периода естественно благоприятная и планета подпериода также естественно благоприятная, то подпериод улучшается. Если планеты главного периода и подпериода естественно неблагоприятные, то подпериод ухудшается.

Чтобы определить дополнительные эффекты в подпериод какой--либо планеты, надо планету главного периода поставить в первый дом и рассмотреть планету подпериода относительно планеты главного периода, учитывая занимаемый дом и то, хозяином каких домов она является. Например, рассмотри главный период Марса, который расположен в Овне в десятом доме, и подпериод Сатурна, находящегося в Козероге в седьмом доме. Помимо того, что надо рассмотреть положение Марса по десятому дому, а Сатурна по седьмому дому, вы также должны определить положение Сатурна относительно позиции Марса, то есть составить дополнительную карту, где Марс как планета главного периода будет находиться в первом доме, тогда Сатурн относительно марса бедт стоять в десятом доме и являться хозяином десятого и одинадцатого домов.

В предыдущей главе мы сделали анализ главных планетных периодов в двух гороскопах. В этой главе мы рассмотрим их планетные подпериоды, определив природу и время событий. 

\subsubsection*{Гороскоп мужчины}

\natal[%
	asc=8,
	two=МАРС\\РАХУ,
	five=ЛУНА,
	eight=КЕТУ\\ЮПИТЕР,
	nine=МЕРКУРИЙ,
	ten=СОЛНЦЕ,
	eleven=ВЕНЕРА,
	twelve=САТУРН
]{}

Рассмотрим подпериод Меркурия в главном периоде Меркурия (возраст 9--10 лет). В это время школьник станет более серьезно относиться к учебе, и его успеваемость повысится. На это указывает расположение Меркурия в пятом доме лунной карты. Так как Меркурий представляет друзей и в карте рождения является хозяином одинадцатого дома (друзья), то в этот период ребенок будет находиться в тесных контактах со своими друзьями и иметь благоприятный шанс обрести новых друзей. Меркурий в карте рождения находится под влиянием естественно неблагоприятного Марса (травмы, ушибы, порезы, хирургическая операция) и является хозяином восьмого дома (критическая ситуация, травмы, хирургическая операция), поэтому он, возможно, в этот период перенесет хирургическую операцию.

Рассмотрим подпериод Луны в главном периоде Меркурия (возраст 16--17лет). Луна расположена в карте рождения в пятом доме (образование), является хозяином девятого дома (удача, путешествие) и находится под аспектом естественно благоприятной Венеры. В лунной карте Луна является хозяином пятого дома (образование). В главный период Меркурия и подпериод Луны человек поступил учиться в высшее учебное заведение. Луна находится в девятом доме (путешествие, удача) от Меркурия и является хозяином первого дома (тело, личность), поэтому в главный период Меркурия и подпериод Луны человек испытывал удовольствие от путешествий и имел хорошее здоровье.

Рассмотрим подпериод Сатурна в главном периоде Меркурия (возраст 23--25 лет). Сатурн в карте рождения расположен в двенадцатом доме (расходы, потери), а в лунной карте --- в восьмом доме (трансформация).

Сатурн находится под влиянием Кету (препятствия, помехи) и Юпитера (дети). В лунной карте Сатурн аспектирует пятый дом (дети), где располагается Меркурий, значит, в главный период Меркурия и подпериод Сатурна произойдет рождение ребенка, что будет сопровождаться большими расходами и трудностями семейной жизни. Сатурн (смерть) стоит в восьмом доме (смерть), что предполагает утрату кого--либо из родных.

В карте рождения Сатурн находится в двенадцатом доме (утрата) и является хозяином четвертого дома (родственники), что подтверждает вероятность смерти кого--либо из родственников. Об этом свидетельствует также и то, что в лунной карте Сатурн является хозяином двенадцатого дома (утрата, потеря). Если учесть, что Меркурий хозяин восьмого дома (смерть, трансформация), которыей несет в себе силы главного периода, то вполне очевидно, что в подпериод Сатурна должна произойти трансформация, которая будет связана со смертью родственника.

\subsubsection*{Гороскоп женщины}

\natal[%
	asc=8,
	one=ЮПИТЕР,
	two=САТУРН,
	three=ЛУНА,
	five=СОЛНЦЕ\\меркурий\\КЕТУ,
	six=ВЕНЕРА,
	eight=МАРС,
	eleven=РАХУ
]{}

Рассмотрим подпериод Марса в главном периоде Луны (возраст 3 года). Марс расположен в восьмом доме (критическая ситуация, травма, и\,т.\,д.) и является хозяином первого (тело) и шестого (болезнь, боль, травма) домов. По своей природе эта планета неблагоприятна и несет идеи, связанные с разрушением. Марс испытывает влияние от естествнно неблагоприятного Сатурна (горе, неудача). Луна как планета главного периода находится под аспектом естественно неблагоприятного Марса (травма, физическая боль), поэтому в главный период Луны подпериод Марса у девочки была травмирована голова. Марс находится в шестом доме (болезнь) от Луны, что является указанием на преодоление физической боли.

Рассмотрим подпериод Раху в главном периоде Луны (возраст 4--5лет). Раху находится в одинадатом доме (друзья, надежды) в карте рождения и в девятом доме (путешествие) в лунной карте. Раху испытывает влияния от Марса, Сатурна, Солнца и Меркурия. Раху в соответствии со своей природой сначала заключает в себя, а потом освобождает и управляет путешествиями. Луна как планета главного периода находится под аспектом Раху, и поэтому в главный период Луны и подпериод Раху произошел переезд в другой город.

Рассмотрим подпериод Сатурна в главном периоде Луны (возраст 8--9 лет). Сатурн расположен во втором доме (семья) в карте рождения и в двенадцатом доме (упадок, изоляция, больница) в лунной карте. Он находится под влиянием естественно неблагоприятного Марса (физическая боль). В главный период Луны и подпериод Сатурна женщина в детском возрасте болела, но так как Луна является хозяином девятого дома (удача), она быстро поправилась, хотя трудности в семье ощущались (Сатурн во втором доме в карте рождения).

На примерах--гороскопах мы описали метод, с помощью которого вы сможете добиться хороших результатов в предсказании будущих событий. Если вы будете знать вышеприведенные правила напамять, это облегчит не только понимание гороскопа в целом, но также приведет вас к знанию высшего закона Кармы.
