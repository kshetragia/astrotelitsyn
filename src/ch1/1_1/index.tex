\section{Расчет гороскопа рождения}

Мы не будем утруждать читателя углублением в астрономию, так как об этой области знаний достаточно много написано, а также и потому, что сами по себе знания астрономии вовсе не являются гарантом удачных астрологических предсказаний. Большинство астрологов, практикующих Индийскую предсказательную астрологию, всю рутинную, кропотливую работу, связанную с расчетом гороскопа, доверяют компьютеру, используя для этого специальные программы, а всю мощь своего интеллекта и подсознания направляют на решение чисто астрологических задач --- интерпретацию полученных данных и предсказания.

Древние риши (видящие) использовали для определения судьбы человека 9 планет, 12 знаков Зодиака, 12 домов, 27 лунных созвездий.

9 планет --- Солнце, Луна, Марс, Меркурий, Юпитер, Венера и Сатурн --- это реально существующие небесные тела, а также Раху\footnote{Раху --- восходящий северный узел} и Кету\footnote{Кету --- нисходящий южный узел.}, которые являются математически вычисленными точками пересечения орбит Луны и Земли\footnote{В астрологии названия планет не соответствуют астрономическим понятиям о небесных телах (например, Солнце --- это звезда, Луна --- спутник) --- примеч.\,ред.}

\begin{table}[tph!]
	\caption{Символические обозначения 12 знаков зодиака.}
	\label{tbl:signs}

	\centering

	% Расширить по вертикали
	\renewcommand{\arraystretch}{2}

	% Разширить вширь
	\setlength{\tabcolsep}{.05\textwidth}

	% Заполним данными
	\begin{tabular}{lll}
		\aries\ --- Овен      & \leo\ --- Лев    & \sagittarius\ --- Стрелец \\
		\taurus\ --- Телец    & \virgo\ --- Дева & \capricornus\ --- Козерог \\
		\gemini\ --- Близнецы & \libra\ --- Весы & \aquarius\ --- Водолей \\
		\cancer\ --- Рак      & \scorpio\ --- Скорпион & \pisces\ --- Рыбы \\
	\end{tabular}
\end{table}

\begin{table}[tph!]
	\caption{Принятые в астрологии символы для обозначения планет.}
	\label{tbl:planets}

	\centering

	% Расширить по вертикали
	\renewcommand{\arraystretch}{2}

	% Разширить вширь
	\setlength{\tabcolsep}{.05\textwidth}

	% Заполним данными
	\begin{tabular}{lll}
		\astrosun\ --- Солнце & \mercury\ --- Меркурий & \saturn\ --- Сатурн \\
		\fullmoon\ --- Луна   & \jupiter\ --- Юпитер   & \ascnode\ --- Раху \\
		\mars\ --- Марс       & \venus\ --- Венера     & \descnode\ --- Кету
	\end{tabular}
\end{table}

Для составления гороскопа нужно провести расчет координат 9 планет и восходящего знака Зодиака в момент рождения человека, который включает:
\begin{myitem}
	\item Дату рождения
	\item Точное время рождения
	\item Место рождения
\end{myitem}

Если человек не знает своего точного времени рождения, он может обратиться в родильный дом, где в книге регистрации новорожденных данные о времени рождения хранятся в течение 75 лет.

Если у вас нет специальной компьютерной программы, то для расчета необходимо также иметь следующую литературу:
\begin{myenum}
	\item Эфемериды --- астрономические таблицы, по которым определяются положения планет в знаках зодиака в момент рождения.
	\item Таблицы домов Плацидуса (или Коха) для определения координат восходящего знака Зодиака в момент рождения.
	\item Справочник ``Координаты городов и изменения исчисления времени на территории СССР с 1917 по 1992 год''. (Воронеж,\,1992)
\end{myenum}

Эфемериды и таблицы домов Плацидуса (или Коха) можно приобрести в розничной книжной торговле или взять в библиотеке.

Существуют две модели Солнечной системы: гелеоцентрическая и геоцентрическая. Гелеоцентрическая модель Солнечной системы --- это та модель, которая реально существует в космическом пространстве: вокруг Солнца движутся планеты. Все астрономические открытия делались на основе гелеоцентрической модели мироздания.

Астрология рассматривает движения планет в основе геоцентрической модели Солнечной системы, то есть в центр Солнечной системы условно ставится Земля и при этом движение планет относительно земного наблюдателя происходит вокруг Земли против часовой стрелки.

Древние риши определили, что весь спектр идей, относящихся к человеку и его деятельности, представлен девятью планетами, поэтому в Индийской предсказательной астрологии используются знания только об этих планетах.

Планеты бывают благоприятными и неблагоприятными по своей природе. Юпитер и Венера --- благоприятные, Сатурн, Марс, Раху, Кету и Солнце --- неблагоприятные. Меньше всего зла исходит от Солнца. Луна и Меркурий считаются нейтральными по своей природе и могут быть благоприятными и неблагоприятными в зависимости от их расположения на карте рождения. Луна растущая благоприятна, убывающая --- наоборот, неблагоприятна. Меркури, находящийся в знаке Зодиака один или в связи с благоприятной планетой, будет благоприятным. Если он находится в одном знаке Зодиака с какой-либо неблагоприятной планетой, то тоже становится неблагоприятным.

Планеты разделены на мужские, женские и нейтральные.

\begin{table}[tph!]
	% Расширить по вертикали
	\renewcommand{\arraystretch}{1}

	% Заполним данными
	\begin{tabular}{l|l|l}
		Мужские планеты & Солнце, Марс, Юпитер & активность, мужское начало \\
		Женские планеты & Луна, Венера         & мягкость, женское начало \\
		Нейтральные планеты & Меркурий, Сатурн & проявляют оба начала в соответствии \\
		                     & Раху и Кету      & с полом знака Зодиака \\
	\end{tabular}
\end{table}


Планеты кроме Раху и Кету соответствуют семи дням недели:

\begin{table}[tph!]
	% Расширить по вертикали
	\renewcommand{\arraystretch}{1}

	% Заполним данными
	\begin{tabular}{ll}
		Воскресенье & Солнце \\
		Понедельник & Луна \\
		Вторник     & Марс \\
		Среда       & Меркурий \\
		Четверг     & Юпитер \\
		Пятница     & Венера \\
		Суббота     & Сатурн \\
	\end{tabular}
\end{table}

В Индийской предсказательной астрологии каждая планета несет определенные идеи в жизнь человека. Знание их необходимо для правильного прочтения карты рождения.

\begin{myenum}[topsep=0]
	\item \textbf{Солнце:}
		\begin{mydescr}
			\item[Физиология] --- сердце, мозг, голова, кости, голосовые связки, правый глаз.
			\item[Направление] --- восток.
			\item[Идея] --- индивидуальная душа (эго), отец, представители власти, способность к самоосознанию, статус, уверенность, благородство, слава, сознание, правда, истина, храбрость, решительность, энергия, политика, химия, медицина, врачи, хирурги, целители, храмы и места поклонений, места жертвоприношений, залы для коронования, осветительные приборы, высокий пост в правительстве.
		\end{mydescr}
	\item \textbf{Луна:}
		\begin{mydescr}
			\item[Физиология] --- молочные железы, матка, кровообращение, левый глаз.
			\item[Направление] --- северо-запад.
			\item[Идея] --- мать, женщина, известность, имя, ум, вода, океан, море, реки, путешествие и путешественники, лекарственные растения, сочные плоды, женственность, фантазии, перемены, молоко, популярные курорты, фармацевты и аптекари, зеркало, рыболов, грациозность, чувства.
		\end{mydescr}
	\item \textbf{Марс:}
		\begin{mydescr}
			\item[Физиология] --- нос, мышечная ткань, сухожилия, половые органы.
			\item[Направление] --- юг.
			\item[Идея] --- братья и сестры, решительность, твердость, действия, благосостояние, недвижимость, энергия, сила, земельный участок, пожар, военные, военная операция, режущие инструменты, инженеры, математика, электроника, калькулятор, хирурги, катастрофы, горы, леса, тяжба, спор, спорт, землетрясение, раны и травмы, жар, сапфиры, кораллы, сокровища, химическая лаборатория, война, битва, соревнование, строительство, сила духа, холодное и огнестрельное оружие, ожоги, должность в армии и полиции, конструктор, хирургическая операция, разрыв, разрушение, несчастный случай.
		\end{mydescr}
	\item \textbf{Меркурий:}
		\begin{mydescr}
			\item[Физиология] --- легкие, кишечник, брюшная полость, язык, кисти рук, нервные центры.
			\item[Направление] --- север.
			\item[Идея] --- интеллект, интеллектуальная деятельность, речь, литературные способности, средства информации и связи, рационализм, секретарская работа, бухгалтер, ученый, учебные заведения, головная и желудочная боль, рекламные агенства, издательство и издатели, исследование, новости, коммерческая деятельность, друзья.
		\end{mydescr}
	\item \textbf{Юпитер:}
		\begin{mydescr}
			\item[Физиология] --- печень, нижняя часть брюшной полости, органы слуха, таз.
			\item[Направление] --- северо-восток.
			\item[Идея] --- поведение, красноречие, мудрость, философия, духовный рост, религия, духовный наставник, учитель, ведические знания, набожность, религиозные учреждения, благотворительность, профессор, склонность к полноте, синтез, аргументация, получивший высокую оценку, дети, банки и банкиры, легальная деятельность, министры, юристы, справедливость, почтенные и уважаемые люди, советники, священники, беременные женщины, муж, поддерживающий жизнь.
		\end{mydescr}
	\item \textbf{Венера:}
		\begin{mydescr}
			\item[Физиология] --- почки, яичники, выделения, половая система.
			\item[Направление] --- юго-восток
			\item[Идея] --- желания, привязанность, жадность, ревность, праздность, красота, живописные места, текстильная продукция, люди искусства, товары ``люкс'', спиртные напитки, окружающая среда, спальня, комфорт, сексуальные удовольствия, венерические болезни, парфюмерия, транспортные средства, предметы роскоши, цветы, ботаника, творческие способности, муж, жена, любовник и любовница, украшения, декорации, театры, музеи, кино, сладости, ювелирные украшения.
		\end{mydescr}
	\item \textbf{Сатурн:}
		\begin{mydescr}
			\item[Физиология] --- ноги, колени, костный мозг, мочевой пузырь, дыхательная система, принцип передвижения.
			\item[Направление] --- запад
			\item[Идея] --- время, вызывающеее изменения, смерть, лишения, результат длительных действий, аскетизм, горе, печаль, подлость, обман, воры, мошенники, слуги, работа в подчинении кого-то, рабочий, самоотречение, пожилые люди, уход в отставку, места захоронения, судебный исполнитель, нищий, настенные или настольные часы, отсрочка, заблуждение, ложный вывод, уголовный розыск, черная магия, тайные знания, практика йоги, изменение обстоятельств, разорение и крах, предательство, сила влияния на других, нефть, железо, люди низкого происхождения, должности, получаемые за услуги, консерватизм, сфера услуг, препятствие.
		\end{mydescr}
	\item \textbf{Раху:}
		\begin{mydescr}
			\item[Физиология] --- скулы, кожа, выделительная система, глотание, пищеварительный тракт, прямая кишка.
			\item[Направление] --- юго-запад
			\item[Идея] --- страсть, толпа, мятеж, восстание, ядовитые рептилии, бедствие, наводнение, люди низкой культуры, воры и мошенники, гнев, гордыня, тупость, глупость, мистические знания, таинства, тайная доктрина, грубость, путешествия, муравейник, злое предсказание, эпидемии, насилие, коррупция, незаконные действия, охотники, тяжба, инфекционные заболевания, припадок, удушье, способность, оказывать влияние на других, лишения, крупное воровство.

Раху действует подобно Сатурну и провляет энергии тех планет, которые оказывают на нее влияние.
		\end{mydescr}
	\item \textbf{Кету:}
		\begin{mydescr}
			\item[Физиология] --- позвоночник, спинно-мозговой канал, нервная система, половая система.
			\item[Направление] --- направления не имеет/
			\item[Идея] --- препятствия и помехи, несчастный случай, духовные силы, психическое состояние, астрология, банкротство, клевета, оккультизм, падение, болезнь и недомогание, огонь, математика и мат. способности, эпидемии, страх, нервозность, испуг, яд, больничная палата, мелкое воровство, неожиданность, рана.

Кету действует подобно Марсу и проявляет энергии тех планет, которые оказывают на нее влияние.
		\end{mydescr}
\end{myenum}

Главные идеи девяти планет очень важно знать напамять при анализе карты рождения. А также следует не забывать, что сильные и удачно расположенные планеты в карте рождения проявят свои хорошие качества, а слабые и неудачно расположенные планеты --- дадут плохой результат.

\subsection{Определение силы планет}

В момент рождения человека планеты находятся в различных знаках Зодиака и в зависимости от этого проявляют свою силу или слабость.o

Существует множество способов для определения силы планет с огромным количеством вычислений и составлением дополнительно шестнадцати различных карт, аналогичных карте навамса, о которой будет рассказано далее. Однако на практике большинство астрологов пользуется для этого следующими правилами:
\begin{myenum}[itemsep=0,parsep=0]
	\item Планета в знаке экзальтации.\label{cost:1}
	\item Планета в мулатриконе.\label{cost:2}
	\item Планета в собственном знаке.\label{cost:3}
	\item Планета в знаке большого друга.\label{cost:4}
	\item Планета в дружественном знаке.\label{cost:5}
	\item Планета в нейтральном знаке.\label{cost:6}
	\item Планета во враждебном знаке.\label{cost:7}
	\item Планета в знаке большого врага.\label{cost:8}
	\item Планета в знаке ослабления.\label{cost:9}
\end{myenum}

Данные девять достоинств расположены в порядке убывания силы планет. Планета имеет наибольшую силу, если находится в знаке экзальтации, тогда она проявляет лучшие свои качества и приносит человеку счастье, удачу и доход. В знаке ослабления планеты теряют силы и сулят человеку горе, страдания и убытки. Промежуточные достоинства планет дают неоднозначный результат, в них проявляется как хорошее, так и плохое. Но чем выше достоинство планеты (мулатрикона, собственный знак, знак большого друга), тем сильнее в ней доброе начало, меньше плохих качеств. И наоборот, чем ниже достоинство планеты (враждебный знак или знак большого врага), тем больше будет проявляться зло и меньше хороших качеств.

Достоинство планет по пунктам \ref{cost:1}, \ref{cost:2}, \ref{cost:3} и \ref{cost:9} легко определить по таблице~\ref{tbl:cost}, где перечислены знаки, которые являются экзальтирующими, знаками мулатриконы, собственными и ослабляющими для каждой планеты.

Сначала надо посмотреть, в каком знаке Зодиака эта планета расположена в карте рождения, а затем по таблице определить, какое достоинство этот знак придает соответствующей планете. Например, если в карте рождения Юпитер находится в Рыбах, значит Юпитер в собственном знаке. Возле каждого знака экзальтации проставлено значение градуса (например, Овен \(10^\circ\)) --- это значит, что если планета в знаке находится именно в таком градусе, она наиболее сильная (то есть это градус наивысшей экзальтации планеты в знаке). Аналогично для знака ослабления --- указанное значение градуса определяет точку наибольшего ослабления соответствующей планеты в данном знаке.

\begin{table}[tph!]
	\caption{Определение достоинств планет}
	\label{tbl:cost}

	\centering

	% Расширить по вертикали
	\renewcommand{\arraystretch}{1}

	% Разширить вширь
	%\setlength{\tabcolsep}{.05\textwidth}

	% Заполним данными
	\begin{tabular}{|l|l|l|l|l|}
		\hline
		Планета & Знак экзальтации & Знак мулатрикона & Собственный знак & Знак ослабления \\
		\hline
		Солнце   & Овен \gradus{10}    & Лев \gradus{0}--\gradus{20}     & Лев & Весы \gradus{10}    \\
		Луна     & Телец \gradus{3}    & Телец \gradus{4}--\gradus{30}   & Рак & Скорпион \gradus{3} \\
		Марс     & Козерог \gradus{28} & Овен \gradus{1}--\gradus{12}    & Овен, Скорпион   & Рак \gradus{28}    \\
		Меркурий & Дева \gradus{15}    & Дева \gradus{16}--\gradus{20}   & Близнецы, Дева   & Рыбы \gradus{15}   \\
		Юпитер   & Рак \gradus{5}      & Стрелец \gradus{1}--\gradus{10} & Стрелец, Рыбы    & Козерог \gradus{5} \\
		Венера   & Рыбы \gradus{27}    & Весы \gradus{0}--\gradus{5}     & Телец, Весы      & Дева \gradus{27}   \\
		Сатурн   & Весы \gradus{20}    & Водолей \gradus{1}--\gradus{20} & Козерог, Водолей & Овен \gradus{20}   \\
		Раху     & Телец               & Близнецы & Водолей & Скорпион \\
		Кету     & Скорпион            & Стрелец  & Лев & Телец \\
		\hline
	\end{tabular}
\end{table}

Возле каждого знака мулатрикона также проставлены значения градусов (например, Лев \(0{^\circ}-20^\circ\)) --- это значит, что достоинство или сила мулатрикона действует в пределах указанных градусов, а в остальном пространстве знака планета будет иметь достоинство знака экзальтации или собственного знака в соответствии с таблицей\footnote{Некоторые индийские астрологические школы указывают позицию мулатрикона для Венеры от \gradus{0} Весов до \gradus{15} Весов.}.

Солнце и луна имеют по одному собственному знаку, а остальные пять планет по два. Индийские астрологи считают, что Раху и Кету не имеют собственных знаков, так как они не обладают физическими телами, но их расположение в знаках Водолея и Льва придает им силу, соответствующую достоинству собственного знака.

По терминологии, принятой в Индии, планета является хозяином собственного знака, то есть если Лев --- собственный знак для Солнца, то Солнце --- хозяин Льва и\,т.\,д.

Для того, чтобы определить достоинства планет по пунктам \ref{cost:4}, \ref{cost:5}, \ref{cost:6}, \ref{cost:7} и \ref{cost:8}, то есть в знаке большого друга, дружественном знаке, нейтральном знаке, враждебном знаке и знаке большого врага, необходимо ввести дополнительные термины, такие как \emph{постоянные} отношения и \emph{временные} отношения между планетами.

Каждая планета, кроме Раху и Кету, имеет дружественные, нейтральные или враждебные отношения с другими планетами. Их можно установить с помощью приведенной таблицы~\ref{tbl:relations}, отражающие постоянные отношения между планетами. Из этой таблицы видно, что для Солнца дружественными являются Луна, Марс и Юпитер, нейтральным --- Меркурий, а враждебными --- Сатурн и Венера. И так для каждой планеты, находящейся в левой колонке. Отношения между планетами устанавливаются через знаки, хозяевами которых они являются. Например, если Солнце находится в Раке, а хозяйкой Рака является Луна, которая дружественных отношениях с Солнцем, Значит Солнце находится в дружественном знаке. Таким образом определяют постоянные отношения для каждой планеты в гороскопе.

\begin{table}[tph!]
	\caption{Характер отношений между планетами}
	\label{tbl:relations}

	\centering

	% Расширить по вертикали
	\renewcommand{\arraystretch}{1}

	% Разширить вширь
	%\setlength{\tabcolsep}{.05\textwidth}

	% Заполним данными
	\begin{tabular}{|l|l|l|l|}
		\hline
		Планета & Дружественные & Нейтральные & Враждебные \\
		\hline
		Солнце   & Луна, Марс, Юпитер & Меркурий & Сатурн, Венера \\
		Луна     & Солнце, Меркурий & Марс, Юпитер, Венера, Сатурн & --- \\
		Марс     & Солнце, Луна, Юпитер & Венера, Сатурн & Меркурий \\
		Меркурий & Солнце, Венера & Марс, Юпитер, Сатурн & Луна \\
		Юпитер   & Солнце, Луна, Марс & Сатурн & Меркурий, Венера \\
		Венера   & Меркурий, Сатурн & Марс, Юпитер & Солнце, Луна \\
		Сатурн   & Меркурий, Венера & Юпитер & Солнце, Луна, Марс \\
		\hline
	\end{tabular}
\end{table}

Временные отношения между планетами определяются по их взаимному расположению в домах. Они бывают двух типов --- временная дружба и временная вражда. Если одна планета расположена относительно другой во 2, 3, 4, 10, 11 или 12-м доме, значит эти две планеты находятся во временных дружественных отношениях. Если две планеты находятся в одном доме или одна планета расположена относительно другой в 5, 6, 7, 8 или 9-м доме, значит эти две планеты находятся во временных враждебных отношениях. Надо обратить внимание на то, что при установлении временных отношений между планетами за первый дом принимается тот дом, где находится планета, достоинство которой вы определяете.

Установив постоянные и временные отношения между планетами в гороскопе, можно приступить к определению достоинств планет в соответствии с пунктами \ref{cost:4}, \ref{cost:5}, \ref{cost:6}, \ref{cost:7} и \ref{cost:8} из правила определения силы планет по девяти достоинствам:

\begin{myenum}[itemsep=0,parsep=0]
	\item Постоянные дружественные отношения + дружба временная = планета в знаке большого друга.
	\item Постоянные нейтральные отношения + дружба временная = планета в дружественном знаке.
	\item Постоянные враждебные отношения + дружба временная = планета в нейтральном знаке.
	\item Постоянные дружественные отношения + вражда временная = планета в нейтральном знаке.
	\item Постоянные нейтральные отношения + вражда временная = планета во враждебном знаке.
	\item Постоянные враждебные отношения + вражда временная = планета в знаке большого врага. 
\end{myenum}

Покажем как определяются достоинства планет в знаках Зодиака на нескольких примерах:

\begin{myenum}
	\item Солнце расположено в Тельце, а Венера в Овне. Солнце находится во враждебном знаке, так как хозяйка Тельца --- Венера, согласно таблице~\ref{tbl:relations}, враждебна к Солнцу. Если считать Телец, в котором находится Солнце, 1-м домом, то Венера относительно Солнца находится в Овне, то есть в 12-м доме, что говорит о временной дружбе этих планет. Теперь складываем постоянные враждебные отношения с временной дружбой и получаем, что Солнце находится в нейтральном знаке.
	\item Марс расположен во Льве, хозяином которого является Солнце. Солнце дружественно Марсу и находится в Деве, то есть во 2-м доме от Марса, что указывает на временные дружественные отношения. Сложив постоянные дружественные отношения со временной дружбой, определяем Марс как находящийся в знаке большого друга.
	\item Венера находится в Раке, хозяином которого является Луна. Луна для Венеры враждебна. Луна находится в Козероге, то есть в 7-ом доме от Венеры. Луна --- временный враг Венеры. Сложив постоянные враждебные отношения со временной враждой, определяем Венеру как находящуюся в знаке большого врага.
\end{myenum}


\subsubsection*{Определение силы планет в зависимости от их расположения в знаке Зодиака.}

Вы уже знаете, что каждый знак Зодиака занимает \gradus{30} пространства в Зодиаке, а также то, что знаки делятся на мужские и женские. Все нечетные знаки --- мужские, четные --- женские.

Если планета расположена в части знака от \gradus{12} до \gradus{18}, то это местонахождение способствует максимальному проявлению силы такой планеты. Расположение планеты в первых \gradus{6} четных женских знаков или последних \gradus{6} нечетных мужских знаков значительно ослабит ее. В остальном пространстве знака планеты чувствуют себя достаточно сильными для проявления своих идей.

Если планета находится на границе между двумя знаками, то она не в состоянии полностью проявить себя, поэтому рассматривается как ослабленная.

Если планета находится не далее чем в \gradus{5} от Солнца, то она считается сгорающей, и сила ее значительно убывает. Меркурий меньше других подвержен сгоранию, так как нахождение вблизи Солнца его естественное состояние. Планеты Раху и Кету не подвержены сгоранию, поскольку не обладают физическими телами.

В ретроградном движении планета проявляет больше силы чем в прямом.

Определение силы планет --- это важнейшая тема в Индийской предсказательной астрологии, и мы будем продолжать развивать ее в следующих главах. Конечно, при первых попытках анализа гороскопа читатель столкнется с некоторыми трудностями в определении силы планет, так как здесь нет четкого выражения относительных сил в процентах, а самих признаков силы и слабости планет достаточно много, но в дальнейшем у вас появится опыт. Практикующий астролог чувствует силу планет даже по внешнему виду человека, с которым беседует. Поэтому мы еще раз подчеркиваем, что очень важно знать напамять максимальное количество информации.

\subsection{Двадцать семь лунных созвездий(накшатр) и их идеи}

Древние риши, наблюдая за движением планет, сделали великое открытие --- определили двадцать семь равных промежутков пространства в Зодиакальном поясе --- накшатр (см. табл. \ref{tbl:nakshatras}). Каждое лунное созвездие равно \coord{13}{20}{} Зодиакального пояса. Риши обнаружили, что планеты в разных частях одного и того же знака, оказывают различное воздействие. К примеру, если Солнце проходит по знаку Овен, оно дает эффекты в зависимости от того, в каком лунном созвездии находится. Три созвездия (третье неполное) в знаке Овен и дополнительные эффекты Солнца в гороскопе должны быть связаны с тем созвездием, в котором оно находилось в момент рождения человека. Солнце в созвездии Ашвини придаст одну окраску событиям и характеру человека, в Бхарани --- другую и в Криттике --- третью.

Каждое лунное созвездие представляет определенные силы природы, которые выражаются в идеях.

\begin{myenum}
	\item \textbf{Ашвини} --- накшатра транспорта. 
		\begin{mydescr}
			\item[Протяженность] --- \signum{0}{}{\aries} -- \signum{13}{20}{\aries}
			\item[Идеи] --- владелец лошадей, наездник, кавалерист, различные средства транспорта, приезждать куда-то, достигать чего-то, посещать, получать, добиваться, совершать благородные деяния, предоставлять помощь, приносить сокровища человеку, врачеватель, обоняние. вдох-выдох, носовые звуки, нечеткое произношение, избегать горестей, уйти от несчастья.
		\end{mydescr}
	\item \textbf{Бхарани} --- накшатра сдерживающего начала.
		\begin{mydescr}
			\item[Протяженность] --- \signum{13}{20}{\aries} -- \signum{26}{40}{\aries}
			\item[Идеи] --- Бхарани заключается в себя, а потом освобождает, действие сдерживания, подавление, дисциплина, самоконтроль, моральный долг, наказывать, подчинять, управлять, злой умысел, покорять, быть преданным и непоколебимым, претерпевать и страдать, нести в чреве, наполнять желудок, питать, пища, иждивенец, чувство тяжести, нанимать кого-то, наемник, война. битва, соревнование, кричать, повышать голос, достижение, завоевание. приз, большое количество, масса, водитель, близнецы.
		\end{mydescr}
	\item \textbf{Криттика} --- накшатра огня.
		\begin{mydescr}
			\item[Протяженность] --- \signum{26}{40}{\aries} -- \signum{10}{}{\taurus}
			\item[Идеи] --- война, битва, командующий, защитник, слава, знаменитый, великие деяния, огонь, аппетит, приготовление пищи, человеческая кожа, кожаные изделия, бумага, документ, белые пятна на коже. яркий, награжденный, богатство, золото, средства передвижения, темная сторона Луны, обильный, большой, усыновленный ребенок.
		\end{mydescr}
	\item \textbf{Рохини} --- накшатра восхождения.
		\begin{mydescr}
			\item[Протяженность] --- \signum{10}{}{\taurus} -- \signum{23}{20}{\taurus}
			\item[Идеи] --- восходить, карабкаться, подниматься, заниматься альпинизмом, высота, продвижение по службе, рост, развитие, рождение, производство, распространение, потомство, сажать, сеять, выращивать, воспаление, заболевание горла, кровь, красный цвет, духи, аромат, домашний скот.
		\end{mydescr}
	\item \textbf{Мригашира} --- накшатра поиска.
		\begin{mydescr}
			\item[Протяженность] --- \signum{23}{20}{\taurus} -- \signum{6}{40}{\gemini}
			\item[Идеи] --- целеустремленно искать, исследовать, находить потерянное, стремиться, достигать, ходатайствовать, очищать, предлагать девушке выйти замуж, выслеживать, дорога, тропинка, путешествие, указывать путь, проводник, лидер, вождь, охотиться, анализировать, изучать, научные исследования.
		\end{mydescr}
	\item \textbf{Ардра} --- накшатра угнетения и притеснения.
		\begin{mydescr}
			\item[Протяженность] --- \signum{6}{40}{\gemini} -- \signum{20}{}{\gemini}
			\item[Идеи] --- угнетать, подавлять, мучительный, разрушать, убивать, раздавать, слезы, печаль, боль, горечь, жадный, жестокий, охотник, влажный, мокрый, нежный, переполненный чувствами, свежий, жидкий.
		\end{mydescr}
	\item \textbf{Пунарвасу} --- накшатра обновления.
		\begin{mydescr}
			\item[Протяженность] --- \signum{20}{}{\gemini} -- \signum{3}{20}{\cancer}
			\item[Идеи] --- место поселения, жилище, родина, восстановление богатства, собственность, ремонт в доме, возвращение из путешествия, пребывание где-то, повторение, приобретать, свободный, свобода, безопасность, бесконечность, неразрывность.
		\end{mydescr}
	\item \textbf{Пушья} --- накшатра цветения
		\begin{mydescr}
			\item[Протяженность] --- \signum{3}{20}{\cancer} -- \signum{16}{40}{\cancer}
			\item[Идеи] --- цветение, цветок, насыщение, процветать, увеличение, приобретать, получать сполна, жирность, богатство, изобилие, полнота, счастливый, благоприятный, молитва, речь, красноречие, мудрость.
		\end{mydescr}
	\item \textbf{Ашлеша} --- цепляющаяся накшатра.
		\begin{mydescr}
			\item[Протяженность] --- \signum{16}{40}{\cancer} -- \signum{0}{}{\leo}
			\item[Идеи] --- соединение, связь, союз, сексуальное партнерство, интимный контакт, объятие, пожатие, сплетение, обвиваться, скручиваться, змея, яд, мучение, жжение, боль.
		\end{mydescr}
	\item \textbf{Магха} --- накшатра восхищения и славы.
		\begin{mydescr}
			\item[Протяженность] --- \signum{0}{}{\leo} -- \signum{13}{20}{\leo}
			\item[Идеи] --- выдающийся, великий, благородный, самый уважаемый, важный, высокий, старик, величество, высочество, могущественный, возбуждать, радовать кого-то, высокая честь, поднимать настроение, богатство, власть, щедрый, отец, родители, предки, благосостояние, совершенство.
		\end{mydescr}
	\item \textbf{Пурвапхалгуни} --- накшатра удачи.
		\begin{mydescr}
			\item[Протяженность] --- \signum{13}{20}{\leo} -- \signum{26}{40}{\leo}
			\item[Идеи] --- любовь, привязанность, страсть, любовные удовольствия, флирт, праздное развлечение, производство плодов, устранение зла, исправление, очищение, получение вознаграждения, усовершенствование, реформация, опыт, следователь, практиковать, выбирать, служить, чтить, уважать.
		\end{mydescr}
	\item \textbf{Уттарпхалгуни} --- накшатра покровителя.
		\begin{mydescr}
			\item[Протяженность] --- \signum{26}{40}{\leo} -- \signum{10}{}{\virgo}
			\item[Идеи] --- покровительство, доброта, благодетельство, Друзья детства, компаньоны. Люди, к которым обращаются за помощью. Друзья, к которым приходят излить душую Лица, оказывающие финансовую поддержку. Целители, от которых ждут облегчения болезней.
		\end{mydescr}
	\item \textbf{Хаста} --- накшатра, означающая ``сжатый кулак''
		\begin{mydescr}
			\item[Протяженность] --- \signum{10}{}{\virgo} -- \signum{23}{20}{\virgo}
			\item[Идеи] --- удерживание в руке, высмеивать, оживлять, шутка, острота, почерк, мастерство, ``золотые'' руки, превосходить кого-то в чем-то, приводить в движение, команидровать, осуществлять контроль, открывать и раскрывать, расширяться, обнажать, резать, убирать урожай, косить, жать, укладывать скирды, веселье, посвящать и освящать.
		\end{mydescr}
	\item \textbf{Читра} --- накшатра ``чудесная''
		\begin{mydescr}
			\item[Протяженность] --- \signum{23}{20}{\virgo} -- \signum{6}{40}{\libra}
			\item[Идеи] --- выдающийся, великолепный, яркоокрашенный, пестрый, пятнистый, многогранный, чудесный, замечательный, многообразие, изумление, зодчий, украшенный орнаментом, картина, чертеж, план, приковывающее внимание.
		\end{mydescr}
	\item \textbf{Свати} --- накшатра самостоятельных действий.
		\begin{mydescr}
			\item[Протяженность] --- \signum{6}{40}{\libra} -- \signum{20}{}{\libra}
			\item[Идеи] --- осознание своего ``я'', идти постоянно самому, достигающий что-то собственными силами, самоподдержва, самостоятельно овладевающий каким-то навыком, провозглашать как свое собственное, материальные блага приходят и уходят сами по себе, владение и потеря богатства происходит неконтролируемым образом, воздух, ветер, шторм.
		\end{mydescr}
	\item \textbf{Висакха} --- накшатра цели.
		\begin{mydescr}
			\item[Протяженность] --- \signum{20}{}{\libra} -- \signum{3}{20}{\scorpio}
			\item[Идеи] --- идущий прямо к достижению цели, достигнутый результат, доказанный вывод, доказательство, доктрина, догма, истреблять ради достижения какой-то цели, повреждать, причинять боль, разрушать, примирять, располагать к себе, воспевать, обожать, конечная цель, просящий милостыню.
		\end{mydescr}
	\item \textbf{Анурадха} --- накшатра, призывающая к деятельности.
		\begin{mydescr}
			\item[Протяженность] --- \signum{3}{20}{\scorpio} -- \signum{16}{40}{\scorpio}
			\item[Идеи] --- друг, союзник, сотрудник, помощник. То, что учреждается и устанавливается. Нечто утверждающееся как сила и власть. Суждение, наблюдение, познание, призыв людей к активности, действовать, будучи объединенными дружбой и общей целью. Союз с кем-то ради общей цели.
		\end{mydescr}
	\item \textbf{Джиешта} --- накшатра ``главная''
		\begin{mydescr}
			\item[Протяженность] --- \signum{16}{40}{\scorpio} -- \signum{0}{}{\sagittarius}
			\item[Идеи] --- великолепный, выдающийся, первый, величайший, тот, которого прославляют, верховная власть, могущественный правитель, провозглашать что-то, старший брат, старший по должности, глава семьи, жена, любимая, возлюбленная.
		\end{mydescr}
	\item \textbf{Мула} --- накшатра ``корневая''
		\begin{mydescr}
			\item[Протяженность] --- \signum{0}{}{\sagittarius} -- \signum{13}{20}{\sagittarius}
			\item[Идеи] --- иметь корни, самая низкая часть чего-нибудь, корень растения, начало, причина, источник знаний, главный город илил столица, оригинальный текст, первоначальный, исходный, беспрецендентный, связывать, привязывать, фиксировать, удерживать, брать в плен, связь, облигация, сдерживать, подавлять, угнетать, неагрессивный, наносящий обиду.
		\end{mydescr}
	\item \textbf{Пурвашадха} --- накшатра ``непобедимая''
		\begin{mydescr}
			\item[Протяженность] --- \signum{13}{20}{\sagittarius} -- \signum{26}{40}{\sagittarius}
			\item[Идеи] --- победоносный, побеждать, превосходить, завоевание и поражение, насилие, воспротивиться чему-то, посстать, сопротивляться, переносить страдания и муки, с пониманием относиться к любому человека, терпеливо ждать подходящего времени, всепрощение, снисхождение, завершение, прийти к концу, поверхность, вода.
		\end{mydescr}
	\item \textbf{Уттарашадха} --- накшатра ``всеобщая''
		\begin{mydescr}
			\item[Протяженность] --- \signum{26}{40}{\sagittarius} -- \signum{10}{}{\capricornus}
			\item[Идеи] --- входить, поселяться, проникать, быть поглощенным, входить в соединение, входить в дом, появляться на сцене, отдыхать, возникновение мысли, принадлежать чему-то, существовать ради кого-то принимать на себя, начинать что-то помнить о каком-то деле, заставлять войти, быть причиной вхождения, универсальность, всеобщность.
		\end{mydescr}
	\item \textbf{Шравана} --- накшатра изучения.
		\begin{mydescr}
			\item[Протяженность] --- \signum{10}{}{\capricornus} -- \signum{23}{20}{\capricornus}
			\item[Идеи] --- тот, кто учился, научный работник, знание, изучение, быть внимательным и послушным, интеллектуальные способности, быть известным и почитаемым, быть услышанным, рассказывать, общаться, священные знания, откровения, таинства, слухи, сообщения новосте, слова и языки, словарный запас, заботиться о ком-то, ученик, последователь, учитель, брать с кого-то пример, ливень, выделения.
		\end{mydescr}
	\item \textbf{Дханишта} --- накшатра симфонии.
		\begin{mydescr}
			\item[Протяженность] --- \signum{23}{20}{\capricornus} -- \signum{6}{40}{\aquarius}
			\item[Идеи] --- музыка, пение, нота, звучание, богатство, драгоценные камин, драгоценности, всё что высоко ценится, влага, потеть, задняя или боковая часть чего-то.
		\end{mydescr}
	\item \textbf{Сатабиша} --- накшатра прикрытия.
		\begin{mydescr}
			\item[Протяженность] --- \signum{6}{40}{\aquarius} -- \signum{20}{}{\aquarius}
			\item[Идеи] --- скрывать, прятать, быть захваченным, удерживать в плену, иметь препятствия, океан, море, озеро, река, пруд, дожди, резервуары с водой, защитник от зла, вооружение, вся верхняя одежда, целитель, врач, лечение, нахождение лечебного средства, неизлечимые заболевания, паралич, водянка, западня.
		\end{mydescr}
	\item \textbf{Пурвабхадрапада} --- накшатра ``пара бешено несущихся лошадей''.
		\begin{mydescr}
			\item[Протяженность] --- \signum{20}{}{\aquarius} -- \signum{3}{20}{\pisces}
			\item[Идеи] --- нестись во всю прыть, гореть, сжигать, пылкий, страстный, порывистый, импульсивный, стремительный, жгучая боль, наказывать, подвергать телесному наказанию, мучать кого-то, угнетать, подавлять, ущемлять, печаль, горе, обида, падать, погибать, уходить.
		\end{mydescr}
	\item \textbf{Уттарабхадрапада} --- накшатра подобна Пурвабхадрападе. Если Пурвабхадрапада вызывает гнев, то Уттарабхадрапада дает силы его контролировать.
		\begin{mydescr}
			\item[Протяженность] --- \signum{3}{20}{\pisces} -- \signum{16}{40}{\pisces}
			\item[Идеи] --- уезжать куда-то, оставлять все дома, личность, знания, мудрость, отерчение. остальные идеи такие же, как у Пурвабхадрапады.
		\end{mydescr}
	\item \textbf{Ревати} --- накшатра указывает на того, кто содержит стадо овец.
		\begin{mydescr}
			\item[Протяженность] --- \signum{16}{40}{\pisces} -- \signum{0}{}{\apies}
			\item[Идеи] --- питание, кормление, выращивать, богатый, процветающий, роскошный, обильный.
		\end{mydescr}
\end{myenum}

К толкованию планет в лунных созвездиях карты рождения надо подходить очень осторожно, так как не все астрологи, практикующие Индийскую предсказательную астрологию, используют идеи накшатр. Для определения будущих событий вполне достаточно сделать анализ и синтез идей планет в знаках и домах гороскопа. Но если вы хотите знать детали этих событий, вы должны аккуратно подойти к использованию идей лунных созвездий. Индийские астрологи были подлинными астерами этого высокого искусства.

