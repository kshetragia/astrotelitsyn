\subsection{Порядок расчета гороскопа}

Чтобы легче усваивалась практика расчета, мы покажем его на примере:
\begin{mylist}
	\item Дата рождения --- 4 декабря 1957г.
	\item Время рождения --- \timeshort{12}{35}{} по местному времени.
	\item Место рождения --- г. Харьков.
\end{mylist}

Прежде всего надо привести местное время в соответствие с Гринвичем. Для этого берем упомянутый справочник ``Координаты гордов...'' и на странице 32 находим город Харьков. В колонке ``Разница с Гринвичем в часах'' указано --- 3 часа. Так как Харьков находится в восточном полушарии, надо от местного времени вычесть 3 часа:
\quotenr{\timeshort{12}{35}{} - \timeshort{3}{}{} = \timeshort{9}{35}{}}

9 часов 35 минут --- это время рождения по Гринвичу.

Если нет под рукой справочника, вы определяете по географическому атласу, что Харьков расположен во втором часовом поясе. Учитывая, что с 16 июня 1930 года по 31 марта 1991 года по всей территории СССР действовал декретный час, к номеру часового пояса прибавляем декретный час и находим разницу с Гринвичем:
\quotenr{\timeshort{2}{}{} + \timeshort{1}{}{} = \timeshort{3}{}{}}

Далее выполняем уже знакомое действие:
\quotenr{\timeshort{12}{35}{} - \timeshort{3}{}{} = \timeshort{9}{35}{} (время рождения по Гринвичу)}

Теперь можно делать расчет координат девяти планет: Солнца, Луны, Марса, Меркурия, Юпитера, Венеры, Сатурна, Раху и Кету. В индийской астрологии транссатурнове планеты не рассматриваются, так как их энергии присутствуют в вышеперечисленных 9 планетах. Для расчета мы будем использовать эфемериды, составленные по Гринвичу на ноль часов каждого дня (Все дальнейшие расчеты производятся по Гринвичу).

Начнем с расчета координат Солнца в нашем примере. Человек родился 4 декабря 1957 года. Откроем страницу эфемерид, на которой указаны координаты планет 4 и 5 декабря 1957 года:
\begin{mylist}
	\item 4 декабря --- \coord{11}{34}{40} Стрельца
	\item 5 декабря --- \coord{12}{35}{32} Стрельца
\end{mylist}

Будем находить координаты Солнца в соответствии со временем рождения --- 9 часов 35 минут. При этом мы должны помнить, что в сутках 24 часа, в часе 60 минут, а в минуте 60 секунд. Математическое действие сводится к решению простой пропроции:

\calc{X = \dfrac{a - b}{24} * t + b} где:

\begin{mylist}[topsep=0]
	\item $a$ --- координаты Солнца 5 декабря 1957 года,
	\item $b$ --- координаты Солнца 4 декабря 1957 года,
	\item $t$ --- время рождения по Гринвичу,
	\item $24$ --- количество часов в сутках,
	\item $X$ --- координаты Солнца в момент рождения.
\end{mylist}

{\parindent=0pt Применяя наши данные, получим:}
\calc{\dfrac{\coord{12}{35}{32} - \coord{11}{34}{40}}{24} * \timemath{9}{35}{} + \coord{11}{34}{40} = \coord{11}{58}{58}} 

Стрельца --- это координаты Солнца, соответствующие времени рождения (\timeshort{9}{35}{}) по Гринвичу. Другими словами, можно сказать, что в момент рождения Солнце находилось в \coord{11}{58}{58} в знаке Зодиака Стрелец.

Далее таким же образом производим расчет координат для Марса, Меркурия, Юпитера, Венеры, Сатурна и Раху. Кету в эфемеридах не обозначена, но она всегда находится в $180^\circ$ от Раху. Поэтому, имея координаты Раху, легко можно определить координаты Кету. Для этого достаточно знать, что Овен расположен в $180^\circ$ от Весов, Телец --- от Скорпиона, Близнецы --- от Стрельца, Рак --- от Козерога, Лев --- от Водолея и Дева --- от Рыб.

В нашем примере мы расчитали, что Раху находится в \coord{17}{03}{00} Весов. В $180^\circ$ от Весов расположен знак Овна. Следовательно, координаты Кету --- \coord{17}{03}{00} Овна.

В эфемеридах координаты Луны расчитаны на полночь и полдень каждого дня, что составляет разницу в 12 часов, поэтому при расчете Луны в соответствии со временем рождения по Гринвичу надо брать во внимание не 24, а 12 часов. Если время рождения по Гринвичу будет в промежутке от полудня до полуночи, то надо от этого времени вычесть 12 часов, а потом с учетом полученного времени сделать расчет координат Луны. Для удобства выпишем данные расчетов координат девяти планет в отдельно:

\begin{table}[tph!]
	\centering

	% Расширить по вертикали
	\renewcommand{\arraystretch}{1.5}

	% Заполним данными
	\begin{tabular}{|ll|}
		\hline
		Солнце   & \coord{11}{58}{58} Стрельца \\
		Луна     & \coord{08}{07}{33} Тельца \\
		Марс     & \coord{17}{12}{08} Скорпиона \\
		Меркурий & \coord{02}{25}{15} Козерога \\
		Юпитер   & \coord{24}{19}{44} Весов \\
		Венера   & \coord{28}{04}{41} Козерога \\
		Сатурн   & \coord{16}{12}{44} Стрельца \\
		Раху     & \coord{10}{19}{38} Скорпиона \\
		Кету     & \coord{10}{19}{38} Тельца \\ \hline
	\end{tabular}
\end{table}

Может случитьcя, что во время расчета координат планет вы встретите в эфемеридах знак ``R'' в колонке под какой-либо из них. Это значит, что данная планета находится в ретроградном движении, то есть относительно наблюдателя с Земли она будет двигаться по часовой стрелке. В то время как прямое направление движения (его начала обозначается буквой ``D'' или вообще не обозначается) происходит против часовой стрелки в Зодиаке. В таком случае при расчете координат планеты, находящейся в ретроградном движении, следует применить формулу:

\calc{X = b - \dfrac{b - a}{24} * t}

Все обозначения в этой формуле такие же, как и в формуле для прямого движения планет.

Далее мы сделаем некоторые пояснения относительно аянамсы\footnote{Аянамса --- разница между координатами фиксированного и подвижного Зодиака, выраженная в градусах, минутах и секундах}. Так как западная астрология использует для прогнозирования событий подвижный Зодиак, то и таблицы эфемерид расчитаны без учета аянамсы. Индийская предсказательная астрология пользуется фиксированным Зодиаком и поэтому в своих расчетах учитывает аянамсу.


\begin{table}[tph!]
	\caption{Аянамса на 1-е января каждого года ХХ столетия}
	\label{tbl:ayanamsa}

	\centering

	% Расширить по вертикали
	\renewcommand{\arraystretch}{1.5}

	% Разширить вширь
	\setlength{\tabcolsep}{.05\textwidth}

	% Заполним данными
	\begin{tabular}{|l|l|l|}
		\hline
		1900 --- \coord{22}{27}{55} & 1934 --- \coord{22}{56}{18} & 1968 --- \coord{23}{24}{29} \\
		1901 --- \coord{22}{28}{43} & 1935 --- \coord{22}{57}{11} & 1969 --- \coord{23}{25}{25} \\
		1902 --- \coord{22}{29}{30} & 1936 --- \coord{22}{58}{04} & 1970 --- \coord{23}{26}{21} \\
		1903 --- \coord{22}{30}{15} & 1937 --- \coord{22}{58}{55} & 1971 --- \coord{23}{27}{17} \\
		1904 --- \coord{22}{30}{59} & 1938 --- \coord{22}{59}{44} & 1972 --- \coord{23}{28}{11} \\
		1905 --- \coord{22}{31}{44} & 1939 --- \coord{23}{00}{30} & 1973 --- \coord{23}{29}{04} \\
		1906 --- \coord{22}{32}{29} & 1940 --- \coord{23}{01}{16} & 1974 --- \coord{23}{29}{55} \\
		1907 --- \coord{22}{33}{15} & 1941 --- \coord{23}{02}{01} & 1975 --- \coord{23}{30}{44} \\
		1908 --- \coord{22}{34}{03} & 1942 --- \coord{23}{02}{45} & 1976 --- \coord{23}{31}{32} \\
		1909 --- \coord{22}{34}{53} & 1943 --- \coord{23}{03}{30} & 1977 --- \coord{23}{32}{17} \\
		1910 --- \coord{22}{35}{45} & 1944 --- \coord{23}{04}{16} & 1978 --- \coord{23}{33}{02} \\
		1911 --- \coord{22}{36}{39} & 1945 --- \coord{23}{05}{04} & 1979 --- \coord{23}{33}{47} \\
		1912 --- \coord{22}{37}{33} & 1946 --- \coord{23}{05}{53} & 1980 --- \coord{23}{34}{31} \\
		1913 --- \coord{22}{38}{29} & 1947 --- \coord{23}{06}{44} & 1981 --- \coord{23}{35}{17} \\
		1914 --- \coord{22}{39}{25} & 1948 --- \coord{23}{07}{38} & 1982 --- \coord{23}{36}{04} \\
		1915 --- \coord{22}{40}{21} & 1949 --- \coord{23}{08}{32} & 1983 --- \coord{23}{36}{53} \\
		1916 --- \coord{22}{41}{15} & 1950 --- \coord{23}{09}{27} & 1984 --- \coord{23}{37}{44} \\
		1917 --- \coord{22}{42}{08} & 1951 --- \coord{23}{10}{23} & 1985 --- \coord{23}{38}{37} \\
		1918 --- \coord{22}{43}{00} & 1952 --- \coord{23}{11}{20} & 1986 --- \coord{23}{39}{31} \\
		1919 --- \coord{22}{43}{49} & 1953 --- \coord{23}{12}{14} & 1987 --- \coord{23}{40}{27} \\
		1920 --- \coord{22}{44}{37} & 1954 --- \coord{23}{13}{08} & 1988 --- \coord{23}{41}{22} \\
		1921 --- \coord{22}{45}{23} & 1955 --- \coord{23}{14}{00} & 1989 --- \coord{23}{42}{18} \\
		1922 --- \coord{22}{46}{08} & 1956 --- \coord{23}{14}{50} & 1990 --- \coord{23}{43}{14} \\
		1923 --- \coord{22}{46}{52} & 1957 --- \coord{23}{15}{38} & 1991 --- \coord{23}{44}{07} \\
		1924 --- \coord{22}{47}{37} & 1958 --- \coord{23}{16}{24} & 1992 --- \coord{23}{44}{59} \\
		1925 --- \coord{22}{48}{23} & 1959 --- \coord{23}{17}{09} & 1993 --- \coord{23}{45}{50} \\
		1926 --- \coord{22}{49}{09} & 1960 --- \coord{23}{17}{54} & 1994 --- \coord{23}{46}{39} \\
		1927 --- \coord{22}{49}{58} & 1961 --- \coord{23}{18}{38} & 1995 --- \coord{23}{47}{25} \\
		1928 --- \coord{22}{50}{48} & 1962 --- \coord{23}{19}{23} & 1996 --- \coord{23}{48}{10} \\
		1929 --- \coord{22}{51}{42} & 1963 --- \coord{23}{20}{10} & 1997 --- \coord{23}{48}{55} \\
		1930 --- \coord{22}{52}{35} & 1964 --- \coord{23}{20}{58} & 1998 --- \coord{23}{49}{40} \\
		1931 --- \coord{22}{53}{30} & 1965 --- \coord{23}{21}{48} & 1999 --- \coord{23}{50}{24} \\
		1932 --- \coord{22}{54}{26} & 1966 --- \coord{23}{22}{40} & 2000 --- \coord{23}{51}{11} \\
		1933 --- \coord{22}{55}{23} & 1967 --- \coord{23}{23}{34} & \\ \hline
	\end{tabular}
\end{table}

\newpage
Выпишем из таблицы~\ref{tbl:ayanamsa} значение аянамсы на 1 января 1957 года и на 1 января 1958 года и путем решения пропорции определим значение аянамсы, соответсвующее дате рождения 4 декабря 1957 года. В нашем примере: аянамса на 1 января 1958 года --- \coord{23}{16}{24}: 

\calc{\dfrac{\coord{23}{16}{24} - \coord{23}{15}{38}}{12} * 11.2 + \coord{23}{15}{38} = \coord{23}{16}{21}} где:

\begin{mylist}[topsep=0]
	\item 12 --- количество месяцев в году
	\item 11.2 месяца --- это 11 месяцев и 4 дня, т.е. 4 декабря.
\end{mylist}

Итак, величина аянамсы в день рождения в нашем примере равна \coord{23}{16}{21}. Вычитаем эту аянамсу из координат девяти планет, приведенных выше и получаем:

\begin{table}[tph!]
	\centering

	% Расширить по вертикали
	\renewcommand{\arraystretch}{1.5}

	% Заполним данными
	\begin{tabular}{|ll|}
		\hline
		Солнце   & \coord{18}{42}{36} Скорпиона \\
		Луна     & \coord{14}{51}{12} Овна \\
		Марс     & \coord{23}{56}{00} Весов \\
		Меркурий & \coord{09}{09}{00} Стрельца \\
		Юпитер   & \coord{01}{03}{00} Весов \\
		Венера   & \coord{04}{48}{00} Козерога \\
		Сатурн   & \coord{22}{56}{00} Скорпиона \\
		Раху     & \coord{17}{03}{00} Весов \\
		Кету     & \coord{17}{03}{00} Овна \\ \hline
	\end{tabular}
\end{table}

Это позиции, или координаты планет в Зодиаке во время рождения человека в нашем гороскопе, приведенные в соответствие с Индийской предсказательной астрологией.
